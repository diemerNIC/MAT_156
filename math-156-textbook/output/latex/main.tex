%********************************************%
%*       Generated from PreTeXt source      *%
%*       on 2022-06-03T17:11:31-07:00       *%
%*   A recent stable commit (2020-08-09):   *%
%* 98f21740783f166a773df4dc83cab5293ab63a4a *%
%*                                          *%
%*         https://pretextbook.org          *%
%*                                          *%
%********************************************%
%% We elect to always write snapshot output into <job>.dep file
\RequirePackage{snapshot}
\documentclass[twoside,10pt,]{book}
%% Custom Preamble Entries, early (use latex.preamble.early)
%% Default LaTeX packages
%%   1.  always employed (or nearly so) for some purpose, or
%%   2.  a stylewriter may assume their presence
\usepackage{geometry}
%% Some aspects of the preamble are conditional,
%% the LaTeX engine is one such determinant
\usepackage{ifthen}
%% etoolbox has a variety of modern conveniences
\usepackage{etoolbox}
\usepackage{ifxetex,ifluatex}
%% Raster graphics inclusion
\usepackage{graphicx}
%% Color support, xcolor package
%% Always loaded, for: add/delete text, author tools
%% Here, since tcolorbox loads tikz, and tikz loads xcolor
\PassOptionsToPackage{usenames,dvipsnames,svgnames,table}{xcolor}
\usepackage{xcolor}
%% begin: defined colors, via xcolor package, for styling
%% end: defined colors, via xcolor package, for styling
%% Colored boxes, and much more, though mostly styling
%% skins library provides "enhanced" skin, employing tikzpicture
%% boxes may be configured as "breakable" or "unbreakable"
%% "raster" controls grids of boxes, aka side-by-side
\usepackage{tcolorbox}
\tcbuselibrary{skins}
\tcbuselibrary{breakable}
\tcbuselibrary{raster}
%% We load some "stock" tcolorbox styles that we use a lot
%% Placement here is provisional, there will be some color work also
%% First, black on white, no border, transparent, but no assumption about titles
\tcbset{ bwminimalstyle/.style={size=minimal, boxrule=-0.3pt, frame empty,
colback=white, colbacktitle=white, coltitle=black, opacityfill=0.0} }
%% Second, bold title, run-in to text/paragraph/heading
%% Space afterwards will be controlled by environment,
%% independent of constructions of the tcb title
%% Places \blocktitlefont onto many block titles
\tcbset{ runintitlestyle/.style={fonttitle=\blocktitlefont\upshape\bfseries, attach title to upper} }
%% Spacing prior to each exercise, anywhere
\tcbset{ exercisespacingstyle/.style={before skip={1.5ex plus 0.5ex}} }
%% Spacing prior to each block
\tcbset{ blockspacingstyle/.style={before skip={2.0ex plus 0.5ex}} }
%% xparse allows the construction of more robust commands,
%% this is a necessity for isolating styling and behavior
%% The tcolorbox library of the same name loads the base library
\tcbuselibrary{xparse}
%% The tcolorbox library loads TikZ, its calc package is generally useful,
%% and is necessary for some smaller documents that use partial tcolor boxes
%% See:  https://github.com/PreTeXtBook/pretext/issues/1624
\usetikzlibrary{calc}
%% Hyperref should be here, but likes to be loaded late
%%
%% Inline math delimiters, \(, \), need to be robust
%% 2016-01-31:  latexrelease.sty  supersedes  fixltx2e.sty
%% If  latexrelease.sty  exists, bugfix is in kernel
%% If not, bugfix is in  fixltx2e.sty
%% See:  https://tug.org/TUGboat/tb36-3/tb114ltnews22.pdf
%% and read "Fewer fragile commands" in distribution's  latexchanges.pdf
\IfFileExists{latexrelease.sty}{}{\usepackage{fixltx2e}}
%% shorter subnumbers in some side-by-side require manipulations
\usepackage{xstring}
%% Footnote counters and part/chapter counters are manipulated
%% April 2018:  chngcntr  commands now integrated into the kernel,
%% but circa 2018/2019 the package would still try to redefine them,
%% so we need to do the work of loading conditionally for old kernels.
%% From version 1.1a,  chngcntr  should detect defintions made by LaTeX kernel.
\ifdefined\counterwithin
\else
    \usepackage{chngcntr}
\fi
%% Text height identically 9 inches, text width varies on point size
%% See Bringhurst 2.1.1 on measure for recommendations
%% 75 characters per line (count spaces, punctuation) is target
%% which is the upper limit of Bringhurst's recommendations
\geometry{letterpaper,total={340pt,9.0in}}
%% Custom Page Layout Adjustments (use latex.geometry)
%% This LaTeX file may be compiled with pdflatex, xelatex, or lualatex executables
%% LuaTeX is not explicitly supported, but we do accept additions from knowledgeable users
%% The conditional below provides  pdflatex  specific configuration last
%% begin: engine-specific capabilities
\ifthenelse{\boolean{xetex} \or \boolean{luatex}}{%
%% begin: xelatex and lualatex-specific default configuration
\ifxetex\usepackage{xltxtra}\fi
%% realscripts is the only part of xltxtra relevant to lualatex 
\ifluatex\usepackage{realscripts}\fi
%% end:   xelatex and lualatex-specific default configuration
}{
%% begin: pdflatex-specific default configuration
%% We assume a PreTeXt XML source file may have Unicode characters
%% and so we ask LaTeX to parse a UTF-8 encoded file
%% This may work well for accented characters in Western language,
%% but not with Greek, Asian languages, etc.
%% When this is not good enough, switch to the  xelatex  engine
%% where Unicode is better supported (encouraged, even)
\usepackage[utf8]{inputenc}
%% end: pdflatex-specific default configuration
}
%% end:   engine-specific capabilities
%%
%% Fonts.  Conditional on LaTex engine employed.
%% Default Text Font: The Latin Modern fonts are
%% "enhanced versions of the [original TeX] Computer Modern fonts."
%% We use them as the default text font for PreTeXt output.
%% Default Monospace font: Inconsolata (aka zi4)
%% Sponsored by TUG: http://levien.com/type/myfonts/inconsolata.html
%% Loaded for documents with intentional objects requiring monospace
%% See package documentation for excellent instructions
%% fontspec will work universally if we use filename to locate OTF files
%% Loads the "upquote" package as needed, so we don't have to
%% Upright quotes might come from the  textcomp  package, which we also use
%% We employ the shapely \ell to match Google Font version
%% pdflatex: "varl" package option produces shapely \ell
%% pdflatex: "var0" package option produces plain zero (not used)
%% pdflatex: "varqu" package option produces best upright quotes
%% xelatex,lualatex: add OTF StylisticSet 1 for shapely \ell
%% xelatex,lualatex: add OTF StylisticSet 2 for plain zero (not used)
%% xelatex,lualatex: add OTF StylisticSet 3 for upright quotes
%%
%% Automatic Font Control
%% Portions of a document, are, or may, be affected by defined commands
%% These are perhaps more flexible when using  xelatex  rather than  pdflatex
%% The following definitions are meant to be re-defined in a style, using \renewcommand
%% They are scoped when employed (in a TeX group), and so should not be defined with an argument
\newcommand{\divisionfont}{\relax}
\newcommand{\blocktitlefont}{\relax}
\newcommand{\contentsfont}{\relax}
\newcommand{\pagefont}{\relax}
\newcommand{\tabularfont}{\relax}
\newcommand{\xreffont}{\relax}
\newcommand{\titlepagefont}{\relax}
%%
\ifthenelse{\boolean{xetex} \or \boolean{luatex}}{%
%% begin: font setup and configuration for use with xelatex
%% Generally, xelatex is necessary for non-Western fonts
%% fontspec package provides extensive control of system fonts,
%% meaning *.otf (OpenType), and apparently *.ttf (TrueType)
%% that live *outside* your TeX/MF tree, and are controlled by your *system*
%% (it is possible that a TeX distribution will place fonts in a system location)
%%
%% The fontspec package is the best vehicle for using different fonts in  xelatex
%% So we load it always, no matter what a publisher or style might want
%%
\usepackage{fontspec}
%%
%% begin: xelatex main font ("font-xelatex-main" template)
%% Latin Modern Roman is the default font for xelatex and so is loaded with a TU encoding
%% *in the format* so we can't touch it, only perhaps adjust it later
%% in one of two ways (then known by NFSS names such as "lmr")
%% (1) via NFSS with font family names such as "lmr" and "lmss"
%% (2) via fontspec with commands like \setmainfont{Latin Modern Roman}
%% The latter requires the font to be known at the system-level by its font name,
%% but will give access to OTF font features through optional arguments
%% https://tex.stackexchange.com/questions/470008/
%% where-and-how-does-fontspec-sty-specify-the-default-font-latin-modern-roman
%% http://tex.stackexchange.com/questions/115321
%% /how-to-optimize-latin-modern-font-with-xelatex
%%
%% end:   xelatex main font ("font-xelatex-main" template)
%% begin: xelatex mono font ("font-xelatex-mono" template)
%% (conditional on non-trivial uses being present in source)
\IfFontExistsTF{Inconsolatazi4-Regular.otf}{}{\GenericError{}{The font "Inconsolatazi4-Regular.otf" requested by PreTeXt output is not available.  Either a file cannot be located in default locations via a filename, or a font is not known by its name as part of your system.}{Consult the PreTeXt Guide for help with LaTeX fonts.}{}}
\IfFontExistsTF{Inconsolatazi4-Bold.otf}{}{\GenericError{}{The font "Inconsolatazi4-Bold.otf" requested by PreTeXt output is not available.  Either a file cannot be located in default locations via a filename, or a font is not known by its name as part of your system.}{Consult the PreTeXt Guide for help with LaTeX fonts.}{}}
\usepackage{zi4}
\setmonofont[BoldFont=Inconsolatazi4-Bold.otf,StylisticSet={1,3}]{Inconsolatazi4-Regular.otf}
%% end:   xelatex mono font ("font-xelatex-mono" template)
%% begin: xelatex font adjustments ("font-xelatex-style" template)
%% end:   xelatex font adjustments ("font-xelatex-style" template)
%%
%% Extensive support for other languages
\usepackage{polyglossia}
%% Set main/default language based on pretext/@xml:lang value
%% document language code is "en-US", US English
%% usmax variant has extra hypenation
\setmainlanguage[variant=usmax]{english}
%% Enable secondary languages based on discovery of @xml:lang values
%% Enable fonts/scripts based on discovery of @xml:lang values
%% Western languages should be ably covered by Latin Modern Roman
%% end:   font setup and configuration for use with xelatex
}{%
%% begin: font setup and configuration for use with pdflatex
%% begin: pdflatex main font ("font-pdflatex-main" template)
\usepackage{lmodern}
\usepackage[T1]{fontenc}
%% end:   pdflatex main font ("font-pdflatex-main" template)
%% begin: pdflatex mono font ("font-pdflatex-mono" template)
%% (conditional on non-trivial uses being present in source)
\usepackage[varqu,varl]{inconsolata}
%% end:   pdflatex mono font ("font-pdflatex-mono" template)
%% begin: pdflatex font adjustments ("font-pdflatex-style" template)
%% end:   pdflatex font adjustments ("font-pdflatex-style" template)
%% end:   font setup and configuration for use with pdflatex
}
%% Micromanage spacing, etc.  The named "microtype-options"
%% template may be employed to fine-tune package behavior
\usepackage{microtype}
%% Symbols, align environment, commutative diagrams, bracket-matrix
\usepackage{amsmath}
\usepackage{amscd}
\usepackage{amssymb}
%% allow page breaks within display mathematics anywhere
%% level 4 is maximally permissive
%% this is exactly the opposite of AMSmath package philosophy
%% there are per-display, and per-equation options to control this
%% split, aligned, gathered, and alignedat are not affected
\allowdisplaybreaks[4]
%% allow more columns to a matrix
%% can make this even bigger by overriding with  latex.preamble.late  processing option
\setcounter{MaxMatrixCols}{30}
%%
%%
%% Division Titles, and Page Headers/Footers
%% titlesec package, loading "titleps" package cooperatively
%% See code comments about the necessity and purpose of "explicit" option.
%% The "newparttoc" option causes a consistent entry for parts in the ToC 
%% file, but it is only effective if there is a \titleformat for \part.
%% "pagestyles" loads the  titleps  package cooperatively.
\usepackage[explicit, newparttoc, pagestyles]{titlesec}
%% The companion titletoc package for the ToC.
\usepackage{titletoc}
%% Fixes a bug with transition from chapters to appendices in a "book"
%% See generating XSL code for more details about necessity
\newtitlemark{\chaptertitlename}
%% begin: customizations of page styles via the modal "titleps-style" template
%% Designed to use commands from the LaTeX "titleps" package
%% Plain pages should have the same font for page numbers
\renewpagestyle{plain}{%
\setfoot{}{\pagefont\thepage}{}%
}%
%% Two-page spread as in default LaTeX
\renewpagestyle{headings}{%
\sethead%
[\pagefont\thepage]%
[]
[\pagefont\slshape\MakeUppercase{\ifthechapter{\chaptertitlename\space\thechapter.\space}{}\chaptertitle}]%
{\pagefont\slshape\MakeUppercase{\ifthesection{Section\space\thesection.\space\sectiontitle}{}}}%
{}%
{\pagefont\thepage}%
}%
\pagestyle{headings}
%% end: customizations of page styles via the modal "titleps-style" template
%%
%% Create globally-available macros to be provided for style writers
%% These are redefined for each occurence of each division
\newcommand{\divisionnameptx}{\relax}%
\newcommand{\titleptx}{\relax}%
\newcommand{\subtitleptx}{\relax}%
\newcommand{\shortitleptx}{\relax}%
\newcommand{\authorsptx}{\relax}%
\newcommand{\epigraphptx}{\relax}%
%% Create environments for possible occurences of each division
%% Environment for a PTX "chapter" at the level of a LaTeX "chapter"
\NewDocumentEnvironment{chapterptx}{mmmmmm}
{%
\renewcommand{\divisionnameptx}{Chapter}%
\renewcommand{\titleptx}{#1}%
\renewcommand{\subtitleptx}{#2}%
\renewcommand{\shortitleptx}{#3}%
\renewcommand{\authorsptx}{#4}%
\renewcommand{\epigraphptx}{#5}%
\chapter[{#3}]{#1}%
\label{#6}%
}{}%
%% Environment for a PTX "section" at the level of a LaTeX "section"
\NewDocumentEnvironment{sectionptx}{mmmmmm}
{%
\renewcommand{\divisionnameptx}{Section}%
\renewcommand{\titleptx}{#1}%
\renewcommand{\subtitleptx}{#2}%
\renewcommand{\shortitleptx}{#3}%
\renewcommand{\authorsptx}{#4}%
\renewcommand{\epigraphptx}{#5}%
\section[{#3}]{#1}%
\label{#6}%
}{}%
%% Environment for a PTX "subsection" at the level of a LaTeX "subsection"
\NewDocumentEnvironment{subsectionptx}{mmmmmm}
{%
\renewcommand{\divisionnameptx}{Subsection}%
\renewcommand{\titleptx}{#1}%
\renewcommand{\subtitleptx}{#2}%
\renewcommand{\shortitleptx}{#3}%
\renewcommand{\authorsptx}{#4}%
\renewcommand{\epigraphptx}{#5}%
\subsection[{#3}]{#1}%
\label{#6}%
}{}%
%% Environment for a PTX "exercises" at the level of a LaTeX "subsection"
\NewDocumentEnvironment{exercises-subsection}{mmmmmm}
{%
\renewcommand{\divisionnameptx}{Exercises}%
\renewcommand{\titleptx}{#1}%
\renewcommand{\subtitleptx}{#2}%
\renewcommand{\shortitleptx}{#3}%
\renewcommand{\authorsptx}{#4}%
\renewcommand{\epigraphptx}{#5}%
\subsection[{#3}]{#1}%
\label{#6}%
}{}%
%% Environment for a PTX "exercises" at the level of a LaTeX "subsection"
\NewDocumentEnvironment{exercises-subsection-numberless}{mmmmmm}
{%
\renewcommand{\divisionnameptx}{Exercises}%
\renewcommand{\titleptx}{#1}%
\renewcommand{\subtitleptx}{#2}%
\renewcommand{\shortitleptx}{#3}%
\renewcommand{\authorsptx}{#4}%
\renewcommand{\epigraphptx}{#5}%
\subsection*{#1}%
\addcontentsline{toc}{subsection}{#3}
\label{#6}%
}{}%
%% Environment for a PTX "solutions" at the level of a LaTeX "subsection"
\NewDocumentEnvironment{solutions-subsection}{mmmmmm}
{%
\renewcommand{\divisionnameptx}{Solutions}%
\renewcommand{\titleptx}{#1}%
\renewcommand{\subtitleptx}{#2}%
\renewcommand{\shortitleptx}{#3}%
\renewcommand{\authorsptx}{#4}%
\renewcommand{\epigraphptx}{#5}%
\subsection[{#3}]{#1}%
\label{#6}%
}{}%
%% Environment for a PTX "solutions" at the level of a LaTeX "subsection"
\NewDocumentEnvironment{solutions-subsection-numberless}{mmmmmm}
{%
\renewcommand{\divisionnameptx}{Solutions}%
\renewcommand{\titleptx}{#1}%
\renewcommand{\subtitleptx}{#2}%
\renewcommand{\shortitleptx}{#3}%
\renewcommand{\authorsptx}{#4}%
\renewcommand{\epigraphptx}{#5}%
\subsection*{#1}%
\addcontentsline{toc}{subsection}{#3}
\label{#6}%
}{}%
%% Environment for a PTX "subsubsection" at the level of a LaTeX "subsubsection"
\NewDocumentEnvironment{subsubsectionptx}{mmmmmm}
{%
\renewcommand{\divisionnameptx}{Subsubsection}%
\renewcommand{\titleptx}{#1}%
\renewcommand{\subtitleptx}{#2}%
\renewcommand{\shortitleptx}{#3}%
\renewcommand{\authorsptx}{#4}%
\renewcommand{\epigraphptx}{#5}%
\subsubsection[{#3}]{#1}%
\label{#6}%
}{}%
%% Environment for a PTX "index" at the level of a LaTeX "chapter"
\NewDocumentEnvironment{indexptx}{mmmmmm}
{%
\renewcommand{\divisionnameptx}{Index}%
\renewcommand{\titleptx}{#1}%
\renewcommand{\subtitleptx}{#2}%
\renewcommand{\shortitleptx}{#3}%
\renewcommand{\authorsptx}{#4}%
\renewcommand{\epigraphptx}{#5}%
\chapter*{#1}%
\addcontentsline{toc}{chapter}{#3}
\label{#6}%
}{}%
%%
%% Styles for six traditional LaTeX divisions
\titleformat{\part}[display]
{\divisionfont\Huge\bfseries\centering}{\divisionnameptx\space\thepart}{30pt}{\Huge#1}
[{\Large\centering\authorsptx}]
\titleformat{\chapter}[display]
{\divisionfont\huge\bfseries}{\divisionnameptx\space\thechapter}{20pt}{\Huge#1}
[{\Large\authorsptx}]
\titleformat{name=\chapter,numberless}[display]
{\divisionfont\huge\bfseries}{}{0pt}{#1}
[{\Large\authorsptx}]
\titlespacing*{\chapter}{0pt}{50pt}{40pt}
\titleformat{\section}[hang]
{\divisionfont\Large\bfseries}{\thesection}{1ex}{#1}
[{\large\authorsptx}]
\titleformat{name=\section,numberless}[block]
{\divisionfont\Large\bfseries}{}{0pt}{#1}
[{\large\authorsptx}]
\titlespacing*{\section}{0pt}{3.5ex plus 1ex minus .2ex}{2.3ex plus .2ex}
\titleformat{\subsection}[hang]
{\divisionfont\large\bfseries}{\thesubsection}{1ex}{#1}
[{\normalsize\authorsptx}]
\titleformat{name=\subsection,numberless}[block]
{\divisionfont\large\bfseries}{}{0pt}{#1}
[{\normalsize\authorsptx}]
\titlespacing*{\subsection}{0pt}{3.25ex plus 1ex minus .2ex}{1.5ex plus .2ex}
\titleformat{\subsubsection}[hang]
{\divisionfont\normalsize\bfseries}{\thesubsubsection}{1em}{#1}
[{\small\authorsptx}]
\titleformat{name=\subsubsection,numberless}[block]
{\divisionfont\normalsize\bfseries}{}{0pt}{#1}
[{\normalsize\authorsptx}]
\titlespacing*{\subsubsection}{0pt}{3.25ex plus 1ex minus .2ex}{1.5ex plus .2ex}
\titleformat{\paragraph}[hang]
{\divisionfont\normalsize\bfseries}{\theparagraph}{1em}{#1}
[{\small\authorsptx}]
\titleformat{name=\paragraph,numberless}[block]
{\divisionfont\normalsize\bfseries}{}{0pt}{#1}
[{\normalsize\authorsptx}]
\titlespacing*{\paragraph}{0pt}{3.25ex plus 1ex minus .2ex}{1.5em}
%%
%% Styles for five traditional LaTeX divisions
\titlecontents{part}%
[0pt]{\contentsmargin{0em}\addvspace{1pc}\contentsfont\bfseries}%
{\Large\thecontentslabel\enspace}{\Large}%
{}%
[\addvspace{.5pc}]%
\titlecontents{chapter}%
[0pt]{\contentsmargin{0em}\addvspace{1pc}\contentsfont\bfseries}%
{\large\thecontentslabel\enspace}{\large}%
{\hfill\bfseries\thecontentspage}%
[\addvspace{.5pc}]%
\dottedcontents{section}[3.8em]{\contentsfont}{2.3em}{1pc}%
\dottedcontents{subsection}[6.1em]{\contentsfont}{3.2em}{1pc}%
\dottedcontents{subsubsection}[9.3em]{\contentsfont}{4.3em}{1pc}%
%%
%% Begin: Semantic Macros
%% To preserve meaning in a LaTeX file
%%
%% \mono macro for content of "c", "cd", "tag", etc elements
%% Also used automatically in other constructions
%% Simply an alias for \texttt
%% Always defined, even if there is no need, or if a specific tt font is not loaded
\newcommand{\mono}[1]{\texttt{#1}}
%%
%% Following semantic macros are only defined here if their
%% use is required only in this specific document
%%
%% Used for warnings, typically bold and italic
\newcommand{\alert}[1]{\textbf{\textit{#1}}}
%% Used for inline definitions of terms
\newcommand{\terminology}[1]{\textbf{#1}}
%% End: Semantic Macros
%% begin: environments for duplicates in solutions divisions
%% Solutions to division exercises, not in exercise group
\tcbset{ divisionsolutionstyle/.style={bwminimalstyle, runintitlestyle, exercisespacingstyle, after title={\space}, breakable, parbox=false } }
\newtcolorbox{divisionsolution}[3]{divisionsolutionstyle, title={\hyperlink{#3}{#1}.\notblank{#2}{\space#2}{}}}
%% Solutions to division exercises, in exercise group, no columns
\tcbset{ divisionsolutionegstyle/.style={bwminimalstyle, runintitlestyle, exercisespacingstyle, after title={\space}, left skip=\egindent, breakable, parbox=false } }
\newtcolorbox{divisionsolutioneg}[3]{divisionsolutionegstyle, title={\hyperlink{#3}{#1}.\notblank{#2}{\space#2}{}}}
%% Divisional exercises (and worksheet) as LaTeX environments
%% Third argument is option for extra workspace in worksheets
%% Hanging indent occupies a 5ex width slot prior to left margin
%% Experimentally this seems just barely sufficient for a bold "888."
%% Division exercises, not in exercise group
\tcbset{ divisionexercisestyle/.style={bwminimalstyle, runintitlestyle, exercisespacingstyle, left=5ex, breakable, parbox=false } }
\newtcolorbox{divisionexercise}[4]{divisionexercisestyle, before title={\hspace{-5ex}\makebox[5ex][l]{#1.}}, title={\notblank{#2}{#2\space}{}}, phantom={\label{#4}\hypertarget{#4}{}}, after={\notblank{#3}{\newline\rule{\workspacestrutwidth}{#3}\newline\vfill}{\par}}}
%% Division exercises, in exercise group, no columns
\tcbset{ divisionexerciseegstyle/.style={bwminimalstyle, runintitlestyle, exercisespacingstyle, left=5ex, left skip=\egindent, breakable, parbox=false } }
\newtcolorbox{divisionexerciseeg}[4]{divisionexerciseegstyle, before title={\hspace{-5ex}\makebox[5ex][l]{#1.}}, title={\notblank{#2}{#2\space}{}}, phantom={\label{#4}\hypertarget{#4}{}}, after={\notblank{#3}{\newline\rule{\workspacestrutwidth}{#3}\newline\vfill}{\par}}}
%% Localize LaTeX supplied names (possibly none)
\renewcommand*{\chaptername}{Chapter}
%% Equation Numbering
%% Controlled by  numbering.equations.level  processing parameter
%% No adjustment here implies document-wide numbering
\numberwithin{equation}{section}
%% "tcolorbox" environment for a single image, occupying entire \linewidth
%% arguments are left-margin, width, right-margin, as multiples of
%% \linewidth, and are guaranteed to be positive and sum to 1.0
\tcbset{ imagestyle/.style={bwminimalstyle} }
\NewTColorBox{image}{mmm}{imagestyle,left skip=#1\linewidth,width=#2\linewidth}
%% For improved tables
\usepackage{array}
%% Some extra height on each row is desirable, especially with horizontal rules
%% Increment determined experimentally
\setlength{\extrarowheight}{0.2ex}
%% Define variable thickness horizontal rules, full and partial
%% Thicknesses are 0.03, 0.05, 0.08 in the  booktabs  package
\newcommand{\hrulethin}  {\noalign{\hrule height 0.04em}}
\newcommand{\hrulemedium}{\noalign{\hrule height 0.07em}}
\newcommand{\hrulethick} {\noalign{\hrule height 0.11em}}
%% We preserve a copy of the \setlength package before other
%% packages (extpfeil) get a chance to load packages that redefine it
\let\oldsetlength\setlength
\newlength{\Oldarrayrulewidth}
\newcommand{\crulethin}[1]%
{\noalign{\global\oldsetlength{\Oldarrayrulewidth}{\arrayrulewidth}}%
\noalign{\global\oldsetlength{\arrayrulewidth}{0.04em}}\cline{#1}%
\noalign{\global\oldsetlength{\arrayrulewidth}{\Oldarrayrulewidth}}}%
\newcommand{\crulemedium}[1]%
{\noalign{\global\oldsetlength{\Oldarrayrulewidth}{\arrayrulewidth}}%
\noalign{\global\oldsetlength{\arrayrulewidth}{0.07em}}\cline{#1}%
\noalign{\global\oldsetlength{\arrayrulewidth}{\Oldarrayrulewidth}}}
\newcommand{\crulethick}[1]%
{\noalign{\global\oldsetlength{\Oldarrayrulewidth}{\arrayrulewidth}}%
\noalign{\global\oldsetlength{\arrayrulewidth}{0.11em}}\cline{#1}%
\noalign{\global\oldsetlength{\arrayrulewidth}{\Oldarrayrulewidth}}}
%% Single letter column specifiers defined via array package
\newcolumntype{A}{!{\vrule width 0.04em}}
\newcolumntype{B}{!{\vrule width 0.07em}}
\newcolumntype{C}{!{\vrule width 0.11em}}
%% tcolorbox to place tabular outside of a sidebyside
\tcbset{ tabularboxstyle/.style={bwminimalstyle,} }
\newtcolorbox{tabularbox}[3]{tabularboxstyle, left skip=#1\linewidth, width=#2\linewidth,}
%% Footnote Numbering
%% Specified by numbering.footnotes.level
%% Undo counter reset by chapter for a book
\counterwithout{footnote}{chapter}
\counterwithin*{footnote}{section}
%% Program listing support: for listings, programs, consoles, and Sage code
\ifthenelse{\boolean{xetex} \or \boolean{luatex}}%
  {\tcbuselibrary{listings}}%
  {\tcbuselibrary{listingsutf8}}%
%% We define the listings font style to be the default "ttfamily"
%% To fix hyphens/dashes rendered in PDF as fancy minus signs by listing
%% http://tex.stackexchange.com/questions/33185/listings-package-changes-hyphens-to-minus-signs
\makeatletter
\lst@CCPutMacro\lst@ProcessOther {"2D}{\lst@ttfamily{-{}}{-{}}}
\@empty\z@\@empty
\makeatother
%% We define a null language, free of any formatting or style
%% for use when a language is not supported, or pseudo-code, or consoles
%% Not necessary for Sage code, so in limited cases included unnecessarily
\lstdefinelanguage{none}{identifierstyle=,commentstyle=,stringstyle=,keywordstyle=}
\ifthenelse{\boolean{xetex}}{}{%
%% begin: pdflatex-specific listings configuration
%% translate U+0080 - U+00F0 to their textmode LaTeX equivalents
%% Data originally from https://www.w3.org/Math/characters/unicode.xml, 2016-07-23
%% Lines marked in XSL with "$" were converted from mathmode to textmode
\lstset{extendedchars=true}
\lstset{literate={ }{{~}}{1}{¡}{{\textexclamdown }}{1}{¢}{{\textcent }}{1}{£}{{\textsterling }}{1}{¤}{{\textcurrency }}{1}{¥}{{\textyen }}{1}{¦}{{\textbrokenbar }}{1}{§}{{\textsection }}{1}{¨}{{\textasciidieresis }}{1}{©}{{\textcopyright }}{1}{ª}{{\textordfeminine }}{1}{«}{{\guillemotleft }}{1}{¬}{{\textlnot }}{1}{­}{{\-}}{1}{®}{{\textregistered }}{1}{¯}{{\textasciimacron }}{1}{°}{{\textdegree }}{1}{±}{{\textpm }}{1}{²}{{\texttwosuperior }}{1}{³}{{\textthreesuperior }}{1}{´}{{\textasciiacute }}{1}{µ}{{\textmu }}{1}{¶}{{\textparagraph }}{1}{·}{{\textperiodcentered }}{1}{¸}{{\c{}}}{1}{¹}{{\textonesuperior }}{1}{º}{{\textordmasculine }}{1}{»}{{\guillemotright }}{1}{¼}{{\textonequarter }}{1}{½}{{\textonehalf }}{1}{¾}{{\textthreequarters }}{1}{¿}{{\textquestiondown }}{1}{À}{{\`{A}}}{1}{Á}{{\'{A}}}{1}{Â}{{\^{A}}}{1}{Ã}{{\~{A}}}{1}{Ä}{{\"{A}}}{1}{Å}{{\AA }}{1}{Æ}{{\AE }}{1}{Ç}{{\c{C}}}{1}{È}{{\`{E}}}{1}{É}{{\'{E}}}{1}{Ê}{{\^{E}}}{1}{Ë}{{\"{E}}}{1}{Ì}{{\`{I}}}{1}{Í}{{\'{I}}}{1}{Î}{{\^{I}}}{1}{Ï}{{\"{I}}}{1}{Ð}{{\DH }}{1}{Ñ}{{\~{N}}}{1}{Ò}{{\`{O}}}{1}{Ó}{{\'{O}}}{1}{Ô}{{\^{O}}}{1}{Õ}{{\~{O}}}{1}{Ö}{{\"{O}}}{1}{×}{{\texttimes }}{1}{Ø}{{\O }}{1}{Ù}{{\`{U}}}{1}{Ú}{{\'{U}}}{1}{Û}{{\^{U}}}{1}{Ü}{{\"{U}}}{1}{Ý}{{\'{Y}}}{1}{Þ}{{\TH }}{1}{ß}{{\ss }}{1}{à}{{\`{a}}}{1}{á}{{\'{a}}}{1}{â}{{\^{a}}}{1}{ã}{{\~{a}}}{1}{ä}{{\"{a}}}{1}{å}{{\aa }}{1}{æ}{{\ae }}{1}{ç}{{\c{c}}}{1}{è}{{\`{e}}}{1}{é}{{\'{e}}}{1}{ê}{{\^{e}}}{1}{ë}{{\"{e}}}{1}{ì}{{\`{\i}}}{1}{í}{{\'{\i}}}{1}{î}{{\^{\i}}}{1}{ï}{{\"{\i}}}{1}{ð}{{\dh }}{1}{ñ}{{\~{n}}}{1}{ò}{{\`{o}}}{1}{ó}{{\'{o}}}{1}{ô}{{\^{o}}}{1}{õ}{{\~{o}}}{1}{ö}{{\"{o}}}{1}{÷}{{\textdiv }}{1}{ø}{{\o }}{1}{ù}{{\`{u}}}{1}{ú}{{\'{u}}}{1}{û}{{\^{u}}}{1}{ü}{{\"{u}}}{1}{ý}{{\'{y}}}{1}{þ}{{\th }}{1}{ÿ}{{\"{y}}}{1}}
%% end: pdflatex-specific listings configuration
}
%% End of generic listing adjustments
%% The listings package as tcolorbox for Sage code
%% We do as much styling as possible with tcolorbox, not listings
%% Sage's blue is 50%, we go way lighter (blue!05 would also work)
%% Note that we defuse listings' default "aboveskip" and "belowskip"
\definecolor{sageblue}{rgb}{0.95,0.95,1}
\tcbset{ sagestyle/.style={left=0pt, right=0pt, top=0ex, bottom=0ex, middle=0pt, toptitle=0pt, bottomtitle=0pt,
boxsep=4pt, listing only, fontupper=\small\ttfamily,
breakable, parbox=false, 
listing options={language=Python,breaklines=true,breakatwhitespace=true, extendedchars=true, aboveskip=0pt, belowskip=0pt}} }
\newtcblisting{sageinput}{sagestyle, colback=sageblue, sharp corners, boxrule=0.5pt, toprule at break=-0.3pt, bottomrule at break=-0.3pt, }
\newtcblisting{sageoutput}{sagestyle, colback=white, colframe=white, frame empty, before skip=0pt, after skip=0pt, }
%% Fancy Verbatim for consoles, preformatted, code display, literate programming
\usepackage{fancyvrb}
%% code display (cd), by analogy with math display (md)
%% (a) indented slightly, so a paragraph appears to hang together
\DefineVerbatimEnvironment{codedisplay}{Verbatim}{xleftmargin=4ex}
%% (b) flush left works well in exceptions like list items, etc
\DefineVerbatimEnvironment{codedisplayleft}{Verbatim}{}
%% More flexible list management, esp. for references
%% But also for specifying labels (i.e. custom order) on nested lists
\usepackage{enumitem}
%% Indented groups of "exercise" within an "exercises" division
%% Lengths control the indentation (always) and gaps (multi-column)
\newlength{\egindent}\setlength{\egindent}{0.05\linewidth}
\newlength{\exggap}\setlength{\exggap}{0.05\linewidth}
%% Thin "xparse" environments will represent the entire exercise
%% group, in the case when it does not hold multiple columns.
\NewDocumentEnvironment{exercisegroup}{}
{}{}
%% Support for index creation
%% imakeidx package does not require extra pass (as with makeidx)
%% Title of the "Index" section set via a keyword
%% Language support for the "see" and "see also" phrases
\usepackage{imakeidx}
\makeindex[title=Index, intoc=true]
\renewcommand{\seename}{See}
\renewcommand{\alsoname}{See also}
%% hyperref driver does not need to be specified, it will be detected
%% Footnote marks in tcolorbox have broken linking under
%% hyperref, so it is necessary to turn off all linking
%% It *must* be given as a package option, not with \hypersetup
\usepackage[hyperfootnotes=false]{hyperref}
%% configure hyperref's  \href{}{}  and  \nolinkurl  to match listings' inline verbatim
\renewcommand\UrlFont{\small\ttfamily}
%% For a print PDF, no surrounding boxes, so simply textcolor (but still active to preserve spacing)
\hypersetup{hidelinks=true}
\hypersetup{pdftitle={Binary, Logic, and More}}
%% If you manually remove hyperref, leave in this next command
%% This will allow LaTeX compilation, employing this no-op command
\providecommand\phantomsection{}
%% Division Numbering: Chapters, Sections, Subsections, etc
%% Division numbers may be turned off at some level ("depth")
%% A section *always* has depth 1, contrary to us counting from the document root
%% The latex default is 3.  If a larger number is present here, then
%% removing this command may make some cross-references ambiguous
%% The precursor variable $numbering-maxlevel is checked for consistency in the common XSL file
\setcounter{secnumdepth}{2}
%%
%% AMS "proof" environment is no longer used, but we leave previously
%% implemented \qedhere in place, should the LaTeX be recycled
\newcommand{\qedhere}{\relax}
%%
%% A faux tcolorbox whose only purpose is to provide common numbering
%% facilities for most blocks (possibly not projects, 2D displays)
%% Controlled by  numbering.theorems.level  processing parameter
\newtcolorbox[auto counter, number within=section]{block}{}
%%
%% This document is set to number PROJECT-LIKE on a separate numbering scheme
%% So, a faux tcolorbox whose only purpose is to provide this numbering
%% Controlled by  numbering.projects.level  processing parameter
\newtcolorbox[auto counter, number within=section]{project-distinct}{}
%% A faux tcolorbox whose only purpose is to provide common numbering
%% facilities for 2D displays which are subnumbered as part of a "sidebyside"
\makeatletter
\newtcolorbox[auto counter, number within=tcb@cnt@block, number freestyle={\noexpand\thetcb@cnt@block(\noexpand\alph{\tcbcounter})}]{subdisplay}{}
\makeatother
%%
%% tcolorbox, with styles, for THEOREM-LIKE
%%
%% theorem: fairly simple numbered block/structure
\tcbset{ theoremstyle/.style={bwminimalstyle, runintitlestyle, blockspacingstyle, after title={\space}, } }
\newtcolorbox[use counter from=block]{theorem}[3]{title={{Theorem~\thetcbcounter\notblank{#1#2}{\space}{}\notblank{#1}{\space#1}{}\notblank{#2}{\space(#2)}{}}}, phantomlabel={#3}, breakable, parbox=false, after={\par}, fontupper=\itshape, theoremstyle, }
%%
%% tcolorbox, with styles, for DEFINITION-LIKE
%%
%% definition: fairly simple numbered block/structure
\tcbset{ definitionstyle/.style={bwminimalstyle, runintitlestyle, blockspacingstyle, after title={\space}, after upper={\space\space\hspace*{\stretch{1}}\(\lozenge\)}, } }
\newtcolorbox[use counter from=block]{definition}[2]{title={{Definition~\thetcbcounter\notblank{#1}{\space\space#1}{}}}, phantomlabel={#2}, breakable, parbox=false, after={\par}, definitionstyle, }
%%
%% tcolorbox, with styles, for EXAMPLE-LIKE
%%
%% example: fairly simple numbered block/structure
\tcbset{ examplestyle/.style={bwminimalstyle, runintitlestyle, blockspacingstyle, after title={\space}, after upper={\space\space\hspace*{\stretch{1}}\(\square\)}, } }
\newtcolorbox[use counter from=block]{example}[2]{title={{Example~\thetcbcounter\notblank{#1}{\space\space#1}{}}}, phantomlabel={#2}, breakable, parbox=false, after={\par}, examplestyle, }
%%
%% tcolorbox, with styles, for FIGURE-LIKE
%%
%% figureptx: 2-D display structure
\tcbset{ figureptxstyle/.style={bwminimalstyle, middle=1ex, blockspacingstyle, fontlower=\blocktitlefont} }
\newtcolorbox[use counter from=block]{figureptx}[3]{lower separated=false, before lower={{\textbf{Figure~\thetcbcounter}\space#1}}, phantomlabel={#2}, unbreakable, parbox=false, figureptxstyle, }
%%
%% xparse environments for introductions and conclusions of divisions
%%
%% introduction: in a structured division
\NewDocumentEnvironment{introduction}{m}
{\notblank{#1}{\noindent\textbf{#1}\space}{}}{\par\medskip}
%%
%% tcolorbox, with styles, for miscellaneous environments
%%
%% assemblage: fairly simple un-numbered block/structure
\tcbset{ assemblagestyle/.style={size=normal, colback=white, colbacktitle=white, coltitle=black, colframe=black, rounded corners, titlerule=0.0pt, center title, fonttitle=\blocktitlefont\bfseries, blockspacingstyle, } }
\newtcolorbox{assemblage}[2]{title={\notblank{#1}{#1}{}}, phantomlabel={#2}, breakable, parbox=false, assemblagestyle}
%% Graphics Preamble Entries
\usepackage{tikz}
\usepackage{venndiagram}
\usepackage{circuitikz}
%% If tikz has been loaded, replace ampersand with \amp macro
\ifdefined\tikzset
    \tikzset{ampersand replacement = \amp}
\fi
%% tcolorbox styles for sidebyside layout
\tcbset{ sbsstyle/.style={raster before skip=2.0ex, raster equal height=rows, raster force size=false} }
\tcbset{ sbspanelstyle/.style={bwminimalstyle, fonttitle=\blocktitlefont} }
%% Enviroments for side-by-side and components
%% Necessary to use \NewTColorBox for boxes of the panels
%% "newfloat" environment to squash page-breaks within a single sidebyside
%% "xparse" environment for entire sidebyside
\NewDocumentEnvironment{sidebyside}{mmmm}
  {\begin{tcbraster}
    [sbsstyle,raster columns=#1,
    raster left skip=#2\linewidth,raster right skip=#3\linewidth,raster column skip=#4\linewidth]}
  {\end{tcbraster}}
%% "tcolorbox" environment for a panel of sidebyside
\NewTColorBox{sbspanel}{mO{top}}{sbspanelstyle,width=#1\linewidth,valign=#2}
%% extpfeil package for certain extensible arrows,
%% as also provided by MathJax extension of the same name
%% NB: this package loads mtools, which loads calc, which redefines
%%     \setlength, so it can be removed if it seems to be in the 
%%     way and your math does not use:
%%     
%%     \xtwoheadrightarrow, \xtwoheadleftarrow, \xmapsto, \xlongequal, \xtofrom
%%     
%%     we have had to be extra careful with variable thickness
%%     lines in tables, and so also load this package late
\usepackage{extpfeil}
%% Custom Preamble Entries, late (use latex.preamble.late)
%% Begin: Author-provided packages
%% (From  docinfo/latex-preamble/package  elements)
%% End: Author-provided packages
%% Begin: Author-provided macros
%% (From  docinfo/macros  element)
%% Plus three from PTX for XML characters
\newcommand{\conditional}{{p {\rightarrow} q}}
\newcommand{\inverse}{{\sim\!{p}{} {\rightarrow} \sim\!{q}{}}}
\newcommand{\converse}{{q {\rightarrow} p}}
\newcommand{\contrapositive}{{\sim\!{q}{} {\rightarrow} \sim\!{p}{}}}
\newcommand{\biconditional}{{p {\leftrightarrow}{} q}}

\newcommand{\intersection}{\cap}
\newcommand{\intersect}{\cap}
\newcommand{\union}{\cup}

\newcommand{\nth}{{n^{\text{th}}}}
\newcommand{\kth}{{k^{\text{th}}}}
\newcommand{\upth}[1]{{#1^{\text{th}}}}
\newcommand{\upst}[1]{{#1^{\text{st}}}}
\newcommand{\upnd}[1]{{#1^{\text{nd}}}}
\newcommand{\uprd}[1]{{#1^{\text{rd}}}}
\newcommand{\infinity}{{\infty}}
\newcommand{\lt}{<}
\newcommand{\gt}{>}
\newcommand{\amp}{&}
%% End: Author-provided macros
\begin{document}
%% bottom alignment is explicit, since it normally depends on oneside, twoside
\raggedbottom
\frontmatter
%% begin: half-title
\thispagestyle{empty}
{\titlepagefont\centering
\vspace*{0.28\textheight}
{\Huge Binary, Logic, and More}\\[2\baselineskip]
{\LARGE Applied Math for Computing}\\
}
\clearpage
%% end:   half-title
%% begin: adcard (empty)
\thispagestyle{empty}
\null%
\clearpage
%% end:   adcard (empty)
%% begin: title page
%% Inspired by Peter Wilson's "titleDB" in "titlepages" CTAN package
\thispagestyle{empty}
{\titlepagefont\centering
\vspace*{0.14\textheight}
%% Target for xref to top-level element is ToC
\addtocontents{toc}{\protect\label{x:book:binary-logic-and-more}\protect\hypertarget{x:book:binary-logic-and-more}{}}
{\Huge Binary, Logic, and More}\\[\baselineskip]
{\LARGE Applied Math for Computing}\\[3\baselineskip]
{\Large Patricia R. Wrean}\\[0.5\baselineskip]
{\Large Camosun College}\\[3\baselineskip]
{\Large With contributions from, and conversion to PreTeXt by}\\[0.5\baselineskip]
{\normalsize Jason J. Diemer}\\[0.25\baselineskip]
North Island College\\
}
\clearpage
%% end:   title page
%% begin: copyright-page
\thispagestyle{empty}
\label{g:colophon:idp225740008}{}\hypertarget{g:colophon:idp225740008}{}\vspace*{\stretch{2}}
\noindent\textcopyright{}2016 \textendash{} 2021\quad{}Patricia R. Wrean\\[0.5\baselineskip]
This work is licensed under the Creative Commons Attribution-NonCommercial-ShareAlike 4.0 International License. You can view a copy of the license \href{http://creativecommons.org/licenses/by-nc-sa/4.0/}{here}\footnote{\nolinkurl{http://creativecommons.org/licenses/by-nc-sa/4.0/}\label{g:fn:idp225752168}}. \begin{image}{0.375}{0.25}{0.375}%
\includegraphics[width=\linewidth]{external/images/by-nc-sa.png}
\end{image}%
\par\medskip
\vspace*{\stretch{1}}
\null\clearpage
%% end:   copyright-page
%% begin: table of contents
%% Adjust Table of Contents
\setcounter{tocdepth}{1}
\renewcommand*\contentsname{Contents}
\tableofcontents
%% end:   table of contents
\mainmatter
%
%
\typeout{************************************************}
\typeout{Chapter 1 Binary, Octal, and Hexadecimal}
\typeout{************************************************}
%
\begin{chapterptx}{Binary, Octal, and Hexadecimal}{}{Binary, Octal, and Hexadecimal}{}{}{x:chapter:binary-octal-hex}
%
%
\typeout{************************************************}
\typeout{Section 1.1 Decimal and Octal}
\typeout{************************************************}
%
\begin{sectionptx}{Decimal and Octal}{}{Decimal and Octal}{}{}{x:section:decimal-octal}
%
%
\typeout{************************************************}
\typeout{Subsection 1.1.1 Review of the Decimal System}
\typeout{************************************************}
%
\begin{subsectionptx}{Review of the Decimal System}{}{Review of the Decimal System}{}{}{x:subsection:decimal-review}
Before we look at the numbering systems commonly used by computers, it will likely be helpful to review the workings of the decimal system, the numbering system commonly used by humans. \index{decimal system}\index{base!10} The decimal system (base 10) is based on ten digits, starting from zero, and uses a positional notation, so called because the magnitude of the number depends not only on what digits are used, but also \emph{where} each digit is located within the number.%
\par
For example, if we start counting from zero upwards, we get%
\begin{equation*}
0, 1, 2, 3, 4, 5, 6, 7, 8, 9, 10
\end{equation*}
Notice that in base ten, we don't have a single digit to denote the number ten.  Instead, we write a zero in the right column and then write a one in the column to the left.  Similarly, when we continue counting up to twenty,%
\begin{equation*}
11, 12, 13, 14, 15, 16, 17, 18, 19, 20
\end{equation*}
once we have written the number nineteen as 19, the next number sets the right digit to zero while incrementing the left column by one to give the number 20 (twenty).%
\par
This means that for the decimal number 179, the digit 9 is in the "ones" place, the digit 7 is in the "tens" place, and the digit 1 is in the "hundreds" place, so we can write%
\begin{equation*}
179=1\times 100+7\times 10+9\times 1
\end{equation*}
or%
\begin{equation*}
179=1\times 10^2+7\times 10^1+9\times 10^0,
\end{equation*}
recalling that \(10^0=1\).%
\begin{example}{}{g:example:idp225759464}%
In the decimal number 386, state which digit is in the%
\begin{enumerate}
\item{}ones place%
\item{}tens place%
\item{}hundreds place%
\end{enumerate}
%
\par\smallskip%
\noindent\textbf{\blocktitlefont Answer}.\label{g:answer:idp225760360}{}\hypertarget{g:answer:idp225760360}{}\quad{}%
\begin{enumerate}
\item{}6, since it is the right-most digit%
\item{}8%
\item{}3%
\end{enumerate}
%
\end{example}
\begin{example}{}{g:example:idp225759848}%
In the decimal number 24680, in what place are the following digits?%
\begin{enumerate}
\item{}8%
\item{}6%
\item{}0%
\item{}2%
\item{}4%
\end{enumerate}
%
\par\smallskip%
\noindent\textbf{\blocktitlefont Answer}.\label{g:answer:idp225755496}{}\hypertarget{g:answer:idp225755496}{}\quad{}%
\begin{enumerate}
\item{}the tens place%
\item{}the hundreds place%
\item{}the ones place%
\item{}the ten thousands place%
\item{}the thousands place%
\end{enumerate}
%
\end{example}
\end{subsectionptx}
%
%
\typeout{************************************************}
\typeout{Subsection 1.1.2 Bases Other Than Ten - How They work}
\typeout{************************************************}
%
\begin{subsectionptx}{Bases Other Than Ten - How They work}{}{Bases Other Than Ten - How They work}{}{}{x:subsection:other-bases}
To put a number into a base other than ten, we use the same ideas as before: \index{base!other than 10}%
\begin{itemize}[label=\textbullet]
\item{}the number of digits available is equal to the base%
\item{}there is no single digit which represents the base, so in order to write the base in that system, set the right column to zero and increment the column to the left by one%
\end{itemize}
%
\par
The best way to understand this is to work through an example, so let us first look at numbers written in base 4.\index{base!4}  This base is \emph{not} commonly used in computing, but it is a useful example nonetheless.  We will use the same ideas as before:%
\begin{itemize}[label=\textbullet]
\item{}there are four digits in total:  0,1,2,3%
\item{}there is no single digit which represents the base, so when we want to write the number ``four'', set the right column to zero and increment the column to the left by one%
\end{itemize}
%
\par
To make it clear which base we are using, any numbers written in a base other than ten will have the base as a subscript.  So the number three in base 4 is \(3_4\).%
\par
Let's contrast counting using base 10 versus base 4 by counting from one to twenty in both bases side-by-side. \index{base!subscript notation}  Notice that the default base is 10, so numbers in the decimal system are written without a subscript, but numbers in base 4 have the base as a subscript.%
\begin{sidebyside}{2}{0.2}{0.2}{0.2}%
\begin{sbspanel}{0.2}%
\resizebox{\ifdim\width > \linewidth\linewidth\else\width\fi}{!}{%
{\centering%
{\tabularfont%
\begin{tabular}{cc}\hrulethick
base 10&base 4\tabularnewline\hrulemedium
\(1\)&\(1_4\)\tabularnewline[0pt]
\(2\)&\(2_4\)\tabularnewline[0pt]
\(3\)&\(3_4\)\tabularnewline[0pt]
\(4\)&\(10_4\)\tabularnewline[0pt]
\(5\)&\(11_4\)\tabularnewline[0pt]
\(6\)&\(12_4\)\tabularnewline[0pt]
\(8\)&\(13_4\)\tabularnewline[0pt]
\(8\)&\(20_4\)\tabularnewline[0pt]
\(9\)&\(21_4\)\tabularnewline[0pt]
\(10\)&\(22_4\)\tabularnewline\hrulethick
\end{tabular}
}%
\par}
}%
\end{sbspanel}%
\begin{sbspanel}{0.2}%
\resizebox{\ifdim\width > \linewidth\linewidth\else\width\fi}{!}{%
{\centering%
{\tabularfont%
\begin{tabular}{cc}\hrulethick
base 10&base 4\tabularnewline\hrulemedium
\(11\)&\(23_4\)\tabularnewline[0pt]
\(12\)&\(30_4\)\tabularnewline[0pt]
\(13\)&\(31_4\)\tabularnewline[0pt]
\(14\)&\(32_4\)\tabularnewline[0pt]
\(15\)&\(33_4\)\tabularnewline[0pt]
\(16\)&\(100_4\)\tabularnewline[0pt]
\(17\)&\(101_4\)\tabularnewline[0pt]
\(18\)&\(102_4\)\tabularnewline[0pt]
\(19\)&\(103_4\)\tabularnewline[0pt]
\(20\)&\(100_4\)\tabularnewline\hrulethick
\end{tabular}
}%
\par}
}%
\end{sbspanel}%
\end{sidebyside}%
\par
Another thing to note is what happens when we try to write the decimal number 16 in base 4.  The previous number, 15, is written as \(33_4\).  When you add one to \(33_4\), the three in the right-hand column increments to four.  The base is four, however, and so we write a zero instead and add one to the next column to the left.  Since the digit is that column is also a three, we write a zero and write a one in the next column to the left.  This yields \(16=100_4\). %
\par
So, looking at the number 14 in decimal, we can think of it as%
\begin{equation*}
14=1\times 10 + 4\times 1
\end{equation*}
but that same number written in base 4 is%
\begin{equation*}
32_4=3\times 4+2\times 1
\end{equation*}
The two numerical representations are equivalent since \(12+2\) is 14.%
\par
 Earlier, we expanded 179 as%
\begin{equation*}
179=1\times 10^2+7\times 10^1+9\times 10^0.
\end{equation*}
We can expand numbers in base 4 in the same way, using powers of the base 4 instead of powers of 10.\index{conversion!from base 4!to decimal}  For example, we can show that \(100_4=16_{10}\) as follows:%
\begin{align*}
100_4 \amp =1\times 4^2+0\times 4^1+0\times 4^0\\
\amp =1\times 16+0\times 4+0\times 1\\
\amp =16+0+0\\
\amp =16
\end{align*}
%
\par
Similarly,%
\begin{align*}
302_4 \amp = 3\times 4^2+0\times 4^1+2\times 4^0\\
\amp = 3\times 16+0\times 4+2\times 1\\
\amp = 48+0+2\\
\amp = 50
\end{align*}
and we can conclude that \(302_4=50_{10}\)%
\begin{example}{}{g:example:idp225810664}%
The number \(1230_4\) can be expanded in base 10 as%
\begin{align*}
1230_4 \amp = 1\times 4^3+2\times 4^2+3\times 4^1+0\times 4^0\\
\amp = 64+32+12+0\\
\amp = 64+32+12+0\\
\amp = 108
\end{align*}
%
\par
Expand the following numbers into base 10 in a similar fashion.  Then perform that calculation to express the number in base 10 (also called \terminology{decimal form}).%
%
\begin{enumerate}
\item{}\(\displaystyle 23_4\)%
\item{}\(\displaystyle 121_4\)%
\item{}\(\displaystyle 30102_4\)%
\item{}\(\displaystyle 2132_4\)%
\end{enumerate}
\par\smallskip%
\noindent\textbf{\blocktitlefont Answer}.\label{g:answer:idp225811432}{}\hypertarget{g:answer:idp225811432}{}\quad{}%
\begin{enumerate}
\item{}\(\displaystyle 23_4=2\times 4^1+3\times 4^0=8+3=11\)%
\item{}\(\displaystyle 121_4=1\times 4^2+2\times 4^1+1\times 4^0=16+8+1=25\)%
\item{}\(\displaystyle 30102_4
=3\times 4^4+0\times 4^3+1\times 4^2+0\times 4^1+2\times 4^0=768+0+16+0+2=786\)%
\item{}\(\displaystyle 2132_4=2\times 4^3+1\times 4^2+3\times 4^1+2\times 4^0=128+16+12+2=158\)%
\end{enumerate}
\end{example}
\end{subsectionptx}
%
%
\typeout{************************************************}
\typeout{Subsection 1.1.3 Octal}
\typeout{************************************************}
%
\begin{subsectionptx}{Octal}{}{Octal}{}{}{x:subsection:octal}
Let us now look at numbers written in base 8, called \terminology{octal}. \index{base!8}\index{octal}  Octal is a base commonly used in computing.  We will use the same ideas as before:%
\begin{itemize}[label=\textbullet]
\item{}there are eight digits in total: 0, 1, 2, 3, 4, 5, 6, 7.%
\item{}when we want to write the number "eight" in a column, set that column to zero and increment the column to the left by one.%
\item{}to make it clear which base we are using, any numbers written in a base other than ten will have the base as a subscript.%
\end{itemize}
So the number eight \footnote{Note that unless explicitly stated otherwise, any number is to be interpreted as being in base ten.\label{g:fn:idp225821032}} is written in base eight as \(10_8\).  The \emph{only} place the digit 8 can appear in a number expressed in base eight is in the subscript indicating the base!%
\par
Let's contrast counting in base 10 versus base 8 by counting from one to twenty in both bases side-by-side.%
\begin{sidebyside}{2}{0.2}{0.2}{0.2}%
\begin{sbspanel}{0.2}%
\resizebox{\ifdim\width > \linewidth\linewidth\else\width\fi}{!}{%
{\centering%
{\tabularfont%
\begin{tabular}{cc}\hrulethick
base 10&base 8\tabularnewline\hrulemedium
\(1\)&\(1_8\)\tabularnewline[0pt]
\(2\)&\(2_8\)\tabularnewline[0pt]
\(3\)&\(3_8\)\tabularnewline[0pt]
\(4\)&\(4_8\)\tabularnewline[0pt]
\(5\)&\(5_8\)\tabularnewline[0pt]
\(6\)&\(6_8\)\tabularnewline[0pt]
\(7\)&\(7_8\)\tabularnewline[0pt]
\(8\)&\(10_8\)\tabularnewline[0pt]
\(9\)&\(11_8\)\tabularnewline[0pt]
\(10\)&\(12_8\)\tabularnewline\hrulethick
\end{tabular}
}%
\par}
}%
\end{sbspanel}%
\begin{sbspanel}{0.2}%
\resizebox{\ifdim\width > \linewidth\linewidth\else\width\fi}{!}{%
{\centering%
{\tabularfont%
\begin{tabular}{cc}\hrulethick
base 10&base 8\tabularnewline\hrulemedium
\(11\)&\(13_8\)\tabularnewline[0pt]
\(12\)&\(14_8\)\tabularnewline[0pt]
\(13\)&\(15_8\)\tabularnewline[0pt]
\(14\)&\(16_8\)\tabularnewline[0pt]
\(15\)&\(17_8\)\tabularnewline[0pt]
\(16\)&\(20_8\)\tabularnewline[0pt]
\(17\)&\(21_8\)\tabularnewline[0pt]
\(18\)&\(22_8\)\tabularnewline[0pt]
\(19\)&\(23_8\)\tabularnewline[0pt]
\(20\)&\(24_8\)\tabularnewline\hrulethick
\end{tabular}
}%
\par}
}%
\end{sbspanel}%
\end{sidebyside}%
\par
Consider the number 14.  We can think of it as%
\begin{equation*}
14=1\times 10+4\times 1
\end{equation*}
but that same number written in octal is%
\begin{equation*}
16_8=1\times 8+6\times 1
\end{equation*}
and since \(8+6\) is 14, you can see that the two numerical representations are equivalent.%
\par
Earlier, we expanded 179 as%
\begin{equation*}
179=1\times 10^2+7\times 10^1+9\times 10^0.
\end{equation*}
We can expand in octal in the same way, using powers of 8 instead of powers of 10: \index{conversion!from octal!to decimal}%
\begin{align*}
263_8 \amp = 2\times 8^2+6\times 8^1+3\times 8^0\\
\amp = 2\times 64+6\times 8+3\times 1\\
\amp = 128+48+3\\
\amp = 179
\end{align*}
So we see that 179 is equivalent to \(263_8\).%
\begin{example}{}{g:example:idp225854952}%
In the number \(135724_8\), state which digit is in the%
\begin{enumerate}
\item{}ones place%
\item{}eights place%
\item{}sixty-fours place%
\item{}\(8^5\) place%
\end{enumerate}
%
\par\smallskip%
\noindent\textbf{\blocktitlefont Answer}.\label{g:answer:idp225864936}{}\hypertarget{g:answer:idp225864936}{}\quad{}%
\begin{enumerate}
\item{}4%
\item{}2%
\item{}7%
\item{}1%
\end{enumerate}
%
\end{example}
\begin{example}{}{g:example:idp225862632}%
The number \(12345_8\) can be expanded in base 8 as \(1\times 8^4+2\times 8^3+3\times 8^2+4\times 8^1+5\times 8^0\).  Expand the following numbers into base 8 in a similar fashion.  Then perform the resulting calculation to find the expression of each in decimal form.%
\begin{enumerate}
\item{}\(\displaystyle 41_8\)%
\item{}\(\displaystyle 764_8\)%
\item{}\(\displaystyle 1011_8\)%
\item{}\(\displaystyle 25073_8\)%
\end{enumerate}
%
\par\smallskip%
\noindent\textbf{\blocktitlefont Answer}.\label{g:answer:idp225866984}{}\hypertarget{g:answer:idp225866984}{}\quad{}%
\begin{enumerate}
\item{}\(\displaystyle 41_8=4\times 8^1+1\times 8^0=32+1=33\)%
\item{}\(\displaystyle 764_8=7\times 8^2+6\times 8^1+4\times 8^0=448+48+4=500\)%
\item{}\(\displaystyle 1011_8=1\times 8^3+0\times 8^2+1\times 8^1+1\times 8^0=512+0+8+1=521\)%
\item{}\(\displaystyle 25073_8=2\times 8^4+5\times 8^3+0\times 8^2+7\times 8^1+3\times 8^0=8192+2560+0+56+3=10811\)%
\end{enumerate}
%
\end{example}
\begin{example}{}{g:example:idp225868648}%
Use the technique of the previous example to write the following numbers in expanded form and then express them in base ten.%
\begin{enumerate}
\item{}\(\displaystyle 210_3\)%
\item{}\(\displaystyle 11001_2\)%
\item{}\(\displaystyle 4135_6\)%
\item{}\(\displaystyle 266_7\)%
\end{enumerate}
%
\par\smallskip%
\noindent\textbf{\blocktitlefont Answer}.\label{g:answer:idp225875176}{}\hypertarget{g:answer:idp225875176}{}\quad{}%
\begin{enumerate}
\item{}\(\displaystyle 210_3=2\times 3^2+1\times 3^1+0\times 3^0=18+3+0=21\)%
\item{}\(\displaystyle 11001_2=1\times 2^4+1\times 2^3+0\times 2^2+0\times 2^1+1\times 2^0=16+8+0+0+1=25\)%
\item{}\(\displaystyle 4135_6=4\times 6^3+1\times 6^2+3\times 6^1+5\times 6^0=864+36+18+5=923\)%
\item{}\(\displaystyle 266_7=2\times 7^2+6\times 7^1+6\times 7^0=98+42+6=146\)%
\end{enumerate}
%
\end{example}
\end{subsectionptx}
%
%
\typeout{************************************************}
\typeout{Exercises 1.1.4 Exercises}
\typeout{************************************************}
%
\begin{exercises-subsection}{Exercises}{}{Exercises}{}{}{g:exercises:idp225873000}
\begin{divisionexercise}{1}{}{}{g:exercise:idp225873640}%
Exercise involves the following table. \begin{center}%
{\tabularfont%
\begin{tabular}{cccccccc}\hrulethick
base 10&base 2&base 3&base 4&base 5&base 6&base 7&base 8\tabularnewline\hrulemedium
\(1\)&&&\(1_4\)&&&&\tabularnewline\hrulethin
\(2\)&&&\(2_4\)&&&&\tabularnewline\hrulethin
\(3\)&&&\(3_4\)&&&&\tabularnewline\hrulethin
\(4\)&&&\(10_4\)&&&&\tabularnewline\hrulethin
\(5\)&&&\(11_4\)&&&&\tabularnewline\hrulethin
\(6\)&&&\(12_4\)&&&&\tabularnewline\hrulethin
\(7\)&&&\(13_4\)&&&&\tabularnewline\hrulethin
\(8\)&&&\(20_4\)&&&&\tabularnewline\hrulethin
\(9\)&&&\(21_4\)&&&&\tabularnewline\hrulethin
\(10\)&&&\(22_4\)&&&&\tabularnewline\hrulethin
\(11\)&&&\(23_4\)&&&&\tabularnewline\hrulethin
\(12\)&&&\(30_4\)&&&&\tabularnewline\hrulethin
\(13\)&&&&&&&\tabularnewline\hrulethin
\(14\)&&&&&&&\tabularnewline\hrulethin
\(15\)&&&&&&&\tabularnewline\hrulethin
\(16\)&&&&&&&\tabularnewline\hrulethin
\(17\)&&&&&&&\tabularnewline\hrulethin
\(18\)&&&&&&&\tabularnewline\hrulethin
\(19\)&&&&&&&\tabularnewline\hrulethin
\(20\)&&&&&&&\tabularnewline\hrulethick
\end{tabular}
}%
\end{center}%
 For the following, complete the specified column in this table.  The fourth column has been started as an example.%
%
\begin{enumerate}[label=(\alph*)]
\item{}\(\ \)base 2%
\item{}\(\ \)base 3%
\item{}\(\ \)base 4%
\item{}\(\ \)base 5%
\item{}\(\ \)base 6%
\item{}\(\ \)base 7%
\item{}\(\ \)base 8%
\end{enumerate}
\end{divisionexercise}%
\par\medskip\noindent%
\textbf{Exercise Group.}\space\space%
In the number \(12345678_{10}\), in what place are the following digits?%
\begin{exercisegroup}
\begin{divisionexerciseeg}{2}{}{}{g:exercise:idp226136168}%
\(\ \)8\end{divisionexerciseeg}%
\begin{divisionexerciseeg}{3}{}{}{g:exercise:idp226136424}%
\(\ \)6\end{divisionexerciseeg}%
\begin{divisionexerciseeg}{4}{}{}{g:exercise:idp226130920}%
\(\ \)5\end{divisionexerciseeg}%
\begin{divisionexerciseeg}{5}{}{}{g:exercise:idp226136552}%
\(\ \)7\end{divisionexerciseeg}%
\begin{divisionexerciseeg}{6}{}{}{g:exercise:idp226135400}%
\(\ \)2\end{divisionexerciseeg}%
\begin{divisionexerciseeg}{7}{}{}{g:exercise:idp226135528}%
\(\ \)1\end{divisionexerciseeg}%
\end{exercisegroup}
\par\medskip\noindent
\par\medskip\noindent%
\textbf{Exercise Group.}\space\space%
In the number \(1234567_8\), which digit is in the%
\begin{exercisegroup}
\begin{divisionexerciseeg}{8}{}{}{g:exercise:idp226136040}%
\(\ \)ones place?\end{divisionexerciseeg}%
\begin{divisionexerciseeg}{9}{}{}{g:exercise:idp226143336}%
\(\ \)eights place?\end{divisionexerciseeg}%
\begin{divisionexerciseeg}{10}{}{}{g:exercise:idp226141928}%
\(\ \)sixty-fours place?\end{divisionexerciseeg}%
\begin{divisionexerciseeg}{11}{}{}{g:exercise:idp226138856}%
\(\ 8^5\) place?\end{divisionexerciseeg}%
\end{exercisegroup}
\par\medskip\noindent
\par\medskip\noindent%
\textbf{Exercise Group.}\space\space%
Write the following numbers in expanded form:%
\begin{exercisegroup}
\begin{divisionexerciseeg}{12}{}{}{g:exercise:idp226145256}%
\(\ 523_8\)\end{divisionexerciseeg}%
\begin{divisionexerciseeg}{13}{}{}{g:exercise:idp226143720}%
\(\ 1011110_2\)\end{divisionexerciseeg}%
\begin{divisionexerciseeg}{14}{}{}{g:exercise:idp226138984}%
\(\ 22013_4\)\end{divisionexerciseeg}%
\begin{divisionexerciseeg}{15}{}{}{g:exercise:idp226140520}%
\(\ 4130_5\)\end{divisionexerciseeg}%
\begin{divisionexerciseeg}{16}{}{}{g:exercise:idp226143976}%
\(\ 987_{10}\)\end{divisionexerciseeg}%
\end{exercisegroup}
\par\medskip\noindent
\par\medskip\noindent%
\textbf{Exercise Group.}\space\space%
Convert the following numbers to base 10.%
\begin{exercisegroup}
\begin{divisionexerciseeg}{17}{}{}{g:exercise:idp226152296}%
\(\ 2031_4\)\end{divisionexerciseeg}%
\begin{divisionexerciseeg}{19}{}{}{g:exercise:idp226147944}%
\(\ 100_8\)\end{divisionexerciseeg}%
\begin{divisionexerciseeg}{20}{}{}{g:exercise:idp226150888}%
\(\ 1005_8\)\end{divisionexerciseeg}%
\begin{divisionexerciseeg}{21}{}{}{g:exercise:idp226151912}%
\(\ 2034_8\)\end{divisionexerciseeg}%
\end{exercisegroup}
\par\medskip\noindent
\end{exercises-subsection}
%
%
\typeout{************************************************}
\typeout{Solutions 1.1.5 Solutions to Section~{\xreffont\ref*{x:section:decimal-octal}} Exercises}
\typeout{************************************************}
%
\begin{solutions-subsection}{Solutions to Section~{\xreffont\ref*{x:section:decimal-octal}} Exercises}{}{Solutions to Section~{\xreffont\ref*{x:section:decimal-octal}} Exercises}{}{}{g:solutions:idp226146920}
\par\medskip
\noindent\textbf{\normalsize{}1.1.4\space\textperiodcentered\space{}Exercises}
\begin{divisionsolution}{1.1.4.1}{}{g:exercise:idp225873640}%
\par\smallskip%
\noindent\hypertarget{g:solution:idp225945960-main}{}\begin{center}%
{\tabularfont%
\begin{tabular}{cccccccc}\hrulethick
base 10&base 2&base 3&base 4&base 5&base 6&base 7&base 8\tabularnewline\hrulemedium
\(1\)&\(1_2\)&\(1_3\)&\(1_4\)&\(1_5\)&\(1_6\)&\(1_7\)&\(1_8\)\tabularnewline\hrulethin
\(2\)&\(10_2\)&\(2_3\)&\(2_4\)&\(2_5\)&\(2_6\)&\(2_7\)&\(2_8\)\tabularnewline\hrulethin
\(3\)&\(11_2\)&\(10_3\)&\(3_4\)&\(3_5\)&\(3_6\)&\(3_7\)&\(3_8\)\tabularnewline\hrulethin
\(4\)&\(100_2\)&\(11_3\)&\(10_4\)&\(4_5\)&\(4_6\)&\(4_7\)&\(4_8\)\tabularnewline\hrulethin
\(5\)&\(101_2\)&\(12_3\)&\(11_4\)&\(10_5\)&\(5_6\)&\(5_7\)&\(5_8\)\tabularnewline\hrulethin
\(6\)&\(110_2\)&\(20_3\)&\(12_4\)&\(11_5\)&\(10_6\)&\(6_7\)&\(6_8\)\tabularnewline\hrulethin
\(7\)&\(111_2\)&\(21_3\)&\(13_4\)&\(12_5\)&\(11_6\)&\(10_7\)&\(7_8\)\tabularnewline\hrulethin
\(8\)&\(1000_2\)&\(22_3\)&\(20_4\)&\(13_5\)&\(12_6\)&\(11_7\)&\(10_8\)\tabularnewline\hrulethin
\(9\)&\(1001_2\)&\(100_3\)&\(21_4\)&\(14_5\)&\(13_6\)&\(12_7\)&\(11_8\)\tabularnewline\hrulethin
\(10\)&\(1010_2\)&\(101_3\)&\(22_4\)&\(20_5\)&\(14_6\)&\(13_7\)&\(12_8\)\tabularnewline\hrulethin
\(11\)&\(1011_2\)&\(102_3\)&\(23_4\)&\(21_5\)&\(15_6\)&\(14_7\)&\(13_8\)\tabularnewline\hrulethin
\(12\)&\(1100_2\)&\(110_3\)&\(30_4\)&\(22_5\)&\(20_6\)&\(15_7\)&\(14_8\)\tabularnewline\hrulethin
\(13\)&\(1101_2\)&\(111_3\)&\(31_4\)&\(23_5\)&\(21_6\)&\(16_7\)&\(15_8\)\tabularnewline\hrulethin
\(14\)&\(1110_2\)&\(112_3\)&\(32_4\)&\(24_5\)&\(22_6\)&\(20_7\)&\(16_8\)\tabularnewline\hrulethin
\(15\)&\(1111_2\)&\(120_3\)&\(33_4\)&\(30_5\)&\(23_6\)&\(21_7\)&\(17_8\)\tabularnewline\hrulethin
\(16\)&\(10000_2\)&\(121_3\)&\(100_4\)&\(31_5\)&\(24_6\)&\(22_7\)&\(20_8\)\tabularnewline\hrulethin
\(17\)&\(10001_2\)&\(122_3\)&\(101_4\)&\(32_5\)&\(25_6\)&\(23_7\)&\(21_8\)\tabularnewline\hrulethin
\(18\)&\(10010_2\)&\(200_3\)&\(102_4\)&\(33_5\)&\(30_6\)&\(24_7\)&\(22_8\)\tabularnewline\hrulethin
\(19\)&\(10011_2\)&\(201_3\)&\(103_4\)&\(34_5\)&\(31_6\)&\(25_7\)&\(23_8\)\tabularnewline\hrulethin
\(20\)&\(10100_2\)&\(202_3\)&\(110_4\)&\(40_5\)&\(32_6\)&\(26_7\)&\(24_8\)\tabularnewline\hrulethick
\end{tabular}
}%
\end{center}%
%
\end{divisionsolution}%
\begin{exercisegroup}
\begin{divisionsolutioneg}{1.1.4.2}{}{g:exercise:idp226136168}%
\par\smallskip%
\noindent\hypertarget{g:solution:idp226132712-main}{}ones\end{divisionsolutioneg}%
\begin{divisionsolutioneg}{1.1.4.3}{}{g:exercise:idp226136424}%
\par\smallskip%
\noindent\hypertarget{g:solution:idp226136936-main}{}hundreds\end{divisionsolutioneg}%
\begin{divisionsolutioneg}{1.1.4.4}{}{g:exercise:idp226130920}%
\par\smallskip%
\noindent\hypertarget{g:solution:idp226133992-main}{}thousands\end{divisionsolutioneg}%
\begin{divisionsolutioneg}{1.1.4.5}{}{g:exercise:idp226136552}%
\par\smallskip%
\noindent\hypertarget{g:solution:idp226134632-main}{}tens\end{divisionsolutioneg}%
\begin{divisionsolutioneg}{1.1.4.6}{}{g:exercise:idp226135400}%
\par\smallskip%
\noindent\hypertarget{g:solution:idp226131944-main}{}millions\end{divisionsolutioneg}%
\begin{divisionsolutioneg}{1.1.4.7}{}{g:exercise:idp226135528}%
\par\smallskip%
\noindent\hypertarget{g:solution:idp226138216-main}{}ten millions\end{divisionsolutioneg}%
\end{exercisegroup}
\par\medskip\noindent
\begin{exercisegroup}
\begin{divisionsolutioneg}{1.1.4.8}{}{g:exercise:idp226136040}%
\par\smallskip%
\noindent\hypertarget{g:solution:idp226134376-main}{}7\end{divisionsolutioneg}%
\begin{divisionsolutioneg}{1.1.4.9}{}{g:exercise:idp226143336}%
\par\smallskip%
\noindent\hypertarget{g:solution:idp226139624-main}{}6\end{divisionsolutioneg}%
\begin{divisionsolutioneg}{1.1.4.10}{}{g:exercise:idp226141928}%
\par\smallskip%
\noindent\hypertarget{g:solution:idp226142568-main}{}5\end{divisionsolutioneg}%
\begin{divisionsolutioneg}{1.1.4.11}{}{g:exercise:idp226138856}%
\par\smallskip%
\noindent\hypertarget{g:solution:idp226144104-main}{}2\end{divisionsolutioneg}%
\end{exercisegroup}
\par\medskip\noindent
\begin{exercisegroup}
\begin{divisionsolutioneg}{1.1.4.12}{}{g:exercise:idp226145256}%
\par\smallskip%
\noindent\hypertarget{g:solution:idp226140136-main}{}\(\ 523_8=5\times 8^2+2\times 8^1+3\times 8^0\)\end{divisionsolutioneg}%
\begin{divisionsolutioneg}{1.1.4.13}{}{g:exercise:idp226143720}%
\par\smallskip%
\noindent\hypertarget{g:solution:idp226141672-main}{}\(\ 1011110_2=1\times2^6+0\times2^5+1\times2^4+1\times2^3+1\times2^2+1\times2^1+0\times2^0\)\end{divisionsolutioneg}%
\begin{divisionsolutioneg}{1.1.4.14}{}{g:exercise:idp226138984}%
\par\smallskip%
\noindent\hypertarget{g:solution:idp226139368-main}{}\(\ 22013_4=2\times 4^4+2\times4^3+0\times4^2+1\times4^1+3\times4^0\)\end{divisionsolutioneg}%
\begin{divisionsolutioneg}{1.1.4.15}{}{g:exercise:idp226140520}%
\par\smallskip%
\noindent\hypertarget{g:solution:idp226140904-main}{}\(\ 4\times5^3+1\times5^2+3\times5^2+0\times5^0\)\end{divisionsolutioneg}%
\begin{divisionsolutioneg}{1.1.4.16}{}{g:exercise:idp226143976}%
\par\smallskip%
\noindent\hypertarget{g:solution:idp226139496-main}{}\(\ 987_{10}=9\times10^2+8\times10^1+7\times10^0\)\end{divisionsolutioneg}%
\end{exercisegroup}
\par\medskip\noindent
\begin{exercisegroup}
\begin{divisionsolutioneg}{1.1.4.17}{}{g:exercise:idp226152296}%
\par\smallskip%
\noindent\hypertarget{g:solution:idp226149608-main}{}\(\ 2031_4=141\)\end{divisionsolutioneg}%
\begin{divisionsolutioneg}{1.1.4.19}{}{g:exercise:idp226147944}%
\par\smallskip%
\noindent\hypertarget{g:solution:idp226154216-main}{}\(\ 100_8=64\)\end{divisionsolutioneg}%
\begin{divisionsolutioneg}{1.1.4.20}{}{g:exercise:idp226150888}%
\par\smallskip%
\noindent\hypertarget{g:solution:idp226147304-main}{}\(\ 1005_8=517\)\end{divisionsolutioneg}%
\begin{divisionsolutioneg}{1.1.4.21}{}{g:exercise:idp226151912}%
\par\smallskip%
\noindent\hypertarget{g:solution:idp226150760-main}{}\(\ 2034_8=1052\)\end{divisionsolutioneg}%
\end{exercisegroup}
\par\medskip\noindent
\end{solutions-subsection}
\end{sectionptx}
%
%
\typeout{************************************************}
\typeout{Section 1.2 Binary and Hexadecimal}
\typeout{************************************************}
%
\begin{sectionptx}{Binary and Hexadecimal}{}{Binary and Hexadecimal}{}{}{x:section:binary-hex}
%
%
\typeout{************************************************}
\typeout{Subsection 1.2.1 Binary}
\typeout{************************************************}
%
\begin{subsectionptx}{Binary}{}{Binary}{}{}{x:subsection:binary}
Let us now look at base 2, called \terminology{binary}: \index{base!2}\index{binary}%
\begin{itemize}[label=\textbullet]
\item{}there are two digits in total: 0 and 1%
\item{}counting up from 1 in a column involves setting that column digit to 0 and incrementing the column to the left by 1%
\end{itemize}
%
\par
Let's compare counting using base 10 and base 2 by counting from one to twenty in both bases side by side: \begin{sidebyside}{2}{0.2}{0.2}{0.2}%
\begin{sbspanel}{0.2}%
\resizebox{\ifdim\width > \linewidth\linewidth\else\width\fi}{!}{%
{\centering%
{\tabularfont%
\begin{tabular}{cc}\hrulethick
base 10&base 2\tabularnewline\hrulemedium
\(1\)&\(1_2\)\tabularnewline[0pt]
\(2\)&\(10_2\)\tabularnewline[0pt]
\(3\)&\(11_2\)\tabularnewline[0pt]
\(4\)&\(100_2\)\tabularnewline[0pt]
\(5\)&\(101_2\)\tabularnewline[0pt]
\(6\)&\(110_2\)\tabularnewline[0pt]
\(7\)&\(111_2\)\tabularnewline[0pt]
\(8\)&\(1000_2\)\tabularnewline[0pt]
\(9\)&\(1001_2\)\tabularnewline[0pt]
\(10\)&\(1010_2\)\tabularnewline\hrulethick
\end{tabular}
}%
\par}
}%
\end{sbspanel}%
\begin{sbspanel}{0.2}%
\resizebox{\ifdim\width > \linewidth\linewidth\else\width\fi}{!}{%
{\centering%
{\tabularfont%
\begin{tabular}{cc}\hrulethick
base 10&base 2\tabularnewline\hrulemedium
\(11\)&\(1011_2\)\tabularnewline[0pt]
\(12\)&\(1100_2\)\tabularnewline[0pt]
\(13\)&\(1101_2\)\tabularnewline[0pt]
\(14\)&\(1110_2\)\tabularnewline[0pt]
\(15\)&\(1111_2\)\tabularnewline[0pt]
\(16\)&\(10000_2\)\tabularnewline[0pt]
\(17\)&\(10001_2\)\tabularnewline[0pt]
\(18\)&\(10010_2\)\tabularnewline[0pt]
\(19\)&\(10011_2\)\tabularnewline[0pt]
\(20\)&\(10100_2\)\tabularnewline\hrulethick
\end{tabular}
}%
\par}
}%
\end{sbspanel}%
\end{sidebyside}%
%
\par
Writing a binary number in expanded form uses powers of 2 instead of powers of 10.  We can determine the equivalent decimal form of a binary number by first writing it in expanded form, just as with other bases.  Let's look at the binary number \(110_2\):%
\begin{align*}
110_2 \amp =1\times 2^2+1\times2^1+0\times2^0\\
\amp = 1\times 4+1\times 2+0\times 1\\
\amp =4+2\\
\amp = 6
\end{align*}
and so the binary number \(110_2\) is 6 in base 10. \index{conversion!from binary!to decimal} Similarly, the number \(10100_2\) is equivalent to 20:%
\begin{align*}
10100_2 \amp =1\times2^4+0\times2^3+1\times2^2+0\times2^1+0\times2^0\\
\amp = 1\times16+0\times 8+1\times 4+0\times 2+0\times 1\\
\amp = 16+4\\
\amp = 20
\end{align*}
%
\par
You can see that binary numbers quickly become difficult to read. So, while computers use only zeros and ones, we generally write numbers in a computing context in bases that are powers of two, like octal and hexadecimal (base 16).  We will see in \hyperref[x:section:convert-bin-oct-hex]{Section~{\xreffont\ref{x:section:convert-bin-oct-hex}}, p.\,\pageref{x:section:convert-bin-oct-hex}} how to quickly convert back and forth between binary, octal, and hexadecimal.%
\begin{example}{}{g:example:idp226192488}%
In the following binary numbers, in what place is the underlined number?%
\begin{enumerate}
\item{}\(\displaystyle \ 11100\underline{1}\)%
\item{}\(\displaystyle \ 1110\underline{0}1\)%
\item{}\(\displaystyle \ 11\underline{1}001\)%
\item{}\(\displaystyle \ \underline{1}11001\)%
\end{enumerate}
%
\par\smallskip%
\noindent\textbf{\blocktitlefont Answer}.\label{g:answer:idp226196840}{}\hypertarget{g:answer:idp226196840}{}\quad{}%
\begin{enumerate}
\item{}\(\ \)the ones place%
\item{}\(\ \)the twos place%
\item{}\(\ \)the eights place (the \(2^3\) place)%
\item{}\(\ \)the thirty-twos place (the \(2^5\) place)%
\end{enumerate}
%
\end{example}
\begin{example}{}{g:example:idp226196072}%
Convert the following numbers to base 10.%
\begin{enumerate}
\item{}\(\displaystyle \ 11_2\)%
\item{}\(\displaystyle \ 11010_2\)%
\item{}\(\displaystyle \ 1010101_2\)%
\end{enumerate}
%
\par\smallskip%
\noindent\textbf{\blocktitlefont Solution}.\label{g:solution:idp226205800}{}\hypertarget{g:solution:idp226205800}{}\quad{}%
\begin{enumerate}
\item{}%
\begin{align*}
11_2 \amp = 1\times2^1+1\times2^0\\
\amp = 1\times2 +1\times 1\\
\amp = 2 + 1\\
\amp =3
\end{align*}
%
\item{}%
\begin{align*}
11010_2 \amp = 1\times 2^4+1\times2^3+0\times2^0+1\times2^1+0\times2^0\\
\amp = 1\times16+1\times8+0\times4+1\times2+0\times1\\
\amp =16+8+0+2+0\\
\amp = 26
\end{align*}
%
\item{}%
\begin{align*}
1010101_2 \amp = 1\times2^6+0\times2^5+1\times2^4+0\times2^3+1\times2^2+1\times2^1+1\times2^0\\
\amp = 1\times64+0\times32+1\times16+0\times8+1\times4+0\times2+1\times1\\
\amp = 64+0+16+0+4+0+1\\
\amp = 85
\end{align*}
%
\end{enumerate}
%
\end{example}
\end{subsectionptx}
%
%
\typeout{************************************************}
\typeout{Subsection 1.2.2 Hexadecimal}
\typeout{************************************************}
%
\begin{subsectionptx}{Hexadecimal}{}{Hexadecimal}{}{}{x:subsection:hexadecimal}
Another base commonly used in computing is base 16, called \terminology{hexadecimal}. \index{hexadecimal}\index{base!16} We will use the same ideas as before:%
\begin{itemize}[label=\textbullet]
\item{}there are sixteen digits in total%
\item{}when we want to count up from 15 in a column, we set the digit in that column to zero and increment the column to the left by 1%
\end{itemize}
%
\par
The problem we have is that we run out of numerical digits when counting in hexadecimal.  We could \emph{invent} new symbols, but instead we will borrow from the alphabet. The numbers 0 to 15 are written in hexadecimal as:%
\begin{equation*}
0,1,2,3,4,5,6,7,8,9,A,B,C,D,E,F
\end{equation*}
The next, corresponding to decimal 16, is \(10_{16}\).%
\par
Let's count up to twenty in hexadecimal: \begin{sidebyside}{2}{0.2}{0.2}{0.2}%
\begin{sbspanel}{0.2}%
\resizebox{\ifdim\width > \linewidth\linewidth\else\width\fi}{!}{%
{\centering%
{\tabularfont%
\begin{tabular}{cc}\hrulethick
base 10&base 16\tabularnewline\hrulemedium
\(1\)&\(1_{16}\)\tabularnewline[0pt]
\(2\)&\(2_{16}\)\tabularnewline[0pt]
\(3\)&\(3_{16}\)\tabularnewline[0pt]
\(4\)&\(4_{16}\)\tabularnewline[0pt]
\(5\)&\(5_{16}\)\tabularnewline[0pt]
\(6\)&\(6_{16}\)\tabularnewline[0pt]
\(7\)&\(7_{16}\)\tabularnewline[0pt]
\(8\)&\(8_{16}\)\tabularnewline[0pt]
\(9\)&\(9_{16}\)\tabularnewline[0pt]
\(10\)&\(A_{16}\)\tabularnewline\hrulethick
\end{tabular}
}%
\par}
}%
\end{sbspanel}%
\begin{sbspanel}{0.2}%
\resizebox{\ifdim\width > \linewidth\linewidth\else\width\fi}{!}{%
{\centering%
{\tabularfont%
\begin{tabular}{cc}\hrulethick
base 10&base 16\tabularnewline\hrulemedium
\(11\)&\(B_{16}\)\tabularnewline[0pt]
\(12\)&\(C_{16}\)\tabularnewline[0pt]
\(13\)&\(D_{16}\)\tabularnewline[0pt]
\(14\)&\(E_{16}\)\tabularnewline[0pt]
\(15\)&\(F_{16}\)\tabularnewline[0pt]
\(16\)&\(10_{16}\)\tabularnewline[0pt]
\(17\)&\(11_{16}\)\tabularnewline[0pt]
\(18\)&\(12_{16}\)\tabularnewline[0pt]
\(19\)&\(13_{16}\)\tabularnewline[0pt]
\(20\)&\(14_{16}\)\tabularnewline\hrulethick
\end{tabular}
}%
\par}
}%
\end{sbspanel}%
\end{sidebyside}%
%
\par
Let's consider the expanded form of \(14_{16}\) to see its equivalence to \(20_{10}\): \index{conversion!from hexadecimal!to decimal}%
\begin{align*}
14_{16} \amp = 1\times16^1+4\times16^0\\
\amp = 1\times 16+4\times 1\\
\amp = 16+4\\
\amp = 20
\end{align*}
%
\begin{example}{}{g:example:idp226247656}%
In the number \(13579BDF_{16}\),  in what place are the following digits? %
\begin{enumerate}
\item{}\(\ \)1%
\item{}\(\ \)D%
\item{}\(\ \)F%
\item{}\(\ \)5%
\end{enumerate}
\par\smallskip%
\noindent\textbf{\blocktitlefont Answer}.\label{g:answer:idp226250984}{}\hypertarget{g:answer:idp226250984}{}\quad{}%
\begin{enumerate}
\item{}\(\ \)the \(16^7\)s place%
\item{}\(\ \)the sixteens place%
\item{}\(\ \)the ones place%
\item{}\(\ \) the \(16^5\)s place%
\end{enumerate}
%
\end{example}
\begin{example}{}{g:example:idp226259048}%
Write the following numbers in expanded form.  When doing so, replace any letters with their base 10 equivalents.  For example, the expanded form of \(2C_{16}\) is \(2\times16^1+12\times16^0\). %
\begin{enumerate}
\item{}\(\displaystyle \ A1_{16}\)%
\item{}\(\displaystyle \ BB8_{16}\)%
\item{}\(\displaystyle \ C1D1_{16}\)%
\item{}\(\displaystyle \ 1FFFFD_{16}\)%
\end{enumerate}
\par\smallskip%
\noindent\textbf{\blocktitlefont Answer}.\label{g:answer:idp226255720}{}\hypertarget{g:answer:idp226255720}{}\quad{}%
\begin{enumerate}
\item{}\(\displaystyle \ A1_{16}=10\times16^1+1\times16^0\)%
\item{}\(\displaystyle \ BB8_{16}=11\times16^2+11\times16^1+8\times16^0\)%
\item{}\(\displaystyle \ C1D1_{16}=12\times16^3+1\times16^2+13\times16^1+1\times16^0\)%
\item{}\(\displaystyle \ 1FFFFD_{16}=1\times16^5+15\times16^4+15\times16^3+15\times16^2+15\times16^1+13\times16^0\)%
\end{enumerate}
\end{example}
\begin{example}{}{g:example:idp226268136}%
Convert the following numbers to base 10: %
\begin{enumerate}
\item{}\(\displaystyle \ 9F0_{16}\)%
\item{}\(\displaystyle \ DE4CD_{16}\)%
\end{enumerate}
\par\smallskip%
\noindent\textbf{\blocktitlefont Solution}.\label{g:solution:idp226264680}{}\hypertarget{g:solution:idp226264680}{}\quad{}%
\begin{enumerate}
\item{}%
\begin{align*}
\ 9F0_{16} \amp = 9\times16^2+15\times16^1+0\times16^0\\
\amp = 9\times256+15\times16+0\times1\\
\amp= 2304+240+0\\
\amp = 2544
\end{align*}
%
\item{}%
\begin{align*}
\ DE4CD_{16} \amp = 13\times16^4+14\times16^3+4\times16^2+12\times16^1+13\times16^0\\
\amp = 841968+57344+1024+192+13\\
\amp = 910541
\end{align*}
%
\end{enumerate}
\end{example}
\end{subsectionptx}
%
%
\typeout{************************************************}
\typeout{Exercises 1.2.3 Exercises}
\typeout{************************************************}
%
\begin{exercises-subsection}{Exercises}{}{Exercises}{}{}{g:exercises:idp226266600}
\par\medskip\noindent%
\textbf{Exercise Group.}\space\space%
In the following binary numbers, in what place is the underlined number?%
\begin{exercisegroup}
\begin{divisionexerciseeg}{1}{}{}{g:exercise:idp226268776}%
\(\ 1001010\underline{1}1\)\end{divisionexerciseeg}%
\begin{divisionexerciseeg}{2}{}{}{g:exercise:idp226275560}%
\(\ 10010101\underline{1}\)\end{divisionexerciseeg}%
\begin{divisionexerciseeg}{3}{}{}{g:exercise:idp226272360}%
\(\ 10\underline{0}101011\)\end{divisionexerciseeg}%
\begin{divisionexerciseeg}{4}{}{}{g:exercise:idp226273768}%
\(\ \underline{1}00101011\)\end{divisionexerciseeg}%
\begin{divisionexerciseeg}{5}{}{}{g:exercise:idp226271080}%
\(\ 1001\underline{0}1011\)\end{divisionexerciseeg}%
\end{exercisegroup}
\par\medskip\noindent
\par\medskip\noindent%
\textbf{Exercise Group.}\space\space%
Write the following numbers in expanded form, and then use that form to express the number in base 10.%
\begin{exercisegroup}
\begin{divisionexerciseeg}{6}{}{}{g:exercise:idp226272104}%
\(\ 10_2\)\end{divisionexerciseeg}%
\begin{divisionexerciseeg}{7}{}{}{g:exercise:idp226274920}%
\(\ 111_2\)\end{divisionexerciseeg}%
\begin{divisionexerciseeg}{8}{}{}{g:exercise:idp226284520}%
\(\ 1011_2\)\end{divisionexerciseeg}%
\begin{divisionexerciseeg}{9}{}{}{g:exercise:idp226281192}%
\(\ 1110111_2\)\end{divisionexerciseeg}%
\end{exercisegroup}
\par\medskip\noindent
\par\medskip\noindent%
\textbf{Exercise Group.}\space\space%
Convert the following numbers to base 10:%
\begin{exercisegroup}
\begin{divisionexerciseeg}{10}{}{}{g:exercise:idp226281320}%
\(\ 1001_2\)\end{divisionexerciseeg}%
\begin{divisionexerciseeg}{11}{}{}{g:exercise:idp226277864}%
\(\ 10110001_2\)\end{divisionexerciseeg}%
\begin{divisionexerciseeg}{12}{}{}{g:exercise:idp226280040}%
\(\ 10101_2\)\end{divisionexerciseeg}%
\end{exercisegroup}
\par\medskip\noindent
\par\medskip\noindent%
\textbf{Exercise Group.}\space\space%
In the number \(1C3D02_{16}\), identify the place value of the following digits:%
\begin{exercisegroup}
\begin{divisionexerciseeg}{13}{}{}{g:exercise:idp226279400}%
\(\ 2\)\end{divisionexerciseeg}%
\begin{divisionexerciseeg}{14}{}{}{g:exercise:idp226290664}%
\(\ 0\)\end{divisionexerciseeg}%
\begin{divisionexerciseeg}{15}{}{}{g:exercise:idp226292200}%
\(\ D\)\end{divisionexerciseeg}%
\begin{divisionexerciseeg}{16}{}{}{g:exercise:idp226290152}%
\(\ 3\)\end{divisionexerciseeg}%
\begin{divisionexerciseeg}{17}{}{}{g:exercise:idp226288744}%
\(\ C\)\end{divisionexerciseeg}%
\begin{divisionexerciseeg}{18}{}{}{g:exercise:idp226286824}%
\(\ 1\)\end{divisionexerciseeg}%
\end{exercisegroup}
\par\medskip\noindent
\par\medskip\noindent%
\textbf{Exercise Group.}\space\space%
Write the following hexadecimal numbers in expanded form.%
\begin{exercisegroup}
\begin{divisionexerciseeg}{19}{}{}{g:exercise:idp226289896}%
\(\ 523_{16}\)\end{divisionexerciseeg}%
\begin{divisionexerciseeg}{20}{}{}{g:exercise:idp226293096}%
\(\ F2_{16}\)\end{divisionexerciseeg}%
\begin{divisionexerciseeg}{21}{}{}{g:exercise:idp226290024}%
\(\ 2A013_{16}\)\end{divisionexerciseeg}%
\begin{divisionexerciseeg}{22}{}{}{g:exercise:idp226297192}%
\(\ BEAD_{16}\)\end{divisionexerciseeg}%
\begin{divisionexerciseeg}{23}{}{}{g:exercise:idp226297064}%
\(\ 9C8_{16}\)\end{divisionexerciseeg}%
\end{exercisegroup}
\par\medskip\noindent
\par\medskip\noindent%
\textbf{Exercise Group.}\space\space%
Convert the following numbers to base 10.%
\begin{exercisegroup}
\begin{divisionexerciseeg}{24}{}{}{g:exercise:idp226294120}%
\(\ AC882_{16}\)\end{divisionexerciseeg}%
\begin{divisionexerciseeg}{25}{}{}{g:exercise:idp226296936}%
\(\ 1000_{16}\)\end{divisionexerciseeg}%
\begin{divisionexerciseeg}{26}{}{}{g:exercise:idp226297960}%
\(\ 2CF_{16}\)\end{divisionexerciseeg}%
\begin{divisionexerciseeg}{27}{}{}{g:exercise:idp226301288}%
\(\ BB8_{16}\)\end{divisionexerciseeg}%
\begin{divisionexerciseeg}{28}{}{}{g:exercise:idp226300648}%
\(\ 7AAA01_{16}\)\end{divisionexerciseeg}%
\begin{divisionexerciseeg}{29}{}{}{g:exercise:idp226302056}%
\(\ 65ABF_{16}\)\end{divisionexerciseeg}%
\end{exercisegroup}
\par\medskip\noindent
\end{exercises-subsection}
%
%
\typeout{************************************************}
\typeout{Solutions 1.2.4 Solutions to Section~{\xreffont\ref*{x:section:binary-hex}} Exercises}
\typeout{************************************************}
%
\begin{solutions-subsection}{Solutions to Section~{\xreffont\ref*{x:section:binary-hex}} Exercises}{}{Solutions to Section~{\xreffont\ref*{x:section:binary-hex}} Exercises}{}{}{g:solutions:idp226296552}
\par\medskip
\noindent\textbf{\normalsize{}1.2.3\space\textperiodcentered\space{}Exercises}
\begin{exercisegroup}
\begin{divisionsolutioneg}{1.2.3.1}{}{g:exercise:idp226268776}%
\par\smallskip%
\noindent\hypertarget{g:solution:idp226269928-main}{}the twos place\end{divisionsolutioneg}%
\begin{divisionsolutioneg}{1.2.3.2}{}{g:exercise:idp226275560}%
\par\smallskip%
\noindent\hypertarget{g:solution:idp226274536-main}{}the ones place\end{divisionsolutioneg}%
\begin{divisionsolutioneg}{1.2.3.3}{}{g:exercise:idp226272360}%
\par\smallskip%
\noindent\hypertarget{g:solution:idp226275304-main}{}the 64s place (\(2^6\))\end{divisionsolutioneg}%
\begin{divisionsolutioneg}{1.2.3.4}{}{g:exercise:idp226273768}%
\par\smallskip%
\noindent\hypertarget{g:solution:idp226269672-main}{}the 256s place (\(2^8\))\end{divisionsolutioneg}%
\begin{divisionsolutioneg}{1.2.3.5}{}{g:exercise:idp226271080}%
\par\smallskip%
\noindent\hypertarget{g:solution:idp226271336-main}{}the sixteens (\(2^4\)) place\end{divisionsolutioneg}%
\end{exercisegroup}
\par\medskip\noindent
\begin{exercisegroup}
\begin{divisionsolutioneg}{1.2.3.6}{}{g:exercise:idp226272104}%
\par\smallskip%
\noindent\hypertarget{g:solution:idp226276200-main}{}%
\begin{align*}
10_2 \amp = 1\times2^1+0\times2^0\\
\amp = 2 + 0\\
\amp = 2
\end{align*}
\end{divisionsolutioneg}%
\begin{divisionsolutioneg}{1.2.3.7}{}{g:exercise:idp226274920}%
\par\smallskip%
\noindent\hypertarget{g:solution:idp226275944-main}{}%
\begin{align*}
111_2 \amp = 1\times2^2+1\times2^1+1\times2^0\\
\amp = 4+2+1\\
\amp = 7
\end{align*}
\end{divisionsolutioneg}%
\begin{divisionsolutioneg}{1.2.3.8}{}{g:exercise:idp226284520}%
\par\smallskip%
\noindent\hypertarget{g:solution:idp226283496-main}{}%
\begin{align*}
1011_2 \amp = 1\times2^3+0\times2^2+1\times2^1+1\times2^0\\
\amp = 8+0+2+1\\
\amp = 11
\end{align*}
\end{divisionsolutioneg}%
\begin{divisionsolutioneg}{1.2.3.9}{}{g:exercise:idp226281192}%
\par\smallskip%
\noindent\hypertarget{g:solution:idp226277992-main}{}%
\begin{align*}
1110111_2 \amp = 1\times2^6+1\times2^5+1\times2^4+0\times2^3+1\times2^2+1\times2^1+1\times2^0\\
\amp = 64+32+16+0+4+2+1\\
\amp = 110
\end{align*}
\end{divisionsolutioneg}%
\end{exercisegroup}
\par\medskip\noindent
\begin{exercisegroup}
\begin{divisionsolutioneg}{1.2.3.10}{}{g:exercise:idp226281320}%
\par\smallskip%
\noindent\hypertarget{g:solution:idp226282728-main}{}\(\ 1001_2=9\)\end{divisionsolutioneg}%
\begin{divisionsolutioneg}{1.2.3.11}{}{g:exercise:idp226277864}%
\par\smallskip%
\noindent\hypertarget{g:solution:idp226285032-main}{}\(\ 10110001_2=177\)\end{divisionsolutioneg}%
\begin{divisionsolutioneg}{1.2.3.12}{}{g:exercise:idp226280040}%
\par\smallskip%
\noindent\hypertarget{g:solution:idp226278248-main}{}\(\ 10101_2=21\)\end{divisionsolutioneg}%
\end{exercisegroup}
\par\medskip\noindent
\begin{exercisegroup}
\begin{divisionsolutioneg}{1.2.3.13}{}{g:exercise:idp226279400}%
\par\smallskip%
\noindent\hypertarget{g:solution:idp226280808-main}{}ones\end{divisionsolutioneg}%
\begin{divisionsolutioneg}{1.2.3.14}{}{g:exercise:idp226290664}%
\par\smallskip%
\noindent\hypertarget{g:solution:idp226293480-main}{}sixteens\end{divisionsolutioneg}%
\begin{divisionsolutioneg}{1.2.3.15}{}{g:exercise:idp226292200}%
\par\smallskip%
\noindent\hypertarget{g:solution:idp226291944-main}{}the \(16^2\)s position\end{divisionsolutioneg}%
\begin{divisionsolutioneg}{1.2.3.16}{}{g:exercise:idp226290152}%
\par\smallskip%
\noindent\hypertarget{g:solution:idp226286568-main}{}the \(16^3\)s position\end{divisionsolutioneg}%
\begin{divisionsolutioneg}{1.2.3.17}{}{g:exercise:idp226288744}%
\par\smallskip%
\noindent\hypertarget{g:solution:idp226285928-main}{}the \(16^4\)s position\end{divisionsolutioneg}%
\begin{divisionsolutioneg}{1.2.3.18}{}{g:exercise:idp226286824}%
\par\smallskip%
\noindent\hypertarget{g:solution:idp226290280-main}{}the \(16^5\)s position\end{divisionsolutioneg}%
\end{exercisegroup}
\par\medskip\noindent
\begin{exercisegroup}
\begin{divisionsolutioneg}{1.2.3.19}{}{g:exercise:idp226289896}%
\par\smallskip%
\noindent\hypertarget{g:solution:idp226291304-main}{}\(\ 523_{16}=5\times16^2+2\times16^1+3\times16^0\)\end{divisionsolutioneg}%
\begin{divisionsolutioneg}{1.2.3.20}{}{g:exercise:idp226293096}%
\par\smallskip%
\noindent\hypertarget{g:solution:idp226293224-main}{}\(\ F2_{16}=15\times16^1+2\times16^0\)\end{divisionsolutioneg}%
\begin{divisionsolutioneg}{1.2.3.21}{}{g:exercise:idp226290024}%
\par\smallskip%
\noindent\hypertarget{g:solution:idp226298216-main}{}\(\ 2A013_{16}=2\times16^4+10\times16^3+0\times16^2+1\times16^1+3\times16^0\)\end{divisionsolutioneg}%
\begin{divisionsolutioneg}{1.2.3.22}{}{g:exercise:idp226297192}%
\par\smallskip%
\noindent\hypertarget{g:solution:idp226299880-main}{}\(\ BEAD_{16}=10\times16^3+14\times16^2+10\times16^2+2\times16^1+13\times16^0\)\end{divisionsolutioneg}%
\begin{divisionsolutioneg}{1.2.3.23}{}{g:exercise:idp226297064}%
\par\smallskip%
\noindent\hypertarget{g:solution:idp226299368-main}{}\(\ 9C8_{16}=9\times16^2+12\times16^1+8\times16^0\)\end{divisionsolutioneg}%
\end{exercisegroup}
\par\medskip\noindent
\begin{exercisegroup}
\begin{divisionsolutioneg}{1.2.3.24}{}{g:exercise:idp226294120}%
\par\smallskip%
\noindent\hypertarget{g:solution:idp226300136-main}{}\(\ AC882_{16}=706690\)\end{divisionsolutioneg}%
\begin{divisionsolutioneg}{1.2.3.25}{}{g:exercise:idp226296936}%
\par\smallskip%
\noindent\hypertarget{g:solution:idp226300008-main}{}\(\ 1000_{16}=4096\)\end{divisionsolutioneg}%
\begin{divisionsolutioneg}{1.2.3.26}{}{g:exercise:idp226297960}%
\par\smallskip%
\noindent\hypertarget{g:solution:idp226297576-main}{}\(\ 2CF_{16}=719\)\end{divisionsolutioneg}%
\begin{divisionsolutioneg}{1.2.3.27}{}{g:exercise:idp226301288}%
\par\smallskip%
\noindent\hypertarget{g:solution:idp226298088-main}{}\(\ BB8_{16}=3000\)\end{divisionsolutioneg}%
\begin{divisionsolutioneg}{1.2.3.28}{}{g:exercise:idp226300648}%
\par\smallskip%
\noindent\hypertarget{g:solution:idp226301416-main}{}\(\ 7AAA01_{16}=8038913\)\end{divisionsolutioneg}%
\begin{divisionsolutioneg}{1.2.3.29}{}{g:exercise:idp226302056}%
\par\smallskip%
\noindent\hypertarget{g:solution:idp226294760-main}{}\(\ 65ABF_{16}=416447\)\end{divisionsolutioneg}%
\end{exercisegroup}
\par\medskip\noindent
\end{solutions-subsection}
\end{sectionptx}
%
%
\typeout{************************************************}
\typeout{Section 1.3 Converting Non-Integer Numbers to Decimal}
\typeout{************************************************}
%
\begin{sectionptx}{Converting Non-Integer Numbers to Decimal}{}{Converting Non-Integer Numbers to Decimal}{}{}{x:section:non-integer-to-decimal}
%
%
\typeout{************************************************}
\typeout{Subsection 1.3.1 Review of the Decimal System for Non-integers}
\typeout{************************************************}
%
\begin{subsectionptx}{Review of the Decimal System for Non-integers}{}{Review of the Decimal System for Non-integers}{}{}{x:subsection:review-non-integer-decimal}
Let's once again review the decimal system, but this time we will consider non-integer numbers.  Recall that integers are numbers that can be written without a fractional part, like 5, -3, and 0.  To write non-integer numbers as a decimal, we again use positional notation, where the fractional part is to the right of the decimal point.%
\par
For example, consider the base-10 number%
\begin{equation*}
8.76
\end{equation*}
The dot is called the decimal point (some cultures use a comma rather than a dot).  The digit to the immediate left (8) is in the ``ones'' place, the first digit to the right (7) is in the ``tenths'' place, and the second digit to the right (6) is in the ``hundredths'' place:%
\begin{equation*}
8.76=8+\frac{7}{10}+\frac{6}{100}
\end{equation*}
%
\par
Recall that \(\tfrac{1}{10}\) can be written as \(10^{-1}\), we can write%
\begin{equation*}
8.76=8\times10^0+7\times10^{-1}+6\times10^{-2},
\end{equation*}
or, equivalently,%
\begin{equation*}
8.76=8+0.7+0.06
\end{equation*}
This may seem redundant, but this representation will come in handy when considering bases other than ten.%
\begin{example}{}{g:example:idp226308712}%
In the decimal number 38.6, state which digit is in the %
\begin{enumerate}
\item{}\(\ \)tens place%
\item{}\(\ \)ones place%
\item{}\(\ \)tenths place%
\end{enumerate}
\par\smallskip%
\noindent\textbf{\blocktitlefont Answer}.\label{g:answer:idp226313320}{}\hypertarget{g:answer:idp226313320}{}\quad{}%
\begin{enumerate}
\item{}\(\ \)3%
\item{}\(\ \)8, since it is to the left of the decimal point%
\item{}\(\ \)6%
\end{enumerate}
\end{example}
\begin{example}{}{g:example:idp226311784}%
In the decimal number 2.4608, in what place are the following digits? %
\begin{enumerate}
\item{}\(\ \)6%
\item{}\(\ \)2%
\item{}\(\ \)0%
\item{}\(\ \)8%
\item{}\(\ \)4%
\end{enumerate}
\par\smallskip%
\noindent\textbf{\blocktitlefont Answer}.\label{g:answer:idp226052584}{}\hypertarget{g:answer:idp226052584}{}\quad{}%
\begin{enumerate}
\item{}\(\ \)the hundredths place%
\item{}\(\ \)the ones place%
\item{}\(\ \)the thousandths place%
\item{}\(\ \)the ten thousandths place%
\item{}\(\ \)the tenths place%
\end{enumerate}
\end{example}
\end{subsectionptx}
%
%
\typeout{************************************************}
\typeout{Subsection 1.3.2 Non-Integers in Bases Other Than Ten}
\typeout{************************************************}
%
\begin{subsectionptx}{Non-Integers in Bases Other Than Ten}{}{Non-Integers in Bases Other Than Ten}{}{}{x:subsection:non-integers-other-bases}
When we consider a number such as%
\begin{equation*}
10.011_2
\end{equation*}
we immediately run into a naming problem.  We can no longer call the dot the \terminology{decimal point} as this is not a decimal number.  For this particular example, we can call the dot the \terminology{binary point}, and in general the dot is called the \terminology{radix point}.%
\par
Now, the digits to the left of the binary point make up the integer part of the number as before, and the digits to the right make up the fractional part.%
\par
Let's look at a number in base 4, so we're not getting confused with all the zeros and ones.  If we consider%
\begin{equation*}
20.31_4
\end{equation*}
we see that the two digits to the left of the radix point, 2 and 0, make up the \terminology{integer part}, while the 3 and the 1 make up the \terminology{fractional part}.  Using positional notation, the 3 is in the \emph{fourths} place while the 1 is in the \emph{sixteenths} place.  we have seen before that \(20_4=2\times4^1+4\times4^0\), so now%
\begin{equation*}
20.31_4=2\times4^1+0\times4^0+3\times4^{-1}+1\times4^{-2}
\end{equation*}
or, if you prefer,%
\begin{equation*}
20.31_4=2\times4+0+\frac{3}{4}+\frac{1}{4^2}.
\end{equation*}
%
\par
Writing these numbers in their decimal equivalents, we see that%
\begin{align*}
20.31_4 \amp =8+0+0.75+0.0625\\
\amp = 8.8125_{10}
\end{align*}
%
\par
Similarly,%
\begin{align*}
F2.B9_{16} \amp =15\times16^1+2\times16^0+11\times16^{-1}+9\times16^{-2}\\
\amp = 15\times16+2\times1+\frac{11}{16}+\frac{9}{16^2}\\
\amp = 240+2+0.6875+0.035156\\
\amp = 242.72265625
\end{align*}
and we can conlude that \(F2.B9_{16}=242.72265625_{10}\)%
\begin{example}{}{g:example:idp226335320}%
In the number \(30.1C_{16}\) state which digit is in the %
\begin{enumerate}
\item{}\(\ \)sixteens place%
\item{}\(\ \)sixteenths place%
\end{enumerate}
\par\smallskip%
\noindent\textbf{\blocktitlefont Answer}.\label{g:answer:idp226335064}{}\hypertarget{g:answer:idp226335064}{}\quad{}%
\begin{enumerate}
\item{}\(\ \)3%
\item{}\(\ \)1%
\end{enumerate}
\end{example}
\begin{example}{}{g:example:idp226342872}%
Write the following numbers in expanded form, and then use the expanded form to determine the decimal form of the number.  If appropriate, round your answer to three decimal places. %
\begin{enumerate}
\item{}\(\displaystyle \ 101.011_2\)%
\item{}\(\displaystyle \ 12.17_8\)%
\item{}\(\displaystyle \ 0.FE_{16}\)%
\item{}\(\displaystyle 2132.43_5\)%
\end{enumerate}
\par\smallskip%
\noindent\textbf{\blocktitlefont Answer}.\label{g:answer:idp226343128}{}\hypertarget{g:answer:idp226343128}{}\quad{}%
\begin{enumerate}
\item{}\(\displaystyle \ 101.011_2=1\times2^2+0\times2^1+1\times2^0+0\times2^{-1}+1\times2^{-2}+1\times2^{-3}=5.375\)%
\item{}\(\displaystyle \ 12.17_8=1\times8^1+2\times8^0+1\times8^{-1}+7\times8^{-2}=10.234\)%
\item{}\(\displaystyle \ 0.FE_{16}=0\times16^0+15\times16^{-1}+14\times16^{-2}=0.992\)%
\item{}\(\displaystyle \ 2132.43_5=2\times5^3+1\times5^2+3\times5^1+2\times5^0+4\times5^{-1}+3\times5^{-2}=292.92\)%
\end{enumerate}
\end{example}
\end{subsectionptx}
%
%
\typeout{************************************************}
\typeout{Exercises 1.3.3 Exercises}
\typeout{************************************************}
%
\begin{exercises-subsection}{Exercises}{}{Exercises}{}{}{g:exercises:idp226348888}
\par\medskip\noindent%
\textbf{Exercise Group.}\space\space%
In the number \(123.45678_{10}\), in what place are the following digits?\begin{exercisegroup}
\begin{divisionexerciseeg}{1}{}{}{g:exercise:idp226348376}%
\(\ \)3\end{divisionexerciseeg}%
\begin{divisionexerciseeg}{2}{}{}{g:exercise:idp226349272}%
\(\ \)6\end{divisionexerciseeg}%
\begin{divisionexerciseeg}{3}{}{}{g:exercise:idp226346200}%
\(\ \)5\end{divisionexerciseeg}%
\begin{divisionexerciseeg}{4}{}{}{g:exercise:idp226344408}%
\(\ \)7\end{divisionexerciseeg}%
\begin{divisionexerciseeg}{5}{}{}{g:exercise:idp226347224}%
\(\ \)2\end{divisionexerciseeg}%
\begin{divisionexerciseeg}{6}{}{}{g:exercise:idp226351704}%
\(\ \)1\end{divisionexerciseeg}%
\end{exercisegroup}
\par\medskip\noindent
\par\medskip\noindent%
\textbf{Exercise Group.}\space\space%
In the number \(1234.567_8\), which digit is in the\begin{exercisegroup}
\begin{divisionexerciseeg}{7}{}{}{g:exercise:idp226352344}%
\(\ \)ones place?\end{divisionexerciseeg}%
\begin{divisionexerciseeg}{8}{}{}{g:exercise:idp226351960}%
\(\ \)eighths place?\end{divisionexerciseeg}%
\begin{divisionexerciseeg}{9}{}{}{g:exercise:idp226355800}%
\(\ \)eights place?\end{divisionexerciseeg}%
\begin{divisionexerciseeg}{10}{}{}{g:exercise:idp226355672}%
\(\ \)sixty-fourths place?\end{divisionexerciseeg}%
\begin{divisionexerciseeg}{11}{}{}{g:exercise:idp226352984}%
\(\ \)sixty-fours place?\end{divisionexerciseeg}%
\end{exercisegroup}
\par\medskip\noindent
\par\medskip\noindent%
\textbf{Exercise Group.}\space\space%
Convert the following numbers to base 10.  When appropriate, round to three decimal places.\begin{exercisegroup}
\begin{divisionexerciseeg}{12}{}{}{g:exercise:idp226359640}%
\(\ 72.31_8\)\end{divisionexerciseeg}%
\begin{divisionexerciseeg}{13}{}{}{g:exercise:idp226363352}%
\(\ 203.1_4\)\end{divisionexerciseeg}%
\begin{divisionexerciseeg}{14}{}{}{g:exercise:idp226365016}%
\(\ 100.111_2\)\end{divisionexerciseeg}%
\begin{divisionexerciseeg}{15}{}{}{g:exercise:idp226363480}%
\(\ 100.5_7\)\end{divisionexerciseeg}%
\begin{divisionexerciseeg}{16}{}{}{g:exercise:idp226362456}%
\(\ 20C4.B7_{16}\)\end{divisionexerciseeg}%
\end{exercisegroup}
\par\medskip\noindent
\end{exercises-subsection}
%
%
\typeout{************************************************}
\typeout{Solutions 1.3.4 Solutions to Section~{\xreffont\ref*{x:section:non-integer-to-decimal}} Exercises}
\typeout{************************************************}
%
\begin{solutions-subsection}{Solutions to Section~{\xreffont\ref*{x:section:non-integer-to-decimal}} Exercises}{}{Solutions to Section~{\xreffont\ref*{x:section:non-integer-to-decimal}} Exercises}{}{}{g:solutions:idp226360792}
\par\medskip
\noindent\textbf{\normalsize{}1.3.3\space\textperiodcentered\space{}Exercises}
\begin{exercisegroup}
\begin{divisionsolutioneg}{1.3.3.1}{}{g:exercise:idp226348376}%
\par\smallskip%
\noindent\hypertarget{g:solution:idp226344792-main}{}\(\ \)ones\end{divisionsolutioneg}%
\begin{divisionsolutioneg}{1.3.3.2}{}{g:exercise:idp226349272}%
\par\smallskip%
\noindent\hypertarget{g:solution:idp226344024-main}{}\(\ \)thousandths\end{divisionsolutioneg}%
\begin{divisionsolutioneg}{1.3.3.3}{}{g:exercise:idp226346200}%
\par\smallskip%
\noindent\hypertarget{g:solution:idp226345304-main}{}\(\ \)hundredths\end{divisionsolutioneg}%
\begin{divisionsolutioneg}{1.3.3.4}{}{g:exercise:idp226344408}%
\par\smallskip%
\noindent\hypertarget{g:solution:idp226348248-main}{}\(\ \)ten thousandths\end{divisionsolutioneg}%
\begin{divisionsolutioneg}{1.3.3.5}{}{g:exercise:idp226347224}%
\par\smallskip%
\noindent\hypertarget{g:solution:idp226347480-main}{}\(\ \)tens\end{divisionsolutioneg}%
\begin{divisionsolutioneg}{1.3.3.6}{}{g:exercise:idp226351704}%
\par\smallskip%
\noindent\hypertarget{g:solution:idp226354008-main}{}\(\ \)hundreds\end{divisionsolutioneg}%
\end{exercisegroup}
\par\medskip\noindent
\begin{exercisegroup}
\begin{divisionsolutioneg}{1.3.3.7}{}{g:exercise:idp226352344}%
\par\smallskip%
\noindent\hypertarget{g:solution:idp226352216-main}{}\(\ \)4\end{divisionsolutioneg}%
\begin{divisionsolutioneg}{1.3.3.8}{}{g:exercise:idp226351960}%
\par\smallskip%
\noindent\hypertarget{g:solution:idp226358744-main}{}\(\ \)5\end{divisionsolutioneg}%
\begin{divisionsolutioneg}{1.3.3.9}{}{g:exercise:idp226355800}%
\par\smallskip%
\noindent\hypertarget{g:solution:idp226355032-main}{}\(\ \)3\end{divisionsolutioneg}%
\begin{divisionsolutioneg}{1.3.3.10}{}{g:exercise:idp226355672}%
\par\smallskip%
\noindent\hypertarget{g:solution:idp226355928-main}{}\(\ \)6\end{divisionsolutioneg}%
\begin{divisionsolutioneg}{1.3.3.11}{}{g:exercise:idp226352984}%
\par\smallskip%
\noindent\hypertarget{g:solution:idp226353624-main}{}\(\ \)2\end{divisionsolutioneg}%
\end{exercisegroup}
\par\medskip\noindent
\begin{exercisegroup}
\begin{divisionsolutioneg}{1.3.3.12}{}{g:exercise:idp226359640}%
\par\smallskip%
\noindent\hypertarget{g:solution:idp226365144-main}{}\(\ 58.391\)\end{divisionsolutioneg}%
\begin{divisionsolutioneg}{1.3.3.13}{}{g:exercise:idp226363352}%
\par\smallskip%
\noindent\hypertarget{g:solution:idp226362840-main}{}\(\ 35.25\)\end{divisionsolutioneg}%
\begin{divisionsolutioneg}{1.3.3.14}{}{g:exercise:idp226365016}%
\par\smallskip%
\noindent\hypertarget{g:solution:idp226360024-main}{}\(\ 4.875\)\end{divisionsolutioneg}%
\begin{divisionsolutioneg}{1.3.3.15}{}{g:exercise:idp226363480}%
\par\smallskip%
\noindent\hypertarget{g:solution:idp226364632-main}{}\(\ 49.714\)\end{divisionsolutioneg}%
\begin{divisionsolutioneg}{1.3.3.16}{}{g:exercise:idp226362456}%
\par\smallskip%
\noindent\hypertarget{g:solution:idp226365656-main}{}\(\ 8388.715\)\end{divisionsolutioneg}%
\end{exercisegroup}
\par\medskip\noindent
\end{solutions-subsection}
\end{sectionptx}
%
%
\typeout{************************************************}
\typeout{Section 1.4 Converting from Decimal}
\typeout{************************************************}
%
\begin{sectionptx}{Converting from Decimal}{}{Converting from Decimal}{}{}{x:section:converting-from-decimal}
\begin{introduction}{}%
As we've seen, converting from a different base back to decimal form can be done by expansion:%
\begin{align*}
316_8 \amp = 3\times8^2+1\times8^1+6\times8^0\\
\amp = 3\times64+1\times8+6\times1\\
\amp = 192+8+6\\
\amp=206
\end{align*}
but how do we go back the other way?  In order to do that, we first need to look as some \terminology{modular arithmetic}.\end{introduction}%
%
%
\typeout{************************************************}
\typeout{Subsection 1.4.1 Modular Arithmetic: Finding Quotient and Remainder}
\typeout{************************************************}
%
\begin{subsectionptx}{Modular Arithmetic: Finding Quotient and Remainder}{}{Modular Arithmetic: Finding Quotient and Remainder}{}{}{x:subsection:modular-arithmetic}
Suppose we wish to divide one integer by another. If the second integer doesn't divide into the first evenly, the result is a real number which we can report either in fraction or decimal form.  For example, if we divide 13 by 5, we get%
\begin{equation*}
13\div 5=\frac{13}{5}=2+\frac{3}{5}=2.6
\end{equation*}
although the second last expression is usually written as \(2\frac{3}{5}\), with the \(+\) symbol suppressed.%
\par
If, for some reason, we wish to stay in the land of integers, we could report the result by saying that 13 divided by 5 equals 2, with 3 left over.  In this example, the 2 is called the \terminology{quotient} and the 3 is called the \terminology{remainder}, and we can write this calculation in the following form: \index{quotient}\index{remainder}%
\begin{equation*}
13=\underbrace{2}_{\text{quotient}}\times 5 + \underbrace{3}_{\text{remainder}}
\end{equation*}
%
\par
This sort of calculation is very helpful when doing unit conversions.\footnote{The metric system has reduced the need for this type of calculation, but since we are still stuck with practical units that are not multiples of 10 (time given in days, hours, and minutes, for example), doing this type of conversion is still necessary.\label{g:fn:idp226371800}}  For example, if we know that a certain length of time is 54 hours long and we would prefer to give it in terms of days and hours, then%
\begin{align*}
54 \text{ hours} \amp =\underbrace{\text{days}}_{\text{quotient}}\times 24+\underbrace{\text{hours}}_{\text{remainder}}\\
\amp = 2\times 24+6
\end{align*}
so 54 hours \(=\) 2 days plus 6 hours.%
\par
But how can we find quotients and remainders with a standard calculator?  If we take the number 54 and divide it by 24, our calculator will tell us 2.25.  There are two ways to go:%
\begin{enumerate}
\item{}Take the integer part of 2.25, which is 2.  Then perform the following calculation:%
\begin{align*}
\text{remainder} \amp = 54-\underbrace{2}_{\text{integer part}}\times 24\\
\amp = 54-48\\
\amp = 6
\end{align*}
%
\item{}Alternatively, you can take the decimal part of 2.25, which is 0.25, and multiply it by the divisor, 24:%
\begin{align*}
\text{remainder} \amp = \underbrace{0.25}_{\text{decimal part}}\times 24\\
\amp = 6
\end{align*}
%
\end{enumerate}
%
\begin{example}{}{g:example:idp226376024}%
Find the quotient and remainder for the following: %
\begin{enumerate}
\item{}\(\displaystyle \ 25\div 4\)%
\item{}\(\displaystyle \ 86\div 3\)%
\item{}\(\displaystyle \ 101\div 12\)%
\item{}\(\displaystyle \ 91\div 8\)%
\end{enumerate}
\par\smallskip%
\noindent\textbf{\blocktitlefont Solution}.\label{g:solution:idp226381784}{}\hypertarget{g:solution:idp226381784}{}\quad{}%
\begin{enumerate}
\item{}%
\begin{align*}
\amp 25\div 4=6.25\\
\amp \text{quotient }=\text{integer part of }6.25=6\\
\amp \text{remainder }=25-\underbrace{6}_{\text{integer part}} \times 4=1\\
\amp \text{or remainder }=\underbrace{0.25}_{\text{decimal part}}\times 4=1
\end{align*}
%
\item{}\(\displaystyle \text{ quotient }=28,\text{ remainder }=2\)%
\item{}\(\displaystyle \text{ quotient }=8,\text{ remainder }=5\)%
\item{}\(\displaystyle \text{ quotient }=11,\text{ remainder }=3\)%
\end{enumerate}
\end{example}
\end{subsectionptx}
%
%
\typeout{************************************************}
\typeout{Subsection 1.4.2 Using Quotient and Remainder to Convert from Decimal Form}
\typeout{************************************************}
%
\begin{subsectionptx}{Using Quotient and Remainder to Convert from Decimal Form}{}{Using Quotient and Remainder to Convert from Decimal Form}{}{}{x:subsection:convert-from-decimal-form}
Now that we know how to find quotients and remainders, let's look at some examples that convert from a decimal number into octal.  Consider the number \(316_8\).  We earlier found that \(316_8=206_{10}\).  Let's now go back the other way, using repeated division and remainders.%
\begin{example}{}{g:example:idp226385112}%
Convert the decimal number 206 to octal.%
\par
The procedure is to divide 206 by the base, which in this case is 8, and write down the quotient and the remainder.  Then divide the quotient by the base and write down the new quotient and remainder.  As you can see below, we first divide 206 by 8 to get a quotient of 25 with remainder 6.  If we then divide 25 by 8, we get a quotient of 3 with remainder 1.  we continue doing this until we get a quotient of zero, as in the table below. \begin{center}%
{\tabularfont%
\begin{tabular}{ccc}\hrulethick
&quotient&remainder\tabularnewline\hrulemedium
\multicolumn{1}{r}{\(206\div 8=\)}&25&6\tabularnewline[0pt]
\multicolumn{1}{r}{\(25\div 8=\)}&3&1\tabularnewline[0pt]
\multicolumn{1}{r}{\(3\div 8 =\)}&0&3\tabularnewline\hrulethick
\end{tabular}
}%
\end{center}%
%
\par
Taken in reverse order, the remainders comprise the digits of the equivalent octal number \(316_8\).%
\end{example}
Let's do some more examples using this same procedure.%
\begin{example}{}{g:example:idp226395096}%
Convert the decimal number 41 to binary.\par\smallskip%
\noindent\textbf{\blocktitlefont Solution}.\label{g:solution:idp226397912}{}\hypertarget{g:solution:idp226397912}{}\quad{}\begin{center}%
{\tabularfont%
\begin{tabular}{ccc}\hrulethick
&quotient&remainder\tabularnewline\hrulemedium
\multicolumn{1}{r}{\(41\div 2=\)}&20&1\tabularnewline[0pt]
\multicolumn{1}{r}{\(20\div 2=\)}&10&0\tabularnewline[0pt]
\multicolumn{1}{r}{\(10\div 2 =\)}&5&0\tabularnewline[0pt]
\multicolumn{1}{r}{\(5\div 2=\)}&2&1\tabularnewline[0pt]
\multicolumn{1}{r}{\(2\div 2=\)}&1&0\tabularnewline[0pt]
\multicolumn{1}{r}{\(1\div 2=\)}&0&1\tabularnewline\hrulethick
\end{tabular}
}%
\end{center}%
 Reading the remainders from bottom to top gives \(41_{10}=101001_2\)\end{example}
\begin{example}{}{g:example:idp226402904}%
Convert the decimal number 24362 to hexadecimal.\par\smallskip%
\noindent\textbf{\blocktitlefont Solution}.\label{g:solution:idp226405848}{}\hypertarget{g:solution:idp226405848}{}\quad{}\begin{center}%
{\tabularfont%
\begin{tabular}{cccc}\hrulethick
&quotient&remainder (base 10)&remainder (base 16)\tabularnewline\hrulemedium
\multicolumn{1}{r}{\(24362\div 16=\)}&1522&10&A\tabularnewline[0pt]
\multicolumn{1}{r}{\(1522\div 16=\)}&95&2&2\tabularnewline[0pt]
\multicolumn{1}{r}{\(95\div 16=\)}&5&15&F\tabularnewline[0pt]
\multicolumn{1}{r}{\(5\div 16=\)}&0&5&5
\end{tabular}
}%
\end{center}%
 So \(24362_{10}=5F2A_{16}\)\end{example}
\end{subsectionptx}
%
%
\typeout{************************************************}
\typeout{Subsection 1.4.3 Converting Non-integer Numbers from Decimal Form}
\typeout{************************************************}
%
\begin{subsectionptx}{Converting Non-integer Numbers from Decimal Form}{}{Converting Non-integer Numbers from Decimal Form}{}{}{x:subsection:convert-nonint-from-decimal-form}
We have just seen that to convert an integer number from decimal form, we divide by the base repeatedly.  To convert the fractional part of a non-integer decimal number to another base, we instead will multiply the base repeatedly.%
\begin{example}{}{g:example:idp226423384}%
Convert the decimal number 0.59375 to octal.%
\par
The procedure is to multiply the decimal number by the base, and split the result into the integer part and the fractional (decimal) part.  Then take the fractional part and write it on the next line.  Mulitply it by the base (8 in this example) and split as before.  Stop when the fractional part goes to zero.  Finally, write down the integer parts from top to bottom after the radix point and affix the subscript to indicate the base. \begin{center}%
{\tabularfont%
\begin{tabular}{ccc}\hrulethick
&integer&fractional\tabularnewline\hrulemedium
\multicolumn{1}{r}{\(0.59375\times 8=\)}&\(4\)&\(+0.75\)\tabularnewline[0pt]
\multicolumn{1}{r}{\(0.75\times 8=\)}&\(6\)&\(+0\)\tabularnewline\hrulethick
\end{tabular}
}%
\end{center}%
 Reading the integer parts from top to bottom gives \(0.59375_{10}=0.46_8\).%
\end{example}
\begin{example}{}{g:example:idp226433240}%
Convert the decimal number 0.625 to binary.\par\smallskip%
\noindent\textbf{\blocktitlefont Solution}.\label{g:solution:idp226425688}{}\hypertarget{g:solution:idp226425688}{}\quad{}\begin{center}%
{\tabularfont%
\begin{tabular}{ccc}\hrulethick
&integer&fractional\tabularnewline\hrulemedium
\multicolumn{1}{r}{\(0.625\times 2=\)}&\(1\)&\(+0.25\)\tabularnewline[0pt]
\multicolumn{1}{r}{\(0.25\times 2=\)}&\(0\)&\(+0.5\)\tabularnewline\hrulethick
\multicolumn{1}{r}{\(0.5\times 2=\)}&\(1\)&\(+0\)\tabularnewline\hrulethick
\end{tabular}
}%
\end{center}%
 We have \(0.625_{10}=0.101_2\).\end{example}
\begin{example}{}{g:example:idp226441304}%
Convert the decimal number 0.6328125 to hexadecimal.\par\smallskip%
\noindent\textbf{\blocktitlefont Solution}.\label{g:solution:idp226433624}{}\hypertarget{g:solution:idp226433624}{}\quad{}\begin{center}%
{\tabularfont%
\begin{tabular}{ccc}\hrulethick
&integer&fractional\tabularnewline\hrulemedium
\multicolumn{1}{r}{\(0.6328125\times 16=\)}&\(10 (A)\)&\(+0.125\)\tabularnewline[0pt]
\multicolumn{1}{r}{\(0.125\times 16=\)}&\(2\)&\(+0\)\tabularnewline\hrulethick
\end{tabular}
}%
\end{center}%
 Reading the integer parts from top to bottom gives \(0.6328125_{10}=0.A2_{16}\)\end{example}
Let's do an example with a twist.%
\begin{example}{}{g:example:idp226442968}%
Convert the decimal number 0.3 to octal.\par\smallskip%
\noindent\textbf{\blocktitlefont Solution}.\label{g:solution:idp226443480}{}\hypertarget{g:solution:idp226443480}{}\quad{}\begin{center}%
{\tabularfont%
\begin{tabular}{ccc}\hrulethick
&integer&fractional\tabularnewline\hrulemedium
\multicolumn{1}{r}{\(0.3\times 8=\)}&\(2\)&\(+0.4\)\tabularnewline[0pt]
\multicolumn{1}{r}{\(0.4\times 8=\)}&\(3\)&\(+0.2\)\tabularnewline[0pt]
\multicolumn{1}{r}{\(0.2\times 8=\)}&\(1\)&\(+0.6\)\tabularnewline[0pt]
\multicolumn{1}{r}{\(0.6\times 8=\)}&\(4\)&\(+0.8\)\tabularnewline[0pt]
\multicolumn{1}{r}{\(0.8\times 8=\)}&\(6\)&\(+0.4\)\tabularnewline\hrulethick
\end{tabular}
}%
\end{center}%
You'll notice that we have a bit of a problem:  if we put the fractional\slash{}decimal part 0.4 on the next line, then we'll have a repeat of the second line, and then the next three lines will also repeat, and so on.  What this means is that \(0.3_{10}\) is a repeating decimal in octal: \(0.3_{10}=0.2\overline{3146}_8\).%
\end{example}
Finally, what if we are converting a non-integer number that has both an integer part and a fractional part?  The answer is to do the two parts separately and then put them together.%
\begin{example}{}{g:example:idp226450904}%
Convert the number 17.375 to binary.\par\smallskip%
\noindent\textbf{\blocktitlefont Solution}.\label{g:solution:idp226453464}{}\hypertarget{g:solution:idp226453464}{}\quad{}First, we'll do the integer part and convert \(17_{10}\) to binary. \begin{center}%
{\tabularfont%
\begin{tabular}{ccc}\hrulethick
&quotient&remainder\tabularnewline\hrulemedium
\multicolumn{1}{r}{\(17\div 2=\)}&\(8\)&\(1\)\tabularnewline[0pt]
\multicolumn{1}{r}{\(8\div 2=\)}&\(4\)&\(0\)\tabularnewline[0pt]
\multicolumn{1}{r}{\(4\div 2=\)}&\(2\)&\(0\)\tabularnewline[0pt]
\multicolumn{1}{r}{\(2\div 2 = \)}&\(1\)&\(0\)\tabularnewline[0pt]
\multicolumn{1}{r}{\(1\div 2=\)}&\(0\)&\(1\)\tabularnewline\hrulethick
\end{tabular}
}%
\end{center}%
 So \(17_{10}=10001_2\)%
\par
Now, convert \(0.375_{10}\) to binary. \begin{center}%
{\tabularfont%
\begin{tabular}{ccc}\hrulethick
&integer&fractional\tabularnewline\hrulemedium
\multicolumn{1}{r}{\(0.375\times 2=\)}&\(0\)&\(+0.75\)\tabularnewline[0pt]
\multicolumn{1}{r}{\(0.75\times 2=\)}&\(1\)&\(+0.5\)\tabularnewline[0pt]
\multicolumn{1}{r}{\(0.5\times 2=\)}&\(1\)&\(+0.0\)\tabularnewline\hrulethick
\end{tabular}
}%
\end{center}%
 So \(0.375_{10}=0.011_2\)%
\par
Putting it all together, we get that \(17.375_{10}=10001.011_2\)%
\end{example}
\end{subsectionptx}
%
%
\typeout{************************************************}
\typeout{Exercises 1.4.4 Exercises}
\typeout{************************************************}
%
\begin{exercises-subsection}{Exercises}{}{Exercises}{}{}{g:exercises:idp226482264}
\par\medskip\noindent%
\textbf{Exercise Group.}\space\space%
Convert the following decimal numbers to octal.\begin{exercisegroup}
\begin{divisionexerciseeg}{1}{}{}{g:exercise:idp226477656}%
\(\ 23\)\end{divisionexerciseeg}%
\begin{divisionexerciseeg}{2}{}{}{g:exercise:idp226480728}%
\(\ 19\)\end{divisionexerciseeg}%
\begin{divisionexerciseeg}{3}{}{}{g:exercise:idp226483032}%
\(\ 49\)\end{divisionexerciseeg}%
\begin{divisionexerciseeg}{4}{}{}{g:exercise:idp226482520}%
\(\ 84\)\end{divisionexerciseeg}%
\end{exercisegroup}
\par\medskip\noindent
\par\medskip\noindent%
\textbf{Exercise Group.}\space\space%
Convert the following decimal numbers to binary.\begin{exercisegroup}
\begin{divisionexerciseeg}{5}{}{}{g:exercise:idp226484056}%
\(\ 7\)\end{divisionexerciseeg}%
\begin{divisionexerciseeg}{6}{}{}{g:exercise:idp226488536}%
\(\ 12\)\end{divisionexerciseeg}%
\begin{divisionexerciseeg}{7}{}{}{g:exercise:idp226483928}%
\(\ 23\)\end{divisionexerciseeg}%
\begin{divisionexerciseeg}{8}{}{}{g:exercise:idp226487128}%
\(\ 30\)\end{divisionexerciseeg}%
\end{exercisegroup}
\par\medskip\noindent
\par\medskip\noindent%
\textbf{Exercise Group.}\space\space%
Convert the following decimal numbers to hexadecimal.\begin{exercisegroup}
\begin{divisionexerciseeg}{9}{}{}{g:exercise:idp226489176}%
\(\ 19\)\end{divisionexerciseeg}%
\begin{divisionexerciseeg}{10}{}{}{g:exercise:idp226483160}%
\(\ 25\)\end{divisionexerciseeg}%
\begin{divisionexerciseeg}{11}{}{}{g:exercise:idp226483416}%
\(\ 48\)\end{divisionexerciseeg}%
\begin{divisionexerciseeg}{12}{}{}{g:exercise:idp226494936}%
\(\ 64\)\end{divisionexerciseeg}%
\end{exercisegroup}
\par\medskip\noindent
\par\medskip\noindent%
\textbf{Exercise Group.}\space\space%
Convert the decimal number 27 to the following bases.\begin{exercisegroup}
\begin{divisionexerciseeg}{13}{}{}{g:exercise:idp226495704}%
\(\ \)binary\end{divisionexerciseeg}%
\begin{divisionexerciseeg}{14}{}{}{g:exercise:idp226495448}%
\(\ \)octal\end{divisionexerciseeg}%
\begin{divisionexerciseeg}{15}{}{}{g:exercise:idp226496344}%
\(\ \)hexadecimal\end{divisionexerciseeg}%
\begin{divisionexerciseeg}{16}{}{}{g:exercise:idp226495320}%
\(\ \)base 7\end{divisionexerciseeg}%
\end{exercisegroup}
\par\medskip\noindent
\par\medskip\noindent%
\textbf{Exercise Group.}\space\space%
Convert the following decimal numbers to octal.\begin{exercisegroup}
\begin{divisionexerciseeg}{17}{}{}{g:exercise:idp226497880}%
\(\ 203\)\end{divisionexerciseeg}%
\begin{divisionexerciseeg}{18}{}{}{g:exercise:idp226493144}%
\(\ 1000\)\end{divisionexerciseeg}%
\begin{divisionexerciseeg}{19}{}{}{g:exercise:idp226506456}%
\(\ 2549\)\end{divisionexerciseeg}%
\begin{divisionexerciseeg}{20}{}{}{g:exercise:idp226499160}%
\(\ 56742\)\end{divisionexerciseeg}%
\end{exercisegroup}
\par\medskip\noindent
\par\medskip\noindent%
\textbf{Exercise Group.}\space\space%
Convert the following decimal numbers to binary.\begin{exercisegroup}
\begin{divisionexerciseeg}{21}{}{}{g:exercise:idp226505944}%
\(\ 47\)\end{divisionexerciseeg}%
\begin{divisionexerciseeg}{22}{}{}{g:exercise:idp226504408}%
\(\ 123\)\end{divisionexerciseeg}%
\begin{divisionexerciseeg}{23}{}{}{g:exercise:idp226506072}%
\(\ 80\)\end{divisionexerciseeg}%
\begin{divisionexerciseeg}{24}{}{}{g:exercise:idp226502744}%
\(\ 35\)\end{divisionexerciseeg}%
\end{exercisegroup}
\par\medskip\noindent
\par\medskip\noindent%
\textbf{Exercise Group.}\space\space%
Convert the following decimal numbers to hexadecimal.\begin{exercisegroup}
\begin{divisionexerciseeg}{25}{}{}{g:exercise:idp226501464}%
\(\ 189\)\end{divisionexerciseeg}%
\begin{divisionexerciseeg}{26}{}{}{g:exercise:idp226505560}%
\(\ 3245\)\end{divisionexerciseeg}%
\begin{divisionexerciseeg}{27}{}{}{g:exercise:idp226506968}%
\(\ 11331\)\end{divisionexerciseeg}%
\begin{divisionexerciseeg}{28}{}{}{g:exercise:idp226513112}%
\(\ 100\)\end{divisionexerciseeg}%
\end{exercisegroup}
\par\medskip\noindent
\par\medskip\noindent%
\textbf{Exercise Group.}\space\space%
Convert the decimal number 1234 to the following bases.\begin{exercisegroup}
\begin{divisionexerciseeg}{29}{}{}{g:exercise:idp226512344}%
\(\ \)binary\end{divisionexerciseeg}%
\begin{divisionexerciseeg}{30}{}{}{g:exercise:idp226511448}%
\(\ \)octal\end{divisionexerciseeg}%
\begin{divisionexerciseeg}{31}{}{}{g:exercise:idp226514392}%
\(\ \)hexadecimal\end{divisionexerciseeg}%
\begin{divisionexerciseeg}{32}{}{}{g:exercise:idp226513368}%
\(\ \)base 7\end{divisionexerciseeg}%
\end{exercisegroup}
\par\medskip\noindent
\par\medskip\noindent%
\textbf{Exercise Group.}\space\space%
Perform the following conversions for non-integer numbers.  Give exact answers (do not round off).\begin{exercisegroup}
\begin{divisionexerciseeg}{33}{}{}{g:exercise:idp226513624}%
\(\ 0.359375\) to octal\end{divisionexerciseeg}%
\begin{divisionexerciseeg}{34}{}{}{g:exercise:idp226509656}%
\(\ 0.8125\) to binary\end{divisionexerciseeg}%
\begin{divisionexerciseeg}{35}{}{}{g:exercise:idp226516056}%
\(\ 0.234375\) to hexadecimal\end{divisionexerciseeg}%
\begin{divisionexerciseeg}{36}{}{}{g:exercise:idp226515672}%
\(\ 0.6\) to octal\end{divisionexerciseeg}%
\begin{divisionexerciseeg}{37}{}{}{g:exercise:idp226520664}%
\(\ 0.3\) to binary\end{divisionexerciseeg}%
\begin{divisionexerciseeg}{38}{}{}{g:exercise:idp226523352}%
\(\ 18.125\) to hexadecimal\end{divisionexerciseeg}%
\begin{divisionexerciseeg}{39}{}{}{g:exercise:idp226519640}%
\(\ 37.875\) to octal\end{divisionexerciseeg}%
\end{exercisegroup}
\par\medskip\noindent
\end{exercises-subsection}
%
%
\typeout{************************************************}
\typeout{Solutions 1.4.5 Solutions to Section~{\xreffont\ref*{x:section:converting-from-decimal}} Exercises}
\typeout{************************************************}
%
\begin{solutions-subsection}{Solutions to Section~{\xreffont\ref*{x:section:converting-from-decimal}} Exercises}{}{Solutions to Section~{\xreffont\ref*{x:section:converting-from-decimal}} Exercises}{}{}{g:solutions:idp226520536}
\par\medskip
\noindent\textbf{\normalsize{}1.4.4\space\textperiodcentered\space{}Exercises}
\begin{exercisegroup}
\begin{divisionsolutioneg}{1.4.4.1}{}{g:exercise:idp226477656}%
\par\smallskip%
\noindent\hypertarget{g:solution:idp226478808-main}{}\(\ 23=27_8\)\end{divisionsolutioneg}%
\begin{divisionsolutioneg}{1.4.4.2}{}{g:exercise:idp226480728}%
\par\smallskip%
\noindent\hypertarget{g:solution:idp226482008-main}{}\(\ 19=23_8\)\end{divisionsolutioneg}%
\begin{divisionsolutioneg}{1.4.4.3}{}{g:exercise:idp226483032}%
\par\smallskip%
\noindent\hypertarget{g:solution:idp226488920-main}{}\(\ 49=61_8\)\end{divisionsolutioneg}%
\begin{divisionsolutioneg}{1.4.4.4}{}{g:exercise:idp226482520}%
\par\smallskip%
\noindent\hypertarget{g:solution:idp226485336-main}{}\(\ 84=124_8\)\end{divisionsolutioneg}%
\end{exercisegroup}
\par\medskip\noindent
\begin{exercisegroup}
\begin{divisionsolutioneg}{1.4.4.5}{}{g:exercise:idp226484056}%
\par\smallskip%
\noindent\hypertarget{g:solution:idp226489560-main}{}\(\ 7=111_2\)\end{divisionsolutioneg}%
\begin{divisionsolutioneg}{1.4.4.6}{}{g:exercise:idp226488536}%
\par\smallskip%
\noindent\hypertarget{g:solution:idp226490200-main}{}\(12=1100_2\)\end{divisionsolutioneg}%
\begin{divisionsolutioneg}{1.4.4.7}{}{g:exercise:idp226483928}%
\par\smallskip%
\noindent\hypertarget{g:solution:idp226486744-main}{}\(23=10111_2\)\end{divisionsolutioneg}%
\begin{divisionsolutioneg}{1.4.4.8}{}{g:exercise:idp226487128}%
\par\smallskip%
\noindent\hypertarget{g:solution:idp226489944-main}{}\(\ 30=11110_2\)\end{divisionsolutioneg}%
\end{exercisegroup}
\par\medskip\noindent
\begin{exercisegroup}
\begin{divisionsolutioneg}{1.4.4.9}{}{g:exercise:idp226489176}%
\par\smallskip%
\noindent\hypertarget{g:solution:idp226490328-main}{}\(\ 19=13_{16}\)\end{divisionsolutioneg}%
\begin{divisionsolutioneg}{1.4.4.10}{}{g:exercise:idp226483160}%
\par\smallskip%
\noindent\hypertarget{g:solution:idp226485080-main}{}\(25=19_{16}\)\end{divisionsolutioneg}%
\begin{divisionsolutioneg}{1.4.4.11}{}{g:exercise:idp226483416}%
\par\smallskip%
\noindent\hypertarget{g:solution:idp226490840-main}{}\(48=30_{16}\)\end{divisionsolutioneg}%
\begin{divisionsolutioneg}{1.4.4.12}{}{g:exercise:idp226494936}%
\par\smallskip%
\noindent\hypertarget{g:solution:idp226493400-main}{}\(\ 64=40_{16}\)\end{divisionsolutioneg}%
\end{exercisegroup}
\par\medskip\noindent
\begin{exercisegroup}
\begin{divisionsolutioneg}{1.4.4.13}{}{g:exercise:idp226495704}%
\par\smallskip%
\noindent\hypertarget{g:solution:idp226493784-main}{}\(\ 27=11011_2\)\end{divisionsolutioneg}%
\begin{divisionsolutioneg}{1.4.4.14}{}{g:exercise:idp226495448}%
\par\smallskip%
\noindent\hypertarget{g:solution:idp226496600-main}{}\(\ 27=33_8\)\end{divisionsolutioneg}%
\begin{divisionsolutioneg}{1.4.4.15}{}{g:exercise:idp226496344}%
\par\smallskip%
\noindent\hypertarget{g:solution:idp226492888-main}{}\(\ 27=1B_{16}\)\end{divisionsolutioneg}%
\begin{divisionsolutioneg}{1.4.4.16}{}{g:exercise:idp226495320}%
\par\smallskip%
\noindent\hypertarget{g:solution:idp226496728-main}{}\(\ 27=36_7\)\end{divisionsolutioneg}%
\end{exercisegroup}
\par\medskip\noindent
\begin{exercisegroup}
\begin{divisionsolutioneg}{1.4.4.17}{}{g:exercise:idp226497880}%
\par\smallskip%
\noindent\hypertarget{g:solution:idp226498648-main}{}\(\ 203=313_8\)\end{divisionsolutioneg}%
\begin{divisionsolutioneg}{1.4.4.18}{}{g:exercise:idp226493144}%
\par\smallskip%
\noindent\hypertarget{g:solution:idp226491480-main}{}\(\ 1000=1750_8\)\end{divisionsolutioneg}%
\begin{divisionsolutioneg}{1.4.4.19}{}{g:exercise:idp226506456}%
\par\smallskip%
\noindent\hypertarget{g:solution:idp226499544-main}{}\(\ 2549=4765_8\)\end{divisionsolutioneg}%
\begin{divisionsolutioneg}{1.4.4.20}{}{g:exercise:idp226499160}%
\par\smallskip%
\noindent\hypertarget{g:solution:idp226504536-main}{}\(\ 56742=156646_8\)\end{divisionsolutioneg}%
\end{exercisegroup}
\par\medskip\noindent
\begin{exercisegroup}
\begin{divisionsolutioneg}{1.4.4.21}{}{g:exercise:idp226505944}%
\par\smallskip%
\noindent\hypertarget{g:solution:idp226499672-main}{}\(\ 47=101111_2\)\end{divisionsolutioneg}%
\begin{divisionsolutioneg}{1.4.4.22}{}{g:exercise:idp226504408}%
\par\smallskip%
\noindent\hypertarget{g:solution:idp226500184-main}{}\(\ 123=1111011_2\)\end{divisionsolutioneg}%
\begin{divisionsolutioneg}{1.4.4.23}{}{g:exercise:idp226506072}%
\par\smallskip%
\noindent\hypertarget{g:solution:idp226503512-main}{}\(\ 80=1010000_2\)\end{divisionsolutioneg}%
\begin{divisionsolutioneg}{1.4.4.24}{}{g:exercise:idp226502744}%
\par\smallskip%
\noindent\hypertarget{g:solution:idp226504664-main}{}\(\ 35=100011_2\)\end{divisionsolutioneg}%
\end{exercisegroup}
\par\medskip\noindent
\begin{exercisegroup}
\begin{divisionsolutioneg}{1.4.4.25}{}{g:exercise:idp226501464}%
\par\smallskip%
\noindent\hypertarget{g:solution:idp226502872-main}{}\(\ 189=BD_{16}\)\end{divisionsolutioneg}%
\begin{divisionsolutioneg}{1.4.4.26}{}{g:exercise:idp226505560}%
\par\smallskip%
\noindent\hypertarget{g:solution:idp226506328-main}{}\(\ 3245=CAD_{16}\)\end{divisionsolutioneg}%
\begin{divisionsolutioneg}{1.4.4.27}{}{g:exercise:idp226506968}%
\par\smallskip%
\noindent\hypertarget{g:solution:idp226507736-main}{}\(\ 2C43_{16}\)\end{divisionsolutioneg}%
\begin{divisionsolutioneg}{1.4.4.28}{}{g:exercise:idp226513112}%
\par\smallskip%
\noindent\hypertarget{g:solution:idp226513240-main}{}\(\ 100=64_{16}\)\end{divisionsolutioneg}%
\end{exercisegroup}
\par\medskip\noindent
\begin{exercisegroup}
\begin{divisionsolutioneg}{1.4.4.29}{}{g:exercise:idp226512344}%
\par\smallskip%
\noindent\hypertarget{g:solution:idp226508888-main}{}\(\ 1234=10011010010_2\)\end{divisionsolutioneg}%
\begin{divisionsolutioneg}{1.4.4.30}{}{g:exercise:idp226511448}%
\par\smallskip%
\noindent\hypertarget{g:solution:idp226513752-main}{}\(\ 1234=2322_8\)\end{divisionsolutioneg}%
\begin{divisionsolutioneg}{1.4.4.31}{}{g:exercise:idp226514392}%
\par\smallskip%
\noindent\hypertarget{g:solution:idp226510168-main}{}\(\ 1234=4D2_{16}\)\end{divisionsolutioneg}%
\begin{divisionsolutioneg}{1.4.4.32}{}{g:exercise:idp226513368}%
\par\smallskip%
\noindent\hypertarget{g:solution:idp226510296-main}{}\(\ 1234=3412_7\)\end{divisionsolutioneg}%
\end{exercisegroup}
\par\medskip\noindent
\begin{exercisegroup}
\begin{divisionsolutioneg}{1.4.4.33}{}{g:exercise:idp226513624}%
\par\smallskip%
\noindent\hypertarget{g:solution:idp226507992-main}{}\(\ 0.27_8\)\end{divisionsolutioneg}%
\begin{divisionsolutioneg}{1.4.4.34}{}{g:exercise:idp226509656}%
\par\smallskip%
\noindent\hypertarget{g:solution:idp226516824-main}{}\(\ 0.1101_2\)\end{divisionsolutioneg}%
\begin{divisionsolutioneg}{1.4.4.35}{}{g:exercise:idp226516056}%
\par\smallskip%
\noindent\hypertarget{g:solution:idp226517976-main}{}\(\ 0.3C_{16}\)\end{divisionsolutioneg}%
\begin{divisionsolutioneg}{1.4.4.36}{}{g:exercise:idp226515672}%
\par\smallskip%
\noindent\hypertarget{g:solution:idp226518488-main}{}\(\ 0.\overline{4631}_8\)\end{divisionsolutioneg}%
\begin{divisionsolutioneg}{1.4.4.37}{}{g:exercise:idp226520664}%
\par\smallskip%
\noindent\hypertarget{g:solution:idp226522456-main}{}\(\ 0.0\overline{1001}_2\) is the simplest, but other equivalent expressions are \(0.01\overline{0011}_2\) and \(0.010\overline{0110}_2\), etc.\end{divisionsolutioneg}%
\begin{divisionsolutioneg}{1.4.4.38}{}{g:exercise:idp226523352}%
\par\smallskip%
\noindent\hypertarget{g:solution:idp226520024-main}{}\(\ 12.2_{16}\)\end{divisionsolutioneg}%
\begin{divisionsolutioneg}{1.4.4.39}{}{g:exercise:idp226519640}%
\par\smallskip%
\noindent\hypertarget{g:solution:idp226516184-main}{}\(\ 45.7_8\)\end{divisionsolutioneg}%
\end{exercisegroup}
\par\medskip\noindent
\end{solutions-subsection}
\end{sectionptx}
%
%
\typeout{************************************************}
\typeout{Section 1.5 Converting between Binary, Octal, and Hexadecimal}
\typeout{************************************************}
%
\begin{sectionptx}{Converting between Binary, Octal, and Hexadecimal}{}{Converting between Binary, Octal, and Hexadecimal}{}{}{x:section:convert-bin-oct-hex}
%
%
\typeout{************************************************}
\typeout{Subsection 1.5.1 Converting Between Binary and Octal}
\typeout{************************************************}
%
\begin{subsectionptx}{Converting Between Binary and Octal}{}{Converting Between Binary and Octal}{}{}{x:subsection:convert-binary-octal}
Let's first count from zero to 7 in both octal and binary. \begin{center}%
{\tabularfont%
\begin{tabular}{cc}\hrulethick
octal&binary\tabularnewline\hrulemedium
\(0_8\)&\(0_2\)\tabularnewline[0pt]
\(1_8\)&\(1_2\)\tabularnewline[0pt]
\(2_8\)&\(10_2\)\tabularnewline[0pt]
\(3_8\)&\(11_2\)\tabularnewline[0pt]
\(4_8\)&\(100_2\)\tabularnewline[0pt]
\(5_8\)&\(101_2\)\tabularnewline[0pt]
\(6_8\)&\(110_2\)\tabularnewline[0pt]
\(7_8\)&\(111_2\)\tabularnewline\hrulethick
\end{tabular}
}%
\end{center}%
%
\par
If we add leading zeros where necessary to bring all binary numbers to three digits, we can see that%
\begin{align*}
6_8 \amp = 110_2\\
3_8 \amp = 011_2\\
4_8 \amp = 100_2
\end{align*}
The reason we are doing this is that if we look at a number expressed both in octal and binary, such as \(634_8=110011100_2\), we can see an interesting pattern.  If the binary number is split into groups of three digits, \emph{starting from the right-hand-side},%
\begin{align*}
634_8 \amp = 110011100_2\\
\amp = 110\ 011\ 110_2\\
\amp = \overbrace{110}^{6}\overbrace{011}^{3}\overbrace{100}^{4}
\end{align*}
then we can see that each group of three digits is equal to the corresponding octal number in that place.%
\par
Let's use this observation to convert from octal to binary. \index{conversion!from octal!to binary} \begin{example}{}{g:example:idp226541912}%
Convert the following octal numbers to binary: %
\begin{enumerate}
\item{}\(\displaystyle \ 1110_2\)%
\item{}\(\displaystyle \ 1010000100_2\)%
\end{enumerate}
\par\smallskip%
\noindent\textbf{\blocktitlefont Solution}.\label{g:solution:idp226547288}{}\hypertarget{g:solution:idp226547288}{}\quad{}%
\begin{enumerate}
\item{}\(\ \)For each digit in the octal number, write the corresponding three-digit binary number (includeing any leading zeros as necessary).  Then put the groups of three all together and, optionally, drop any leading zeros.%
\begin{align*}
23_8 \amp = \overbrace{010}^2\overbrace{011}^3\\
\amp = 010\ 011\\
\amp = 10011_2,\text{ dropping the leading zero}
\end{align*}
%
\item{}%
\begin{align*}
671_8 \amp = \overbrace{110}^6\overbrace{111}^7\overbrace{001}^1\\
\amp = 110\ 111\ 001\\
\amp = 110111001_2
\end{align*}
%
\end{enumerate}
\end{example}
%
\par
We can use a similar technique to convert from binary to octal. \index{conversion!from binary!to octal} \begin{example}{}{g:example:idp226554584}%
Convert the following binary numbers to octal. %
\begin{enumerate}
\item{}\(\displaystyle \ 11110_2\)%
\item{}\(\displaystyle \ 1010000100_2\)%
\end{enumerate}
\par\smallskip%
\noindent\textbf{\blocktitlefont Solution}.\label{g:solution:idp226555992}{}\hypertarget{g:solution:idp226555992}{}\quad{}First, group the binary digits into threes, starting from the right-hand side.  The last group might need leading zeros to be added to make a group of three.  Then rewrite each set of three into the corresponding digit in octal. %
\begin{enumerate}
\item{}\(\displaystyle \ 11110_2=011\ 110=\overbrace{011}^3\overbrace{110}^6=36_8\)%
\item{}\(\displaystyle \ 1010000100_2=001\ 010\ 000\ 100=\overbrace{001}^1\overbrace{010}^2\overbrace{000}^0\overbrace{100}^4=1204_8\)%
\end{enumerate}
\end{example}
%
\end{subsectionptx}
%
%
\typeout{************************************************}
\typeout{Subsection 1.5.2 Converting Between Binary and Hexadecimal}
\typeout{************************************************}
%
\begin{subsectionptx}{Converting Between Binary and Hexadecimal}{}{Converting Between Binary and Hexadecimal}{}{}{x:subsection:convert-binary-hexadecimal}
As before, let's first count from zero to fifteen in both hexadecimal and binary.%
\begin{sidebyside}{2}{0.2}{0.2}{0.2}%
\begin{sbspanel}{0.2}%
\resizebox{\ifdim\width > \linewidth\linewidth\else\width\fi}{!}{%
{\centering%
{\tabularfont%
\begin{tabular}{cc}\hrulethick
hexadecimal&binary\tabularnewline\hrulemedium
\(0_{16}\)&\(0_2\)\tabularnewline[0pt]
\(1_{16}\)&\(1_2\)\tabularnewline[0pt]
\(2_{16}\)&\(10_2\)\tabularnewline[0pt]
\(3_{16}\)&\(11_2\)\tabularnewline[0pt]
\(4_{16}\)&\(100_2\)\tabularnewline[0pt]
\(5_{16}\)&\(101_2\)\tabularnewline[0pt]
\(6_{16}\)&\(110_2\)\tabularnewline[0pt]
\(7_{16}\)&\(111_2\)\tabularnewline\hrulethick
\end{tabular}
}%
\par}
}%
\end{sbspanel}%
\begin{sbspanel}{0.2}%
\resizebox{\ifdim\width > \linewidth\linewidth\else\width\fi}{!}{%
{\centering%
{\tabularfont%
\begin{tabular}{cc}\hrulethick
hexadecimal&binary\tabularnewline\hrulemedium
\(8_{16}\)&\(1000_2\)\tabularnewline[0pt]
\(9_{16}\)&\(1001_2\)\tabularnewline[0pt]
\(A_{16}\)&\(1010_2\)\tabularnewline[0pt]
\(B_{16}\)&\(1011_2\)\tabularnewline[0pt]
\(C_{16}\)&\(1100_2\)\tabularnewline[0pt]
\(D_{16}\)&\(1101_2\)\tabularnewline[0pt]
\(E_{16}\)&\(1110_2\)\tabularnewline[0pt]
\(F_{16}\)&\(1111_2\)\tabularnewline\hrulethick
\end{tabular}
}%
\par}
}%
\end{sbspanel}%
\end{sidebyside}%
\par
If we add leading zeros where necessary to write all binary numbers with four digits, we can see that, for example,%
\begin{align*}
9_{16} \amp = 1001_2\\
D_{16} \amp = 1101_2\\
2_{16} \amp = 0010_2
\end{align*}
Now consider the number \(9D2_{16}\).  To convert this to binary, write each hexadecimal digit in 4-digit binary (padding with leading zeros as necessary, as shown above) from left to right:%
\begin{align*}
9D2_{16} \amp = \overbrace{1001}^9\overbrace{1101}^D\overbrace{0010}^2\\
\amp = 100111010010_2
\end{align*}
%
\par
Let's work through more examples.%
\begin{example}{}{g:example:idp226583256}%
Convert the following hexadecimal numbers to binary. \index{conversion!from hexadecimal!to binary}%
%
\begin{enumerate}
\item{}\(\displaystyle \ 3A_{16}\)%
\item{}\(\displaystyle \ F02C_{16}\)%
\end{enumerate}
\par\smallskip%
\noindent\textbf{\blocktitlefont Solution}.\label{g:solution:idp226322776}{}\hypertarget{g:solution:idp226322776}{}\quad{}%
\begin{enumerate}
\item{}%
\begin{align*}
3A_{16} \amp = \overbrace{0011}^3\overbrace{1010}^A\\
\amp = 111010_2,\text{ dropping the leading zeros}
\end{align*}
%
\item{}%
\begin{align*}
F02C_{16} \amp = \overbrace{1111}^F\overbrace{0000}^0\overbrace{0010}^2\overbrace{1100}^C\\
\amp = 1111000000101100_2
\end{align*}
%
\end{enumerate}
\end{example}
\begin{example}{}{g:example:idp226320344}%
Convert the following binary numbers to hexadecimal. %
\begin{enumerate}
\item{}\(\displaystyle \ 11110_2\)%
\item{}\(\displaystyle \ 1010001111_2\)%
\end{enumerate}
\par\smallskip%
\noindent\textbf{\blocktitlefont Solution}.\label{g:solution:idp226321624}{}\hypertarget{g:solution:idp226321624}{}\quad{}First, group the binary digits into sets of four, \emph{starting from the right-hand side}.  Then rewrite each set of four into the corresponding hexadecimal digit. %
\begin{enumerate}
\item{}\(\displaystyle \ 11110_2 = 0001\ 1110 = \overbrace{0001}^1\overbrace{1110}^{14}=1E_{16}\)%
\item{}\(\displaystyle \ 1010001111_2 = 0010\ 1000\ 1111=\overbrace{0010}^2\overbrace{1000}^8\overbrace{1111}^{15}=28F_{16}\)%
\end{enumerate}
Padding the left-most group of four with leading zeros is done for clarity, rather than for necessity.  You can skip that part of the process and instead write, for example,%
\begin{equation*}
11110_2=\overbrace{1}^1\overbrace{1110}^{14}=1E_{16}
\end{equation*}
%
\end{example}
\end{subsectionptx}
%
%
\typeout{************************************************}
\typeout{Subsection 1.5.3 Converting Between Octal and Hexadecimal}
\typeout{************************************************}
%
\begin{subsectionptx}{Converting Between Octal and Hexadecimal}{}{Converting Between Octal and Hexadecimal}{}{}{x:subsection:convert-octal-hexadecimal}
The fastest way to do this is to convert into binary first, then regroup the binary digits.%
\begin{example}{}{g:example:idp226589416}%
Convert the following octal numbers to hexadecimal. %
\begin{enumerate}
\item{}\(\displaystyle \ 72_8\)%
\item{}\(\displaystyle \ 333_8\)%
\end{enumerate}
\par\smallskip%
\noindent\textbf{\blocktitlefont Solution}.\label{g:solution:idp226591848}{}\hypertarget{g:solution:idp226591848}{}\quad{}%
\begin{enumerate}
\item{}%
\begin{align*}
72_8 \amp = \overbrace{111}^7\overbrace{010}^2\\
\amp = 111010_2\\
\amp = 11\ 1010\\
\amp = \overbrace{11}^3\overbrace{1010}^{A}\\
\amp = 3A_{16}
\end{align*}
%
\item{}%
\begin{align*}
333_8 \amp =\overbrace{011}^3\overbrace{011}^3\overbrace{011}^3\\
\amp = 011\ 011\ 011\\
\amp = 11011011_2\\
\amp = 1101\ 1011\\
\amp = \overbrace{1101}^{D}\overbrace{1011}^{B}\\
\amp = DB_{16}
\end{align*}
%
\end{enumerate}
\end{example}
\begin{example}{}{g:example:idp226596712}%
Convert the following hexadecimal numbers to octal. %
\begin{enumerate}
\item{}\(\displaystyle \ 9A_{16}\)%
\item{}\(\displaystyle \ 4E59_{16}\)%
\end{enumerate}
\par\smallskip%
\noindent\textbf{\blocktitlefont Solution}.\label{g:solution:idp226604008}{}\hypertarget{g:solution:idp226604008}{}\quad{}%
\begin{enumerate}
\item{}%
\begin{align*}
9A_{16} \amp = \overbrace{1001}^9\overbrace{1010}^A\\
\amp = 10011010_2\\
\amp = 10\ 011\ 010\\
\amp = \overbrace{10}^2\overbrace{011}^3\overbrace{010}^2\\
\amp = 232_8
\end{align*}
%
\item{}%
\begin{align*}
4E59_{16} \amp = \overbrace{0100}^4\overbrace{1110}^E\overbrace{0101}^5\overbrace{1001}^9\\
\amp = 0100\ 1110\ 0101\ 1001\\
\amp = 100111001011001_2\\
\amp = 100\ 111\ 001\ 011\ 001\\
\amp = \overbrace{100}^4\overbrace{111}^7\overbrace{001}^1\overbrace{011}^3\overbrace{001}^1\\
\amp = 47131_8
\end{align*}
%
\end{enumerate}
\end{example}
\end{subsectionptx}
%
%
\typeout{************************************************}
\typeout{Subsection 1.5.4 Converting Non-integer Numbers}
\typeout{************************************************}
%
\begin{subsectionptx}{Converting Non-integer Numbers}{}{Converting Non-integer Numbers}{}{}{x:subsection:converting-non-integers}
Converting non-integer numbers between binary, octal, and hexadecimal works in the way you'd expect.%
\begin{example}{}{g:example:idp226597992}%
Convert the following octal numbers to binary. %
\begin{enumerate}
\item{}\(\displaystyle \ 0.32_8\)%
\item{}\(\displaystyle \ 6.71_8\)%
\end{enumerate}
\par\smallskip%
\noindent\textbf{\blocktitlefont Solution}.\label{g:solution:idp226610920}{}\hypertarget{g:solution:idp226610920}{}\quad{}%
\begin{enumerate}
\item{}For each digit in the octal number, write the corresponding three-digit binary number (with leading zeros if necessary).  Then put the groups of three all together and, optionally, drop any leading zeros to the left of the radix point and any trailing zeros to the right of the radix point.%
%
\begin{align*}
0.32_8 \amp = 0.\overbrace{011}^3\overbrace{010}^2\\
\amp = 0.011\ 010\\
\amp = 0.01101_2,\text{ dropping the trailing zero}
\end{align*}
\item{}%
\begin{align*}
6.71_8 \amp = \overbrace{110}^6.\overbrace{111}^7\overbrace{001}^1\\
\amp = 110.111\ 001\\
\amp = 110.111001_2
\end{align*}
%
\end{enumerate}
\end{example}
\begin{example}{}{g:example:idp226610664}%
Convert the following binary numbers to octal: %
\begin{enumerate}
\item{}\(\displaystyle \ 11.11_2\)%
\item{}\(\displaystyle \ 1010.000101_2\)%
\end{enumerate}
\par\smallskip%
\noindent\textbf{\blocktitlefont Solution}.\label{g:solution:idp226613224}{}\hypertarget{g:solution:idp226613224}{}\quad{}First, group the binary digits into threes, starting from the radix point, adding extra leading and\slash{}or trailing zeros as appropriate.  Then rewrite each set of three as the corresponding octal digit.%
%
\begin{enumerate}
\item{}\(\displaystyle \ 11.11_2 = 011.110=\overbrace{011}^3.\overbrace{110}^6=3.6_8\)%
\item{}\(\displaystyle \ 1010.000101_2=1\ 010.000\ 101=\overbrace{1}^1\overbrace{010}^2.\overbrace{000}^0\overbrace{101}^5=12.05_8\)%
\end{enumerate}
\end{example}
\begin{example}{}{g:example:idp226608360}%
Convert the following hexadecimal numbers to binary. %
\begin{enumerate}
\item{}\(\displaystyle \ 3.A_{16}\)%
\item{}\(\displaystyle \ F.02C_{16}\)%
\end{enumerate}
\par\smallskip%
\noindent\textbf{\blocktitlefont Solution}.\label{g:solution:idp226617704}{}\hypertarget{g:solution:idp226617704}{}\quad{}%
\begin{enumerate}
\item{}%
\begin{align*}
3.A_{16} \amp = \overbrace{0011}^3.\overbrace{1010}^A\\
\amp = 11.101_2,\text{ dropping the leading and trailing zeros}
\end{align*}
%
\item{}%
\begin{align*}
F.02C_{16} \amp = \overbrace{1111}^F.\overbrace{0000}^0\overbrace{0010}^2\overbrace{1100}^C\\
\amp = 1111.0000 0010 11_2
\end{align*}
%
\end{enumerate}
\end{example}
We use the same techniques as before to convert non-integer numbers between octal and hexadecimal: convert to binary as an intermediate step.%
\begin{example}{}{g:example:idp226617576}%
Convert the following hexadecimal numbers to octal. %
\begin{enumerate}
\item{}\(\displaystyle \ 9.A_{16}\)%
\item{}\(\displaystyle \ 4.E59_{16}\)%
\end{enumerate}
\par\smallskip%
\noindent\textbf{\blocktitlefont Solution}.\label{g:solution:idp226621160}{}\hypertarget{g:solution:idp226621160}{}\quad{}%
\begin{enumerate}
\item{}%
\begin{align*}
9A_{16} \amp = \overbrace{1001}^9.\overbrace{1010}^A\\
\amp = 1001.1010_2\\
\amp = 1\ 001.101\ 0\\
\amp = \overbrace{1}^1\overbrace{001}^1.\overbrace{101}^5\\
\amp = 11.5_8
\end{align*}
%
\item{}%
\begin{align*}
4.E59_{16} \amp = \overbrace{0100}^4.\overbrace{1110}^E\overbrace{0101}^5\overbrace{1001}^9\\
\amp = 0100.1110\ 0101\ 1001\\
\amp = 100.111001011001_2\\
\amp = 100.111\ 001\ 011\ 001\\
\amp = \overbrace{100}^4.\overbrace{111}^7\overbrace{001}^1\overbrace{011}^3\overbrace{001}^1\\
\amp = 4.7131_8
\end{align*}
%
\end{enumerate}
\end{example}
\end{subsectionptx}
%
%
\typeout{************************************************}
\typeout{Exercises 1.5.5 Exercises}
\typeout{************************************************}
%
\begin{exercises-subsection}{Exercises}{}{Exercises}{}{}{g:exercises:idp226626920}
\par\medskip\noindent%
\textbf{Exercise Group.}\space\space%
Convert the following octal numbers to binary:\begin{exercisegroup}
\begin{divisionexerciseeg}{1}{}{}{g:exercise:idp226627048}%
\(\ 30_8\)\end{divisionexerciseeg}%
\begin{divisionexerciseeg}{2}{}{}{g:exercise:idp226622184}%
\(\ 21_8\)\end{divisionexerciseeg}%
\begin{divisionexerciseeg}{3}{}{}{g:exercise:idp226626280}%
\(\ 113_8\)\end{divisionexerciseeg}%
\begin{divisionexerciseeg}{4}{}{}{g:exercise:idp226623976}%
\(\ 201_8\)\end{divisionexerciseeg}%
\begin{divisionexerciseeg}{5}{}{}{g:exercise:idp226633832}%
\(\ 340_8\)\end{divisionexerciseeg}%
\begin{divisionexerciseeg}{6}{}{}{g:exercise:idp226630632}%
\(\ 1104_8\)\end{divisionexerciseeg}%
\end{exercisegroup}
\par\medskip\noindent
\par\medskip\noindent%
\textbf{Exercise Group.}\space\space%
Convert the following hexadecimal numbers to binary.\begin{exercisegroup}
\begin{divisionexerciseeg}{7}{}{}{g:exercise:idp226637672}%
\(\ 13_{16}\)\end{divisionexerciseeg}%
\begin{divisionexerciseeg}{8}{}{}{g:exercise:idp226632168}%
\(\ 2B_{16}\)\end{divisionexerciseeg}%
\begin{divisionexerciseeg}{9}{}{}{g:exercise:idp226637800}%
\(\ 1B_{16}\)\end{divisionexerciseeg}%
\begin{divisionexerciseeg}{10}{}{}{g:exercise:idp226632296}%
\(\ 56_{16}\)\end{divisionexerciseeg}%
\begin{divisionexerciseeg}{11}{}{}{g:exercise:idp226633960}%
\(\ 9A_{16}\)\end{divisionexerciseeg}%
\begin{divisionexerciseeg}{12}{}{}{g:exercise:idp226631912}%
\(\ 29A_{16}\)\end{divisionexerciseeg}%
\end{exercisegroup}
\par\medskip\noindent
\par\medskip\noindent%
\textbf{Exercise Group.}\space\space%
Convert the following binary numbers to octal.\begin{exercisegroup}
\begin{divisionexerciseeg}{13}{}{}{g:exercise:idp226629992}%
\(\ 1100_2\)\end{divisionexerciseeg}%
\begin{divisionexerciseeg}{14}{}{}{g:exercise:idp226643688}%
\(\ 10100_2\)\end{divisionexerciseeg}%
\begin{divisionexerciseeg}{15}{}{}{g:exercise:idp226639592}%
\(\ 100011_2\)\end{divisionexerciseeg}%
\begin{divisionexerciseeg}{16}{}{}{g:exercise:idp226644840}%
\(\ 110010_2\)\end{divisionexerciseeg}%
\begin{divisionexerciseeg}{17}{}{}{g:exercise:idp226645736}%
\(\ 1001100_2\)\end{divisionexerciseeg}%
\begin{divisionexerciseeg}{18}{}{}{g:exercise:idp226638824}%
\(\ 1100110_2\)\end{divisionexerciseeg}%
\end{exercisegroup}
\par\medskip\noindent
\par\medskip\noindent%
\textbf{Exercise Group.}\space\space%
Convert the following binary numbers to hexadecimal.\begin{exercisegroup}
\begin{divisionexerciseeg}{19}{}{}{g:exercise:idp226639848}%
\(\ 10011_2\)\end{divisionexerciseeg}%
\begin{divisionexerciseeg}{20}{}{}{g:exercise:idp226642280}%
\(\ 11010_2\)\end{divisionexerciseeg}%
\begin{divisionexerciseeg}{21}{}{}{g:exercise:idp226645224}%
\(\ 101001_2\)\end{divisionexerciseeg}%
\begin{divisionexerciseeg}{22}{}{}{g:exercise:idp226650600}%
\(\ 1000000_2\)\end{divisionexerciseeg}%
\begin{divisionexerciseeg}{23}{}{}{g:exercise:idp226647400}%
\(\ 1100011_2\)\end{divisionexerciseeg}%
\begin{divisionexerciseeg}{24}{}{}{g:exercise:idp226649320}%
\(\ 1101111_2\)\end{divisionexerciseeg}%
\end{exercisegroup}
\par\medskip\noindent
\par\medskip\noindent%
\textbf{Exercise Group.}\space\space%
Convert the following octal numbers to hexadecimal:\begin{exercisegroup}
\begin{divisionexerciseeg}{25}{}{}{g:exercise:idp226654312}%
\(\ 16_8\)\end{divisionexerciseeg}%
\begin{divisionexerciseeg}{26}{}{}{g:exercise:idp226648680}%
\(\ 53_8\)\end{divisionexerciseeg}%
\begin{divisionexerciseeg}{27}{}{}{g:exercise:idp226651624}%
\(\ 73_8\)\end{divisionexerciseeg}%
\begin{divisionexerciseeg}{28}{}{}{g:exercise:idp226652264}%
\(\ 142_8\)\end{divisionexerciseeg}%
\begin{divisionexerciseeg}{29}{}{}{g:exercise:idp226647912}%
\(\ 2457_8\)\end{divisionexerciseeg}%
\begin{divisionexerciseeg}{30}{}{}{g:exercise:idp226648424}%
\(\ 5002_8\)\end{divisionexerciseeg}%
\end{exercisegroup}
\par\medskip\noindent
\par\medskip\noindent%
\textbf{Exercise Group.}\space\space%
Convert the following hexadecimal numbers to octal.\begin{exercisegroup}
\begin{divisionexerciseeg}{31}{}{}{g:exercise:idp226654824}%
\(\ F_{16}\)\end{divisionexerciseeg}%
\begin{divisionexerciseeg}{32}{}{}{g:exercise:idp226657896}%
\(\ 30_{16}\)\end{divisionexerciseeg}%
\begin{divisionexerciseeg}{33}{}{}{g:exercise:idp226659560}%
\(\ 5F_{16}\)\end{divisionexerciseeg}%
\begin{divisionexerciseeg}{34}{}{}{g:exercise:idp226656872}%
\(\ C2_{16}\)\end{divisionexerciseeg}%
\begin{divisionexerciseeg}{35}{}{}{g:exercise:idp226659944}%
\(\ 1D07_{16}\)\end{divisionexerciseeg}%
\begin{divisionexerciseeg}{36}{}{}{g:exercise:idp226658280}%
\(\ A2E6_{16}\)\end{divisionexerciseeg}%
\end{exercisegroup}
\par\medskip\noindent
\par\medskip\noindent%
\textbf{Exercise Group.}\space\space%
Perform the following conversions for non-integer numbers.\begin{exercisegroup}
\begin{divisionexerciseeg}{37}{}{}{g:exercise:idp226661480}%
\(\ E.15_{16}\) to binary.\end{divisionexerciseeg}%
\begin{divisionexerciseeg}{38}{}{}{g:exercise:idp226656616}%
\(\ 4.702_8\) to binary\end{divisionexerciseeg}%
\begin{divisionexerciseeg}{39}{}{}{g:exercise:idp226668776}%
\(\ 10.011_2\) to hexadecimal\end{divisionexerciseeg}%
\begin{divisionexerciseeg}{40}{}{}{g:exercise:idp226664040}%
\(\ 110.1_2\) to octal\end{divisionexerciseeg}%
\begin{divisionexerciseeg}{41}{}{}{g:exercise:idp226668136}%
\(\ 7B.B_{16}\) to octal\end{divisionexerciseeg}%
\begin{divisionexerciseeg}{42}{}{}{g:exercise:idp226670056}%
\(\ 4.1702_8\) to hexadecimal\end{divisionexerciseeg}%
\end{exercisegroup}
\par\medskip\noindent
\end{exercises-subsection}
%
%
\typeout{************************************************}
\typeout{Solutions 1.5.6 Solutions to Section~{\xreffont\ref*{x:section:convert-bin-oct-hex}} Exercises}
\typeout{************************************************}
%
\begin{solutions-subsection}{Solutions to Section~{\xreffont\ref*{x:section:convert-bin-oct-hex}} Exercises}{}{Solutions to Section~{\xreffont\ref*{x:section:convert-bin-oct-hex}} Exercises}{}{}{g:solutions:idp226667368}
\par\medskip
\noindent\textbf{\normalsize{}1.5.5\space\textperiodcentered\space{}Exercises}
\begin{exercisegroup}
\begin{divisionsolutioneg}{1.5.5.1}{}{g:exercise:idp226627048}%
\par\smallskip%
\noindent\hypertarget{g:solution:idp226628328-main}{}\(\ 30_8=11000_2\)\end{divisionsolutioneg}%
\begin{divisionsolutioneg}{1.5.5.2}{}{g:exercise:idp226622184}%
\par\smallskip%
\noindent\hypertarget{g:solution:idp226623464-main}{}\(\ 21_8=10001_2\)\end{divisionsolutioneg}%
\begin{divisionsolutioneg}{1.5.5.3}{}{g:exercise:idp226626280}%
\par\smallskip%
\noindent\hypertarget{g:solution:idp226621800-main}{}\(\ 113_8=1001011_2\)\end{divisionsolutioneg}%
\begin{divisionsolutioneg}{1.5.5.4}{}{g:exercise:idp226623976}%
\par\smallskip%
\noindent\hypertarget{g:solution:idp226624744-main}{}\(\ 201_8=10000001_2\)\end{divisionsolutioneg}%
\begin{divisionsolutioneg}{1.5.5.5}{}{g:exercise:idp226633832}%
\par\smallskip%
\noindent\hypertarget{g:solution:idp226633576-main}{}\(\ 340_8=11100000_2\)\end{divisionsolutioneg}%
\begin{divisionsolutioneg}{1.5.5.6}{}{g:exercise:idp226630632}%
\par\smallskip%
\noindent\hypertarget{g:solution:idp226636392-main}{}\(\ 1104_8=1001000100_2\)\end{divisionsolutioneg}%
\end{exercisegroup}
\par\medskip\noindent
\begin{exercisegroup}
\begin{divisionsolutioneg}{1.5.5.7}{}{g:exercise:idp226637672}%
\par\smallskip%
\noindent\hypertarget{g:solution:idp226631784-main}{}\(\ 13_{16}=10011_2\)\end{divisionsolutioneg}%
\begin{divisionsolutioneg}{1.5.5.8}{}{g:exercise:idp226632168}%
\par\smallskip%
\noindent\hypertarget{g:solution:idp226633448-main}{}\(\ 2B_{16}=101011_2\)\end{divisionsolutioneg}%
\begin{divisionsolutioneg}{1.5.5.9}{}{g:exercise:idp226637800}%
\par\smallskip%
\noindent\hypertarget{g:solution:idp226633704-main}{}\(\ 1B_{16}=11011_2\)\end{divisionsolutioneg}%
\begin{divisionsolutioneg}{1.5.5.10}{}{g:exercise:idp226632296}%
\par\smallskip%
\noindent\hypertarget{g:solution:idp226632552-main}{}\(\ 56_{16}=1010110_2\)\end{divisionsolutioneg}%
\begin{divisionsolutioneg}{1.5.5.11}{}{g:exercise:idp226633960}%
\par\smallskip%
\noindent\hypertarget{g:solution:idp226636264-main}{}\(\ 9A_{16}=10011010_2\)\end{divisionsolutioneg}%
\begin{divisionsolutioneg}{1.5.5.12}{}{g:exercise:idp226631912}%
\par\smallskip%
\noindent\hypertarget{g:solution:idp226638056-main}{}\(\ 29A_{16}=1010011010_2\)\end{divisionsolutioneg}%
\end{exercisegroup}
\par\medskip\noindent
\begin{exercisegroup}
\begin{divisionsolutioneg}{1.5.5.13}{}{g:exercise:idp226629992}%
\par\smallskip%
\noindent\hypertarget{g:solution:idp226638952-main}{}\(\ 1100_2=14_8\)\end{divisionsolutioneg}%
\begin{divisionsolutioneg}{1.5.5.14}{}{g:exercise:idp226643688}%
\par\smallskip%
\noindent\hypertarget{g:solution:idp226639336-main}{}\(\ 10100_2=24_8\)\end{divisionsolutioneg}%
\begin{divisionsolutioneg}{1.5.5.15}{}{g:exercise:idp226639592}%
\par\smallskip%
\noindent\hypertarget{g:solution:idp226640360-main}{}\(\ 100011_2=43_8\)\end{divisionsolutioneg}%
\begin{divisionsolutioneg}{1.5.5.16}{}{g:exercise:idp226644840}%
\par\smallskip%
\noindent\hypertarget{g:solution:idp226645992-main}{}\(\ 110010_2=62_8\)\end{divisionsolutioneg}%
\begin{divisionsolutioneg}{1.5.5.17}{}{g:exercise:idp226645736}%
\par\smallskip%
\noindent\hypertarget{g:solution:idp226638696-main}{}\(\ 1001100_2=114_8\)\end{divisionsolutioneg}%
\begin{divisionsolutioneg}{1.5.5.18}{}{g:exercise:idp226638824}%
\par\smallskip%
\noindent\hypertarget{g:solution:idp226642536-main}{}\(\ 1100110_2=146_8\)\end{divisionsolutioneg}%
\end{exercisegroup}
\par\medskip\noindent
\begin{exercisegroup}
\begin{divisionsolutioneg}{1.5.5.19}{}{g:exercise:idp226639848}%
\par\smallskip%
\noindent\hypertarget{g:solution:idp226643816-main}{}\(\ 10011_2=13_{16}\)\end{divisionsolutioneg}%
\begin{divisionsolutioneg}{1.5.5.20}{}{g:exercise:idp226642280}%
\par\smallskip%
\noindent\hypertarget{g:solution:idp226643560-main}{}\(\ 11010_2=1A_{16}\)\end{divisionsolutioneg}%
\begin{divisionsolutioneg}{1.5.5.21}{}{g:exercise:idp226645224}%
\par\smallskip%
\noindent\hypertarget{g:solution:idp226640744-main}{}\(\ 101001_2=29_{16}\)\end{divisionsolutioneg}%
\begin{divisionsolutioneg}{1.5.5.22}{}{g:exercise:idp226650600}%
\par\smallskip%
\noindent\hypertarget{g:solution:idp226651368-main}{}\(\ 1000000_2=40_{16}\)\end{divisionsolutioneg}%
\begin{divisionsolutioneg}{1.5.5.23}{}{g:exercise:idp226647400}%
\par\smallskip%
\noindent\hypertarget{g:solution:idp226649960-main}{}\(\ 1100011_2=63_{16}\)\end{divisionsolutioneg}%
\begin{divisionsolutioneg}{1.5.5.24}{}{g:exercise:idp226649320}%
\par\smallskip%
\noindent\hypertarget{g:solution:idp226654184-main}{}\(\ 1101111_2=6F_{16}\)\end{divisionsolutioneg}%
\end{exercisegroup}
\par\medskip\noindent
\begin{exercisegroup}
\begin{divisionsolutioneg}{1.5.5.25}{}{g:exercise:idp226654312}%
\par\smallskip%
\noindent\hypertarget{g:solution:idp226654440-main}{}\(\ 16_8=E_{16}\)\end{divisionsolutioneg}%
\begin{divisionsolutioneg}{1.5.5.26}{}{g:exercise:idp226648680}%
\par\smallskip%
\noindent\hypertarget{g:solution:idp226647528-main}{}\(\ 53_8=2B_{16}\)\end{divisionsolutioneg}%
\begin{divisionsolutioneg}{1.5.5.27}{}{g:exercise:idp226651624}%
\par\smallskip%
\noindent\hypertarget{g:solution:idp226649576-main}{}\(\ 73_8=3B_{16}\)\end{divisionsolutioneg}%
\begin{divisionsolutioneg}{1.5.5.28}{}{g:exercise:idp226652264}%
\par\smallskip%
\noindent\hypertarget{g:solution:idp226652648-main}{}\(\ 142_8=62_{16}\)\end{divisionsolutioneg}%
\begin{divisionsolutioneg}{1.5.5.29}{}{g:exercise:idp226647912}%
\par\smallskip%
\noindent\hypertarget{g:solution:idp226653544-main}{}\(\ 2457_8=52F_{16}\)\end{divisionsolutioneg}%
\begin{divisionsolutioneg}{1.5.5.30}{}{g:exercise:idp226648424}%
\par\smallskip%
\noindent\hypertarget{g:solution:idp226656232-main}{}\(\ 5002_8=A02_{16}\)\end{divisionsolutioneg}%
\end{exercisegroup}
\par\medskip\noindent
\begin{exercisegroup}
\begin{divisionsolutioneg}{1.5.5.31}{}{g:exercise:idp226654824}%
\par\smallskip%
\noindent\hypertarget{g:solution:idp226658920-main}{}\(\ F_{16}=17_8\)\end{divisionsolutioneg}%
\begin{divisionsolutioneg}{1.5.5.32}{}{g:exercise:idp226657896}%
\par\smallskip%
\noindent\hypertarget{g:solution:idp226657128-main}{}\(\ 30_{16}=60_8\)\end{divisionsolutioneg}%
\begin{divisionsolutioneg}{1.5.5.33}{}{g:exercise:idp226659560}%
\par\smallskip%
\noindent\hypertarget{g:solution:idp226662632-main}{}\(\ 5F_{16}=137_8\)\end{divisionsolutioneg}%
\begin{divisionsolutioneg}{1.5.5.34}{}{g:exercise:idp226656872}%
\par\smallskip%
\noindent\hypertarget{g:solution:idp226656104-main}{}\(\ C2_{16}=302_8\)\end{divisionsolutioneg}%
\begin{divisionsolutioneg}{1.5.5.35}{}{g:exercise:idp226659944}%
\par\smallskip%
\noindent\hypertarget{g:solution:idp226660584-main}{}\(\ 1D07_{16}=16407_8\)\end{divisionsolutioneg}%
\begin{divisionsolutioneg}{1.5.5.36}{}{g:exercise:idp226658280}%
\par\smallskip%
\noindent\hypertarget{g:solution:idp226658408-main}{}\(\ A2E6_{16}=121346_8\)\end{divisionsolutioneg}%
\end{exercisegroup}
\par\medskip\noindent
\begin{exercisegroup}
\begin{divisionsolutioneg}{1.5.5.37}{}{g:exercise:idp226661480}%
\par\smallskip%
\noindent\hypertarget{g:solution:idp226655720-main}{}\(\ 1110.00010101_2\)\end{divisionsolutioneg}%
\begin{divisionsolutioneg}{1.5.5.38}{}{g:exercise:idp226656616}%
\par\smallskip%
\noindent\hypertarget{g:solution:idp226666216-main}{}\(\ 100.11100001_2\)\end{divisionsolutioneg}%
\begin{divisionsolutioneg}{1.5.5.39}{}{g:exercise:idp226668776}%
\par\smallskip%
\noindent\hypertarget{g:solution:idp226665704-main}{}\(\ 2.6_{16}\)\end{divisionsolutioneg}%
\begin{divisionsolutioneg}{1.5.5.40}{}{g:exercise:idp226664040}%
\par\smallskip%
\noindent\hypertarget{g:solution:idp226668520-main}{}\(\ 6.4_8\)\end{divisionsolutioneg}%
\begin{divisionsolutioneg}{1.5.5.41}{}{g:exercise:idp226668136}%
\par\smallskip%
\noindent\hypertarget{g:solution:idp226666728-main}{}\(\ 173.54_8\)\end{divisionsolutioneg}%
\begin{divisionsolutioneg}{1.5.5.42}{}{g:exercise:idp226670056}%
\par\smallskip%
\noindent\hypertarget{g:solution:idp226667240-main}{}\(\ 4.3C2_{16}\)\end{divisionsolutioneg}%
\end{exercisegroup}
\par\medskip\noindent
\end{solutions-subsection}
\end{sectionptx}
\end{chapterptx}
%
%
\typeout{************************************************}
\typeout{Chapter 2 Logic}
\typeout{************************************************}
%
\begin{chapterptx}{Logic}{}{Logic}{}{}{x:chapter:logic}
%
%
\typeout{************************************************}
\typeout{Section 2.1 Introduction to Logic}
\typeout{************************************************}
%
\begin{sectionptx}{Introduction to Logic}{}{Introduction to Logic}{}{}{x:section:sec-intro-to-logic}
%
%
\typeout{************************************************}
\typeout{Subsection 2.1.1 Propositions}
\typeout{************************************************}
%
\begin{subsectionptx}{Propositions}{}{Propositions}{}{}{x:subsection:propositions}
In logic, a proposition is a statement that is either true or false but both. \index{proposition}\index{logical proposition} The statement must also be unambiguous.%
\par
Examples of statements that are propositions:%
\begin{enumerate}
\item{}Trevor Noah is the host of the Daily Show on the Comedy Network.%
\item{}Lego Star Wars is a video game.%
\item{}The number \(\pi\) is exactly equal to 3.%
\end{enumerate}
A proposition can be clearly false, as in the last statement.%
\par
Examples of statements that are \emph{not} propositions:%
\begin{enumerate}
\item{}Will you do your homework tonight?%
\item{}Please pass the butter.%
\item{}She was late for class this morning.%
\end{enumerate}
The first is not a  proposition because questions cannot be propositions. (The \emph{answer} to the question very well may be a proposition.)  The second one is a command and cannot be said to be either true or false.  The third of these examples is not a proposition because, taking the statement on its own, the truth value depends on who \emph{she} is.  If, however, that statement were expanded to become, "My roommate's name is Laura and she was late for class this morning," then \emph{she} is clearly defined to be Laura and the whole sentence is a proposition.%
\par
Taking this idea one step further, we can consider \emph{she} in the third example to behave like a variable, and whether the full statement "she was late" is true or false must depend on what the value of the variable \emph{she} is.  Similarly, in programming it is very common to evaluate the value (true or false) of propositions like \(x=3\) or \(y \lt 5\) in statements like%
\begin{codedisplay}

              if x=3 then print ''Hello World''
            
\end{codedisplay}
provided that, like she\slash{}Laura, the value of \(x\) has been defined previously.%
\par
Writing propositions in sentences is unwieldy, so we use variables to denote propositions.  In symbolic logic it is conventional to use the letters \(p,\, q,\, r,\, s,\, t,\,\) etc., for propositions.  Each of these variables can then have one of two values: true or false.  For example, let \(p =\,\)\emph{Lego Star Wars is a video game} and \(q =\,\)\emph{The number \(\pi\) is exactly equal to 3.}  In this instance, the proposition \(p\) is true, since there is a video game called Lego Star Wars.  The proposition \(q\) is false, since \(\pi\) is the irrational number \(3.1415926\ldots\), whose decimal representation neither repeats nor terminates.%
\end{subsectionptx}
%
%
\typeout{************************************************}
\typeout{Subsection 2.1.2 Operators}
\typeout{************************************************}
%
\begin{subsectionptx}{Operators}{}{Operators}{}{}{x:subsection:operators}
%
%
\typeout{************************************************}
\typeout{Subsubsection  Negation}
\typeout{************************************************}
%
\begin{subsubsectionptx}{Negation}{}{Negation}{}{}{x:subsubsection:negation}
The negation of any proposition \(p\) is called \emph{not p} and is written using the tilde symbol, \(\sim\!{p}\). \index{negation}\index{not}\index{negative} The tilde can be found on a standard keyboard as the shifted key to the left of the 1.  You may also see negations in logic written using this symbol, \(\neg p\), or using a bar or overline, \(\overline{p}\).  In computing, you will also see \(!p\) as the negation.  We will use the \(\sim\!{p}\) because it uses a character on a conventional keyboard, so it is easier to type.%
\par
Note that you should be a little careful when negating sentences.  For example, the negation of ``Pat is happy'' is \emph{not} ``Pat is unhappy.''  There are many other emotions that Pat could have (anger, fear, boredom, etc.).  If the first statement is false, then its negationv must be true, so between the two you need to cover all possible situations that could arise. It would be safe to say that the negation of ``Pat is happy'' is ``Pat is not happy'', though.%
\par
Another situation requiring care involves statements such as ``All the students have coding experience.''  The negation of this statement is \emph{not} ``None of the students have coding experience.''  The original statement is false even if only one of ``the students'' has no coding experience.  So the negation of ``All the students have coding experience'' is ``At least one of the students does not have coding experience.''  Another way to write the negation is ``Some of the students don't have coding experience.''%
\begin{example}{}{g:example:idp226694760}%
Are these two sentences negations of each other?%
\begin{itemize}[label=\textbullet]
\item{}"The number of students in Math 156 is even."%
\item{}"The number of students in Math 156 is odd."%
\end{itemize}
Answer: Yes, these two are negations of each other.  Since we never have fractions of students in class, the number of students must be either zero or a natural number (that is, a whole number).  Since whole numbers and natural numbers are either even or odd, all possible situations are accounted by these two sentences.  (Zero is an even number.)%
\end{example}
\begin{example}{}{g:example:idp226698216}%
Are these two sentences negations of each other?%
\begin{itemize}[label=\textbullet]
\item{}"Pat's Visa account balance is positive."%
\item{}"Pat's Visa account balance is negative."%
\end{itemize}
Answer: No, these two statements are not negations of each other.  There is a third possible case, "Pat's Visa account balance is zero."  Zero is an unsigned number, so the two statements don't account for all possible situations regarding Pat's Visa account balance. \index{zero!as unsigned number} If the second statement was "Pat's Visa account balance is negative or zero," then the second statement would be the negation of the first one (and vice versa).%
\end{example}
\end{subsubsectionptx}
\end{subsectionptx}
%
%
\typeout{************************************************}
\typeout{Subsection 2.1.3 Combining Propositions Using Connectives}
\typeout{************************************************}
%
\begin{subsectionptx}{Combining Propositions Using Connectives}{}{Combining Propositions Using Connectives}{}{}{x:subsection:connectives}
\begin{introduction}{}%
Propositions may be combined using logical operators called \terminology{connectives}, and the result is called a \terminology{compound proposition}.  There are three basic connectives that we will study: \terminology{and}, \terminology{or}, and \terminology{exclusive or}.  (Oddly, the \terminology{not} operator is also called a connective, even though it acts on only one statement rather than combining two.)\end{introduction}%
%
%
\typeout{************************************************}
\typeout{Subsubsection  Conjunction}
\typeout{************************************************}
%
\begin{subsubsectionptx}{Conjunction}{}{Conjunction}{}{}{x:subsubsection:and}
If we connect the propositions \(p\) and \(q\) with \emph{and}, then the result is called the \terminology{conjunction} of \(p\) and \(q\). \index{and}\index{conjunction} The conjunction of \(p\) and \(q\) is true provided \emph{both} \(p\) and \(q\) are true.  Otherwise it is false. Symbolically, we write%
\begin{equation*}
p {\wedge} q
\end{equation*}
%
\begin{example}{}{g:example:idp226705256}%
Under what conditions is the statement "Jason does his marking and goes to a movie" true?%
\par
Answer: "Jason does his marking and goes to a movie" is true provided that Jason does his marking and also goes to a movie. If he does one or the other \emph{but not both}, then the statement "Jason does his marking and goes to a movie" is false.  It is also false if Jason does neither action.%
\end{example}
\end{subsubsectionptx}
%
%
\typeout{************************************************}
\typeout{Subsubsection  Inclusive Disjunction}
\typeout{************************************************}
%
\begin{subsubsectionptx}{Inclusive Disjunction}{}{Inclusive Disjunction}{}{}{x:subsubsection:inclusive-disjunction}
If we connect propositions \(p\) and \(q\) with \emph{or}, then the result is called the \terminology{inclusive disjunction} of \(p\) and \(q\). \index{or!inclusive}\index{inclusive or}\index{disjunction!inclusive}\index{inclusive disjunction} The inclusive disjunction is true if at least one of \(p\) and \(q\) is true.  It is false only if both \(p\) and \(q\) are false.  Symbolically, we write%
\begin{equation*}
p {\vee} q
\end{equation*}
%
\begin{example}{}{g:example:idp226712552}%
Under what conditions is the statement ``Jason does his marking or goes to a movie'' true?%
\par
Answer: ``Jason does his marking or goes to a movie'' is true provided at least one of the statements ``Jason does his marking'' or ``Jason goes to a movie'' is true.  If Jason does his marking, then the compound sentence is true regardless of whether Jason goes to a movie or not.  Similarly, if Jason goes to a movie, then the compound sentence is true whether Jason does his marking or not.  In complete detail, the inclusive disjunction is true if \emph{any} of the following are true:%
\begin{itemize}[label=\textbullet]
\item{}Jason does his marking and also goes to a movie.%
\item{}Jason does his marking but does not go to a movie.%
\item{}Jason goes to a movie but does not do his marking.%
\end{itemize}
The only case in which "Jason does his marking or goes to a movie" is false is if Jason neither goes to a movie nor does his marking.%
\end{example}
\end{subsubsectionptx}
%
%
\typeout{************************************************}
\typeout{Subsubsection  Exclusive Disjunction}
\typeout{************************************************}
%
\begin{subsubsectionptx}{Exclusive Disjunction}{}{Exclusive Disjunction}{}{}{x:subsubsection:exclusive-disjunction}
If we connect the propositions \(p\) and \(q\) with \terminology{exclusive or} (often written \mono{XOR}) then the result is called the \terminology{exclusive disjunction} of \(p\) and \(q\). \index{or!exclusive}\index{exclusive or}\index{disjunction!exclusive}\index{exclusive disjunction}\index{XOR} Symbolically, we write%
\begin{equation*}
p{\oplus} q
\end{equation*}
The exclusive disjunction \(p{\oplus} q\) is true provided \emph{exactly one} of the statements is true.%
\end{subsubsectionptx}
%
%
\typeout{************************************************}
\typeout{Subsubsection  Or vs. XOR}
\typeout{************************************************}
%
\begin{subsubsectionptx}{Or vs. XOR}{}{Or vs. XOR}{}{}{x:subsubsection:inclusive-vs-exclusive-disjunction}
In ordinary language, the word \emph{or} can mean either the \emph{inclusive or} or the \emph{exclusive or}, \index{or!vs. XOR}\index{XOR!vs. or} and it is usually up to the reader or listener to decide which one is appropriate from the context.%
\begin{example}{}{g:example:idp226733032}%
Which \emph{or} is meant in the following sentences or phrases?%
\begin{enumerate}
\item{}Would you like milk or sugar in your tea?%
\item{}Wanted dead or alive.%
\end{enumerate}
%
\par
Answer:%
\begin{enumerate}
\item{}The answer could be "milk", "sugar", "both", or "neither".  Since "both" is an option, the \emph{inclusive or} is intended.%
\item{}The person who is wanted in one of these two states will either be dead or alive but not both, so \emph{exclusive or} is the best interpretation.%
\end{enumerate}
%
\end{example}
To unambiguously state which \emph{or} is meant in everyday language, the word \emph{or} can be replaced by slightly wordier constructions.  The sentence "Would you like milk or sugar or both in your tea?" makes it clear that the \emph{inclusive or} is meant.  Replacing \emph{or} with \emph{and\slash{}or} has the same result.  Using the phrase "but not both" is a signal that the \emph{exclusive or} is meant.%
\par
In general, if a statement is ambiguous, it is best to seek clarification.  If that is not possible, then assuming that \emph{or} means \emph{inclusive or} is generally the safest.  For the rest of this course, we will use \emph{or} to mean the \emph{inclusive or}.%
\begin{example}{}{g:example:idp226739816}%
Under what conditions is the statement "Jason does his marking or goes to a movie but not both" true?\par\smallskip%
\noindent\textbf{\blocktitlefont Answer}.\label{g:answer:idp226740712}{}\hypertarget{g:answer:idp226740712}{}\quad{}"Jason does his marking or goes to a movie but not both" is true provided exactly one of the two conditions is true.  If he does his marking, then he cannot also go to a movie or else the exclusive disjunction is false.  If he goes to a movie, then he cannot also do his marking or the statement is false.  Truth of the statement means he doesn't do both, and he doesn't do neither.\end{example}
\end{subsubsectionptx}
%
%
\typeout{************************************************}
\typeout{Subsubsection  Logical Connectives and Precedence}
\typeout{************************************************}
%
\begin{subsubsectionptx}{Logical Connectives and Precedence}{}{Logical Connectives and Precedence}{}{}{x:subsubsection:connective-precedence}
When you are doing arithmetic, to evaluate the expression%
\begin{equation*}
4+3\times 2^2
\end{equation*}
you need to know which operation (addition, multiplication, exponentiation) should come first.  In the same way, there is an order of operations, or \emph{precedence}, in logic:%
\begin{itemize}[label=\textbullet]
\item{}negation, \(\sim\), is done first%
\item{}and, \({\wedge}\), is done next%
\item{}or, \({\vee}\), is done last%
\end{itemize}
and brackets () override the default precedence.  If you like, you can think of \emph{not} like exponents, \emph{and} like multiplication, and \emph{or} like addition (more on this later).%
\par
To evaluate the proposition \(p{\vee} q{\wedge} r\), you would evaluate \(q{\wedge} r\) first, and then form the inclusive disjunction with \(p\) (informally "'or it' with \(p\)").  So \(p{\vee} q{\wedge} r\) is the same as \(p{\vee}(q{\wedge} r)\).  The precedence of the connectives makes the brackets redundant, but when you are starting out it might be helpful to include them.%
\par
To evaluate the proposition \(\sim\!{q}{\wedge} p\), you'd negate \(q\) first, and then form the conjunction with \(p\).  You could write \((\sim\!{q}){\wedge} p\), but the brackets are redundant.  If you want the conjunction to occur before the negation, then you must write \(\sim\!(q{\wedge} p)\).  It is very important that you recognize that \(\sim\!{q}{\wedge} p\) is \alert{not} equivalent to \(\sim\! (q{\wedge} p)\).%
\end{subsubsectionptx}
\end{subsectionptx}
%
%
\typeout{************************************************}
\typeout{Exercises 2.1.4 Exercises}
\typeout{************************************************}
%
\begin{exercises-subsection}{Exercises}{}{Exercises}{}{}{g:exercises:idp226751976}
\par\medskip\noindent%
\textbf{Exercise Group.}\space\space%
State whether the following sentences are propositions.\begin{exercisegroup}
\begin{divisionexerciseeg}{1}{}{}{g:exercise:idp226747112}%
On September 6, 2006, mathematicians proved that \(2^{32582657}-1\) is a prime number.\end{divisionexerciseeg}%
\begin{divisionexerciseeg}{2}{}{}{g:exercise:idp226759272}%
\(\ \)Will you marry me?\end{divisionexerciseeg}%
\begin{divisionexerciseeg}{3}{}{}{g:exercise:idp226757352}%
\(\ \)Python is her favourite computing language.\end{divisionexerciseeg}%
\begin{divisionexerciseeg}{4}{}{}{g:exercise:idp226754280}%
\(\ \)What is your favourite computing language?\end{divisionexerciseeg}%
\begin{divisionexerciseeg}{5}{}{}{g:exercise:idp226753000}%
\(\ \)Please bring me a textbook.\end{divisionexerciseeg}%
\begin{divisionexerciseeg}{6}{}{}{g:exercise:idp226755304}%
\(\ \)The University of Victoria is located in Alberta.\end{divisionexerciseeg}%
\end{exercisegroup}
\par\medskip\noindent
\par\medskip\noindent%
\textbf{Exercise Group.}\space\space%
Let \(p\) be "Rich is seven feet tall" and \(q\) be "Susan has brown hair." Translate the following sentences into logical notation.\begin{exercisegroup}
\begin{divisionexerciseeg}{7}{}{}{g:exercise:idp226756200}%
\(\ \)Rich is seven feet tall or he is seven feet tall.\end{divisionexerciseeg}%
\begin{divisionexerciseeg}{8}{}{}{g:exercise:idp226754664}%
\(\ \)Either Rich is not seven feet tall or Susan does not have brown hair.\end{divisionexerciseeg}%
\begin{divisionexerciseeg}{9}{}{}{g:exercise:idp226767080}%
\(\ \)It is not true that Rich is seven feet tall or Susan has brown hair.\end{divisionexerciseeg}%
\begin{divisionexerciseeg}{10}{}{}{g:exercise:idp226767336}%
\(\ \)Rich is seven feet tall and Susan has brown hair.\end{divisionexerciseeg}%
\begin{divisionexerciseeg}{11}{}{}{g:exercise:idp226761192}%
\(\ \)Either Rich is seven feet tall or Susan does not have brown hair, but not both.\end{divisionexerciseeg}%
\end{exercisegroup}
\par\medskip\noindent
\par\medskip\noindent%
\textbf{Exercise Group.}\space\space%
Decide if the inclusive or exclusive disjunction is meant in the following sentences.\begin{exercisegroup}
\begin{divisionexerciseeg}{12}{}{}{g:exercise:idp226762728}%
\(\ \)I will sit inside or outside.\end{divisionexerciseeg}%
\begin{divisionexerciseeg}{13}{}{}{g:exercise:idp226762984}%
\(\ \)During their movie marathon, my friends watched the Harry Potter movies or the Transformer movies.\end{divisionexerciseeg}%
\begin{divisionexerciseeg}{14}{}{}{g:exercise:idp226776552}%
\(\ \)I will get an A or a B in the course.\end{divisionexerciseeg}%
\begin{divisionexerciseeg}{15}{}{}{g:exercise:idp226772200}%
\(\ \)My answer is correct or incorrect.\end{divisionexerciseeg}%
\begin{divisionexerciseeg}{16}{}{}{g:exercise:idp226772968}%
\(\ \)We need someone who speaks French or German.\end{divisionexerciseeg}%
\end{exercisegroup}
\par\medskip\noindent
\par\medskip\noindent%
\textbf{Exercise Group.}\space\space%
Let \(p\) be "The moon is made of green cheese" and \(q\) be "The earth is made of green cheese."  Translate the following English sentences into logical notation.\begin{exercisegroup}
\begin{divisionexerciseeg}{17}{}{}{g:exercise:idp226773352}%
\(\ \)Either the moon is made of green cheese or both the moon and the earth are made of green cheese.\end{divisionexerciseeg}%
\begin{divisionexerciseeg}{18}{}{}{g:exercise:idp226775144}%
\(\ \)The earth is made of green cheese and either the earth or the moon is made of green cheese.\end{divisionexerciseeg}%
\begin{divisionexerciseeg}{19}{}{}{g:exercise:idp226775912}%
\(\ \)Either the earth is made of green cheese while the moon is not, or the moon is made of green cheese.\end{divisionexerciseeg}%
\begin{divisionexerciseeg}{20}{}{}{g:exercise:idp226781160}%
\(\ \)The earth is made of green cheese and either the moon is made of green cheese or the earth is not.\end{divisionexerciseeg}%
\end{exercisegroup}
\par\medskip\noindent
\par\medskip\noindent%
\textbf{Exercise Group.}\space\space%
Let \(p=\,\)"Jane did their homework" and \(q=\,\)"Jane went for a jog."  Translate the following logical propositions into English sentences.\begin{exercisegroup}
\begin{divisionexerciseeg}{21}{}{}{g:exercise:idp226779880}%
\(\ p{\wedge} q\)\end{divisionexerciseeg}%
\begin{divisionexerciseeg}{22}{}{}{g:exercise:idp226780776}%
\(\ \sim\!(p{\wedge} q)\)\end{divisionexerciseeg}%
\begin{divisionexerciseeg}{23}{}{}{g:exercise:idp226782568}%
\(\ q\,{\wedge}\sim\!{p}\)\end{divisionexerciseeg}%
\begin{divisionexerciseeg}{24}{}{}{g:exercise:idp226782696}%
\(\ \sim\!{q}\,{\vee} \sim\!{p}\)\end{divisionexerciseeg}%
\begin{divisionexerciseeg}{25}{}{}{g:exercise:idp226783848}%
\(\ \sim\!(\sim\!{p})\)\end{divisionexerciseeg}%
\begin{divisionexerciseeg}{26}{}{}{g:exercise:idp226784616}%
\(\ q\,{\oplus}\sim\!{q}\)\end{divisionexerciseeg}%
\end{exercisegroup}
\par\medskip\noindent
\par\medskip\noindent%
\textbf{Exercise Group.}\space\space%
For each pair of sentences below, is the second sentence the negation of the first?\begin{exercisegroup}
\begin{divisionexerciseeg}{27}{}{}{g:exercise:idp226786664}%
\(\ \)Pat owes Peter money.  Peter owes Pat money.\end{divisionexerciseeg}%
\begin{divisionexerciseeg}{28}{}{}{g:exercise:idp226792424}%
\(\ \)The number of students in Math 156 is greater than 25.  The number of students in Math 156 is less than 25.\end{divisionexerciseeg}%
\begin{divisionexerciseeg}{29}{}{}{g:exercise:idp226790504}%
\(\ \)Jason, the math instructor, is fabulously rich.  Jason, the math instructor, is financially poor.\end{divisionexerciseeg}%
\end{exercisegroup}
\par\medskip\noindent
\par\medskip\noindent%
\textbf{Exercise Group.}\space\space%
Answer the questions assuming that each of the given statements is true.  If you cannot answer the question, state whether "the situation is not possible" or "there's not enough information."\begin{exercisegroup}
\begin{divisionexerciseeg}{30}{}{}{g:exercise:idp226786920}%
\(\ \)"Pearl went for a jog and did her homework."  Did Pearl go for a jog?\end{divisionexerciseeg}%
\begin{divisionexerciseeg}{31}{}{}{g:exercise:idp226788584}%
\(\ \)"Pearl went for a jog or did her homework."  Did Pearl do her homework?\end{divisionexerciseeg}%
\begin{divisionexerciseeg}{32}{}{}{g:exercise:idp226791528}%
\(\ \)"Pearl went for a jog."  Did Pearl go for a jog and do her homework?\end{divisionexerciseeg}%
\begin{divisionexerciseeg}{33}{}{}{g:exercise:idp226790760}%
\(\ \)"Pearl did not go for a jog."  Did Pearl go for a jog and do her homework?\end{divisionexerciseeg}%
\end{exercisegroup}
\par\medskip\noindent
\end{exercises-subsection}
%
%
\typeout{************************************************}
\typeout{Solutions 2.1.5 Solutions to Section~{\xreffont\ref*{x:section:sec-intro-to-logic}} Exercises}
\typeout{************************************************}
%
\begin{solutions-subsection}{Solutions to Section~{\xreffont\ref*{x:section:sec-intro-to-logic}} Exercises}{}{Solutions to Section~{\xreffont\ref*{x:section:sec-intro-to-logic}} Exercises}{}{}{g:solutions:idp226792808}
\par\medskip
\noindent\textbf{\normalsize{}2.1.4\space\textperiodcentered\space{}Exercises}
\begin{exercisegroup}
\begin{divisionsolutioneg}{2.1.4.1}{}{g:exercise:idp226747112}%
\par\smallskip%
\noindent\hypertarget{g:solution:idp226757864-main}{}Yes\end{divisionsolutioneg}%
\begin{divisionsolutioneg}{2.1.4.2}{}{g:exercise:idp226759272}%
\par\smallskip%
\noindent\hypertarget{g:solution:idp226760936-main}{}No.\end{divisionsolutioneg}%
\begin{divisionsolutioneg}{2.1.4.3}{}{g:exercise:idp226757352}%
\par\smallskip%
\noindent\hypertarget{g:solution:idp226758120-main}{}No\end{divisionsolutioneg}%
\begin{divisionsolutioneg}{2.1.4.4}{}{g:exercise:idp226754280}%
\par\smallskip%
\noindent\hypertarget{g:solution:idp226753512-main}{}No\end{divisionsolutioneg}%
\begin{divisionsolutioneg}{2.1.4.5}{}{g:exercise:idp226753000}%
\par\smallskip%
\noindent\hypertarget{g:solution:idp226755688-main}{}No.\end{divisionsolutioneg}%
\begin{divisionsolutioneg}{2.1.4.6}{}{g:exercise:idp226755304}%
\par\smallskip%
\noindent\hypertarget{g:solution:idp226758760-main}{}Yes.\end{divisionsolutioneg}%
\end{exercisegroup}
\par\medskip\noindent
\begin{exercisegroup}
\begin{divisionsolutioneg}{2.1.4.7}{}{g:exercise:idp226756200}%
\par\smallskip%
\noindent\hypertarget{g:solution:idp226759400-main}{}\(\ \)Symbolically, this silly statement is \(p{\vee} p\).\end{divisionsolutioneg}%
\begin{divisionsolutioneg}{2.1.4.8}{}{g:exercise:idp226754664}%
\par\smallskip%
\noindent\hypertarget{g:solution:idp226764776-main}{}From the context, you could go with either \(\sim\!{p}\,{\vee} \sim\!{q}\) or \(\sim\!{p}\,{\oplus}\sim\!{q}\)\end{divisionsolutioneg}%
\begin{divisionsolutioneg}{2.1.4.9}{}{g:exercise:idp226767080}%
\par\smallskip%
\noindent\hypertarget{g:solution:idp226764008-main}{}The particular phrasing in this sentence indicates that care is being taken to \emph{avoid} interpreting this as "Rich is not seven feet tall or Susan has brown hair."  That statement would be represented as \(\sim\!{p}{\vee} q\) (or, more redundantly, \((\sim\!{p}){\vee} q\)).%
\par
Given the way the sentence is written, we should interpret it as \(\sim\! (p{\vee} q)\).  Later, when you study De Morgan's rules, you will learn a way to write \(\sim\!(p{\vee} q)\) in English that avoids any chance of misinterpretation.%
\end{divisionsolutioneg}%
\begin{divisionsolutioneg}{2.1.4.10}{}{g:exercise:idp226767336}%
\par\smallskip%
\noindent\hypertarget{g:solution:idp226765544-main}{}\(p{\wedge} q\)\end{divisionsolutioneg}%
\begin{divisionsolutioneg}{2.1.4.11}{}{g:exercise:idp226761192}%
\par\smallskip%
\noindent\hypertarget{g:solution:idp226766440-main}{}\(p{\oplus} \sim\!{q}\)\end{divisionsolutioneg}%
\end{exercisegroup}
\par\medskip\noindent
\begin{exercisegroup}
\begin{divisionsolutioneg}{2.1.4.12}{}{g:exercise:idp226762728}%
\par\smallskip%
\noindent\hypertarget{g:solution:idp226761576-main}{}exclusive - you cannot sit inside and outside simultaneously.\end{divisionsolutioneg}%
\begin{divisionsolutioneg}{2.1.4.13}{}{g:exercise:idp226762984}%
\par\smallskip%
\noindent\hypertarget{g:solution:idp226774632-main}{}It is conceivable that they watched both series, so the inclusive disjunction should be assumed.\end{divisionsolutioneg}%
\begin{divisionsolutioneg}{2.1.4.14}{}{g:exercise:idp226776552}%
\par\smallskip%
\noindent\hypertarget{g:solution:idp226772072-main}{}exclusive - only one grade is assigned to a course.\end{divisionsolutioneg}%
\begin{divisionsolutioneg}{2.1.4.15}{}{g:exercise:idp226772200}%
\par\smallskip%
\noindent\hypertarget{g:solution:idp226770664-main}{}exclusive - your answer cannot be both correct and incorrect.\end{divisionsolutioneg}%
\begin{divisionsolutioneg}{2.1.4.16}{}{g:exercise:idp226772968}%
\par\smallskip%
\noindent\hypertarget{g:solution:idp226772712-main}{}inclusive - a person who speaks both would be a good fit, too!\end{divisionsolutioneg}%
\end{exercisegroup}
\par\medskip\noindent
\begin{exercisegroup}
\begin{divisionsolutioneg}{2.1.4.17}{}{g:exercise:idp226773352}%
\par\smallskip%
\noindent\hypertarget{g:solution:idp226771432-main}{}\(p{\vee}(p{\wedge} q)\)\end{divisionsolutioneg}%
\begin{divisionsolutioneg}{2.1.4.18}{}{g:exercise:idp226775144}%
\par\smallskip%
\noindent\hypertarget{g:solution:idp226773992-main}{}The disjunction in this statement is the inclusive disjunction, since both could conceivably be made of green cheese.  Therefore, the translation is \(q{\wedge}(q{\vee} p)\)\end{divisionsolutioneg}%
\begin{divisionsolutioneg}{2.1.4.19}{}{g:exercise:idp226775912}%
\par\smallskip%
\noindent\hypertarget{g:solution:idp226774376-main}{}The word \emph{while} can be replaced with \emph{and} without changing the meaning of the statement. The translation is \((q{\wedge} \sim\!{p}){\vee} p\).  The exclusive disjunction is also acceptable, since the two "disjunctives" cannot both be true, but the answer given above is consistent with the statement given earlier that the inclusive disjunction is the assumption when there is ambiguity.\end{divisionsolutioneg}%
\begin{divisionsolutioneg}{2.1.4.20}{}{g:exercise:idp226781160}%
\par\smallskip%
\noindent\hypertarget{g:solution:idp226781672-main}{}\(q{\wedge} (p{\vee}\sim\!{q})\)\end{divisionsolutioneg}%
\end{exercisegroup}
\par\medskip\noindent
\begin{exercisegroup}
\begin{divisionsolutioneg}{2.1.4.21}{}{g:exercise:idp226779880}%
\par\smallskip%
\noindent\hypertarget{g:solution:idp226778216-main}{}Jane did their homework and went for a jog.\end{divisionsolutioneg}%
\begin{divisionsolutioneg}{2.1.4.22}{}{g:exercise:idp226780776}%
\par\smallskip%
\noindent\hypertarget{g:solution:idp226779368-main}{}It is not the case that Jane did their homework and went for a jog.\end{divisionsolutioneg}%
\begin{divisionsolutioneg}{2.1.4.23}{}{g:exercise:idp226782568}%
\par\smallskip%
\noindent\hypertarget{g:solution:idp226781544-main}{}Jane went for a jog but did not do their homework.\end{divisionsolutioneg}%
\begin{divisionsolutioneg}{2.1.4.24}{}{g:exercise:idp226782696}%
\par\smallskip%
\noindent\hypertarget{g:solution:idp226782824-main}{}Jane didn't go for a jog or they didn't do their homework.\end{divisionsolutioneg}%
\begin{divisionsolutioneg}{2.1.4.25}{}{g:exercise:idp226783848}%
\par\smallskip%
\noindent\hypertarget{g:solution:idp226784360-main}{}It is not the case that Jane didn't do their homework.\end{divisionsolutioneg}%
\begin{divisionsolutioneg}{2.1.4.26}{}{g:exercise:idp226784616}%
\par\smallskip%
\noindent\hypertarget{g:solution:idp226785000-main}{}Either Jane went for a jog, or didn't go for a jog, but not both.\end{divisionsolutioneg}%
\end{exercisegroup}
\par\medskip\noindent
\begin{exercisegroup}
\begin{divisionsolutioneg}{2.1.4.27}{}{g:exercise:idp226786664}%
\par\smallskip%
\noindent\hypertarget{g:solution:idp226791400-main}{}No.  Neither might owe the other any money.\end{divisionsolutioneg}%
\begin{divisionsolutioneg}{2.1.4.28}{}{g:exercise:idp226792424}%
\par\smallskip%
\noindent\hypertarget{g:solution:idp226787048-main}{}No.  The number of students in Math 156 could be equal to 25.\end{divisionsolutioneg}%
\begin{divisionsolutioneg}{2.1.4.29}{}{g:exercise:idp226790504}%
\par\smallskip%
\noindent\hypertarget{g:solution:idp226791784-main}{}No. Jason might be in the middle income class - neither rich nor poor.\end{divisionsolutioneg}%
\end{exercisegroup}
\par\medskip\noindent
\begin{exercisegroup}
\begin{divisionsolutioneg}{2.1.4.30}{}{g:exercise:idp226786920}%
\par\smallskip%
\noindent\hypertarget{g:solution:idp226785896-main}{}Yes, Pearl went for a jog or else the given statement would be false.\end{divisionsolutioneg}%
\begin{divisionsolutioneg}{2.1.4.31}{}{g:exercise:idp226788584}%
\par\smallskip%
\noindent\hypertarget{g:solution:idp226787816-main}{}There's not enough information.  If Pearl went for a jog, then the statement is true regardless of whether she did her homework or not.\end{divisionsolutioneg}%
\begin{divisionsolutioneg}{2.1.4.32}{}{g:exercise:idp226791528}%
\par\smallskip%
\noindent\hypertarget{g:solution:idp226786024-main}{}There's not enough information.  We're not told if she did her homework or not.\end{divisionsolutioneg}%
\begin{divisionsolutioneg}{2.1.4.33}{}{g:exercise:idp226790760}%
\par\smallskip%
\noindent\hypertarget{g:solution:idp226791912-main}{}No.\end{divisionsolutioneg}%
\end{exercisegroup}
\par\medskip\noindent
\end{solutions-subsection}
\end{sectionptx}
%
%
\typeout{************************************************}
\typeout{Section 2.2 Venn Diagrams}
\typeout{************************************************}
%
\begin{sectionptx}{Venn Diagrams}{}{Venn Diagrams}{}{}{x:section:sec-venn-diagrams}
%
%
\typeout{************************************************}
\typeout{Subsection 2.2.1 Venn Diagrams with One Proposition}
\typeout{************************************************}
%
\begin{subsectionptx}{Venn Diagrams with One Proposition}{}{Venn Diagrams with One Proposition}{}{}{x:subsection:one-proposition-diagrams}
One way to visualize operations on propositions is to use a Venn diagram.  Although Venn diagrams are more commonly used with sets, there are many commonalities between the operations on sets and on logical propositions.  The Venn diagram for a single logical proposition \(p\) is shown below.%
\par
\begin{image}{0.3}{0.4}{0.3}%
\resizebox{\linewidth}{!}{%
\begin{tikzpicture}
  \draw (-1.618,-1) rectangle (1.618,1);
  \draw (0,0) circle  [radius=0.8];
  \node[above](0,0){$p$};
\end{tikzpicture}
}%
\end{image}%
%
\par
In this diagram, the rectangle stands for the universe, while the circle denotes the logical proposition \(p\).  We then shade in regions of the diagram to indicate the regions of interest.  For example, when we want to indicate the proposition \(p\), we shade the inside of the circle, as shown in the left diagram.  If instead we want to show the proposition \(\sim\!{p}\), we shade outside of the circle, as in the diagram on the right.%
\par
\begin{sidebyside}{2}{0.05}{0.05}{0.1}%
\begin{sbspanel}{0.4}%
\begin{figureptx}{Shading for \(p\)}{x:figure:shading-for-p}{}%
\resizebox{\linewidth}{!}{%
\begin{tikzpicture}
  \draw (-1.618,-1) rectangle (1.618,1);
  \filldraw[fill=black!20] (0,0) circle  [radius=0.8];
  \node[above](0,0){$p$};
\end{tikzpicture}
}%
\tcblower
\end{figureptx}%
\end{sbspanel}%
\begin{sbspanel}{0.4}%
\begin{figureptx}{Shading for \(\sim\!{p}\)}{x:figure:shading-for-not-p}{}%
\resizebox{\linewidth}{!}{%
\begin{tikzpicture}
  \filldraw[fill=black!20] (-1.618,-1) rectangle (1.618,1);
  \filldraw[fill=white] (0,0) circle  [radius=0.8];
  \node[above](0,0){$p$};
\end{tikzpicture}
}%
\tcblower
\end{figureptx}%
\end{sbspanel}%
\end{sidebyside}%
%
\end{subsectionptx}
%
%
\typeout{************************************************}
\typeout{Subsection 2.2.2 Venn Diagrams with Two Propositions}
\typeout{************************************************}
%
\begin{subsectionptx}{Venn Diagrams with Two Propositions}{}{Venn Diagrams with Two Propositions}{}{}{x:subsection:two-proposition-diagrams}
Venn diagrams with only one proposition don't generally contain much information, as it's usually pretty easy to visualize what \(p\) and \(\sim\!{p}\) mean when you have only the one proposition.  It gets more interesting when you have propositions \(p\) and \(q\) in the same diagram, as you can see in the next figure.%
\par
\begin{image}{0.3}{0.4}{0.3}%
\resizebox{\linewidth}{!}{%
\begin{venndiagram2sets}[labelA={$p$},labelB={$q$}]
\end{venndiagram2sets}
}%
\end{image}%
%
\par
Let's try doing some shading to represent operations on the propositions \(p\) and \(q\).  To begin with, let's examine the shading for \(p\) as shown in \hyperref[x:figure:shading-for-p2]{Figure~{\xreffont\ref{x:figure:shading-for-p2}}, p.\,\pageref{x:figure:shading-for-p2}}.  It looks very similar to the shading for the one-proposition diagram, but you should notice that in order to shade in all of \(p\), \alert{two} regions have been shaded in: the crescent-moon shaped part which represent the part of \(p\) that \emph{does not} overlap with \(q\) and the lozenge-shaped part which represents the part of \(p\) that \emph{does} overlap with \(q\).  Similarly, \(q\) is shown in \hyperref[x:figure:shading-for-q2]{Figure~{\xreffont\ref{x:figure:shading-for-q2}}, p.\,\pageref{x:figure:shading-for-q2}}.%
\par
\begin{sidebyside}{2}{0.05}{0.05}{0.1}%
\begin{sbspanel}{0.4}%
\begin{figureptx}{Shading for \(p\)}{x:figure:shading-for-p2}{}%
\resizebox{\linewidth}{!}{%
\begin{venndiagram2sets}[labelA={$p$},labelB={$q$}]
  \fillA
\end{venndiagram2sets}
}%
\tcblower
\end{figureptx}%
\end{sbspanel}%
\begin{sbspanel}{0.4}%
\begin{figureptx}{Shading for \(q\)}{x:figure:shading-for-q2}{}%
\resizebox{\linewidth}{!}{%
\begin{venndiagram2sets}[labelA={$p$},labelB={$q$}]
  \fillB
\end{venndiagram2sets}
}%
\tcblower
\end{figureptx}%
\end{sbspanel}%
\end{sidebyside}%
%
\par
To represent compound propositions using a Venn diagram, it is helpful first to consider the \terminology{basic regions} of the diagram.  In the figure below, the four basic regions are numbered 1 through 4:%
\par
\begin{figureptx}{Basic regions of a two-proposition Venn diagram.}{x:figure:basic-regions}{}%
\begin{image}{0.3}{0.4}{0.3}%
\resizebox{\linewidth}{!}{%
\begin{venndiagram2sets}[labelA={$p$},labelB={$q$},labelOnlyA={1},labelOnlyB={2},labelAB={3},labelNotAB={4}]
\end{venndiagram2sets}
}%
\end{image}%
\tcblower
\end{figureptx}%
%
\par
For example, if you wished to show the Venn diagram for \(p{\vee} q\), recall that \emph{or} means \emph{one or the other or both}.  Looking at \hyperref[x:figure:shading-for-p2]{Figure~{\xreffont\ref{x:figure:shading-for-p2}}, p.\,\pageref{x:figure:shading-for-p2}} and \hyperref[x:figure:shading-for-q2]{Figure~{\xreffont\ref{x:figure:shading-for-q2}}, p.\,\pageref{x:figure:shading-for-q2}}, we see that we should shade regions 1, 2, and 3:%
\par
\begin{figureptx}{Venn diagram for the compound statement \(p{\vee} q\)}{x:figure:p-or-q}{}%
\begin{image}{0.3}{0.4}{0.3}%
\resizebox{\linewidth}{!}{%
\begin{venndiagram2sets}[labelA={$p$},labelB={$q$}]
  \fillA
  \fillB
\end{venndiagram2sets}
}%
\end{image}%
\tcblower
\end{figureptx}%
%
\par
To show the diagram for \(p{\wedge} q\), shade the basic regions that are shaded in both of \hyperref[x:figure:shading-for-p2]{Figure~{\xreffont\ref{x:figure:shading-for-p2}}, p.\,\pageref{x:figure:shading-for-p2}} and \hyperref[x:figure:shading-for-q2]{Figure~{\xreffont\ref{x:figure:shading-for-q2}}, p.\,\pageref{x:figure:shading-for-q2}}.  This means that you would shade only region 3:%
\par
\begin{figureptx}{Venn diagram for the compound statement \(p{\wedge} q\)}{x:figure:p-and-q}{}%
\begin{image}{0.3}{0.4}{0.3}%
\resizebox{\linewidth}{!}{%
\begin{venndiagram2sets}[labelA={$p$},labelB={$q$}]
  \fillACapB
\end{venndiagram2sets}
}%
\end{image}%
\tcblower
\end{figureptx}%
%
\end{subsectionptx}
%
%
\typeout{************************************************}
\typeout{Subsection 2.2.3 More Complications}
\typeout{************************************************}
%
\begin{subsectionptx}{More Complications}{}{More Complications}{}{}{x:subsection:more-complications}
Suppose you wished to create a Venn diagram representing \(\sim\!{p}\,{\wedge} q\).  One way to do this is by constructing the diagram for \(\sim\!{p}\) and the diagram for \(q\), and then consider conjunction.  Let's call this the \emph{visual step-by-step} method.%
\par
Here are the diagrams for \(\sim\!{p}\) and for \(q\): \begin{sidebyside}{2}{0.05}{0.05}{0.1}%
\begin{sbspanel}{0.4}%
\begin{figureptx}{Venn diagram for \(\sim\!{p}\)}{x:figure:not-p}{}%
\resizebox{\linewidth}{!}{%
\begin{venndiagram2sets}[labelA={$p$},labelB={$q$}]
  \fillNotA
\end{venndiagram2sets}
}%
\tcblower
\end{figureptx}%
\end{sbspanel}%
\begin{sbspanel}{0.4}%
\begin{figureptx}{Shading for \(q\)}{x:figure:shading-for-q2-again}{}%
\resizebox{\linewidth}{!}{%
\begin{venndiagram2sets}[labelA={$p$},labelB={$q$}]
  \fillB
\end{venndiagram2sets}
}%
\tcblower
\end{figureptx}%
\end{sbspanel}%
\end{sidebyside}%
%
\par
Now form the conjunction of the two diagrams by shading only the regions that are shaded in \emph{both} of the figures:%
\par
\begin{figureptx}{Venn diagram for \(\sim\!{p}{\wedge} q\)}{x:figure:not-p-and-q}{}%
\begin{image}{0.3}{0.4}{0.3}%
\resizebox{\linewidth}{!}{%
\begin{venndiagram2sets}[labelA={$p$},labelB={$q$}]
  \fillOnlyB
\end{venndiagram2sets}
}%
\end{image}%
\tcblower
\end{figureptx}%
%
\par
Another way to do this is to list the basic regions associated with each of the statements appearing in the compound statement, and from them determine which basic regions are associated with the compound statement itself.  Referring to \hyperref[x:figure:basic-regions]{Figure~{\xreffont\ref{x:figure:basic-regions}}, p.\,\pageref{x:figure:basic-regions}}, we see that%
\begin{itemize}[label=\textbullet]
\item{}\(\sim\!{p}\) is associated with regions 2 and 4%
\item{}\(q\) is associated with regions 2 and 3%
\end{itemize}
The conjunction of these two statements involves the basic region(s) that are common to both of the statements.  In this case, only basic region 2 is common.  This leads to \hyperref[x:figure:not-p-and-q]{Figure~{\xreffont\ref{x:figure:not-p-and-q}}, p.\,\pageref{x:figure:not-p-and-q}}.%
\par
To find \(\sim\!{p}{\vee} q\) using the visual step-by-step method, you'd take the diagrams in \hyperref[x:figure:not-p]{Figure~{\xreffont\ref{x:figure:not-p}}, p.\,\pageref{x:figure:not-p}} and \hyperref[x:figure:shading-for-q2-again]{Figure~{\xreffont\ref{x:figure:shading-for-q2-again}}, p.\,\pageref{x:figure:shading-for-q2-again}} and then shade the regions that are shaded in at least one of those two diagrams.  Alternatively, you can list the basic regions associated with each of the diagrams and then identify the basic regions that appear in at least one of the two lists.  Referring to the enumerations given earlier, we see that regions 2, 3, and 4 appear on those lists (either on one or the other or both).  So those are the regions that are shaded.  Either way, the result is as follows:%
\par
\begin{figureptx}{Venn diagram for the compound statement \(\sim\!{p}{\vee} q\)}{x:figure:not-p-or-q}{}%
\begin{image}{0.3}{0.4}{0.3}%
\resizebox{\linewidth}{!}{%
\begin{venndiagram2sets}[labelA={$p$},labelB={$q$}]
  \fillNotA
  \fillB
\end{venndiagram2sets}
}%
\end{image}%
\tcblower
\end{figureptx}%
%
\begin{example}{}{g:example:idp226847464}%
Shade a Venn diagram corresponding to \(\sim\!{p}\,{\vee}\sim\!{q}\).\par\smallskip%
\noindent\textbf{\blocktitlefont Solution}.\label{g:solution:idp226843496}{}\hypertarget{g:solution:idp226843496}{}\quad{}Using the visual step-by-step method, first draw the diagrams for \(\sim\!{p}\) and for \(\sim\!{q}\).  The diagram for \(\sim\!{p}\) is in \hyperref[x:figure:not-p]{Figure~{\xreffont\ref{x:figure:not-p}}, p.\,\pageref{x:figure:not-p}}, and here is the diagram for \(\sim\!{q}\):%
\par
\begin{figureptx}{Venn diagram for \(\sim\!{q}\)}{x:figure:not-q}{}%
\begin{image}{0.3}{0.4}{0.3}%
\resizebox{\linewidth}{!}{%
\begin{venndiagram2sets}[labelA={$p$},labelB={$q$}]
  \fillNotB
\end{venndiagram2sets}
}%
\end{image}%
\tcblower
\end{figureptx}%
%
\par
Now shade the regions that are shaded in at least one of those two diagrams to get the following:%
\par
\begin{figureptx}{Venn diagram for \(\sim\!{p}\,{\vee}\sim\!{q}\)}{x:figure:not-p-or-not-q}{}%
\begin{image}{0.3}{0.4}{0.3}%
\resizebox{\linewidth}{!}{%
\begin{venndiagram2sets}[labelA={$p$},labelB={$q$}]
  \fillNotAorNotB
\end{venndiagram2sets}
}%
\end{image}%
\tcblower
\end{figureptx}%
%
\par
Alternatively, we can look at the basic regions:%
\begin{itemize}[label=\textbullet]
\item{}\(\sim\!{p}\): regions 2 and 4%
\item{}\(\sim\!{q}\): regions 1 and 4%
\end{itemize}
We are interested in the inclusive disjunction, so shade any region that appears at least once in those lists: regions 1, 2, and 4.  This gives the figure above.%
\end{example}
\end{subsectionptx}
%
%
\typeout{************************************************}
\typeout{Subsection 2.2.4 Negation and De Morgan's Laws}
\typeout{************************************************}
%
\begin{subsectionptx}{Negation and De Morgan's Laws}{}{Negation and De Morgan's Laws}{}{}{x:subsection:negation-deMorgan}
Consider the proposition \(\sim\!(p {\wedge} q)\).  The brackets indicate that we must find \(p {\wedge} q\) first, and then form the negation.  The diagram for \(p {\wedge} q\) is shown in \hyperref[x:figure:p-and-q]{Figure~{\xreffont\ref{x:figure:p-and-q}}, p.\,\pageref{x:figure:p-and-q}}.  The diagram for \(\sim\!(p {\wedge} q)\) is obtained by "reversing": shade all previously unshaded regions, and unshade the shaded region.  This yields the following:%
\par
\begin{figureptx}{Venn diagram for \(\sim\!(p {\wedge} q)\)}{x:figure:neg-of-p-and-q}{}%
\begin{image}{0.3}{0.4}{0.3}%
\resizebox{\linewidth}{!}{%
\begin{venndiagram2sets}[labelA={$p$},labelB={$q$}]
  \fillNotAorNotB
\end{venndiagram2sets}
}%
\end{image}%
\tcblower
\end{figureptx}%
%
\par
This is identical to the Venn diagram for \(\sim\!{p}\,{\vee} \sim\!{q}\), so we can conclude that the compound statement \(\sim\!{p}\,{\vee} \sim\!{q}\) is equivalent to the compound statement \(\sim\!(p {\wedge} q)\).  Note that the particular nature of statements \(p\) and \(q\) do not affect this equivalence.  Similarly, it can be shown that \(\sim\!{p}\,{\wedge}\sim\!{q}\) is equivalent to \(\sim\!(p{\vee} q)\).  Together, these two assertions of equivalence are known as \terminology{De Morgan's Laws}: \begin{theorem}{De Morgan's Laws.}{}{g:theorem:idp226870616}%
Let \(p\) and \(q\) be any propositions.  Then%
\par
%
\begin{gather*}
\sim\!(p{\wedge} q)\Leftrightarrow\,\sim\!{p}\,{\vee}\sim\!{q}\\
\sim\!(p{\vee} q)\Leftrightarrow\,\sim\!{p}\,{\wedge}\sim\!{q}
\end{gather*}
where the symbol \(\Leftrightarrow\) is to be read as "is logically equivalent to".%
\end{theorem}
%
\end{subsectionptx}
%
%
\typeout{************************************************}
\typeout{Subsection 2.2.5 Venn Diagrams with Three Propositions}
\typeout{************************************************}
%
\begin{subsectionptx}{Venn Diagrams with Three Propositions}{}{Venn Diagrams with Three Propositions}{}{}{x:subsection:three-proposition-diagrams}
We can construct Venn diagrams for compound statements containing three propositions:%
\par
\begin{image}{0.3}{0.4}{0.3}%
\resizebox{\linewidth}{!}{%
\begin{venndiagram3sets}[labelA={$p$},labelB={$q$},labelC={$r$}]
\end{venndiagram3sets}
}%
\end{image}%
%
\par
Notice that there is a circle for each statement, and that there are regions where some or all of the circles overlap.  To find out how to shade the diagram for compound statements such as \((p {\vee} q) {\wedge} r\), use either the visual step-by-step method, or the method of enumerating basic regions.%
\par
Here are the diagrams for each of \(p\), \(q\), and \(r\):%
\par
\begin{sidebyside}{3}{0.1}{0.1}{0.025}%
\begin{sbspanel}{0.25}%
\resizebox{\linewidth}{!}{%
\begin{venndiagram3sets}[labelA={$p$},labelB={$q$},labelC={$r$}]
  \fillA
  \setpostvennhook
  {
    \draw (labelC)++(-90:1cm) node[below] {$p$};
  }
\end{venndiagram3sets}
}%
\end{sbspanel}%
\begin{sbspanel}{0.25}%
\resizebox{\linewidth}{!}{%
\begin{venndiagram3sets}[labelA={$p$},labelB={$q$},labelC={$r$}]
  \fillB
  \setpostvennhook
  {
    \draw (labelC)++(-90:1cm) node[below] {$q$};
  }
\end{venndiagram3sets}
}%
\end{sbspanel}%
\begin{sbspanel}{0.25}%
\resizebox{\linewidth}{!}{%
\begin{venndiagram3sets}[labelA={$p$},labelB={$q$},labelC={$r$}]
  \fillC
  \setpostvennhook
  {
    \draw (labelC)++(-90:1cm) node[below] {$r$};
  }
\end{venndiagram3sets}
}%
\end{sbspanel}%
\end{sidebyside}%
%
\par
Then \(p{\vee} q\) gives%
\par
\begin{image}{0.3}{0.4}{0.3}%
\resizebox{\linewidth}{!}{%
\begin{venndiagram3sets}[labelA={$p$},labelB={$q$},labelC={$r$}]
  \fillA
  \fillB
  \setpostvennhook
  {
    \draw (labelC)++(-90:1cm) node[below] {$p{\vee} q$};
  }
\end{venndiagram3sets}
}%
\end{image}%
%
\par
Now form the intersection of this with the diagram for \(r\):%
\par
\begin{figureptx}{\((p{\vee} q){\wedge} r\)}{x:figure:r-and-porq}{}%
\begin{image}{0.3}{0.4}{0.3}%
\resizebox{\linewidth}{!}{%
\begin{venndiagram3sets}[labelA={$p$},labelB={$q$},labelC={$r$}]
  \fillACapC
  \fillBCapC
\end{venndiagram3sets}
}%
\end{image}%
\tcblower
\end{figureptx}%
%
\par
Here is an enumeration of the basic regions.  It doesn't really matter how you assign labels to the regions, as long as you are consistent in the analysis you perform using the enumeration.%
\par
\begin{figureptx}{Basic regions in a three-proposition Venn diagram}{x:figure:basic-regions-3}{}%
\begin{image}{0.3}{0.4}{0.3}%
\resizebox{\linewidth}{!}{%
\begin{venndiagram3sets}[labelA={$p$},labelB={$q$},labelC={$r$},labelOnlyA={1},labelOnlyB={2},labelOnlyC={3},labelOnlyAB={4},labelOnlyAC={5},labelOnlyBC={6},labelABC={7},labelNotABC={8}]
\end{venndiagram3sets}
}%
\end{image}%
\tcblower
\end{figureptx}%
%
\par
Using the enumeration method, we see that%
\begin{itemize}[label=\textbullet]
\item{}\(p\) consists of regions 1, 4, 5, and 7%
\item{}\(q\) consists of 2, 4, 6, and 7%
\item{}\(r\) consists of 3, 5, 6, and 7%
\item{}Therefore, \(p{\vee} q\) consists of regions 1, 2, 4, 5, 6, and 7 since these regions appear in at least one of the two lists of regions for \(p\) and \(q\).%
\item{}Therefore, \((p{\vee} q){\wedge} r\) consists of regions 5, 6, and 7 because those regions appear on the lists for \emph{both} \(r\) and \(p{\vee} q\).%
\end{itemize}
This leads to \hyperref[x:figure:r-and-porq]{Figure~{\xreffont\ref{x:figure:r-and-porq}}, p.\,\pageref{x:figure:r-and-porq}}.%
\par
Two other important cases are shown below.  Note that the captions are indicated without brackets because%
\begin{equation*}
(p{\vee} q){\vee} r=p{\vee}(q{\vee} r)=p{\vee} q{\vee} r
\end{equation*}
and%
\begin{equation*}
(p{\wedge} q){\wedge} r=p{\wedge}(q{\wedge} r)=p{\wedge} q{\wedge} r
\end{equation*}
%
\par
\begin{sidebyside}{2}{0.1}{0.1}{0}%
\begin{sbspanel}{0.4}%
\resizebox{\linewidth}{!}{%
\begin{venndiagram3sets}[labelA={$p$},labelB={$q$},labelC={$r$}]
  \fillA
  \fillB
  \fillC
  \setpostvennhook
  {
    \draw (labelC)++(-90:1cm) node[below] {$p{\vee} q{\vee} r$};
  }
\end{venndiagram3sets}
}%
\end{sbspanel}%
\begin{sbspanel}{0.4}%
\resizebox{\linewidth}{!}{%
\begin{venndiagram3sets}[labelA={$p$},labelB={$q$},labelC={$r$}]
  \fillACapBCapC
  \setpostvennhook
  {
    \draw (labelC)++(-90:1cm) node[below] {$p{\wedge} q{\wedge} r$};
  }
\end{venndiagram3sets}
}%
\end{sbspanel}%
\end{sidebyside}%
%
\end{subsectionptx}
%
%
\typeout{************************************************}
\typeout{Exercises 2.2.6 Exercises}
\typeout{************************************************}
%
\begin{exercises-subsection}{Exercises}{}{Exercises}{}{}{g:exercises:idp226905432}
\par\medskip\noindent%
\textbf{Exercise Group.}\space\space%
Draw Venn diagrams using two propositions \(p\) and \(q\), shading in the appropriate regions for the following situations.\begin{exercisegroup}
\begin{divisionexerciseeg}{1}{}{}{g:exercise:idp226906456}%
\(\ p{\vee} q\)\end{divisionexerciseeg}%
\begin{divisionexerciseeg}{2}{}{}{g:exercise:idp226901336}%
\(\ p{\wedge} \sim\!{q}\)\end{divisionexerciseeg}%
\begin{divisionexerciseeg}{3}{}{}{g:exercise:idp226902360}%
\(\ \sim\!{p}{\wedge} \sim\!{q}\)\end{divisionexerciseeg}%
\begin{divisionexerciseeg}{4}{}{}{g:exercise:idp226911448}%
\(\ \sim\!(p{\wedge} \sim\!{q})\) (Compare with \#2.)\end{divisionexerciseeg}%
\begin{divisionexerciseeg}{5}{}{}{g:exercise:idp226909144}%
\(\ \sim\!(p{\vee} q)\)\end{divisionexerciseeg}%
\begin{divisionexerciseeg}{6}{}{}{g:exercise:idp226916184}%
\(\ p{\wedge}(\sim\!{p}{\vee} q)\)\end{divisionexerciseeg}%
\begin{divisionexerciseeg}{7}{}{}{g:exercise:idp226914136}%
\(\ p{\vee}(p{\wedge} q)\)\end{divisionexerciseeg}%
\end{exercisegroup}
\par\medskip\noindent
\par\medskip\noindent%
\textbf{Exercise Group.}\space\space%
Draw Venn diagrams using three propositions \(p\), \(q\), and \(r\), shading in the appropriate regions for the following situations.\begin{exercisegroup}
\begin{divisionexerciseeg}{8}{}{}{g:exercise:idp226921176}%
\(\ p{\vee} q{\vee} r\)\end{divisionexerciseeg}%
\begin{divisionexerciseeg}{9}{}{}{g:exercise:idp226921048}%
\(\ (p{\wedge} q){\vee} r\)\end{divisionexerciseeg}%
\begin{divisionexerciseeg}{10}{}{}{g:exercise:idp226923736}%
\(\ p{\wedge} (q{\vee} r\)\end{divisionexerciseeg}%
\begin{divisionexerciseeg}{11}{}{}{g:exercise:idp226920408}%
\(\ p{\vee}\sim\!{q}{\vee} r\)\end{divisionexerciseeg}%
\begin{divisionexerciseeg}{12}{}{}{g:exercise:idp226924120}%
\(\ \sim\!{p}{\wedge} q{\wedge}\sim\!{r}\)\end{divisionexerciseeg}%
\begin{divisionexerciseeg}{13}{}{}{g:exercise:idp226926552}%
\(\ (p{\wedge} q){\vee}\sim\!{r}\)\end{divisionexerciseeg}%
\begin{divisionexerciseeg}{14}{}{}{g:exercise:idp226926808}%
\(\ \sim\!{q}{\wedge}(\sim\!{p}{\vee} r)\)\end{divisionexerciseeg}%
\end{exercisegroup}
\par\medskip\noindent
\end{exercises-subsection}
%
%
\typeout{************************************************}
\typeout{Solutions 2.2.7 Solutions to Section~{\xreffont\ref*{x:section:sec-venn-diagrams}} Exercises}
\typeout{************************************************}
%
\begin{solutions-subsection}{Solutions to Section~{\xreffont\ref*{x:section:sec-venn-diagrams}} Exercises}{}{Solutions to Section~{\xreffont\ref*{x:section:sec-venn-diagrams}} Exercises}{}{}{g:solutions:idp226927192}
\par\medskip
\noindent\textbf{\normalsize{}2.2.6\space\textperiodcentered\space{}Exercises}
\begin{exercisegroup}
\begin{divisionsolutioneg}{2.2.6.1}{}{g:exercise:idp226906456}%
\par\smallskip%
\noindent\hypertarget{g:solution:idp226903128-main}{}\begin{image}{0.3}{0.4}{0.3}%
\resizebox{\linewidth}{!}{%
\begin{venndiagram2sets}[labelA={$p$},labelB={$q$}]
  \fillA
  \fillB
\end{venndiagram2sets}
}%
\end{image}%
\end{divisionsolutioneg}%
\begin{divisionsolutioneg}{2.2.6.2}{}{g:exercise:idp226901336}%
\par\smallskip%
\noindent\hypertarget{g:solution:idp226903768-main}{}\begin{image}{0.3}{0.4}{0.3}%
\resizebox{\linewidth}{!}{%
\begin{venndiagram2sets}[labelA={$p$},labelB={$q$}]
  \fillOnlyA
\end{venndiagram2sets}
}%
\end{image}%
\end{divisionsolutioneg}%
\begin{divisionsolutioneg}{2.2.6.3}{}{g:exercise:idp226902360}%
\par\smallskip%
\noindent\hypertarget{g:solution:idp226905816-main}{}\begin{image}{0.3}{0.4}{0.3}%
\resizebox{\linewidth}{!}{%
\begin{venndiagram2sets}[labelA={$p$},labelB={$q$}]
  \fillNotAorB
\end{venndiagram2sets}
}%
\end{image}%
\end{divisionsolutioneg}%
\begin{divisionsolutioneg}{2.2.6.4}{}{g:exercise:idp226911448}%
\par\smallskip%
\noindent\hypertarget{g:solution:idp226913112-main}{}\begin{image}{0.3}{0.4}{0.3}%
\resizebox{\linewidth}{!}{%
\begin{venndiagram2sets}[labelA={$p$},labelB={$q$}]
  \fillNotA
  \fillB
\end{venndiagram2sets}
}%
\end{image}%
\end{divisionsolutioneg}%
\begin{divisionsolutioneg}{2.2.6.5}{}{g:exercise:idp226909144}%
\par\smallskip%
\noindent\hypertarget{g:solution:idp226909784-main}{}\begin{image}{0.3}{0.4}{0.3}%
\resizebox{\linewidth}{!}{%
\begin{venndiagram2sets}[labelA={$p$},labelB={$q$}]
  \fillNotAorB
\end{venndiagram2sets}
}%
\end{image}%
\end{divisionsolutioneg}%
\begin{divisionsolutioneg}{2.2.6.6}{}{g:exercise:idp226916184}%
\par\smallskip%
\noindent\hypertarget{g:solution:idp226913624-main}{}\begin{image}{0.3}{0.4}{0.3}%
\resizebox{\linewidth}{!}{%
\begin{venndiagram2sets}[labelA={$p$},labelB={$q$}]
  \fillACapB
\end{venndiagram2sets}
}%
\end{image}%
\end{divisionsolutioneg}%
\begin{divisionsolutioneg}{2.2.6.7}{}{g:exercise:idp226914136}%
\par\smallskip%
\noindent\hypertarget{g:solution:idp226914520-main}{}\begin{image}{0.3}{0.4}{0.3}%
\resizebox{\linewidth}{!}{%
\begin{venndiagram2sets}[labelA={$p$},labelB={$q$}]
  \fillA
\end{venndiagram2sets}
}%
\end{image}%
\end{divisionsolutioneg}%
\end{exercisegroup}
\par\medskip\noindent
\begin{exercisegroup}
\begin{divisionsolutioneg}{2.2.6.8}{}{g:exercise:idp226921176}%
\par\smallskip%
\noindent\hypertarget{g:solution:idp226917720-main}{}\begin{image}{0.3}{0.4}{0.3}%
\resizebox{\linewidth}{!}{%
\begin{venndiagram3sets}[labelA={$p$},labelB={$q$},labelC={$r$}]
  \fillA
  \fillB
  \fillC
\end{venndiagram3sets}
}%
\end{image}%
\end{divisionsolutioneg}%
\begin{divisionsolutioneg}{2.2.6.9}{}{g:exercise:idp226921048}%
\par\smallskip%
\noindent\hypertarget{g:solution:idp226920536-main}{}\begin{image}{0.3}{0.4}{0.3}%
\resizebox{\linewidth}{!}{%
\begin{venndiagram3sets}[labelA={$p$},labelB={$q$},labelC={$r$}]
  \fillACapB
  \fillC
\end{venndiagram3sets}
}%
\end{image}%
\end{divisionsolutioneg}%
\begin{divisionsolutioneg}{2.2.6.10}{}{g:exercise:idp226923736}%
\par\smallskip%
\noindent\hypertarget{g:solution:idp226922584-main}{}\begin{image}{0.3}{0.4}{0.3}%
\resizebox{\linewidth}{!}{%
\begin{venndiagram3sets}[labelA={$p$},labelB={$q$},labelC={$r$}]
  \fillACapB
  \fillACapC
\end{venndiagram3sets}
}%
\end{image}%
\end{divisionsolutioneg}%
\begin{divisionsolutioneg}{2.2.6.11}{}{g:exercise:idp226920408}%
\par\smallskip%
\noindent\hypertarget{g:solution:idp226916952-main}{}\begin{image}{0.3}{0.4}{0.3}%
\resizebox{\linewidth}{!}{%
\begin{venndiagram3sets}[labelA={$p$},labelB={$q$},labelC={$r$}]
  \fillA
  \fillNotB
  \fillC
\end{venndiagram3sets}
}%
\end{image}%
\end{divisionsolutioneg}%
\begin{divisionsolutioneg}{2.2.6.12}{}{g:exercise:idp226924120}%
\par\smallskip%
\noindent\hypertarget{g:solution:idp226917080-main}{}\begin{image}{0.3}{0.4}{0.3}%
\resizebox{\linewidth}{!}{%
\begin{venndiagram3sets}[labelA={$p$},labelB={$q$},labelC={$r$}]
  \fillOnlyB
\end{venndiagram3sets}
}%
\end{image}%
\end{divisionsolutioneg}%
\begin{divisionsolutioneg}{2.2.6.13}{}{g:exercise:idp226926552}%
\par\smallskip%
\noindent\hypertarget{g:solution:idp226926680-main}{}\begin{image}{0.3}{0.4}{0.3}%
\resizebox{\linewidth}{!}{%
\begin{venndiagram3sets}[labelA={$p$},labelB={$q$},labelC={$r$}]
  \fillACapB
  \fillNotC
\end{venndiagram3sets}
}%
\end{image}%
\end{divisionsolutioneg}%
\begin{divisionsolutioneg}{2.2.6.14}{}{g:exercise:idp226926808}%
\par\smallskip%
\noindent\hypertarget{g:solution:idp226927576-main}{}\begin{image}{0.3}{0.4}{0.3}%
\resizebox{\linewidth}{!}{%
\begin{venndiagram3sets}[labelA={$p$},labelB={$q$},labelC={$r$}]
  \fillCNotB
  \fillNotABC
\end{venndiagram3sets}
}%
\end{image}%
\end{divisionsolutioneg}%
\end{exercisegroup}
\par\medskip\noindent
\end{solutions-subsection}
\end{sectionptx}
%
%
\typeout{************************************************}
\typeout{Section 2.3 Logical Equivalence}
\typeout{************************************************}
%
\begin{sectionptx}{Logical Equivalence}{}{Logical Equivalence}{}{}{x:section:sec-logical-equivalence}
%
%
\typeout{************************************************}
\typeout{Subsection 2.3.1 Truth Tables}
\typeout{************************************************}
%
\begin{subsectionptx}{Truth Tables}{}{Truth Tables}{}{}{x:subsection:truth-tables}
%
%
\typeout{************************************************}
\typeout{Subsubsection  Truth Tables with Two Variables}
\typeout{************************************************}
%
\begin{subsubsectionptx}{Truth Tables with Two Variables}{}{Truth Tables with Two Variables}{}{}{x:subsubsection:truth-tables-2vars}
Let us consider the propositions \(p\) and \(q\).  Since they are propositions, \(p\) is either true or false and \(q\) is also either true or false.  This leads us to four possible combinations of \(p\) and \(q\):%
\begin{enumerate}
\item{}\(p\) and \(q\) are both false.%
\item{}\(p\) is false and \(q\) is true%
\item{}\(p\) is true and \(q\) is false%
\item{}\(p\) and \(q\) are both true%
\end{enumerate}
We can combine these possibilities into a table called a truth table. \index{truth table!for propositions}\index{proposition!truth table for} We can add further columns to find out what the value of other compound propositions for each combination of \(p\) and \(q\) as well.  Suppose we wished to find out what the truth table was for \(p {\wedge} q\).  Then the table would look like the following. \begin{center}%
{\tabularfont%
\begin{tabular}{Bccc}\hrulethick
\multicolumn{1}{BcB}{\(p\)}&\multicolumn{1}{cB}{\(q\)}&\multicolumn{1}{cB}{\(p{\wedge} q\)}\tabularnewline\hrulemedium
\multicolumn{1}{BcB}{F}&\multicolumn{1}{cB}{F}&\multicolumn{1}{cB}{F}\tabularnewline[0pt]
\multicolumn{1}{BcB}{F}&\multicolumn{1}{cB}{T}&\multicolumn{1}{cB}{F}\tabularnewline[0pt]
\multicolumn{1}{BcB}{T}&\multicolumn{1}{cB}{F}&\multicolumn{1}{cB}{F}\tabularnewline[0pt]
\multicolumn{1}{BcB}{T}&\multicolumn{1}{cB}{T}&\multicolumn{1}{cB}{T}\tabularnewline\hrulethick
\end{tabular}
}%
\end{center}%
 For example, when \(p\) is false and \(q\) is true (the second row, where \(p\) = F and \(q\) = T), then \(p {\wedge} q\) is false because one of them is false (they are not both true).%
\par
Similarly, the truth table for \(p {\vee} q\) is \begin{center}%
{\tabularfont%
\begin{tabular}{Bccc}\hrulethick
\multicolumn{1}{BcB}{\(p\)}&\multicolumn{1}{cB}{\(q\)}&\multicolumn{1}{cB}{\(p{\vee} q\)}\tabularnewline\hrulemedium
\multicolumn{1}{BcB}{F}&\multicolumn{1}{cB}{F}&\multicolumn{1}{cB}{F}\tabularnewline[0pt]
\multicolumn{1}{BcB}{F}&\multicolumn{1}{cB}{T}&\multicolumn{1}{cB}{T}\tabularnewline[0pt]
\multicolumn{1}{BcB}{T}&\multicolumn{1}{cB}{F}&\multicolumn{1}{cB}{T}\tabularnewline[0pt]
\multicolumn{1}{BcB}{T}&\multicolumn{1}{cB}{T}&\multicolumn{1}{cB}{T}\tabularnewline\hrulethick
\end{tabular}
}%
\end{center}%
%
\par
If we like, we can combine the two tables above into a single table like so: \begin{center}%
{\tabularfont%
\begin{tabular}{Bcccc}\hrulethick
\multicolumn{1}{BcB}{\(p\)}&\multicolumn{1}{cB}{\(q\)}&\multicolumn{1}{cB}{\(p{\wedge} q\)}&\multicolumn{1}{cB}{\(p{\vee} q\)}\tabularnewline\hrulemedium
\multicolumn{1}{BcB}{F}&\multicolumn{1}{cB}{F}&\multicolumn{1}{cB}{F}&\multicolumn{1}{cB}{F}\tabularnewline[0pt]
\multicolumn{1}{BcB}{F}&\multicolumn{1}{cB}{T}&\multicolumn{1}{cB}{F}&\multicolumn{1}{cB}{T}\tabularnewline[0pt]
\multicolumn{1}{BcB}{T}&\multicolumn{1}{cB}{F}&\multicolumn{1}{cB}{F}&\multicolumn{1}{cB}{T}\tabularnewline[0pt]
\multicolumn{1}{BcB}{T}&\multicolumn{1}{cB}{T}&\multicolumn{1}{cB}{T}&\multicolumn{1}{cB}{T}\tabularnewline\hrulethick
\end{tabular}
}%
\end{center}%
%
\par
Another way to indicate F and T in a truth table is to use 0 and 1, respectively.  This mode of representation will be used later on, so it is a good idea to practise it now.  Here is the previous table re-written with this alternate notation. \begin{center}%
{\tabularfont%
\begin{tabular}{Bcccc}\hrulethick
\multicolumn{1}{BcB}{\(p\)}&\multicolumn{1}{cB}{\(q\)}&\multicolumn{1}{cB}{\(p{\wedge} q\)}&\multicolumn{1}{cB}{\(p{\vee} q\)}\tabularnewline\hrulemedium
\multicolumn{1}{BcB}{0}&\multicolumn{1}{cB}{0}&\multicolumn{1}{cB}{0}&\multicolumn{1}{cB}{0}\tabularnewline[0pt]
\multicolumn{1}{BcB}{0}&\multicolumn{1}{cB}{1}&\multicolumn{1}{cB}{0}&\multicolumn{1}{cB}{1}\tabularnewline[0pt]
\multicolumn{1}{BcB}{1}&\multicolumn{1}{cB}{0}&\multicolumn{1}{cB}{0}&\multicolumn{1}{cB}{1}\tabularnewline[0pt]
\multicolumn{1}{BcB}{1}&\multicolumn{1}{cB}{1}&\multicolumn{1}{cB}{1}&\multicolumn{1}{cB}{1}\tabularnewline\hrulethick
\end{tabular}
}%
\end{center}%
%
\end{subsubsectionptx}
%
%
\typeout{************************************************}
\typeout{Subsubsection  Truth Tables with One Variable}
\typeout{************************************************}
%
\begin{subsubsectionptx}{Truth Tables with One Variable}{}{Truth Tables with One Variable}{}{}{x:subsubsection:truth-tables-one-var}
What if we were interested in the truth table for \(p {\vee} p\)?  Since this truth table contains only one variable \(p\) rather than two, it will only have two rows since \(p\) can either be true or false (and yes, you could omit the second column if you wish). \begin{center}%
{\tabularfont%
\begin{tabular}{Bccc}\hrulethick
\multicolumn{1}{BcB}{\(p\)}&\multicolumn{1}{cB}{\(p\)}&\multicolumn{1}{cB}{\(p{\vee} p\)}\tabularnewline\hrulemedium
\multicolumn{1}{BcB}{0}&\multicolumn{1}{cB}{0}&\multicolumn{1}{cB}{0}\tabularnewline[0pt]
\multicolumn{1}{BcB}{1}&\multicolumn{1}{cB}{1}&\multicolumn{1}{cB}{1}\tabularnewline\hrulethick
\end{tabular}
}%
\end{center}%
%
\par
Let's now look at how to write the truth table for \(p{\wedge} 1\), where 1 means a statement that is always true.  (Be careful!  The number 1 is a constant, not a variable.  It never takes the value of zero.)  The table would look like: \begin{center}%
{\tabularfont%
\begin{tabular}{Bccc}\hrulethick
\multicolumn{1}{BcB}{\(p\)}&\multicolumn{1}{cB}{\(1\)}&\multicolumn{1}{cB}{\(p{\wedge} 1\)}\tabularnewline\hrulemedium
\multicolumn{1}{BcB}{\(0\)}&\multicolumn{1}{cB}{\(1\)}&\multicolumn{1}{cB}{\(0\)}\tabularnewline[0pt]
\multicolumn{1}{BcB}{\(1\)}&\multicolumn{1}{cB}{\(1\)}&\multicolumn{1}{cB}{\(1\)}\tabularnewline\hrulethick
\end{tabular}
}%
\end{center}%
 Notice that the last column looks like the first, so \(p{\wedge} 1\) has the same values as \(p\).  We say, then, that \(p{\wedge} 1\) is \terminology{logically equivalent} to \(p\). \index{logical equivalence!of propositions} The topic of logical equivalence will be examined in more detail later on.%
\end{subsubsectionptx}
%
%
\typeout{************************************************}
\typeout{Subsubsection  Negations in Truth Tables}
\typeout{************************************************}
%
\begin{subsubsectionptx}{Negations in Truth Tables}{}{Negations in Truth Tables}{}{}{x:subsubsection:negation-in-truth-tables}
To negate a variable, we simply switch the value of each entry in that column from its previous value.  For example, consider the truth table for \(p{\vee}\,\sim\!{p}\).  We start with a column for \(p\) and one for \(\sim\!{p}\).  To populate the latter, we have to negate \(p\).  To do this, we take every entry in the \(p\) column and switch all the zeros to ones and the ones to zeros.  We'll then form the inclusive disjunction of those two columns. \begin{center}%
{\tabularfont%
\begin{tabular}{Bccc}\hrulethick
\multicolumn{1}{BcB}{\(p\)}&\multicolumn{1}{cB}{\(\sim\!{p}\)}&\multicolumn{1}{cB}{\(p\,{\vee}\,\sim\!{p}\)}\tabularnewline\hrulemedium
\multicolumn{1}{BcB}{0}&\multicolumn{1}{cB}{1}&\multicolumn{1}{cB}{1}\tabularnewline[0pt]
\multicolumn{1}{BcB}{1}&\multicolumn{1}{cB}{0}&\multicolumn{1}{cB}{1}\tabularnewline\hrulethick
\end{tabular}
}%
\end{center}%
%
\par
Notice that \(p{\vee}\,\sim\!{p}\) is always true.  A concrete example will hopefully illustrate the reasonableness of this: "Jason's shirt is either plaid or not plaid."  This statement is true regardless of the pattern on Jason's shirt.%
\par
To negate an expession, we use the same idea and switch the value of each entry in the column for that expression, whether that expression is simple or compound.  For example, here is the truth table for \(\sim\!(p{\wedge} 1)\): \begin{center}%
{\tabularfont%
\begin{tabular}{Bcccc}\hrulethick
\multicolumn{1}{BcB}{\(p\)}&\multicolumn{1}{cB}{1}&\multicolumn{1}{cB}{\(p{\wedge} 1\)}&\multicolumn{1}{cB}{\(\sim\!(p{\wedge} 1)\)}\tabularnewline\hrulemedium
\multicolumn{1}{BcB}{0}&\multicolumn{1}{cB}{1}&\multicolumn{1}{cB}{0}&\multicolumn{1}{cB}{1}\tabularnewline[0pt]
\multicolumn{1}{BcB}{1}&\multicolumn{1}{cB}{1}&\multicolumn{1}{cB}{1}&\multicolumn{1}{cB}{0}\tabularnewline\hrulethick
\end{tabular}
}%
\end{center}%
 The fourth column is the negation of the third column: the values in the fourth column are obtained by applying the switches \(0\to 1\) and \(1\to 0\) to the values in the third column.%
\end{subsubsectionptx}
%
%
\typeout{************************************************}
\typeout{Subsubsection  Truth Tables with Three Variables}
\typeout{************************************************}
%
\begin{subsubsectionptx}{Truth Tables with Three Variables}{}{Truth Tables with Three Variables}{}{}{x:subsubsection:truth-table-3-vars}
A truth table with three variables will contain 8 rows in order to display all the possible truth values for the three statements \(p\), \(q\), and \(r\).  For example, the truth table for \(\sim\!{p}\,{\wedge}(q\,{\vee} \sim\!{r})\) is \begin{center}%
{\tabularfont%
\begin{tabular}{Bccccccc}\hrulethick
\multicolumn{1}{BcB}{\(p\)}&\multicolumn{1}{cB}{\(q\)}&\multicolumn{1}{cB}{\(r\)}&\multicolumn{1}{cB}{\(\sim\!{r}\)}&\multicolumn{1}{cB}{\(q\,{\vee}\sim\!{r}\)}&\multicolumn{1}{cB}{\(\sim\!{p}\)}&\multicolumn{1}{cB}{\(\sim\!{p}\,{\wedge}(q\,{\vee} \sim\!{r})\)}\tabularnewline\hrulemedium
\multicolumn{1}{BcB}{0}&\multicolumn{1}{cB}{0}&\multicolumn{1}{cB}{0}&\multicolumn{1}{cB}{1}&\multicolumn{1}{cB}{1}&\multicolumn{1}{cB}{1}&\multicolumn{1}{cB}{1}\tabularnewline[0pt]
\multicolumn{1}{BcB}{0}&\multicolumn{1}{cB}{0}&\multicolumn{1}{cB}{1}&\multicolumn{1}{cB}{0}&\multicolumn{1}{cB}{0}&\multicolumn{1}{cB}{1}&\multicolumn{1}{cB}{0}\tabularnewline[0pt]
\multicolumn{1}{BcB}{0}&\multicolumn{1}{cB}{1}&\multicolumn{1}{cB}{0}&\multicolumn{1}{cB}{1}&\multicolumn{1}{cB}{1}&\multicolumn{1}{cB}{1}&\multicolumn{1}{cB}{1}\tabularnewline[0pt]
\multicolumn{1}{BcB}{0}&\multicolumn{1}{cB}{1}&\multicolumn{1}{cB}{1}&\multicolumn{1}{cB}{0}&\multicolumn{1}{cB}{1}&\multicolumn{1}{cB}{1}&\multicolumn{1}{cB}{1}\tabularnewline[0pt]
\multicolumn{1}{BcB}{1}&\multicolumn{1}{cB}{0}&\multicolumn{1}{cB}{0}&\multicolumn{1}{cB}{1}&\multicolumn{1}{cB}{1}&\multicolumn{1}{cB}{0}&\multicolumn{1}{cB}{0}\tabularnewline[0pt]
\multicolumn{1}{BcB}{1}&\multicolumn{1}{cB}{0}&\multicolumn{1}{cB}{1}&\multicolumn{1}{cB}{0}&\multicolumn{1}{cB}{0}&\multicolumn{1}{cB}{0}&\multicolumn{1}{cB}{0}\tabularnewline[0pt]
\multicolumn{1}{BcB}{1}&\multicolumn{1}{cB}{1}&\multicolumn{1}{cB}{0}&\multicolumn{1}{cB}{1}&\multicolumn{1}{cB}{1}&\multicolumn{1}{cB}{0}&\multicolumn{1}{cB}{0}\tabularnewline[0pt]
\multicolumn{1}{BcB}{1}&\multicolumn{1}{cB}{1}&\multicolumn{1}{cB}{1}&\multicolumn{1}{cB}{0}&\multicolumn{1}{cB}{1}&\multicolumn{1}{cB}{0}&\multicolumn{1}{cB}{0}\tabularnewline\hrulethick
\end{tabular}
}%
\end{center}%
%
\par
It is important to note that the actual order of the rows doesn't matter for the truth table to be complete.  However, if you write out the table with the rows in a random order, it is very easy to duplicate one of the previous rows.  If you continue to fill the table until you have the correct \alert{total} number of rows, then the duplication means that one of the combinations of variables is missing.%
\par
Therefore, it is a good idea to fill the table in a systematic way.  Look at the \emph{column entries}, starting with the leftmost column.  Then look at the next column, then the next.  Note the pattern of the entries in each column.  If you mimic that pattern in your own work, you will minimize the chances of missing a combination variables.%
\par
Another common mistake is to take a shortcut and start the truth table with one of the columns being, for example, \(\sim\!{p}\).  This is not correct.  Truth tables must always start with unnegated variables.%
\end{subsubsectionptx}
\end{subsectionptx}
%
%
\typeout{************************************************}
\typeout{Subsection 2.3.2 Logical Equivalence}
\typeout{************************************************}
%
\begin{subsectionptx}{Logical Equivalence}{}{Logical Equivalence}{}{}{x:subsection:subsec-logical-equivalence}
Two logical expressions are said to be \terminology{logically equivalent} if they have the same values in their columns in the truth table. \index{logical equivalence!of propositions}\index{equivalence, logical!of propositions} We saw in our examples above that \(p\,{\vee}\sim\!{p}\) was logically equivalent to 1\textbackslash{}, and \(p{\wedge} 1\) was logically equivalent to \(p\).  The symbol for \emph{logically equivalent to} is \(\Leftrightarrow\), so \(p\,{\vee}\sim\!{p}\Leftrightarrow 1\) and \(p{\wedge} 1\Leftrightarrow p\).%
\begin{example}{}{g:example:idp227062104}%
Is \(p{\wedge} (q{\vee} r)\) logically equivalent to \((p{\wedge} q){\vee} r\)?\par\smallskip%
\noindent\textbf{\blocktitlefont Solution}.\label{g:solution:idp227061848}{}\hypertarget{g:solution:idp227061848}{}\quad{}\begin{center}%
{\tabularfont%
\begin{tabular}{Bccccccc}\hrulethick
\multicolumn{1}{BcB}{\(p\)}&\multicolumn{1}{cB}{\(q\)}&\multicolumn{1}{cB}{\(t\)}&\multicolumn{1}{cB}{\(q {\vee} r\)}&\multicolumn{1}{cB}{\(p{\wedge}(q {\vee} r)\)}&\multicolumn{1}{cB}{\(p{\wedge} q\)}&\multicolumn{1}{cB}{\((p{\wedge} q){\vee} r\)}\tabularnewline\hrulemedium
\multicolumn{1}{BcB}{0}&\multicolumn{1}{cB}{0}&\multicolumn{1}{cB}{0}&\multicolumn{1}{cB}{0}&\multicolumn{1}{cB}{0}&\multicolumn{1}{cB}{0}&\multicolumn{1}{cB}{0}\tabularnewline[0pt]
\multicolumn{1}{BcB}{0}&\multicolumn{1}{cB}{0}&\multicolumn{1}{cB}{1}&\multicolumn{1}{cB}{1}&\multicolumn{1}{cB}{0}&\multicolumn{1}{cB}{0}&\multicolumn{1}{cB}{1}\tabularnewline[0pt]
\multicolumn{1}{BcB}{0}&\multicolumn{1}{cB}{1}&\multicolumn{1}{cB}{0}&\multicolumn{1}{cB}{1}&\multicolumn{1}{cB}{0}&\multicolumn{1}{cB}{0}&\multicolumn{1}{cB}{0}\tabularnewline[0pt]
\multicolumn{1}{BcB}{0}&\multicolumn{1}{cB}{1}&\multicolumn{1}{cB}{1}&\multicolumn{1}{cB}{1}&\multicolumn{1}{cB}{0}&\multicolumn{1}{cB}{0}&\multicolumn{1}{cB}{1}\tabularnewline[0pt]
\multicolumn{1}{BcB}{1}&\multicolumn{1}{cB}{0}&\multicolumn{1}{cB}{0}&\multicolumn{1}{cB}{0}&\multicolumn{1}{cB}{0}&\multicolumn{1}{cB}{0}&\multicolumn{1}{cB}{0}\tabularnewline[0pt]
\multicolumn{1}{BcB}{1}&\multicolumn{1}{cB}{0}&\multicolumn{1}{cB}{1}&\multicolumn{1}{cB}{1}&\multicolumn{1}{cB}{1}&\multicolumn{1}{cB}{0}&\multicolumn{1}{cB}{1}\tabularnewline[0pt]
\multicolumn{1}{BcB}{1}&\multicolumn{1}{cB}{1}&\multicolumn{1}{cB}{0}&\multicolumn{1}{cB}{1}&\multicolumn{1}{cB}{1}&\multicolumn{1}{cB}{1}&\multicolumn{1}{cB}{1}\tabularnewline[0pt]
\multicolumn{1}{BcB}{1}&\multicolumn{1}{cB}{1}&\multicolumn{1}{cB}{1}&\multicolumn{1}{cB}{1}&\multicolumn{1}{cB}{1}&\multicolumn{1}{cB}{1}&\multicolumn{1}{cB}{1}\tabularnewline\hrulethick
\end{tabular}
}%
\end{center}%
 These two expressions are \emph{not} logically equivalent because their columns in the truth table, the 5th and 7th columns, are not identical.  This example shows once more that order of operations is important.%
\end{example}
\begin{example}{}{g:example:idp227090776}%
Simplify \((p{\wedge} q){\vee}(\sim\!{p}\,{\wedge} q)\).\par\smallskip%
\noindent\textbf{\blocktitlefont Solution}.\label{g:solution:idp227092952}{}\hypertarget{g:solution:idp227092952}{}\quad{}\begin{center}%
{\tabularfont%
\begin{tabular}{Bcccccc}\hrulethick
\multicolumn{1}{BcB}{\(p\)}&\multicolumn{1}{cB}{\(q\)}&\multicolumn{1}{cB}{\(\sim\!{p}\)}&\multicolumn{1}{cB}{\(p{\wedge} q\)}&\multicolumn{1}{cB}{\(\sim\!{p}\,{\wedge} q\)}&\multicolumn{1}{cB}{\((p{\wedge} q){\vee}(\sim\!{p}\,{\wedge} q)\)}\tabularnewline\hrulemedium
\multicolumn{1}{BcB}{0}&\multicolumn{1}{cB}{0}&\multicolumn{1}{cB}{1}&\multicolumn{1}{cB}{0}&\multicolumn{1}{cB}{0}&\multicolumn{1}{cB}{0}\tabularnewline[0pt]
\multicolumn{1}{BcB}{0}&\multicolumn{1}{cB}{1}&\multicolumn{1}{cB}{1}&\multicolumn{1}{cB}{0}&\multicolumn{1}{cB}{1}&\multicolumn{1}{cB}{1}\tabularnewline[0pt]
\multicolumn{1}{BcB}{1}&\multicolumn{1}{cB}{0}&\multicolumn{1}{cB}{0}&\multicolumn{1}{cB}{0}&\multicolumn{1}{cB}{0}&\multicolumn{1}{cB}{0}\tabularnewline[0pt]
\multicolumn{1}{BcB}{1}&\multicolumn{1}{cB}{1}&\multicolumn{1}{cB}{0}&\multicolumn{1}{cB}{1}&\multicolumn{1}{cB}{0}&\multicolumn{1}{cB}{1}\tabularnewline\hrulethick
\end{tabular}
}%
\end{center}%
 Notice that the last column is identical to the column for \(q\).  Therefore, \((p{\wedge} q){\vee}(\sim\!{p}\,{\wedge} q)\Leftrightarrow q\), and so the simplified expression is \(q\)%
\end{example}
\begin{example}{}{g:example:idp227110872}%
Is \(p {\oplus} q\) logically equivalent to \((p{\wedge}\sim\!{q}){\vee} (\sim\!{p}\,{\wedge} q)\)?\par\smallskip%
\noindent\textbf{\blocktitlefont Solution}.\label{g:solution:idp227113176}{}\hypertarget{g:solution:idp227113176}{}\quad{}\begin{center}%
{\tabularfont%
\begin{tabular}{Bcccccccc}\hrulethick
\multicolumn{1}{BcB}{\(p\)}&\multicolumn{1}{cB}{\(q\)}&\multicolumn{1}{cB}{\(p {\oplus} q\)}&\multicolumn{1}{cB}{\(\sim\!{p}\)}&\multicolumn{1}{cB}{\(\sim\!{q}\)}&\multicolumn{1}{cB}{\(p\,{\wedge}\sim\!{q}\)}&\multicolumn{1}{cB}{\(\sim\!{p} {\wedge} q\)}&\multicolumn{1}{cB}{\((p{\wedge}\sim\!{q}){\vee} (\sim\!{p}\,{\wedge} q)\)}\tabularnewline\hrulemedium
\multicolumn{1}{BcB}{0}&\multicolumn{1}{cB}{0}&\multicolumn{1}{cB}{0}&\multicolumn{1}{cB}{1}&\multicolumn{1}{cB}{1}&\multicolumn{1}{cB}{0}&\multicolumn{1}{cB}{0}&\multicolumn{1}{cB}{0}\tabularnewline[0pt]
\multicolumn{1}{BcB}{0}&\multicolumn{1}{cB}{1}&\multicolumn{1}{cB}{1}&\multicolumn{1}{cB}{1}&\multicolumn{1}{cB}{0}&\multicolumn{1}{cB}{0}&\multicolumn{1}{cB}{1}&\multicolumn{1}{cB}{1}\tabularnewline[0pt]
\multicolumn{1}{BcB}{1}&\multicolumn{1}{cB}{0}&\multicolumn{1}{cB}{1}&\multicolumn{1}{cB}{0}&\multicolumn{1}{cB}{1}&\multicolumn{1}{cB}{1}&\multicolumn{1}{cB}{0}&\multicolumn{1}{cB}{1}\tabularnewline[0pt]
\multicolumn{1}{BcB}{1}&\multicolumn{1}{cB}{1}&\multicolumn{1}{cB}{0}&\multicolumn{1}{cB}{0}&\multicolumn{1}{cB}{0}&\multicolumn{1}{cB}{0}&\multicolumn{1}{cB}{0}&\multicolumn{1}{cB}{0}\tabularnewline\hrulethick
\end{tabular}
}%
\end{center}%
 Comparing columns 3 and 8, we see that the two statements are logically equivalent.%
\end{example}
\end{subsectionptx}
%
%
\typeout{************************************************}
\typeout{Exercises 2.3.3 Exercises}
\typeout{************************************************}
%
\begin{exercises-subsection}{Exercises}{}{Exercises}{}{}{g:exercises:idp226865496}
\par\medskip\noindent%
\textbf{Exercise Group.}\space\space%
Give the truth tables for the following logical expressions.\begin{exercisegroup}
\begin{divisionexerciseeg}{1}{}{}{g:exercise:idp226860504}%
\(\ p\,{\wedge}\sim\!{p}\)\end{divisionexerciseeg}%
\begin{divisionexerciseeg}{2}{}{}{g:exercise:idp227171160}%
\(\ p{\vee} 1\)\end{divisionexerciseeg}%
\begin{divisionexerciseeg}{3}{}{}{g:exercise:idp227181016}%
\(\ p\,{\wedge}\sim\!{q}\)\end{divisionexerciseeg}%
\begin{divisionexerciseeg}{4}{}{}{g:exercise:idp227190872}%
\(\sim\!(p{\vee} q)\)\end{divisionexerciseeg}%
\begin{divisionexerciseeg}{5}{}{}{g:exercise:idp227205080}%
\(\ p\,{\oplus}\sim\!{q}\)\end{divisionexerciseeg}%
\begin{divisionexerciseeg}{6}{}{}{g:exercise:idp227212120}%
\(\ p{\vee}(\sim\!{p}\,{\wedge} q)\)\end{divisionexerciseeg}%
\begin{divisionexerciseeg}{7}{}{}{g:exercise:idp227228888}%
\(\ (p{\vee} q){\wedge} r\)\end{divisionexerciseeg}%
\begin{divisionexerciseeg}{8}{}{}{g:exercise:idp227259736}%
\(\ p\,{\vee} q\,{\vee}\, \sim\!{r}\)\end{divisionexerciseeg}%
\begin{divisionexerciseeg}{9}{}{}{g:exercise:idp227280088}%
\(\ (p{\wedge} q)\,{\vee}\,\sim\!(p\,{\vee}\,\sim\!{q})\)\end{divisionexerciseeg}%
\begin{divisionexerciseeg}{10}{}{}{g:exercise:idp227298520}%
\((\sim\!{p}\,{\vee}\,\sim\!{q})\,{\wedge}\,(\sim\!{p}\,{\vee}\,q)\)\end{divisionexerciseeg}%
\end{exercisegroup}
\par\medskip\noindent
\par\medskip\noindent%
\textbf{Exercise Group.}\space\space%
Are the two expressions logically equivalent?\begin{exercisegroup}
\begin{divisionexerciseeg}{11}{}{}{g:exercise:idp227312344}%
\(\ \sim\!(p{\wedge} q)\) and \(\sim\!{p}\,{\wedge}\,\sim\!{q}\)\end{divisionexerciseeg}%
\begin{divisionexerciseeg}{12}{}{}{g:exercise:idp227342552}%
\(\ \sim\!(p{\vee} q)\) and \(\sim\!{p}\,{\wedge}\,\sim\!{q}\)\end{divisionexerciseeg}%
\begin{divisionexerciseeg}{13}{}{}{g:exercise:idp227351128}%
\(\ p{\oplus} q\) and \(\sim\!{p}\,{\oplus}\,\sim\!{q}\)\end{divisionexerciseeg}%
\begin{divisionexerciseeg}{14}{}{}{g:exercise:idp227370840}%
\(\ p{\vee}(q{\wedge} r)\) and \((p{\vee} q){\wedge} r\)\end{divisionexerciseeg}%
\begin{divisionexerciseeg}{15}{}{}{g:exercise:idp227404760}%
\(\ p{\vee}(p{\wedge} q)\) and \(p\)\end{divisionexerciseeg}%
\begin{divisionexerciseeg}{16}{}{}{x:exercise:exer-logic-commutative}%
\(\ (p{\vee} q){\vee} r\) and \(p{\vee} (q{\vee} r)\)\end{divisionexerciseeg}%
\begin{divisionexerciseeg}{17}{}{}{g:exercise:idp227817304}%
\(\ p{\oplus} q\) and \((p{\wedge} q){\vee} (\sim\!{p}\,{\wedge}\,\sim\!{q})\)\end{divisionexerciseeg}%
\end{exercisegroup}
\par\medskip\noindent
\par\medskip\noindent%
\textbf{Exercise Group.}\space\space%
Simplify each statement.\begin{exercisegroup}
\begin{divisionexerciseeg}{18}{}{}{g:exercise:idp227834072}%
\(\ p{\wedge} p\)\end{divisionexerciseeg}%
\begin{divisionexerciseeg}{19}{}{}{g:exercise:idp227837784}%
\(\ p\,{\vee}\sim\!{p}\)\end{divisionexerciseeg}%
\begin{divisionexerciseeg}{20}{}{}{g:exercise:idp227849176}%
\(\ p{\wedge} 0\)\end{divisionexerciseeg}%
\begin{divisionexerciseeg}{21}{}{}{g:exercise:idp227855320}%
\(\sim\!{p}\,{\oplus}\,p\)\end{divisionexerciseeg}%
\begin{divisionexerciseeg}{22}{}{}{g:exercise:idp227864792}%
\(\ (p{\oplus} q){\wedge}(p\,{\oplus} \sim\!{q})\)\end{divisionexerciseeg}%
\begin{divisionexerciseeg}{23}{}{}{g:exercise:idp227881048}%
\(\ p{\vee}(p{\wedge} q)\)\end{divisionexerciseeg}%
\begin{divisionexerciseeg}{24}{}{}{g:exercise:idp227889240}%
\(\ q{\wedge}(p{\vee} q)\)\end{divisionexerciseeg}%
\begin{divisionexerciseeg}{25}{}{}{g:exercise:idp227902040}%
\(\ p{\wedge}(\sim\!{p}\,{\vee}\,q)\)\end{divisionexerciseeg}%
\begin{divisionexerciseeg}{26}{}{}{g:exercise:idp227920472}%
\(\ p{\vee}(\sim\!{p}\,{\wedge}\,q)\)\end{divisionexerciseeg}%
\end{exercisegroup}
\par\medskip\noindent
\end{exercises-subsection}
%
%
\typeout{************************************************}
\typeout{Solutions 2.3.4 Solutions to Section~{\xreffont\ref*{x:section:sec-logical-equivalence}} Exercises}
\typeout{************************************************}
%
\begin{solutions-subsection}{Solutions to Section~{\xreffont\ref*{x:section:sec-logical-equivalence}} Exercises}{}{Solutions to Section~{\xreffont\ref*{x:section:sec-logical-equivalence}} Exercises}{}{}{g:solutions:idp227940056}
\par\medskip
\noindent\textbf{\normalsize{}2.3.3\space\textperiodcentered\space{}Exercises}
\begin{exercisegroup}
\begin{divisionsolutioneg}{2.3.3.1}{}{g:exercise:idp226860504}%
\par\smallskip%
\noindent\hypertarget{g:solution:idp226860632-main}{}\begin{center}%
{\tabularfont%
\begin{tabular}{Bccc}\hrulethick
\multicolumn{1}{BcB}{\(p\)}&\multicolumn{1}{cB}{\(\sim\!{p}\)}&\multicolumn{1}{cB}{\(p\,{\wedge}\sim\!{p}\)}\tabularnewline\hrulemedium
\multicolumn{1}{BcB}{0}&\multicolumn{1}{cB}{1}&\multicolumn{1}{cB}{0}\tabularnewline[0pt]
\multicolumn{1}{BcB}{1}&\multicolumn{1}{cB}{0}&\multicolumn{1}{cB}{0}\tabularnewline\hrulethick
\end{tabular}
}%
\end{center}%
\end{divisionsolutioneg}%
\begin{divisionsolutioneg}{2.3.3.2}{}{g:exercise:idp227171160}%
\par\smallskip%
\noindent\hypertarget{g:solution:idp227177048-main}{}\begin{center}%
{\tabularfont%
\begin{tabular}{Bccc}\hrulethick
\multicolumn{1}{BcB}{\(p\)}&\multicolumn{1}{cB}{\(1\)}&\multicolumn{1}{cB}{\(p {\vee} 1\)}\tabularnewline\hrulemedium
\multicolumn{1}{BcB}{0}&\multicolumn{1}{cB}{1}&\multicolumn{1}{cB}{1}\tabularnewline[0pt]
\multicolumn{1}{BcB}{1}&\multicolumn{1}{cB}{1}&\multicolumn{1}{cB}{1}\tabularnewline\hrulethick
\end{tabular}
}%
\end{center}%
\end{divisionsolutioneg}%
\begin{divisionsolutioneg}{2.3.3.3}{}{g:exercise:idp227181016}%
\par\smallskip%
\noindent\hypertarget{g:solution:idp227186520-main}{}\begin{center}%
{\tabularfont%
\begin{tabular}{Bcccc}\hrulethick
\multicolumn{1}{BcB}{\(p\)}&\multicolumn{1}{cB}{\(q\)}&\multicolumn{1}{cB}{\(\sim\!{p}\)}&\multicolumn{1}{cB}{\(p {\wedge} \sim\!{q}\)}\tabularnewline\hrulemedium
\multicolumn{1}{BcB}{0}&\multicolumn{1}{cB}{0}&\multicolumn{1}{cB}{1}&\multicolumn{1}{cB}{0}\tabularnewline[0pt]
\multicolumn{1}{BcB}{0}&\multicolumn{1}{cB}{1}&\multicolumn{1}{cB}{0}&\multicolumn{1}{cB}{0}\tabularnewline[0pt]
\multicolumn{1}{BcB}{1}&\multicolumn{1}{cB}{0}&\multicolumn{1}{cB}{1}&\multicolumn{1}{cB}{1}\tabularnewline[0pt]
\multicolumn{1}{BcB}{1}&\multicolumn{1}{cB}{1}&\multicolumn{1}{cB}{0}&\multicolumn{1}{cB}{0}\tabularnewline\hrulethick
\end{tabular}
}%
\end{center}%
\end{divisionsolutioneg}%
\begin{divisionsolutioneg}{2.3.3.4}{}{g:exercise:idp227190872}%
\par\smallskip%
\noindent\hypertarget{g:solution:idp227187800-main}{}\begin{center}%
{\tabularfont%
\begin{tabular}{Bcccc}\hrulethick
\multicolumn{1}{BcB}{\(p\)}&\multicolumn{1}{cB}{\(q\)}&\multicolumn{1}{cB}{\(p{\vee} q\)}&\multicolumn{1}{cB}{\(\sim\!(p{\vee} q)\)}\tabularnewline\hrulemedium
\multicolumn{1}{BcB}{0}&\multicolumn{1}{cB}{0}&\multicolumn{1}{cB}{0}&\multicolumn{1}{cB}{1}\tabularnewline[0pt]
\multicolumn{1}{BcB}{0}&\multicolumn{1}{cB}{1}&\multicolumn{1}{cB}{1}&\multicolumn{1}{cB}{0}\tabularnewline[0pt]
\multicolumn{1}{BcB}{1}&\multicolumn{1}{cB}{0}&\multicolumn{1}{cB}{1}&\multicolumn{1}{cB}{0}\tabularnewline[0pt]
\multicolumn{1}{BcB}{1}&\multicolumn{1}{cB}{1}&\multicolumn{1}{cB}{0}&\multicolumn{1}{cB}{0}\tabularnewline\hrulethick
\end{tabular}
}%
\end{center}%
\end{divisionsolutioneg}%
\begin{divisionsolutioneg}{2.3.3.5}{}{g:exercise:idp227205080}%
\par\smallskip%
\noindent\hypertarget{g:solution:idp227203672-main}{}\begin{center}%
{\tabularfont%
\begin{tabular}{Bcccc}\hrulethick
\multicolumn{1}{BcB}{\(p\)}&\multicolumn{1}{cB}{\(q\)}&\multicolumn{1}{cB}{\(\sim\!{q}\)}&\multicolumn{1}{cB}{\(p\,{\oplus}\sim\!{q}\)}\tabularnewline\hrulemedium
\multicolumn{1}{BcB}{0}&\multicolumn{1}{cB}{0}&\multicolumn{1}{cB}{1}&\multicolumn{1}{cB}{1}\tabularnewline[0pt]
\multicolumn{1}{BcB}{0}&\multicolumn{1}{cB}{1}&\multicolumn{1}{cB}{0}&\multicolumn{1}{cB}{0}\tabularnewline[0pt]
\multicolumn{1}{BcB}{1}&\multicolumn{1}{cB}{0}&\multicolumn{1}{cB}{1}&\multicolumn{1}{cB}{0}\tabularnewline[0pt]
\multicolumn{1}{BcB}{1}&\multicolumn{1}{cB}{1}&\multicolumn{1}{cB}{0}&\multicolumn{1}{cB}{1}\tabularnewline\hrulethick
\end{tabular}
}%
\end{center}%
\end{divisionsolutioneg}%
\begin{divisionsolutioneg}{2.3.3.6}{}{g:exercise:idp227212120}%
\par\smallskip%
\noindent\hypertarget{g:solution:idp227218136-main}{}\begin{center}%
{\tabularfont%
\begin{tabular}{Bccccc}\hrulethick
\multicolumn{1}{BcB}{\(p\)}&\multicolumn{1}{cB}{\(q\)}&\multicolumn{1}{cB}{\(\sim\!{p}\)}&\multicolumn{1}{cB}{\(\sim\!{p}\,{\wedge} q\)}&\multicolumn{1}{cB}{\(p{\vee}(\sim\!{p}\,{\wedge} q)\)}\tabularnewline\hrulemedium
\multicolumn{1}{BcB}{0}&\multicolumn{1}{cB}{0}&\multicolumn{1}{cB}{1}&\multicolumn{1}{cB}{0}&\multicolumn{1}{cB}{0}\tabularnewline[0pt]
\multicolumn{1}{BcB}{0}&\multicolumn{1}{cB}{1}&\multicolumn{1}{cB}{1}&\multicolumn{1}{cB}{1}&\multicolumn{1}{cB}{1}\tabularnewline[0pt]
\multicolumn{1}{BcB}{1}&\multicolumn{1}{cB}{0}&\multicolumn{1}{cB}{0}&\multicolumn{1}{cB}{0}&\multicolumn{1}{cB}{1}\tabularnewline[0pt]
\multicolumn{1}{BcB}{1}&\multicolumn{1}{cB}{1}&\multicolumn{1}{cB}{0}&\multicolumn{1}{cB}{0}&\multicolumn{1}{cB}{1}\tabularnewline\hrulethick
\end{tabular}
}%
\end{center}%
\end{divisionsolutioneg}%
\begin{divisionsolutioneg}{2.3.3.7}{}{g:exercise:idp227228888}%
\par\smallskip%
\noindent\hypertarget{g:solution:idp227234264-main}{}\begin{center}%
{\tabularfont%
\begin{tabular}{Bccccc}\hrulethick
\multicolumn{1}{BcB}{\(p\)}&\multicolumn{1}{cB}{\(q\)}&\multicolumn{1}{cB}{\(r\)}&\multicolumn{1}{cB}{\(p{\vee} q\)}&\multicolumn{1}{cB}{\((p{\vee} q){\wedge} r\)}\tabularnewline\hrulemedium
\multicolumn{1}{BcB}{0}&\multicolumn{1}{cB}{0}&\multicolumn{1}{cB}{0}&\multicolumn{1}{cB}{0}&\multicolumn{1}{cB}{0}\tabularnewline[0pt]
\multicolumn{1}{BcB}{0}&\multicolumn{1}{cB}{0}&\multicolumn{1}{cB}{1}&\multicolumn{1}{cB}{0}&\multicolumn{1}{cB}{0}\tabularnewline[0pt]
\multicolumn{1}{BcB}{0}&\multicolumn{1}{cB}{1}&\multicolumn{1}{cB}{0}&\multicolumn{1}{cB}{1}&\multicolumn{1}{cB}{0}\tabularnewline[0pt]
\multicolumn{1}{BcB}{0}&\multicolumn{1}{cB}{1}&\multicolumn{1}{cB}{1}&\multicolumn{1}{cB}{1}&\multicolumn{1}{cB}{1}\tabularnewline[0pt]
\multicolumn{1}{BcB}{1}&\multicolumn{1}{cB}{0}&\multicolumn{1}{cB}{0}&\multicolumn{1}{cB}{1}&\multicolumn{1}{cB}{0}\tabularnewline[0pt]
\multicolumn{1}{BcB}{1}&\multicolumn{1}{cB}{0}&\multicolumn{1}{cB}{1}&\multicolumn{1}{cB}{1}&\multicolumn{1}{cB}{1}\tabularnewline[0pt]
\multicolumn{1}{BcB}{1}&\multicolumn{1}{cB}{1}&\multicolumn{1}{cB}{0}&\multicolumn{1}{cB}{1}&\multicolumn{1}{cB}{0}\tabularnewline[0pt]
\multicolumn{1}{BcB}{1}&\multicolumn{1}{cB}{1}&\multicolumn{1}{cB}{1}&\multicolumn{1}{cB}{1}&\multicolumn{1}{cB}{1}\tabularnewline\hrulethick
\end{tabular}
}%
\end{center}%
\end{divisionsolutioneg}%
\begin{divisionsolutioneg}{2.3.3.8}{}{g:exercise:idp227259736}%
\par\smallskip%
\noindent\hypertarget{g:solution:idp227254616-main}{}\begin{center}%
{\tabularfont%
\begin{tabular}{Bccccc}\hrulethick
\multicolumn{1}{BcB}{\(p\)}&\multicolumn{1}{cB}{\(q\)}&\multicolumn{1}{cB}{\(r\)}&\multicolumn{1}{cB}{\(\sim\!{r}\)}&\multicolumn{1}{cB}{\(p\,{\vee}\,q\sim\!{r}\)}\tabularnewline\hrulemedium
\multicolumn{1}{BcB}{0}&\multicolumn{1}{cB}{0}&\multicolumn{1}{cB}{0}&\multicolumn{1}{cB}{1}&\multicolumn{1}{cB}{1}\tabularnewline[0pt]
\multicolumn{1}{BcB}{0}&\multicolumn{1}{cB}{0}&\multicolumn{1}{cB}{1}&\multicolumn{1}{cB}{0}&\multicolumn{1}{cB}{0}\tabularnewline[0pt]
\multicolumn{1}{BcB}{0}&\multicolumn{1}{cB}{1}&\multicolumn{1}{cB}{0}&\multicolumn{1}{cB}{1}&\multicolumn{1}{cB}{1}\tabularnewline[0pt]
\multicolumn{1}{BcB}{0}&\multicolumn{1}{cB}{1}&\multicolumn{1}{cB}{1}&\multicolumn{1}{cB}{0}&\multicolumn{1}{cB}{1}\tabularnewline[0pt]
\multicolumn{1}{BcB}{1}&\multicolumn{1}{cB}{0}&\multicolumn{1}{cB}{0}&\multicolumn{1}{cB}{1}&\multicolumn{1}{cB}{1}\tabularnewline[0pt]
\multicolumn{1}{BcB}{1}&\multicolumn{1}{cB}{0}&\multicolumn{1}{cB}{1}&\multicolumn{1}{cB}{0}&\multicolumn{1}{cB}{1}\tabularnewline[0pt]
\multicolumn{1}{BcB}{1}&\multicolumn{1}{cB}{1}&\multicolumn{1}{cB}{0}&\multicolumn{1}{cB}{1}&\multicolumn{1}{cB}{1}\tabularnewline[0pt]
\multicolumn{1}{BcB}{1}&\multicolumn{1}{cB}{1}&\multicolumn{1}{cB}{1}&\multicolumn{1}{cB}{0}&\multicolumn{1}{cB}{1}\tabularnewline\hrulethick
\end{tabular}
}%
\end{center}%
\end{divisionsolutioneg}%
\begin{divisionsolutioneg}{2.3.3.9}{}{g:exercise:idp227280088}%
\par\smallskip%
\noindent\hypertarget{g:solution:idp227281368-main}{}\begin{center}%
{\tabularfont%
\begin{tabular}{Bccccccc}\hrulethick
\multicolumn{1}{BcB}{\(p\)}&\multicolumn{1}{cB}{\(q\)}&\multicolumn{1}{cB}{\(\sim\!{q}\)}&\multicolumn{1}{cB}{\(p{\wedge} q\)}&\multicolumn{1}{cB}{\(p\,{\vee}\,\sim\!{q}\)}&\multicolumn{1}{cB}{\(\sim\!(p\,{\vee}\,\sim\!{q})\)}&\multicolumn{1}{cB}{\((p{\wedge} q)\,{\vee}\,\sim\!(p\,{\vee}\,\sim\!{q})\)}\tabularnewline\hrulemedium
\multicolumn{1}{BcB}{0}&\multicolumn{1}{cB}{0}&\multicolumn{1}{cB}{1}&\multicolumn{1}{cB}{0}&\multicolumn{1}{cB}{1}&\multicolumn{1}{cB}{0}&\multicolumn{1}{cB}{0}\tabularnewline[0pt]
\multicolumn{1}{BcB}{0}&\multicolumn{1}{cB}{1}&\multicolumn{1}{cB}{0}&\multicolumn{1}{cB}{0}&\multicolumn{1}{cB}{0}&\multicolumn{1}{cB}{1}&\multicolumn{1}{cB}{1}\tabularnewline[0pt]
\multicolumn{1}{BcB}{1}&\multicolumn{1}{cB}{0}&\multicolumn{1}{cB}{1}&\multicolumn{1}{cB}{0}&\multicolumn{1}{cB}{1}&\multicolumn{1}{cB}{0}&\multicolumn{1}{cB}{0}\tabularnewline[0pt]
\multicolumn{1}{BcB}{1}&\multicolumn{1}{cB}{1}&\multicolumn{1}{cB}{0}&\multicolumn{1}{cB}{1}&\multicolumn{1}{cB}{1}&\multicolumn{1}{cB}{0}&\multicolumn{1}{cB}{1}\tabularnewline\hrulethick
\end{tabular}
}%
\end{center}%
\end{divisionsolutioneg}%
\begin{divisionsolutioneg}{2.3.3.10}{}{g:exercise:idp227298520}%
\par\smallskip%
\noindent\hypertarget{g:solution:idp227299928-main}{}\begin{center}%
{\tabularfont%
\begin{tabular}{Bccccccc}\hrulethick
\multicolumn{1}{BcB}{\(p\)}&\multicolumn{1}{cB}{\(q\)}&\multicolumn{1}{cB}{\(\sim\!{p}\)}&\multicolumn{1}{cB}{\(\sim\!{q}\)}&\multicolumn{1}{cB}{\(\sim\!{p}\,{\vee}\,\sim\!{q}\)}&\multicolumn{1}{cB}{\(\sim\!{p}\,{\vee}\,q\)}&\multicolumn{1}{cB}{\((\sim\!{p}\,{\vee}\,\sim\!{q})\,{\wedge}\,(\sim\!{p}\,{\vee}\,q)\)}\tabularnewline\hrulemedium
\multicolumn{1}{BcB}{0}&\multicolumn{1}{cB}{0}&\multicolumn{1}{cB}{1}&\multicolumn{1}{cB}{1}&\multicolumn{1}{cB}{1}&\multicolumn{1}{cB}{1}&\multicolumn{1}{cB}{1}\tabularnewline[0pt]
\multicolumn{1}{BcB}{0}&\multicolumn{1}{cB}{1}&\multicolumn{1}{cB}{1}&\multicolumn{1}{cB}{0}&\multicolumn{1}{cB}{1}&\multicolumn{1}{cB}{1}&\multicolumn{1}{cB}{1}\tabularnewline[0pt]
\multicolumn{1}{BcB}{1}&\multicolumn{1}{cB}{0}&\multicolumn{1}{cB}{0}&\multicolumn{1}{cB}{1}&\multicolumn{1}{cB}{1}&\multicolumn{1}{cB}{0}&\multicolumn{1}{cB}{0}\tabularnewline[0pt]
\multicolumn{1}{BcB}{1}&\multicolumn{1}{cB}{1}&\multicolumn{1}{cB}{0}&\multicolumn{1}{cB}{0}&\multicolumn{1}{cB}{0}&\multicolumn{1}{cB}{1}&\multicolumn{1}{cB}{0}\tabularnewline\hrulethick
\end{tabular}
}%
\end{center}%
\end{divisionsolutioneg}%
\end{exercisegroup}
\par\medskip\noindent
\begin{exercisegroup}
\begin{divisionsolutioneg}{2.3.3.11}{}{g:exercise:idp227312344}%
\par\smallskip%
\noindent\hypertarget{g:solution:idp227317720-main}{}\begin{center}%
{\tabularfont%
\begin{tabular}{Bccccccc}\hrulethick
\multicolumn{1}{BcB}{\(p\)}&\multicolumn{1}{cB}{\(q\)}&\multicolumn{1}{cB}{\(p{\wedge} q\)}&\multicolumn{1}{cB}{\(\sim\!(p{\wedge} q)\)}&\multicolumn{1}{cB}{\(\sim\!{p}\)}&\multicolumn{1}{cB}{\(\sim\!{q}\)}&\multicolumn{1}{cB}{\(\sim\!{p}\,{\wedge}\,\sim\!{q}\)}\tabularnewline\hrulemedium
\multicolumn{1}{BcB}{0}&\multicolumn{1}{cB}{0}&\multicolumn{1}{cB}{0}&\multicolumn{1}{cB}{1}&\multicolumn{1}{cB}{1}&\multicolumn{1}{cB}{1}&\multicolumn{1}{cB}{1}\tabularnewline[0pt]
\multicolumn{1}{BcB}{0}&\multicolumn{1}{cB}{1}&\multicolumn{1}{cB}{0}&\multicolumn{1}{cB}{1}&\multicolumn{1}{cB}{1}&\multicolumn{1}{cB}{0}&\multicolumn{1}{cB}{0}\tabularnewline[0pt]
\multicolumn{1}{BcB}{1}&\multicolumn{1}{cB}{0}&\multicolumn{1}{cB}{0}&\multicolumn{1}{cB}{1}&\multicolumn{1}{cB}{0}&\multicolumn{1}{cB}{1}&\multicolumn{1}{cB}{0}\tabularnewline[0pt]
\multicolumn{1}{BcB}{1}&\multicolumn{1}{cB}{1}&\multicolumn{1}{cB}{1}&\multicolumn{1}{cB}{0}&\multicolumn{1}{cB}{0}&\multicolumn{1}{cB}{0}&\multicolumn{1}{cB}{0}\tabularnewline\hrulethick
\end{tabular}
}%
\end{center}%
 No, because the 4th and the 7th columns are not the same.\end{divisionsolutioneg}%
\begin{divisionsolutioneg}{2.3.3.12}{}{g:exercise:idp227342552}%
\par\smallskip%
\noindent\hypertarget{g:solution:idp227339480-main}{}\begin{center}%
{\tabularfont%
\begin{tabular}{Bccccccc}\hrulethick
\multicolumn{1}{BcB}{\(p\)}&\multicolumn{1}{cB}{\(q\)}&\multicolumn{1}{cB}{\(p{\vee} q\)}&\multicolumn{1}{cB}{\(\sim\!(p{\vee} q)\)}&\multicolumn{1}{cB}{\(\sim\!{p}\)}&\multicolumn{1}{cB}{\(\sim\!{q}\)}&\multicolumn{1}{cB}{\(\sim\!{p}\,{\wedge}\,\sim\!{q}\)}\tabularnewline\hrulemedium
\multicolumn{1}{BcB}{0}&\multicolumn{1}{cB}{0}&\multicolumn{1}{cB}{0}&\multicolumn{1}{cB}{1}&\multicolumn{1}{cB}{1}&\multicolumn{1}{cB}{1}&\multicolumn{1}{cB}{1}\tabularnewline[0pt]
\multicolumn{1}{BcB}{0}&\multicolumn{1}{cB}{1}&\multicolumn{1}{cB}{1}&\multicolumn{1}{cB}{0}&\multicolumn{1}{cB}{1}&\multicolumn{1}{cB}{0}&\multicolumn{1}{cB}{0}\tabularnewline[0pt]
\multicolumn{1}{BcB}{1}&\multicolumn{1}{cB}{0}&\multicolumn{1}{cB}{1}&\multicolumn{1}{cB}{0}&\multicolumn{1}{cB}{0}&\multicolumn{1}{cB}{1}&\multicolumn{1}{cB}{0}\tabularnewline[0pt]
\multicolumn{1}{BcB}{1}&\multicolumn{1}{cB}{1}&\multicolumn{1}{cB}{1}&\multicolumn{1}{cB}{0}&\multicolumn{1}{cB}{0}&\multicolumn{1}{cB}{0}&\multicolumn{1}{cB}{0}\tabularnewline\hrulethick
\end{tabular}
}%
\end{center}%
 Yes, because the 4th and the 7th columns (the columns corresponding to the two statements of interest) are identical.\end{divisionsolutioneg}%
\begin{divisionsolutioneg}{2.3.3.13}{}{g:exercise:idp227351128}%
\par\smallskip%
\noindent\hypertarget{g:solution:idp227356248-main}{}\begin{center}%
{\tabularfont%
\begin{tabular}{Bcccccc}\hrulethick
\multicolumn{1}{BcB}{\(p\)}&\multicolumn{1}{cB}{\(q\)}&\multicolumn{1}{cB}{\(p{\oplus} q\)}&\multicolumn{1}{cB}{\(\sim\!{p}\)}&\multicolumn{1}{cB}{\(\sim\!{q}\)}&\multicolumn{1}{cB}{\(\sim\!{p}\,{\oplus}\,\sim\!{q}\)}\tabularnewline\hrulemedium
\multicolumn{1}{BcB}{0}&\multicolumn{1}{cB}{0}&\multicolumn{1}{cB}{0}&\multicolumn{1}{cB}{1}&\multicolumn{1}{cB}{1}&\multicolumn{1}{cB}{0}\tabularnewline[0pt]
\multicolumn{1}{BcB}{0}&\multicolumn{1}{cB}{1}&\multicolumn{1}{cB}{1}&\multicolumn{1}{cB}{1}&\multicolumn{1}{cB}{0}&\multicolumn{1}{cB}{1}\tabularnewline[0pt]
\multicolumn{1}{BcB}{1}&\multicolumn{1}{cB}{0}&\multicolumn{1}{cB}{1}&\multicolumn{1}{cB}{0}&\multicolumn{1}{cB}{1}&\multicolumn{1}{cB}{1}\tabularnewline[0pt]
\multicolumn{1}{BcB}{1}&\multicolumn{1}{cB}{1}&\multicolumn{1}{cB}{0}&\multicolumn{1}{cB}{0}&\multicolumn{1}{cB}{0}&\multicolumn{1}{cB}{0}\tabularnewline\hrulethick
\end{tabular}
}%
\end{center}%
 Yes, because the columns corresponding to the statements of interest (the 3rd and the 6th) are identical.\end{divisionsolutioneg}%
\begin{divisionsolutioneg}{2.3.3.14}{}{g:exercise:idp227370840}%
\par\smallskip%
\noindent\hypertarget{g:solution:idp227368664-main}{}\begin{center}%
{\tabularfont%
\begin{tabular}{Bccccccc}\hrulethick
\multicolumn{1}{BcB}{\(p\)}&\multicolumn{1}{cB}{\(q\)}&\multicolumn{1}{cB}{\(r\)}&\multicolumn{1}{cB}{\(q{\wedge} r\)}&\multicolumn{1}{cB}{\(p{\vee}(q{\wedge} r)\)}&\multicolumn{1}{cB}{\(p{\vee} q\)}&\multicolumn{1}{cB}{\((p{\vee} q){\wedge} r\)}\tabularnewline\hrulemedium
\multicolumn{1}{BcB}{0}&\multicolumn{1}{cB}{0}&\multicolumn{1}{cB}{0}&\multicolumn{1}{cB}{0}&\multicolumn{1}{cB}{0}&\multicolumn{1}{cB}{0}&\multicolumn{1}{cB}{0}\tabularnewline[0pt]
\multicolumn{1}{BcB}{0}&\multicolumn{1}{cB}{0}&\multicolumn{1}{cB}{1}&\multicolumn{1}{cB}{0}&\multicolumn{1}{cB}{0}&\multicolumn{1}{cB}{0}&\multicolumn{1}{cB}{0}\tabularnewline[0pt]
\multicolumn{1}{BcB}{0}&\multicolumn{1}{cB}{1}&\multicolumn{1}{cB}{0}&\multicolumn{1}{cB}{0}&\multicolumn{1}{cB}{0}&\multicolumn{1}{cB}{1}&\multicolumn{1}{cB}{0}\tabularnewline[0pt]
\multicolumn{1}{BcB}{0}&\multicolumn{1}{cB}{1}&\multicolumn{1}{cB}{1}&\multicolumn{1}{cB}{1}&\multicolumn{1}{cB}{1}&\multicolumn{1}{cB}{1}&\multicolumn{1}{cB}{1}\tabularnewline[0pt]
\multicolumn{1}{BcB}{1}&\multicolumn{1}{cB}{0}&\multicolumn{1}{cB}{0}&\multicolumn{1}{cB}{0}&\multicolumn{1}{cB}{1}&\multicolumn{1}{cB}{1}&\multicolumn{1}{cB}{0}\tabularnewline[0pt]
\multicolumn{1}{BcB}{1}&\multicolumn{1}{cB}{0}&\multicolumn{1}{cB}{1}&\multicolumn{1}{cB}{0}&\multicolumn{1}{cB}{1}&\multicolumn{1}{cB}{1}&\multicolumn{1}{cB}{1}\tabularnewline[0pt]
\multicolumn{1}{BcB}{1}&\multicolumn{1}{cB}{1}&\multicolumn{1}{cB}{0}&\multicolumn{1}{cB}{0}&\multicolumn{1}{cB}{1}&\multicolumn{1}{cB}{1}&\multicolumn{1}{cB}{0}\tabularnewline[0pt]
\multicolumn{1}{BcB}{1}&\multicolumn{1}{cB}{1}&\multicolumn{1}{cB}{1}&\multicolumn{1}{cB}{1}&\multicolumn{1}{cB}{1}&\multicolumn{1}{cB}{1}&\multicolumn{1}{cB}{1}\tabularnewline\hrulethick
\end{tabular}
}%
\end{center}%
 No, because the corresponding columns (columns 5 and 7) are not identical.\end{divisionsolutioneg}%
\begin{divisionsolutioneg}{2.3.3.15}{}{g:exercise:idp227404760}%
\par\smallskip%
\noindent\hypertarget{g:solution:idp227404888-main}{}\begin{center}%
{\tabularfont%
\begin{tabular}{Bcccc}\hrulethick
\multicolumn{1}{BcB}{\(p\)}&\multicolumn{1}{cB}{\(q\)}&\multicolumn{1}{cB}{\(p{\wedge} q\)}&\multicolumn{1}{cB}{\(p{\vee}(p{\wedge} q)\)}\tabularnewline\hrulemedium
\multicolumn{1}{BcB}{0}&\multicolumn{1}{cB}{0}&\multicolumn{1}{cB}{0}&\multicolumn{1}{cB}{0}\tabularnewline[0pt]
\multicolumn{1}{BcB}{0}&\multicolumn{1}{cB}{1}&\multicolumn{1}{cB}{0}&\multicolumn{1}{cB}{0}\tabularnewline[0pt]
\multicolumn{1}{BcB}{1}&\multicolumn{1}{cB}{0}&\multicolumn{1}{cB}{0}&\multicolumn{1}{cB}{1}\tabularnewline[0pt]
\multicolumn{1}{BcB}{1}&\multicolumn{1}{cB}{1}&\multicolumn{1}{cB}{1}&\multicolumn{1}{cB}{1}\tabularnewline\hrulethick
\end{tabular}
}%
\end{center}%
 Yes, because the corresponding columns (1 and 4) are identical.\end{divisionsolutioneg}%
\begin{divisionsolutioneg}{2.3.3.16}{}{x:exercise:exer-logic-commutative}%
\par\smallskip%
\noindent\hypertarget{g:solution:idp227424344-main}{}\begin{center}%
{\tabularfont%
\begin{tabular}{Bccccccc}\hrulethick
\multicolumn{1}{BcB}{\(p\)}&\multicolumn{1}{cB}{\(q\)}&\multicolumn{1}{cB}{\(r\)}&\multicolumn{1}{cB}{\(p{\vee} q\)}&\multicolumn{1}{cB}{\((p{\vee} q){\vee} r\)}&\multicolumn{1}{cB}{\(q{\vee} r\)}&\multicolumn{1}{cB}{\(p{\vee} (q{\vee} r)\)}\tabularnewline\hrulemedium
\multicolumn{1}{BcB}{0}&\multicolumn{1}{cB}{0}&\multicolumn{1}{cB}{0}&\multicolumn{1}{cB}{0}&\multicolumn{1}{cB}{0}&\multicolumn{1}{cB}{0}&\multicolumn{1}{cB}{0}\tabularnewline[0pt]
\multicolumn{1}{BcB}{0}&\multicolumn{1}{cB}{0}&\multicolumn{1}{cB}{1}&\multicolumn{1}{cB}{0}&\multicolumn{1}{cB}{1}&\multicolumn{1}{cB}{1}&\multicolumn{1}{cB}{1}\tabularnewline[0pt]
\multicolumn{1}{BcB}{0}&\multicolumn{1}{cB}{1}&\multicolumn{1}{cB}{0}&\multicolumn{1}{cB}{1}&\multicolumn{1}{cB}{1}&\multicolumn{1}{cB}{1}&\multicolumn{1}{cB}{1}\tabularnewline[0pt]
\multicolumn{1}{BcB}{0}&\multicolumn{1}{cB}{1}&\multicolumn{1}{cB}{1}&\multicolumn{1}{cB}{1}&\multicolumn{1}{cB}{1}&\multicolumn{1}{cB}{1}&\multicolumn{1}{cB}{1}\tabularnewline[0pt]
\multicolumn{1}{BcB}{1}&\multicolumn{1}{cB}{0}&\multicolumn{1}{cB}{0}&\multicolumn{1}{cB}{1}&\multicolumn{1}{cB}{1}&\multicolumn{1}{cB}{0}&\multicolumn{1}{cB}{1}\tabularnewline[0pt]
\multicolumn{1}{BcB}{1}&\multicolumn{1}{cB}{0}&\multicolumn{1}{cB}{1}&\multicolumn{1}{cB}{1}&\multicolumn{1}{cB}{1}&\multicolumn{1}{cB}{1}&\multicolumn{1}{cB}{1}\tabularnewline[0pt]
\multicolumn{1}{BcB}{1}&\multicolumn{1}{cB}{1}&\multicolumn{1}{cB}{0}&\multicolumn{1}{cB}{1}&\multicolumn{1}{cB}{1}&\multicolumn{1}{cB}{1}&\multicolumn{1}{cB}{1}\tabularnewline[0pt]
\multicolumn{1}{BcB}{1}&\multicolumn{1}{cB}{1}&\multicolumn{1}{cB}{1}&\multicolumn{1}{cB}{1}&\multicolumn{1}{cB}{1}&\multicolumn{1}{cB}{1}&\multicolumn{1}{cB}{1}\tabularnewline\hrulethick
\end{tabular}
}%
\end{center}%
 Yes, because the corresponding columns are identical (5th and 7th).\end{divisionsolutioneg}%
\begin{divisionsolutioneg}{2.3.3.17}{}{g:exercise:idp227817304}%
\par\smallskip%
\noindent\hypertarget{g:solution:idp227815128-main}{}\begin{center}%
{\tabularfont%
\begin{tabular}{Bcccccccc}\hrulethick
\multicolumn{1}{BcB}{\(p\)}&\multicolumn{1}{cB}{\(q\)}&\multicolumn{1}{cB}{\(p{\oplus} q\)}&\multicolumn{1}{cB}{\(\sim\!{p}\)}&\multicolumn{1}{cB}{\(\sim\!{q}\)}&\multicolumn{1}{cB}{\(p{\wedge} q\)}&\multicolumn{1}{cB}{\(\sim\!{p}\,{\wedge}\,\sim\!{q}\)}&\multicolumn{1}{cB}{\((p{\wedge} q){\vee} (\sim\!{p}\,{\wedge}\,\sim\!{q})\)}\tabularnewline\hrulemedium
\multicolumn{1}{BcB}{0}&\multicolumn{1}{cB}{0}&\multicolumn{1}{cB}{0}&\multicolumn{1}{cB}{1}&\multicolumn{1}{cB}{1}&\multicolumn{1}{cB}{0}&\multicolumn{1}{cB}{1}&\multicolumn{1}{cB}{1}\tabularnewline[0pt]
\multicolumn{1}{BcB}{0}&\multicolumn{1}{cB}{1}&\multicolumn{1}{cB}{1}&\multicolumn{1}{cB}{1}&\multicolumn{1}{cB}{0}&\multicolumn{1}{cB}{0}&\multicolumn{1}{cB}{0}&\multicolumn{1}{cB}{0}\tabularnewline[0pt]
\multicolumn{1}{BcB}{1}&\multicolumn{1}{cB}{0}&\multicolumn{1}{cB}{1}&\multicolumn{1}{cB}{0}&\multicolumn{1}{cB}{1}&\multicolumn{1}{cB}{0}&\multicolumn{1}{cB}{0}&\multicolumn{1}{cB}{0}\tabularnewline[0pt]
\multicolumn{1}{BcB}{1}&\multicolumn{1}{cB}{1}&\multicolumn{1}{cB}{0}&\multicolumn{1}{cB}{0}&\multicolumn{1}{cB}{0}&\multicolumn{1}{cB}{1}&\multicolumn{1}{cB}{0}&\multicolumn{1}{cB}{1}\tabularnewline\hrulethick
\end{tabular}
}%
\end{center}%
 No, because the corresponding columns (3 and 8) are not identical.  Can you see how the two statements are related?\end{divisionsolutioneg}%
\end{exercisegroup}
\par\medskip\noindent
\begin{exercisegroup}
\begin{divisionsolutioneg}{2.3.3.18}{}{g:exercise:idp227834072}%
\par\smallskip%
\noindent\hypertarget{g:solution:idp227828952-main}{}\begin{center}%
{\tabularfont%
\begin{tabular}{Bccc}\hrulethick
\multicolumn{1}{BcB}{\(p\)}&\multicolumn{1}{cB}{\(p\)}&\multicolumn{1}{cB}{\(p{\wedge} p\)}\tabularnewline\hrulemedium
\multicolumn{1}{BcB}{0}&\multicolumn{1}{cB}{0}&\multicolumn{1}{cB}{0}\tabularnewline[0pt]
\multicolumn{1}{BcB}{1}&\multicolumn{1}{cB}{1}&\multicolumn{1}{cB}{1}\tabularnewline\hrulethick
\end{tabular}
}%
\end{center}%
 So \(p{\wedge} p\Leftrightarrow p\).  (You can omit the second column of this table, if you wish.)\end{divisionsolutioneg}%
\begin{divisionsolutioneg}{2.3.3.19}{}{g:exercise:idp227837784}%
\par\smallskip%
\noindent\hypertarget{g:solution:idp227838040-main}{}\begin{center}%
{\tabularfont%
\begin{tabular}{Bccc}\hrulethick
\multicolumn{1}{BcB}{\(p\)}&\multicolumn{1}{cB}{\(\sim\!{p}\)}&\multicolumn{1}{cB}{\(p\,{\vee}\sim\!{p}\)}\tabularnewline\hrulemedium
\multicolumn{1}{BcB}{0}&\multicolumn{1}{cB}{1}&\multicolumn{1}{cB}{1}\tabularnewline[0pt]
\multicolumn{1}{BcB}{1}&\multicolumn{1}{cB}{0}&\multicolumn{1}{cB}{1}\tabularnewline\hrulethick
\end{tabular}
}%
\end{center}%
 This statement is always true.  It is logically equivalent to 1.\end{divisionsolutioneg}%
\begin{divisionsolutioneg}{2.3.3.20}{}{g:exercise:idp227849176}%
\par\smallskip%
\noindent\hypertarget{g:solution:idp227844824-main}{}\begin{center}%
{\tabularfont%
\begin{tabular}{Bccc}\hrulethick
\multicolumn{1}{BcB}{\(p\)}&\multicolumn{1}{cB}{\(0\)}&\multicolumn{1}{cB}{\(p{\wedge} 0\)}\tabularnewline\hrulemedium
\multicolumn{1}{BcB}{0}&\multicolumn{1}{cB}{0}&\multicolumn{1}{cB}{0}\tabularnewline[0pt]
\multicolumn{1}{BcB}{1}&\multicolumn{1}{cB}{0}&\multicolumn{1}{cB}{0}\tabularnewline\hrulethick
\end{tabular}
}%
\end{center}%
 This statement is always false.  It is logically equivalent to 0.\end{divisionsolutioneg}%
\begin{divisionsolutioneg}{2.3.3.21}{}{g:exercise:idp227855320}%
\par\smallskip%
\noindent\hypertarget{g:solution:idp227852248-main}{}\begin{center}%
{\tabularfont%
\begin{tabular}{Bccc}\hrulethick
\multicolumn{1}{BcB}{\(p\)}&\multicolumn{1}{cB}{\(\sim\!{p}\)}&\multicolumn{1}{cB}{\(\sim\!{p}\,{\oplus} p\)}\tabularnewline\hrulemedium
\multicolumn{1}{BcB}{0}&\multicolumn{1}{cB}{1}&\multicolumn{1}{cB}{1}\tabularnewline[0pt]
\multicolumn{1}{BcB}{1}&\multicolumn{1}{cB}{0}&\multicolumn{1}{cB}{1}\tabularnewline\hrulethick
\end{tabular}
}%
\end{center}%
 This expression is logically equivalent to 1.\end{divisionsolutioneg}%
\begin{divisionsolutioneg}{2.3.3.22}{}{g:exercise:idp227864792}%
\par\smallskip%
\noindent\hypertarget{g:solution:idp227864408-main}{}\begin{center}%
{\tabularfont%
\begin{tabular}{Bcccccc}\hrulethick
\multicolumn{1}{BcB}{\(p\)}&\multicolumn{1}{cB}{\(q\)}&\multicolumn{1}{cB}{\(p{\oplus} q\)}&\multicolumn{1}{cB}{\(\sim\!{q}\)}&\multicolumn{1}{cB}{\(p\,{\oplus}\sim\!{q}\)}&\multicolumn{1}{cB}{\((p{\oplus} q){\wedge}(p\,{\oplus}\,\sim\!{q})\)}\tabularnewline\hrulemedium
\multicolumn{1}{BcB}{0}&\multicolumn{1}{cB}{0}&\multicolumn{1}{cB}{0}&\multicolumn{1}{cB}{1}&\multicolumn{1}{cB}{1}&\multicolumn{1}{cB}{0}\tabularnewline[0pt]
\multicolumn{1}{BcB}{0}&\multicolumn{1}{cB}{1}&\multicolumn{1}{cB}{1}&\multicolumn{1}{cB}{0}&\multicolumn{1}{cB}{0}&\multicolumn{1}{cB}{0}\tabularnewline[0pt]
\multicolumn{1}{BcB}{1}&\multicolumn{1}{cB}{0}&\multicolumn{1}{cB}{1}&\multicolumn{1}{cB}{1}&\multicolumn{1}{cB}{0}&\multicolumn{1}{cB}{0}\tabularnewline[0pt]
\multicolumn{1}{BcB}{1}&\multicolumn{1}{cB}{1}&\multicolumn{1}{cB}{0}&\multicolumn{1}{cB}{0}&\multicolumn{1}{cB}{1}&\multicolumn{1}{cB}{0}\tabularnewline\hrulethick
\end{tabular}
}%
\end{center}%
 The expression simplifies to 0.\end{divisionsolutioneg}%
\begin{divisionsolutioneg}{2.3.3.23}{}{g:exercise:idp227881048}%
\par\smallskip%
\noindent\hypertarget{g:solution:idp227879256-main}{}\begin{center}%
{\tabularfont%
\begin{tabular}{Bcccc}\hrulethick
\multicolumn{1}{BcB}{\(p\)}&\multicolumn{1}{cB}{\(q\)}&\multicolumn{1}{cB}{\(p{\wedge} q\)}&\multicolumn{1}{cB}{\(p{\vee}(p{\wedge} q)\)}\tabularnewline\hrulemedium
\multicolumn{1}{BcB}{0}&\multicolumn{1}{cB}{0}&\multicolumn{1}{cB}{0}&\multicolumn{1}{cB}{0}\tabularnewline[0pt]
\multicolumn{1}{BcB}{0}&\multicolumn{1}{cB}{1}&\multicolumn{1}{cB}{0}&\multicolumn{1}{cB}{0}\tabularnewline[0pt]
\multicolumn{1}{BcB}{1}&\multicolumn{1}{cB}{0}&\multicolumn{1}{cB}{0}&\multicolumn{1}{cB}{1}\tabularnewline[0pt]
\multicolumn{1}{BcB}{1}&\multicolumn{1}{cB}{1}&\multicolumn{1}{cB}{1}&\multicolumn{1}{cB}{1}\tabularnewline\hrulethick
\end{tabular}
}%
\end{center}%
 Comparing the first and last columns, we see that \(p{\vee}(p{\wedge} q)\Leftrightarrow p\)\end{divisionsolutioneg}%
\begin{divisionsolutioneg}{2.3.3.24}{}{g:exercise:idp227889240}%
\par\smallskip%
\noindent\hypertarget{g:solution:idp227891160-main}{}\begin{center}%
{\tabularfont%
\begin{tabular}{Bcccc}\hrulethick
\multicolumn{1}{BcB}{\(p\)}&\multicolumn{1}{cB}{\(q\)}&\multicolumn{1}{cB}{\(p{\vee} q\)}&\multicolumn{1}{cB}{\(q{\wedge}(p{\vee} q)\)}\tabularnewline\hrulemedium
\multicolumn{1}{BcB}{0}&\multicolumn{1}{cB}{0}&\multicolumn{1}{cB}{0}&\multicolumn{1}{cB}{0}\tabularnewline[0pt]
\multicolumn{1}{BcB}{0}&\multicolumn{1}{cB}{1}&\multicolumn{1}{cB}{1}&\multicolumn{1}{cB}{1}\tabularnewline[0pt]
\multicolumn{1}{BcB}{1}&\multicolumn{1}{cB}{0}&\multicolumn{1}{cB}{1}&\multicolumn{1}{cB}{0}\tabularnewline[0pt]
\multicolumn{1}{BcB}{1}&\multicolumn{1}{cB}{1}&\multicolumn{1}{cB}{1}&\multicolumn{1}{cB}{1}\tabularnewline\hrulethick
\end{tabular}
}%
\end{center}%
 Comparing columns 2 and 4, we see that \(q{\wedge}(p{\vee} q)\Leftrightarrow q\)\end{divisionsolutioneg}%
\begin{divisionsolutioneg}{2.3.3.25}{}{g:exercise:idp227902040}%
\par\smallskip%
\noindent\hypertarget{g:solution:idp227906776-main}{}\begin{center}%
{\tabularfont%
\begin{tabular}{Bccccc}\hrulethick
\multicolumn{1}{BcB}{\(p\)}&\multicolumn{1}{cB}{\(q\)}&\multicolumn{1}{cB}{\(\sim\!{p}\)}&\multicolumn{1}{cB}{\(\sim\!{p}\,{\vee}\,q\)}&\multicolumn{1}{cB}{\(p{\wedge}(\sim\!{p}\,{\vee}\,q)\)}\tabularnewline\hrulemedium
\multicolumn{1}{BcB}{0}&\multicolumn{1}{cB}{0}&\multicolumn{1}{cB}{1}&\multicolumn{1}{cB}{1}&\multicolumn{1}{cB}{0}\tabularnewline[0pt]
\multicolumn{1}{BcB}{0}&\multicolumn{1}{cB}{1}&\multicolumn{1}{cB}{1}&\multicolumn{1}{cB}{1}&\multicolumn{1}{cB}{0}\tabularnewline[0pt]
\multicolumn{1}{BcB}{1}&\multicolumn{1}{cB}{0}&\multicolumn{1}{cB}{0}&\multicolumn{1}{cB}{0}&\multicolumn{1}{cB}{0}\tabularnewline[0pt]
\multicolumn{1}{BcB}{1}&\multicolumn{1}{cB}{1}&\multicolumn{1}{cB}{0}&\multicolumn{1}{cB}{1}&\multicolumn{1}{cB}{1}\tabularnewline\hrulethick
\end{tabular}
}%
\end{center}%
 This statement is true only if both \(p\) and \(q\) are true.  That is, it is equivalent to \(p{\wedge} q\).\end{divisionsolutioneg}%
\begin{divisionsolutioneg}{2.3.3.26}{}{g:exercise:idp227920472}%
\par\smallskip%
\noindent\hypertarget{g:solution:idp227923032-main}{}\begin{center}%
{\tabularfont%
\begin{tabular}{Bccccc}\hrulethick
\multicolumn{1}{BcB}{\(p\)}&\multicolumn{1}{cB}{\(q\)}&\multicolumn{1}{cB}{\(\sim\!{p}\)}&\multicolumn{1}{cB}{\(\sim\!{p}
\,{\wedge}\,q\)}&\multicolumn{1}{cB}{\(p{\vee}(\sim\!{p}\,{\wedge}\,q)\)}\tabularnewline\hrulemedium
\multicolumn{1}{BcB}{0}&\multicolumn{1}{cB}{0}&\multicolumn{1}{cB}{1}&\multicolumn{1}{cB}{0}&\multicolumn{1}{cB}{0}\tabularnewline[0pt]
\multicolumn{1}{BcB}{0}&\multicolumn{1}{cB}{1}&\multicolumn{1}{cB}{1}&\multicolumn{1}{cB}{1}&\multicolumn{1}{cB}{1}\tabularnewline[0pt]
\multicolumn{1}{BcB}{1}&\multicolumn{1}{cB}{0}&\multicolumn{1}{cB}{0}&\multicolumn{1}{cB}{0}&\multicolumn{1}{cB}{1}\tabularnewline[0pt]
\multicolumn{1}{BcB}{1}&\multicolumn{1}{cB}{1}&\multicolumn{1}{cB}{0}&\multicolumn{1}{cB}{0}&\multicolumn{1}{cB}{1}\tabularnewline\hrulethick
\end{tabular}
}%
\end{center}%
 This statement is false only if both \(p\) and \(q\) are false.  It is equivalent to \(p{\vee} q\).\end{divisionsolutioneg}%
\end{exercisegroup}
\par\medskip\noindent
\end{solutions-subsection}
\end{sectionptx}
%
%
\typeout{************************************************}
\typeout{Section 2.4 Boolean Algebra}
\typeout{************************************************}
%
\begin{sectionptx}{Boolean Algebra}{}{Boolean Algebra}{}{}{x:section:sec-boolean-algebra}
%
%
\typeout{************************************************}
\typeout{Subsection 2.4.1 Logic Circuits}
\typeout{************************************************}
%
\begin{subsectionptx}{Logic Circuits}{}{Logic Circuits}{}{}{x:subsection:ssec-logic-circuits}
\begin{introduction}{}%
A logic circuit or digital circuit is an electrical circuit based on a discrete number of voltage levels, usually two. \index{logic circuit}\index{digital circuit} Two-level circuits usually have one voltage set at zero volts, and the circuit then behaves like a switch, being either \emph{on} or \emph{off}. \index{switch} A nice diagram for a switch looks like this: \begin{image}{0.35}{0.3}{0.35}%
\resizebox{\linewidth}{!}{%
\begin{tikzpicture}
  \draw[thick] (0,0) to[nos=$A$]  (4,0);
\end{tikzpicture}
}%
\end{image}%
 so that when the switch is open, as in the diagram, no current exists and the switch is \emph{off}.  When the switch closes and there's a clear path from the left side to the right side, the switch is \emph{on}.%
\par
A digital circuit then makes logical decisions, based on the input to the circuit.  The simplest logic circuits are called \emph{gates}. \index{gate} Physically, a \terminology{gate} is a transistor circuit which takes one or more voltage inputs and gives a single voltage output.%
\par
One way to represent the action of a gate is by using a truth table.  As is usual in a truth table, all possible combinations of the input voltages are given, as well as the output of the gate for each set of inputs.  Each input voltage is given a symbol, such as \(A\).  When the input signal is off, the value of \(A\) is given as 0, and when it is on, the value of \(A\) is 1.  The associated truth table then looks exactly like those we studied with logical propositions.%
\end{introduction}%
%
%
\typeout{************************************************}
\typeout{Subsubsection  \emph{and} gate}
\typeout{************************************************}
%
\begin{subsubsectionptx}{\emph{and} gate}{}{\emph{and} gate}{}{}{x:subsubsection:sssec-and-gate}
The \terminology{switch representation} of an \emph{and} gate looks like this: \index{switch representation!and} \begin{image}{0.35}{0.3}{0.35}%
\resizebox{\linewidth}{!}{%
\begin{tikzpicture}
  \draw[thick] (0,0) to[nos=$A$]  (2,0) to[nos=$B$] (4,0);
\end{tikzpicture}
}%
\end{image}%
 It is a series circuit, and both switches must be closed (switched \emph{on}) for the circuit to carry current.  You can see, then, that this is the same as \(A\) \emph{and} \(B\), since \(A\) \emph{and} \(B\) is true only when both \(A\) and \(B\) are true.%
\par
Another common representation is the \terminology{gate representation}, which looks like this: \index{gate representation!and} \begin{image}{0.35}{0.3}{0.35}%
\resizebox{\linewidth}{!}{%
\begin{circuitikz}
  \ctikzset{logic ports=ieee}
  \draw (0,0) node[and port] (myand) {}
    (myand.in 1) node[anchor=east] {$A$}
    (myand.in 2) node[anchor=east] {$B$}
    (myand.out) node[anchor=west] {$AB$};
\end{circuitikz}
}%
\end{image}%
 In symbols, we write \(A\) \emph{and} \(B\) as \(A\cdot B\) or just \(AB\).%
\end{subsubsectionptx}
%
%
\typeout{************************************************}
\typeout{Subsubsection  \emph{or} gate}
\typeout{************************************************}
%
\begin{subsubsectionptx}{\emph{or} gate}{}{\emph{or} gate}{}{}{x:subsubsection:sssec-or-gate}
The switch representation of an \emph{or} gate looks like this: \index{switch representation!or} \begin{image}{0.35}{0.3}{0.35}%
\resizebox{\linewidth}{!}{%
\begin{tikzpicture}
  \draw[thick] (0,0)--(1,0)--(1,1) to[nos=$A$]  (3,1)--(3,0)--(4,0);
  \draw[thick] (1,0)--(1,-1) to[nos=$B$] (3,-1)--(3,0);
\end{tikzpicture}
}%
\end{image}%
 It is a parallel circuit, and at least one switch must be closed (on) for the circuit to carry current (say from left to right).  You can see, then, that this is the same as \(A\) \emph{or} \(B\), since \(A\) \emph{or} \(B\) is true when either \(A\) is true or \(B\) is true, or both.%
\par
The gate representation looks like this: \index{gate representation!or} \begin{image}{0.35}{0.3}{0.35}%
\resizebox{\linewidth}{!}{%
\begin{circuitikz}
  \ctikzset{logic ports=ieee}
  \draw (0,0) node[or port] (myor) {}
    (myor.in 1) node[anchor=east] {$A$}
    (myor.in 2) node[anchor=east] {$B$}
    (myor.out) node[anchor=west] {$A+B$};
\end{circuitikz}
}%
\end{image}%
 In symbols we write \(A\) \emph{or} \(B\) as \(A+B\).%
\end{subsubsectionptx}
%
%
\typeout{************************************************}
\typeout{Subsubsection  \emph{not} gate}
\typeout{************************************************}
%
\begin{subsubsectionptx}{\emph{not} gate}{}{\emph{not} gate}{}{}{x:subsubsection:sssec-not-gate}
The \emph{not} gate, or \emph{inverter}, has the diagram \index{gate representation!not} \begin{image}{0.4}{0.2}{0.4}%
\resizebox{\linewidth}{!}{%
\begin{circuitikz}
  \ctikzset{logic ports=ieee}
  \draw (0,0) node[not port] (not1) {}
   (not1.in) node[anchor=east] {$A$}
   (not1.out) node[anchor=west] {$\overline{A}$};
\end{circuitikz}
}%
\end{image}%
 and we write \emph{not} \(A\) as \(\overline{A}\), just as in set notation.  If the negation happens in combination with another gate, we usually omit the triangle and just have a little circle to show the negation, as in the next subsection.%
\end{subsubsectionptx}
\end{subsectionptx}
%
%
\typeout{************************************************}
\typeout{Subsection 2.4.2 Gate Representations of Logic Circuits}
\typeout{************************************************}
%
\begin{subsectionptx}{Gate Representations of Logic Circuits}{}{Gate Representations of Logic Circuits}{}{}{x:subsection:ssec-gatereps-logicircs}
The gate representation of the logic circuit for \(A\overline{B}\) is then \begin{image}{0.35}{0.3}{0.35}%
\resizebox{\linewidth}{!}{%
\begin{circuitikz}
  \ctikzset{logic ports=ieee}
  \draw (0,0) node[and port] (myand) {}
   (myand.in 1) node[anchor=east] {$A$}
   (myand.in 2) node[anchor=east] {$B$}
   (myand.bin 2) node[notcirc,left] {}
   (myand.out) node[anchor=west] {$A\overline{B}$};
\end{circuitikz}
}%
\end{image}%
 with the round circle on the input \(B\) negating it, so that the two inputs to the \emph{and} gate are then \(A\) and \(\overline{B}\).%
\par
The gate representation for \(\overline{A+B}\) is \begin{image}{0.35}{0.3}{0.35}%
\resizebox{\linewidth}{!}{%
\begin{circuitikz}
  \ctikzset{logic ports=ieee}
  \draw (0,0) node[or port] (myor) {}
   (myor.in 1) node[anchor=east] {$A$}
   (myor.in 2) node[anchor=east] {$B$}
   (myor.bout) node[notcirc,right] {}
   (myor.out) node[anchor=west] {$\overline{A+B}$};
\end{circuitikz}
}%
\end{image}%
 The \emph{or} gate gives \(A+B\), and the little round circle on the output indicates the negation.%
\begin{example}{}{x:example:ex-not-notAandB}%
What is the logic circuit expression for the following gate diagram? \begin{image}{0.4}{0.2}{0.4}%
\resizebox{\linewidth}{!}{%
\begin{circuitikz}
  \ctikzset{logic ports=ieee}
  \draw (0,0) node[and port] (myand) {}
   (myand.in 1) node[anchor=east] {$A$}
   (myand.bin 1) node[notcirc,left] {}
   (myand.in 2) node[anchor=east] {$B$}
   (myand.bout) node[notcirc,right] {};
\end{circuitikz}
}%
\end{image}%
\par\smallskip%
\noindent\textbf{\blocktitlefont Answer}.\label{g:answer:idp227976408}{}\hypertarget{g:answer:idp227976408}{}\quad{}\(\overline{\overline{A}{}B}\)\end{example}
\end{subsectionptx}
%
%
\typeout{************************************************}
\typeout{Subsection 2.4.3 Boolean Algebra}
\typeout{************************************************}
%
\begin{subsectionptx}{Boolean Algebra}{}{Boolean Algebra}{}{}{x:subsection:ssec-boolean-algebra}
The symbols used for circuits, \(AB\), \(A+B\), and \(\overline{A}\), are the same symbols as used in Boolean algebra. \index{Boolean algebra} In this type of algebra, each variable (\(A\), \(B\), etc.) can take only the value 0 or 1.%
\par
Truth tables in Boolean algebra then look very similar to the truth tables that we've studied in logic. \index{truth table!for Boolean algebra}\index{Boolean algebra!truth table for} For example, the truth table showing \(AB\) and \(A+B\) is: \begin{center}%
{\tabularfont%
\begin{tabular}{Bcccc}\hrulethick
\multicolumn{1}{BcB}{\(A\)}&\multicolumn{1}{cB}{\(B\)}&\multicolumn{1}{cB}{\(AB\)}&\multicolumn{1}{cB}{\(A+B\)}\tabularnewline\hrulemedium
\multicolumn{1}{BcB}{0}&\multicolumn{1}{cB}{0}&\multicolumn{1}{cB}{0}&\multicolumn{1}{cB}{0}\tabularnewline[0pt]
\multicolumn{1}{BcB}{0}&\multicolumn{1}{cB}{1}&\multicolumn{1}{cB}{0}&\multicolumn{1}{cB}{1}\tabularnewline[0pt]
\multicolumn{1}{BcB}{1}&\multicolumn{1}{cB}{0}&\multicolumn{1}{cB}{0}&\multicolumn{1}{cB}{1}\tabularnewline[0pt]
\multicolumn{1}{BcB}{1}&\multicolumn{1}{cB}{1}&\multicolumn{1}{cB}{1}&\multicolumn{1}{cB}{1}\tabularnewline\hrulethick
\end{tabular}
}%
\end{center}%
%
\par
If you have more than one operation happening in a Boolean expression, the order of operations is very similar to the order of operations in arithmetic. \index{order of operations!Boolean algebra}\index{Boolean algebra!order of operations} For example, if you have the locical expression \(AB+C\), in arithmetic you multiply before you add.  In Boolean algebra, you perform the \emph{and} before the \emph{or}.  Negation applies to the entire variable or expression.%
\par
The order of Boolean operations is indicated in the following examples:%
\begin{itemize}[label=\textbullet]
\item{}\(AB+C\qquad\) First evaluate \(AB\), then use the result as the first variable in the \emph{or} operation with \(C\)%
\item{}\(A+BC\qquad\) First evaluate \(BC\), then use that result as the second variable in the \emph{or} operation with \(A\).%
\item{}\(\overline{A}{}B\qquad\) First evaluate the negation of \(A\), then use that result in the \emph{and} operation with \(B\).%
\item{}\(\overline{A+B}\qquad\) First evaluate \(A+B\), then negate the result.  Note the implied grouping of \(A+B\).%
\item{}\(A(B+C)\qquad\) First evaluate the grouped expression, then use the result in the \emph{and} operation with \(A\).%
\end{itemize}
%
\par
Truth tables can be used to demonstrate logical equivalence between Boolean expressions.%
\begin{example}{}{g:example:idp228005592}%
Is \(AB+C\) logically equivalent to \(A(B+C)\)?\par\smallskip%
\noindent\textbf{\blocktitlefont Answer}.\label{g:answer:idp228010456}{}\hypertarget{g:answer:idp228010456}{}\quad{}\begin{center}%
{\tabularfont%
\begin{tabular}{Bccccccc}\hrulethick
\multicolumn{1}{BcB}{\(A\)}&\multicolumn{1}{cB}{\(B\)}&\multicolumn{1}{cB}{\(C\)}&\multicolumn{1}{cB}{\(AB\)}&\multicolumn{1}{cB}{\(AB+C\)}&\multicolumn{1}{cB}{\(B+C\)}&\multicolumn{1}{cB}{\(A(B+C)\)}\tabularnewline\hrulemedium
\multicolumn{1}{BcB}{0}&\multicolumn{1}{cB}{0}&\multicolumn{1}{cB}{0}&\multicolumn{1}{cB}{0}&\multicolumn{1}{cB}{0}&\multicolumn{1}{cB}{0}&\multicolumn{1}{cB}{0}\tabularnewline[0pt]
\multicolumn{1}{BcB}{0}&\multicolumn{1}{cB}{0}&\multicolumn{1}{cB}{1}&\multicolumn{1}{cB}{0}&\multicolumn{1}{cB}{1}&\multicolumn{1}{cB}{1}&\multicolumn{1}{cB}{0}\tabularnewline[0pt]
\multicolumn{1}{BcB}{0}&\multicolumn{1}{cB}{1}&\multicolumn{1}{cB}{0}&\multicolumn{1}{cB}{0}&\multicolumn{1}{cB}{0}&\multicolumn{1}{cB}{1}&\multicolumn{1}{cB}{0}\tabularnewline[0pt]
\multicolumn{1}{BcB}{0}&\multicolumn{1}{cB}{1}&\multicolumn{1}{cB}{1}&\multicolumn{1}{cB}{0}&\multicolumn{1}{cB}{1}&\multicolumn{1}{cB}{1}&\multicolumn{1}{cB}{0}\tabularnewline[0pt]
\multicolumn{1}{BcB}{1}&\multicolumn{1}{cB}{0}&\multicolumn{1}{cB}{0}&\multicolumn{1}{cB}{0}&\multicolumn{1}{cB}{0}&\multicolumn{1}{cB}{0}&\multicolumn{1}{cB}{0}\tabularnewline[0pt]
\multicolumn{1}{BcB}{1}&\multicolumn{1}{cB}{0}&\multicolumn{1}{cB}{1}&\multicolumn{1}{cB}{0}&\multicolumn{1}{cB}{1}&\multicolumn{1}{cB}{1}&\multicolumn{1}{cB}{1}\tabularnewline[0pt]
\multicolumn{1}{BcB}{1}&\multicolumn{1}{cB}{1}&\multicolumn{1}{cB}{0}&\multicolumn{1}{cB}{1}&\multicolumn{1}{cB}{1}&\multicolumn{1}{cB}{1}&\multicolumn{1}{cB}{1}\tabularnewline[0pt]
\multicolumn{1}{BcB}{1}&\multicolumn{1}{cB}{1}&\multicolumn{1}{cB}{1}&\multicolumn{1}{cB}{1}&\multicolumn{1}{cB}{1}&\multicolumn{1}{cB}{1}&\multicolumn{1}{cB}{1}\tabularnewline\hrulethick
\end{tabular}
}%
\end{center}%
 The 5th and the 7th columns correspond to the two expressions of interest.  They are not identical, so these two expressions are not logically equivalent. \emph{Order of operations is important.}\end{example}
\end{subsectionptx}
%
%
\typeout{************************************************}
\typeout{Subsection 2.4.4 Boolean Syntax in Python}
\typeout{************************************************}
%
\begin{subsectionptx}{Boolean Syntax in Python}{}{Boolean Syntax in Python}{}{}{x:subsection:ssec-boolean-syntax-python}
Python allows you to perform logical operations on Boolean variables in the way that you would expect, as you can see in the interactive window below. (The print() statements are needed to convince the underlying interpreter to display output.  In a typical python shell, they are not needed.) \leavevmode%
\begin{sageinput}
print(True or False)
print(True and False)
print(not True)
print(not False)
print(1 and 0)
print(1 or 0)
\end{sageinput}
\begin{sageoutput}
True
False
False
True
0
1
\end{sageoutput}
%
\end{subsectionptx}
%
%
\typeout{************************************************}
\typeout{Exercises 2.4.5 Exercises}
\typeout{************************************************}
%
\begin{exercises-subsection}{Exercises}{}{Exercises}{}{}{x:exercises:exers-sec-boolean-algebra}
\par\medskip\noindent%
\textbf{Exercise Group.}\space\space%
Draw the gate representation for the following logical expressions.\begin{exercisegroup}
\begin{divisionexerciseeg}{1}{}{}{g:exercise:idp228045144}%
\(\ A+\overline{B}\)\end{divisionexerciseeg}%
\begin{divisionexerciseeg}{2}{}{}{g:exercise:idp228043736}%
\(\ \overline{A+B}\)\end{divisionexerciseeg}%
\begin{divisionexerciseeg}{3}{}{}{g:exercise:idp228040408}%
\(\ \overline{A}{}B\)\end{divisionexerciseeg}%
\begin{divisionexerciseeg}{4}{}{}{g:exercise:idp228042328}%
\(\ \overline{A}\,\overline{B}\)\end{divisionexerciseeg}%
\begin{divisionexerciseeg}{5}{}{}{g:exercise:idp228041816}%
\(\ \overline{A\overline{B}}\)\end{divisionexerciseeg}%
\begin{divisionexerciseeg}{6}{}{}{g:exercise:idp228039256}%
\(\ A\overline{B}+C\)\end{divisionexerciseeg}%
\begin{divisionexerciseeg}{7}{}{}{g:exercise:idp228050776}%
\(\ A(B+\overline{C})\)\end{divisionexerciseeg}%
\begin{divisionexerciseeg}{8}{}{}{g:exercise:idp228051672}%
\(\ \overline{AB}C\)\end{divisionexerciseeg}%
\begin{divisionexerciseeg}{9}{}{}{g:exercise:idp228051416}%
\(\ \overline{\overline{\overline{A}\,\overline{B}}+\overline{C}}\)\end{divisionexerciseeg}%
\end{exercisegroup}
\par\medskip\noindent
\par\medskip\noindent%
\textbf{Exercise Group.}\space\space%
Write the Boolean expression which corresponds to the following gates.\begin{exercisegroup}
\begin{divisionexerciseeg}{10}{}{}{g:exercise:idp228051032}%
\begin{image}{0.4}{0.2}{0.4}%
\resizebox{\linewidth}{!}{%
\begin{circuitikz}
\ctikzset{logic ports=ieee}
\draw
	(0,0) node[and port] (myand1) {}
	(myand1.in 1) node[anchor=east] {$A$}
	(myand1.in 2) node[anchor=east] {$B$}
	(myand1.bin 1) node[notcirc,left] {}
	(myand1.bin 2) node[notcirc,left] {};
\end{circuitikz}
}%
\end{image}%
\end{divisionexerciseeg}%
\begin{divisionexerciseeg}{11}{}{}{g:exercise:idp228053720}%
\begin{image}{0.4}{0.2}{0.4}%
\resizebox{\linewidth}{!}{%
\begin{circuitikz}
\ctikzset{logic ports=ieee}
\draw
	(0,0) node[or port] (myor) {}
	(myor.in 1) node[anchor=east] {$A$}
	(myor.in 2) node[anchor=east] {$B$}
	(myor.bin 1) node[notcirc,left] {}
	(myor.bout) node[notcirc,right] {};
\end{circuitikz}
}%
\end{image}%
\end{divisionexerciseeg}%
\begin{divisionexerciseeg}{12}{}{}{g:exercise:idp228053976}%
\begin{image}{0.4}{0.2}{0.4}%
\resizebox{\linewidth}{!}{%
\begin{circuitikz}
\ctikzset{logic ports=ieee}
\draw
	(0,0) node[or port] (myor) {}
	(myor.in 1) node[anchor=east] {$A$}
	(myor.in 2) node[anchor=east] {$B$}
	(myor.bin 1) node[notcirc,left] {}
	(myor.bin 2) node[notcirc,left] {}
	(myor.bout) node[notcirc,right] {};
\end{circuitikz}
}%
\end{image}%
\end{divisionexerciseeg}%
\begin{divisionexerciseeg}{13}{}{}{g:exercise:idp228054104}%
\begin{image}{0.4}{0.2}{0.4}%
\resizebox{\linewidth}{!}{%
\begin{circuitikz}
\ctikzset{logic ports=ieee}
\draw
	(0,0) node[and port] (mygate) {}
	(mygate.in 1) node[anchor=east] {$A$}
	(mygate.in 2) node[anchor=east] {$B$}
	(mygate.bout) node[notcirc,right] {};
\end{circuitikz}
}%
\end{image}%
\end{divisionexerciseeg}%
\begin{divisionexerciseeg}{14}{}{}{g:exercise:idp228057944}%
\begin{image}{0.3}{0.4}{0.3}%
\resizebox{\linewidth}{!}{%
\begin{circuitikz}
\ctikzset{logic ports=ieee}
\draw
	(0,0) node[or port] (mygate1) {} (mygate1.out) node[and port, anchor=in 1] (mygate2) {}
	(mygate1.in 1) node[anchor=east] {$A$}
	(mygate1.in 2) node[anchor=east] {$B$}
	(mygate2.in 2) node[anchor=east] {$C$};
\end{circuitikz}
}%
\end{image}%
\end{divisionexerciseeg}%
\begin{divisionexerciseeg}{15}{}{}{g:exercise:idp228055512}%
\begin{image}{0.3}{0.4}{0.3}%
\resizebox{\linewidth}{!}{%
\begin{circuitikz}
\ctikzset{logic ports=ieee}
\draw
	(0,0) node[and port] (mygate1) {} (mygate1.out) node[or port, anchor=in 2] (mygate2) {}
	(mygate1.in 1) node[anchor=east] {$B$}
	(mygate1.in 2) node[anchor=east] {$C$}
	(mygate1.bin 1) node[notcirc,left] {}
	(mygate1.bout) node[notcirc,right] {}
	(mygate2.in 1) node[anchor=east] {$A$}
;
\end{circuitikz}
}%
\end{image}%
\end{divisionexerciseeg}%
\begin{divisionexerciseeg}{16}{}{}{g:exercise:idp227794264}%
\begin{image}{0.3}{0.4}{0.3}%
\resizebox{\linewidth}{!}{%
\begin{circuitikz}
\ctikzset{logic ports=ieee}
\draw
	(0,0) node[or port] (mygate1) {} (mygate1.out) node[or port, anchor=in 2] (mygate2) {}
	(mygate1.bout) node[notcirc,right] {}
	(mygate1.in 1) node[anchor=east] {$B$}
	(mygate1.in 2) node[anchor=east] {$C$}
	(mygate2.in 1) node[anchor=east] {$A$}
	(mygate2.bout) node[notcirc,right] {};
;
\end{circuitikz}
}%
\end{image}%
\end{divisionexerciseeg}%
\begin{divisionexerciseeg}{17}{}{}{g:exercise:idp227793496}%
\begin{image}{0.3}{0.4}{0.3}%
\resizebox{\linewidth}{!}{%
\begin{circuitikz}
\ctikzset{logic ports=ieee}
\draw
	(0,0) node[and port] (mygate1) {} (mygate1.out) node[and port, anchor=in 1] (mygate2) {}
	(mygate1.bout) node[notcirc,right] {}
	(mygate1.in 1) node[anchor=east] {$A$}
	(mygate1.in 2) node[anchor=east] {$B$}
	(mygate1.bin 1) node[notcirc,left] {}
	(mygate1.bin 2) node[notcirc,left] {}
	(mygate2.in 2) node[anchor=east] {$C$}
	(mygate2.bin 2) node[notcirc,left] {}
;
\end{circuitikz}
}%
\end{image}%
\end{divisionexerciseeg}%
\end{exercisegroup}
\par\medskip\noindent
\par\medskip\noindent%
\textbf{Exercise Group.}\space\space%
Give the truth tables for the following expressions.\begin{exercisegroup}
\begin{divisionexerciseeg}{18}{}{}{g:exercise:idp227794776}%
\(\ A\overline{A}\)\end{divisionexerciseeg}%
\begin{divisionexerciseeg}{19}{}{}{g:exercise:idp228088696}%
\(\ A+1\)\end{divisionexerciseeg}%
\begin{divisionexerciseeg}{20}{}{}{g:exercise:idp228090360}%
\(\ A\overline{B}\)\end{divisionexerciseeg}%
\begin{divisionexerciseeg}{21}{}{}{g:exercise:idp228108152}%
\(\ \overline{A+B}\)\end{divisionexerciseeg}%
\begin{divisionexerciseeg}{22}{}{}{g:exercise:idp228115832}%
\(\ A+\overline{A}{}B\)\end{divisionexerciseeg}%
\begin{divisionexerciseeg}{23}{}{}{g:exercise:idp228131704}%
\(\ (A+B)C\)\end{divisionexerciseeg}%
\begin{divisionexerciseeg}{24}{}{}{g:exercise:idp228148472}%
\(\ A+B+\overline{C}\)\end{divisionexerciseeg}%
\end{exercisegroup}
\par\medskip\noindent
\par\medskip\noindent%
\textbf{Exercise Group.}\space\space%
Are the two expressions logically equivalent?  Justify your answer by giving a truth table.\begin{exercisegroup}
\begin{divisionexerciseeg}{25}{}{}{g:exercise:idp228180600}%
\(\ \overline{AB}\) and \(\overline{A}\,\overline{B}\)\end{divisionexerciseeg}%
\begin{divisionexerciseeg}{26}{}{}{g:exercise:idp228198776}%
\(\ \overline{A+B}\) and \(\overline{A}\,\overline{B}\)\end{divisionexerciseeg}%
\begin{divisionexerciseeg}{27}{}{}{g:exercise:idp228214520}%
\(\ A+BC\) and \((A+B)C\)\end{divisionexerciseeg}%
\begin{divisionexerciseeg}{28}{}{}{g:exercise:idp228250744}%
\(\ A+AB\) and \(A\)\end{divisionexerciseeg}%
\begin{divisionexerciseeg}{29}{}{}{g:exercise:idp228261624}%
\(\ (A+B)+C\) and \(A+(B+C)\)\end{divisionexerciseeg}%
\end{exercisegroup}
\par\medskip\noindent
\par\medskip\noindent%
\textbf{Exercise Group.}\space\space%
Simplify the following logical expressions using truth tables.\begin{exercisegroup}
\begin{divisionexerciseeg}{30}{}{}{g:exercise:idp228285176}%
\(\ AA\)\end{divisionexerciseeg}%
\begin{divisionexerciseeg}{31}{}{}{g:exercise:idp228300280}%
\(\ A+A\)\end{divisionexerciseeg}%
\begin{divisionexerciseeg}{32}{}{}{g:exercise:idp228306552}%
\(\ A+0\)\end{divisionexerciseeg}%
\begin{divisionexerciseeg}{33}{}{}{g:exercise:idp228310904}%
\(\ A+AB\)\end{divisionexerciseeg}%
\begin{divisionexerciseeg}{34}{}{}{g:exercise:idp228325240}%
\(\ A(\overline{A}+B)\) - this one is a bit trickier.  If you are stuck, try writing the truth tables for combinations of \(A\) and \(B\), like \((A+B)\) for example, to find one that fits.\end{divisionexerciseeg}%
\end{exercisegroup}
\par\medskip\noindent
\end{exercises-subsection}
%
%
\typeout{************************************************}
\typeout{Solutions 2.4.6 Solutions to Section~{\xreffont\ref*{x:section:sec-boolean-algebra}} Exercises}
\typeout{************************************************}
%
\begin{solutions-subsection}{Solutions to Section~{\xreffont\ref*{x:section:sec-boolean-algebra}} Exercises}{}{Solutions to Section~{\xreffont\ref*{x:section:sec-boolean-algebra}} Exercises}{}{}{g:solutions:idp228340600}
\par\medskip
\noindent\textbf{\normalsize{}2.4.5\space\textperiodcentered\space{}Exercises}
\begin{exercisegroup}
\begin{divisionsolutioneg}{2.4.5.1}{}{g:exercise:idp228045144}%
\par\smallskip%
\noindent\hypertarget{g:solution:idp228042968-main}{}\begin{image}{0.35}{0.3}{0.35}%
\resizebox{\linewidth}{!}{%
\begin{circuitikz}
  \ctikzset{logic ports=ieee}
  \draw (0,0) node[or port] (myor) {}
   (myor.in 1) node[anchor=east] {$A$}
   (myor.in 2) node[anchor=east] {$B$}
   (myor.bin 2) node[notcirc,left] {}
   (myor.out) node[anchor=west] {$A+\overline{B}$};
\end{circuitikz}
}%
\end{image}%
\end{divisionsolutioneg}%
\begin{divisionsolutioneg}{2.4.5.2}{}{g:exercise:idp228043736}%
\par\smallskip%
\noindent\hypertarget{g:solution:idp228040280-main}{}\begin{image}{0.35}{0.3}{0.35}%
\resizebox{\linewidth}{!}{%
\begin{circuitikz}
  \ctikzset{logic ports=ieee}
  \draw (0,0) node[or port] (myor) {}
   (myor.in 1) node[anchor=east] {$A$}
   (myor.in 2) node[anchor=east] {$B$}
   (myor.bout) node[notcirc,right] {}
   (myor.out) node[anchor=west] {$\overline{A+B}$};
\end{circuitikz}
}%
\end{image}%
\end{divisionsolutioneg}%
\begin{divisionsolutioneg}{2.4.5.3}{}{g:exercise:idp228040408}%
\par\smallskip%
\noindent\hypertarget{g:solution:idp228045400-main}{}\begin{image}{0.35}{0.3}{0.35}%
\resizebox{\linewidth}{!}{%
\begin{circuitikz}
  \ctikzset{logic ports=ieee}
  \draw (0,0) node[and port] (mygate) {}
   (mygate.in 1) node[anchor=east] {$A$}
   (mygate.in 2) node[anchor=east] {$B$}
   (mygate.bin 1) node[notcirc,left] {}
   (mygate.out) node[anchor=west] {$\overline{A}{}B$};
\end{circuitikz}
}%
\end{image}%
\end{divisionsolutioneg}%
\begin{divisionsolutioneg}{2.4.5.4}{}{g:exercise:idp228042328}%
\par\smallskip%
\noindent\hypertarget{g:solution:idp228040664-main}{}\begin{image}{0.35}{0.3}{0.35}%
\resizebox{\linewidth}{!}{%
\begin{circuitikz}
  \ctikzset{logic ports=ieee}
  \draw (0,0) node[and port] (mygate) {}
   (mygate.in 1) node[anchor=east] {$A$}
   (mygate.in 2) node[anchor=east] {$B$}
   (mygate.bin 1) node[notcirc,left] {}
   (mygate.bin 2) node[notcirc,left] {}
   (mygate.out) node[anchor=west] {$\overline{A}\,\overline{B}$};
\end{circuitikz}
}%
\end{image}%
\end{divisionsolutioneg}%
\begin{divisionsolutioneg}{2.4.5.5}{}{g:exercise:idp228041816}%
\par\smallskip%
\noindent\hypertarget{g:solution:idp228045912-main}{}\begin{image}{0.35}{0.3}{0.35}%
\resizebox{\linewidth}{!}{%
\begin{circuitikz}
  \ctikzset{logic ports=ieee}
  \draw (0,0) node[or port] (mygate) {}
   (mygate.in 1) node[anchor=east] {$A$}
   (mygate.in 2) node[anchor=east] {$B$}
   (mygate.bin 2) node[notcirc,left] {}
   (mygate.bout) node[notcirc,right] {}
   (mygate.out) node[anchor=west] {$\overline{A+\overline{B}}$};
\end{circuitikz}
}%
\end{image}%
\end{divisionsolutioneg}%
\begin{divisionsolutioneg}{2.4.5.6}{}{g:exercise:idp228039256}%
\par\smallskip%
\noindent\hypertarget{g:solution:idp228042712-main}{}\begin{image}{0.25}{0.5}{0.25}%
\resizebox{\linewidth}{!}{%
\begin{circuitikz}
	\ctikzset{logic ports=ieee}
	\draw
  		(0,0) node[and port] (myand) {} (myand.out) node[or port, anchor=in 1] (myor) {}
     (myand.in 1) node[anchor=east] {$A$}
     (myand.in 2) node[anchor=east] {$B$}
     (myor.in 2) node[anchor=east] {$C$}
     (myand.bin 2) node[notcirc,left] {}
     (myor.out) node[anchor=west] {$A\overline{B}+C$};
\end{circuitikz}
}%
\end{image}%
\end{divisionsolutioneg}%
\begin{divisionsolutioneg}{2.4.5.7}{}{g:exercise:idp228050776}%
\par\smallskip%
\noindent\hypertarget{g:solution:idp228050264-main}{}\begin{image}{0.25}{0.5}{0.25}%
\resizebox{\linewidth}{!}{%
\begin{circuitikz}
  \ctikzset{logic ports=ieee}
  \draw
      (0,0) node[or port] (myor) {} (myor.out) node[and port, anchor=in 2] (myand) {}
     (myor.in 1) node[anchor=east] {$B$}
     (myor.in 2) node[anchor=east] {$C$}
     (myand.in 1) node[anchor=east] {$A$}
     (myor.bin 2) node[notcirc,left] {}
     (myand.out) node[anchor=west] {$A\left(B+\overline{C}\right)$};
\end{circuitikz}
}%
\end{image}%
\end{divisionsolutioneg}%
\begin{divisionsolutioneg}{2.4.5.8}{}{g:exercise:idp228051672}%
\par\smallskip%
\noindent\hypertarget{g:solution:idp228053336-main}{}\begin{image}{0.25}{0.5}{0.25}%
\resizebox{\linewidth}{!}{%
\begin{circuitikz}
\ctikzset{logic ports=ieee}
\draw
	(0,0) node[and port] (myand1) {} (myand1.out) node[and port, anchor=in 1] (myand2) {}
	(myand1.bout) node[notcirc,right] {}
	(myand1.in 1) node[anchor=east] {$A$}
	(myand1.in 2) node[anchor=east] {$B$}
	(myand2.in 2) node[anchor=east] {$C$}
	(myand2.out) node[anchor=west] {$\overline{AB}C$};
\end{circuitikz}
}%
\end{image}%
\end{divisionsolutioneg}%
\begin{divisionsolutioneg}{2.4.5.9}{}{g:exercise:idp228051416}%
\par\smallskip%
\noindent\hypertarget{g:solution:idp228049496-main}{}\begin{image}{0.25}{0.5}{0.25}%
\resizebox{\linewidth}{!}{%
\begin{circuitikz}
  \ctikzset{logic ports=ieee}
  \draw
  	(0,0) node[and port] (myand1) {} (myand1.out) node[or port, anchor=in 1] (myor) {}
  	(myand1.bout) node[notcirc,right] {}
  	(myand1.in 1) node[anchor=east] {$A$}
  	(myand1.in 2) node[anchor=east] {$B$}
  	(myand1.bin 1) node[notcirc,left] {}
  	(myand1.bin 2) node[notcirc,left] {}
  	(myor.in 2) node[anchor=east] {$C$}
  	(myor.bin 2) node[notcirc,left] {}
  	(myor.bout) node[notcirc,right] {}
  	(myor.out) node[anchor=west] {$\overline{\overline{\overline{A}\,\overline{B}}+\overline{C}}$};
\end{circuitikz}
}%
\end{image}%
\end{divisionsolutioneg}%
\end{exercisegroup}
\par\medskip\noindent
\begin{exercisegroup}
\begin{divisionsolutioneg}{2.4.5.10}{}{g:exercise:idp228051032}%
\par\smallskip%
\noindent\hypertarget{g:solution:idp228047704-main}{}\(\overline{A}\,\overline{B}\)\end{divisionsolutioneg}%
\begin{divisionsolutioneg}{2.4.5.11}{}{g:exercise:idp228053720}%
\par\smallskip%
\noindent\hypertarget{g:solution:idp228049112-main}{}\(\overline{\overline{A}+B}\)\end{divisionsolutioneg}%
\begin{divisionsolutioneg}{2.4.5.12}{}{g:exercise:idp228053976}%
\par\smallskip%
\noindent\hypertarget{g:solution:idp228051800-main}{}\(\overline{\overline{A}+\overline{B}}\)\end{divisionsolutioneg}%
\begin{divisionsolutioneg}{2.4.5.13}{}{g:exercise:idp228054104}%
\par\smallskip%
\noindent\hypertarget{g:solution:idp228056408-main}{}\(\overline{AB}\)\end{divisionsolutioneg}%
\begin{divisionsolutioneg}{2.4.5.14}{}{g:exercise:idp228057944}%
\par\smallskip%
\noindent\hypertarget{g:solution:idp228057560-main}{}\((A+B)C\)\end{divisionsolutioneg}%
\begin{divisionsolutioneg}{2.4.5.15}{}{g:exercise:idp228055512}%
\par\smallskip%
\noindent\hypertarget{g:solution:idp228057048-main}{}\(A+\overline{\overline{B}{}C}\)\end{divisionsolutioneg}%
\begin{divisionsolutioneg}{2.4.5.16}{}{g:exercise:idp227794264}%
\par\smallskip%
\noindent\hypertarget{g:solution:idp227794648-main}{}\(\overline{A+\overline{B+C}}\)\end{divisionsolutioneg}%
\begin{divisionsolutioneg}{2.4.5.17}{}{g:exercise:idp227793496}%
\par\smallskip%
\noindent\hypertarget{g:solution:idp227800792-main}{}\(\overline{\overline{A}\,\overline{B}}\,\overline{C}\)\end{divisionsolutioneg}%
\end{exercisegroup}
\par\medskip\noindent
\begin{exercisegroup}
\begin{divisionsolutioneg}{2.4.5.18}{}{g:exercise:idp227794776}%
\par\smallskip%
\noindent\hypertarget{g:solution:idp227793240-main}{}\begin{center}%
{\tabularfont%
\begin{tabular}{Bccc}\hrulethick
\multicolumn{1}{BcB}{\(A\)}&\multicolumn{1}{cB}{\(\overline{A}\)}&\multicolumn{1}{cB}{\(A\overline{A}\)}\tabularnewline\hrulemedium
\multicolumn{1}{BcB}{0}&\multicolumn{1}{cB}{1}&\multicolumn{1}{cB}{0}\tabularnewline[0pt]
\multicolumn{1}{BcB}{1}&\multicolumn{1}{cB}{0}&\multicolumn{1}{cB}{0}\tabularnewline\hrulethick
\end{tabular}
}%
\end{center}%
\end{divisionsolutioneg}%
\begin{divisionsolutioneg}{2.4.5.19}{}{g:exercise:idp228088696}%
\par\smallskip%
\noindent\hypertarget{g:solution:idp228095224-main}{}\begin{center}%
{\tabularfont%
\begin{tabular}{Bccc}\hrulethick
\multicolumn{1}{BcB}{\(A\)}&\multicolumn{1}{cB}{\(1\)}&\multicolumn{1}{cB}{\(A+1\)}\tabularnewline\hrulemedium
\multicolumn{1}{BcB}{0}&\multicolumn{1}{cB}{1}&\multicolumn{1}{cB}{1}\tabularnewline[0pt]
\multicolumn{1}{BcB}{1}&\multicolumn{1}{cB}{1}&\multicolumn{1}{cB}{1}\tabularnewline\hrulethick
\end{tabular}
}%
\end{center}%
\end{divisionsolutioneg}%
\begin{divisionsolutioneg}{2.4.5.20}{}{g:exercise:idp228090360}%
\par\smallskip%
\noindent\hypertarget{g:solution:idp228091512-main}{}\begin{center}%
{\tabularfont%
\begin{tabular}{Bcccc}\hrulethick
\multicolumn{1}{BcB}{\(A\)}&\multicolumn{1}{cB}{\(B\)}&\multicolumn{1}{cB}{\(\overline{B}\)}&\multicolumn{1}{cB}{\(A\overline{B}\)}\tabularnewline\hrulemedium
\multicolumn{1}{BcB}{0}&\multicolumn{1}{cB}{0}&\multicolumn{1}{cB}{1}&\multicolumn{1}{cB}{0}\tabularnewline[0pt]
\multicolumn{1}{BcB}{0}&\multicolumn{1}{cB}{1}&\multicolumn{1}{cB}{0}&\multicolumn{1}{cB}{0}\tabularnewline[0pt]
\multicolumn{1}{BcB}{1}&\multicolumn{1}{cB}{0}&\multicolumn{1}{cB}{1}&\multicolumn{1}{cB}{1}\tabularnewline[0pt]
\multicolumn{1}{BcB}{1}&\multicolumn{1}{cB}{1}&\multicolumn{1}{cB}{0}&\multicolumn{1}{cB}{0}\tabularnewline\hrulethick
\end{tabular}
}%
\end{center}%
\end{divisionsolutioneg}%
\begin{divisionsolutioneg}{2.4.5.21}{}{g:exercise:idp228108152}%
\par\smallskip%
\noindent\hypertarget{g:solution:idp228107640-main}{}\begin{center}%
{\tabularfont%
\begin{tabular}{Bcccc}\hrulethick
\multicolumn{1}{BcB}{\(A\)}&\multicolumn{1}{cB}{\(B\)}&\multicolumn{1}{cB}{\(A+B\)}&\multicolumn{1}{cB}{\(\overline{A+B}\)}\tabularnewline\hrulemedium
\multicolumn{1}{BcB}{0}&\multicolumn{1}{cB}{0}&\multicolumn{1}{cB}{0}&\multicolumn{1}{cB}{1}\tabularnewline[0pt]
\multicolumn{1}{BcB}{0}&\multicolumn{1}{cB}{1}&\multicolumn{1}{cB}{1}&\multicolumn{1}{cB}{0}\tabularnewline[0pt]
\multicolumn{1}{BcB}{1}&\multicolumn{1}{cB}{0}&\multicolumn{1}{cB}{1}&\multicolumn{1}{cB}{0}\tabularnewline[0pt]
\multicolumn{1}{BcB}{1}&\multicolumn{1}{cB}{1}&\multicolumn{1}{cB}{1}&\multicolumn{1}{cB}{0}\tabularnewline\hrulethick
\end{tabular}
}%
\end{center}%
\end{divisionsolutioneg}%
\begin{divisionsolutioneg}{2.4.5.22}{}{g:exercise:idp228115832}%
\par\smallskip%
\noindent\hypertarget{g:solution:idp228119288-main}{}\begin{center}%
{\tabularfont%
\begin{tabular}{Blllll}\hrulethick
\multicolumn{1}{BlB}{\(A\)}&\multicolumn{1}{lB}{\(B\)}&\multicolumn{1}{lB}{\(\overline{A}\)}&\multicolumn{1}{lB}{\(\overline{A}{}B\)}&\multicolumn{1}{lB}{\(A+\overline{A}{}B\)}\tabularnewline\hrulemedium
\multicolumn{1}{BlB}{0}&\multicolumn{1}{lB}{0}&\multicolumn{1}{lB}{1}&\multicolumn{1}{lB}{0}&\multicolumn{1}{lB}{0}\tabularnewline[0pt]
\multicolumn{1}{BlB}{0}&\multicolumn{1}{lB}{1}&\multicolumn{1}{lB}{1}&\multicolumn{1}{lB}{1}&\multicolumn{1}{lB}{1}\tabularnewline[0pt]
\multicolumn{1}{BlB}{1}&\multicolumn{1}{lB}{0}&\multicolumn{1}{lB}{0}&\multicolumn{1}{lB}{0}&\multicolumn{1}{lB}{1}\tabularnewline[0pt]
\multicolumn{1}{BlB}{1}&\multicolumn{1}{lB}{1}&\multicolumn{1}{lB}{0}&\multicolumn{1}{lB}{0}&\multicolumn{1}{lB}{1}\tabularnewline\hrulethick
\end{tabular}
}%
\end{center}%
\end{divisionsolutioneg}%
\begin{divisionsolutioneg}{2.4.5.23}{}{g:exercise:idp228131704}%
\par\smallskip%
\noindent\hypertarget{g:solution:idp228130296-main}{}\begin{center}%
{\tabularfont%
\begin{tabular}{Bccccc}\hrulethick
\multicolumn{1}{BcB}{\(A\)}&\multicolumn{1}{cB}{\(B\)}&\multicolumn{1}{cB}{\(C\)}&\multicolumn{1}{cB}{\(A+B\)}&\multicolumn{1}{cB}{\((A+B)C\)}\tabularnewline\hrulemedium
\multicolumn{1}{BcB}{0}&\multicolumn{1}{cB}{0}&\multicolumn{1}{cB}{0}&\multicolumn{1}{cB}{0}&\multicolumn{1}{cB}{0}\tabularnewline[0pt]
\multicolumn{1}{BcB}{0}&\multicolumn{1}{cB}{0}&\multicolumn{1}{cB}{1}&\multicolumn{1}{cB}{0}&\multicolumn{1}{cB}{0}\tabularnewline[0pt]
\multicolumn{1}{BcB}{0}&\multicolumn{1}{cB}{1}&\multicolumn{1}{cB}{0}&\multicolumn{1}{cB}{1}&\multicolumn{1}{cB}{0}\tabularnewline[0pt]
\multicolumn{1}{BcB}{0}&\multicolumn{1}{cB}{1}&\multicolumn{1}{cB}{1}&\multicolumn{1}{cB}{1}&\multicolumn{1}{cB}{1}\tabularnewline[0pt]
\multicolumn{1}{BcB}{1}&\multicolumn{1}{cB}{0}&\multicolumn{1}{cB}{0}&\multicolumn{1}{cB}{1}&\multicolumn{1}{cB}{0}\tabularnewline[0pt]
\multicolumn{1}{BcB}{1}&\multicolumn{1}{cB}{0}&\multicolumn{1}{cB}{1}&\multicolumn{1}{cB}{1}&\multicolumn{1}{cB}{1}\tabularnewline[0pt]
\multicolumn{1}{BcB}{1}&\multicolumn{1}{cB}{1}&\multicolumn{1}{cB}{0}&\multicolumn{1}{cB}{1}&\multicolumn{1}{cB}{0}\tabularnewline[0pt]
\multicolumn{1}{BcB}{1}&\multicolumn{1}{cB}{1}&\multicolumn{1}{cB}{1}&\multicolumn{1}{cB}{1}&\multicolumn{1}{cB}{1}\tabularnewline\hrulethick
\end{tabular}
}%
\end{center}%
\end{divisionsolutioneg}%
\begin{divisionsolutioneg}{2.4.5.24}{}{g:exercise:idp228148472}%
\par\smallskip%
\noindent\hypertarget{g:solution:idp228158456-main}{}\begin{center}%
{\tabularfont%
\begin{tabular}{Bcccccc}\hrulethick
\multicolumn{1}{BcB}{\(A\)}&\multicolumn{1}{cB}{\(B\)}&\multicolumn{1}{cB}{\(C\)}&\multicolumn{1}{cB}{\(\overline{C}\)}&\multicolumn{1}{cB}{\(A+B\)}&\multicolumn{1}{cB}{\(A+B+\overline{C}\)}\tabularnewline\hrulemedium
\multicolumn{1}{BcB}{0}&\multicolumn{1}{cB}{0}&\multicolumn{1}{cB}{0}&\multicolumn{1}{cB}{1}&\multicolumn{1}{cB}{0}&\multicolumn{1}{cB}{1}\tabularnewline[0pt]
\multicolumn{1}{BcB}{0}&\multicolumn{1}{cB}{0}&\multicolumn{1}{cB}{1}&\multicolumn{1}{cB}{0}&\multicolumn{1}{cB}{0}&\multicolumn{1}{cB}{0}\tabularnewline[0pt]
\multicolumn{1}{BcB}{0}&\multicolumn{1}{cB}{1}&\multicolumn{1}{cB}{0}&\multicolumn{1}{cB}{1}&\multicolumn{1}{cB}{1}&\multicolumn{1}{cB}{1}\tabularnewline[0pt]
\multicolumn{1}{BcB}{0}&\multicolumn{1}{cB}{1}&\multicolumn{1}{cB}{1}&\multicolumn{1}{cB}{0}&\multicolumn{1}{cB}{1}&\multicolumn{1}{cB}{1}\tabularnewline[0pt]
\multicolumn{1}{BcB}{1}&\multicolumn{1}{cB}{0}&\multicolumn{1}{cB}{0}&\multicolumn{1}{cB}{1}&\multicolumn{1}{cB}{1}&\multicolumn{1}{cB}{1}\tabularnewline[0pt]
\multicolumn{1}{BcB}{1}&\multicolumn{1}{cB}{0}&\multicolumn{1}{cB}{1}&\multicolumn{1}{cB}{0}&\multicolumn{1}{cB}{1}&\multicolumn{1}{cB}{1}\tabularnewline[0pt]
\multicolumn{1}{BcB}{1}&\multicolumn{1}{cB}{1}&\multicolumn{1}{cB}{0}&\multicolumn{1}{cB}{1}&\multicolumn{1}{cB}{1}&\multicolumn{1}{cB}{1}\tabularnewline[0pt]
\multicolumn{1}{BcB}{1}&\multicolumn{1}{cB}{1}&\multicolumn{1}{cB}{1}&\multicolumn{1}{cB}{0}&\multicolumn{1}{cB}{1}&\multicolumn{1}{cB}{1}\tabularnewline\hrulethick
\end{tabular}
}%
\end{center}%
\end{divisionsolutioneg}%
\end{exercisegroup}
\par\medskip\noindent
\begin{exercisegroup}
\begin{divisionsolutioneg}{2.4.5.25}{}{g:exercise:idp228180600}%
\par\smallskip%
\noindent\hypertarget{g:solution:idp228182264-main}{}No:%
\par
\begin{center}%
{\tabularfont%
\begin{tabular}{Bccccccc}\hrulethick
\multicolumn{1}{BcB}{\(A\)}&\multicolumn{1}{cB}{\(B\)}&\multicolumn{1}{cB}{\(AB\)}&\multicolumn{1}{cB}{\(\overline{AB}\)}&\multicolumn{1}{cB}{\(\overline{A}\)}&\multicolumn{1}{cB}{\(\overline{B}\)}&\multicolumn{1}{cB}{\(\overline{A}\,\overline{B}\)}\tabularnewline\hrulemedium
\multicolumn{1}{BcB}{0}&\multicolumn{1}{cB}{0}&\multicolumn{1}{cB}{0}&\multicolumn{1}{cB}{1}&\multicolumn{1}{cB}{1}&\multicolumn{1}{cB}{1}&\multicolumn{1}{cB}{1}\tabularnewline[0pt]
\multicolumn{1}{BcB}{0}&\multicolumn{1}{cB}{1}&\multicolumn{1}{cB}{0}&\multicolumn{1}{cB}{1}&\multicolumn{1}{cB}{1}&\multicolumn{1}{cB}{0}&\multicolumn{1}{cB}{0}\tabularnewline[0pt]
\multicolumn{1}{BcB}{1}&\multicolumn{1}{cB}{0}&\multicolumn{1}{cB}{0}&\multicolumn{1}{cB}{1}&\multicolumn{1}{cB}{0}&\multicolumn{1}{cB}{1}&\multicolumn{1}{cB}{0}\tabularnewline[0pt]
\multicolumn{1}{BcB}{1}&\multicolumn{1}{cB}{1}&\multicolumn{1}{cB}{1}&\multicolumn{1}{cB}{0}&\multicolumn{1}{cB}{0}&\multicolumn{1}{cB}{0}&\multicolumn{1}{cB}{0}\tabularnewline\hrulethick
\end{tabular}
}%
\end{center}%
 The associated columns (4th and 7th) are not identical.%
\end{divisionsolutioneg}%
\begin{divisionsolutioneg}{2.4.5.26}{}{g:exercise:idp228198776}%
\par\smallskip%
\noindent\hypertarget{g:solution:idp228199672-main}{}Yes:%
\par
\begin{center}%
{\tabularfont%
\begin{tabular}{Bccccccc}\hrulethick
\multicolumn{1}{BcB}{\(A\)}&\multicolumn{1}{cB}{\(B\)}&\multicolumn{1}{cB}{\(A+B\)}&\multicolumn{1}{cB}{\(\overline{A+B}\)}&\multicolumn{1}{cB}{\(\overline{A}\)}&\multicolumn{1}{cB}{\(\overline{B}\)}&\multicolumn{1}{cB}{\(\overline{A}\,\overline{B}\)}\tabularnewline\hrulemedium
\multicolumn{1}{BcB}{0}&\multicolumn{1}{cB}{0}&\multicolumn{1}{cB}{0}&\multicolumn{1}{cB}{1}&\multicolumn{1}{cB}{1}&\multicolumn{1}{cB}{1}&\multicolumn{1}{cB}{1}\tabularnewline[0pt]
\multicolumn{1}{BcB}{0}&\multicolumn{1}{cB}{1}&\multicolumn{1}{cB}{1}&\multicolumn{1}{cB}{0}&\multicolumn{1}{cB}{1}&\multicolumn{1}{cB}{0}&\multicolumn{1}{cB}{0}\tabularnewline[0pt]
\multicolumn{1}{BcB}{1}&\multicolumn{1}{cB}{0}&\multicolumn{1}{cB}{1}&\multicolumn{1}{cB}{0}&\multicolumn{1}{cB}{0}&\multicolumn{1}{cB}{1}&\multicolumn{1}{cB}{0}\tabularnewline[0pt]
\multicolumn{1}{BcB}{1}&\multicolumn{1}{cB}{1}&\multicolumn{1}{cB}{1}&\multicolumn{1}{cB}{0}&\multicolumn{1}{cB}{0}&\multicolumn{1}{cB}{0}&\multicolumn{1}{cB}{0}\tabularnewline\hrulethick
\end{tabular}
}%
\end{center}%
 The associated columns (4th and 7th) are identical.%
\end{divisionsolutioneg}%
\begin{divisionsolutioneg}{2.4.5.27}{}{g:exercise:idp228214520}%
\par\smallskip%
\noindent\hypertarget{g:solution:idp228217592-main}{}No:%
\par
\begin{center}%
{\tabularfont%
\begin{tabular}{Bccccccc}\hrulethick
\multicolumn{1}{BcB}{\(A\)}&\multicolumn{1}{cB}{\(B\)}&\multicolumn{1}{cB}{\(C\)}&\multicolumn{1}{cB}{\(BC\)}&\multicolumn{1}{cB}{\(A+BC\)}&\multicolumn{1}{cB}{\(A+B\)}&\multicolumn{1}{cB}{\((A+B)C\)}\tabularnewline\hrulemedium
\multicolumn{1}{BcB}{0}&\multicolumn{1}{cB}{0}&\multicolumn{1}{cB}{0}&\multicolumn{1}{cB}{0}&\multicolumn{1}{cB}{0}&\multicolumn{1}{cB}{0}&\multicolumn{1}{cB}{0}\tabularnewline[0pt]
\multicolumn{1}{BcB}{0}&\multicolumn{1}{cB}{0}&\multicolumn{1}{cB}{1}&\multicolumn{1}{cB}{0}&\multicolumn{1}{cB}{0}&\multicolumn{1}{cB}{0}&\multicolumn{1}{cB}{0}\tabularnewline[0pt]
\multicolumn{1}{BcB}{0}&\multicolumn{1}{cB}{1}&\multicolumn{1}{cB}{0}&\multicolumn{1}{cB}{0}&\multicolumn{1}{cB}{0}&\multicolumn{1}{cB}{1}&\multicolumn{1}{cB}{0}\tabularnewline[0pt]
\multicolumn{1}{BcB}{0}&\multicolumn{1}{cB}{1}&\multicolumn{1}{cB}{1}&\multicolumn{1}{cB}{1}&\multicolumn{1}{cB}{1}&\multicolumn{1}{cB}{1}&\multicolumn{1}{cB}{1}\tabularnewline[0pt]
\multicolumn{1}{BcB}{1}&\multicolumn{1}{cB}{0}&\multicolumn{1}{cB}{0}&\multicolumn{1}{cB}{0}&\multicolumn{1}{cB}{1}&\multicolumn{1}{cB}{1}&\multicolumn{1}{cB}{0}\tabularnewline[0pt]
\multicolumn{1}{BcB}{1}&\multicolumn{1}{cB}{0}&\multicolumn{1}{cB}{1}&\multicolumn{1}{cB}{0}&\multicolumn{1}{cB}{1}&\multicolumn{1}{cB}{1}&\multicolumn{1}{cB}{1}\tabularnewline[0pt]
\multicolumn{1}{BcB}{1}&\multicolumn{1}{cB}{1}&\multicolumn{1}{cB}{0}&\multicolumn{1}{cB}{0}&\multicolumn{1}{cB}{1}&\multicolumn{1}{cB}{1}&\multicolumn{1}{cB}{0}\tabularnewline[0pt]
\multicolumn{1}{BcB}{1}&\multicolumn{1}{cB}{1}&\multicolumn{1}{cB}{1}&\multicolumn{1}{cB}{1}&\multicolumn{1}{cB}{1}&\multicolumn{1}{cB}{1}&\multicolumn{1}{cB}{1}\tabularnewline\hrulethick
\end{tabular}
}%
\end{center}%
 The associated columns (5th and 7th) are not identical.%
\end{divisionsolutioneg}%
\begin{divisionsolutioneg}{2.4.5.28}{}{g:exercise:idp228250744}%
\par\smallskip%
\noindent\hypertarget{g:solution:idp228249336-main}{}Yes:%
\par
\begin{center}%
{\tabularfont%
\begin{tabular}{Bcccc}\hrulethick
\multicolumn{1}{BcB}{\(A\)}&\multicolumn{1}{cB}{\(B\)}&\multicolumn{1}{cB}{\(AB\)}&\multicolumn{1}{cB}{\(A+AB\)}\tabularnewline\hrulemedium
\multicolumn{1}{BcB}{0}&\multicolumn{1}{cB}{0}&\multicolumn{1}{cB}{0}&\multicolumn{1}{cB}{0}\tabularnewline[0pt]
\multicolumn{1}{BcB}{0}&\multicolumn{1}{cB}{1}&\multicolumn{1}{cB}{0}&\multicolumn{1}{cB}{0}\tabularnewline[0pt]
\multicolumn{1}{BcB}{1}&\multicolumn{1}{cB}{0}&\multicolumn{1}{cB}{0}&\multicolumn{1}{cB}{1}\tabularnewline[0pt]
\multicolumn{1}{BcB}{1}&\multicolumn{1}{cB}{1}&\multicolumn{1}{cB}{1}&\multicolumn{1}{cB}{1}\tabularnewline\hrulethick
\end{tabular}
}%
\end{center}%
 The first and last columns are identical.%
\end{divisionsolutioneg}%
\begin{divisionsolutioneg}{2.4.5.29}{}{g:exercise:idp228261624}%
\par\smallskip%
\noindent\hypertarget{g:solution:idp228262136-main}{}Yes:%
\par
\begin{center}%
{\tabularfont%
\begin{tabular}{Bccccccc}\hrulethick
\multicolumn{1}{BcB}{\(A\)}&\multicolumn{1}{cB}{\(B\)}&\multicolumn{1}{cB}{\(C\)}&\multicolumn{1}{cB}{\(A+B\)}&\multicolumn{1}{cB}{\((A+B)+C\)}&\multicolumn{1}{cB}{\((B+C)\)}&\multicolumn{1}{cB}{\(A+(B+C)\)}\tabularnewline\hrulemedium
\multicolumn{1}{BcB}{0}&\multicolumn{1}{cB}{0}&\multicolumn{1}{cB}{0}&\multicolumn{1}{cB}{0}&\multicolumn{1}{cB}{0}&\multicolumn{1}{cB}{0}&\multicolumn{1}{cB}{0}\tabularnewline[0pt]
\multicolumn{1}{BcB}{0}&\multicolumn{1}{cB}{0}&\multicolumn{1}{cB}{1}&\multicolumn{1}{cB}{0}&\multicolumn{1}{cB}{1}&\multicolumn{1}{cB}{1}&\multicolumn{1}{cB}{1}\tabularnewline[0pt]
\multicolumn{1}{BcB}{0}&\multicolumn{1}{cB}{1}&\multicolumn{1}{cB}{0}&\multicolumn{1}{cB}{1}&\multicolumn{1}{cB}{1}&\multicolumn{1}{cB}{1}&\multicolumn{1}{cB}{1}\tabularnewline[0pt]
\multicolumn{1}{BcB}{0}&\multicolumn{1}{cB}{1}&\multicolumn{1}{cB}{1}&\multicolumn{1}{cB}{1}&\multicolumn{1}{cB}{1}&\multicolumn{1}{cB}{1}&\multicolumn{1}{cB}{1}\tabularnewline[0pt]
\multicolumn{1}{BcB}{1}&\multicolumn{1}{cB}{0}&\multicolumn{1}{cB}{0}&\multicolumn{1}{cB}{1}&\multicolumn{1}{cB}{1}&\multicolumn{1}{cB}{0}&\multicolumn{1}{cB}{1}\tabularnewline[0pt]
\multicolumn{1}{BcB}{1}&\multicolumn{1}{cB}{0}&\multicolumn{1}{cB}{1}&\multicolumn{1}{cB}{1}&\multicolumn{1}{cB}{1}&\multicolumn{1}{cB}{1}&\multicolumn{1}{cB}{1}\tabularnewline[0pt]
\multicolumn{1}{BcB}{1}&\multicolumn{1}{cB}{1}&\multicolumn{1}{cB}{0}&\multicolumn{1}{cB}{1}&\multicolumn{1}{cB}{1}&\multicolumn{1}{cB}{1}&\multicolumn{1}{cB}{1}\tabularnewline[0pt]
\multicolumn{1}{BcB}{1}&\multicolumn{1}{cB}{1}&\multicolumn{1}{cB}{1}&\multicolumn{1}{cB}{1}&\multicolumn{1}{cB}{1}&\multicolumn{1}{cB}{1}&\multicolumn{1}{cB}{1}\tabularnewline\hrulethick
\end{tabular}
}%
\end{center}%
 The 5th and 7th columns are identical.%
\end{divisionsolutioneg}%
\end{exercisegroup}
\par\medskip\noindent
\begin{exercisegroup}
\begin{divisionsolutioneg}{2.4.5.30}{}{g:exercise:idp228285176}%
\par\smallskip%
\noindent\hypertarget{g:solution:idp228288760-main}{}\begin{center}%
{\tabularfont%
\begin{tabular}{Bccc}\hrulethick
\multicolumn{1}{BcB}{\(A\)}&\multicolumn{1}{cB}{\(A\)}&\multicolumn{1}{cB}{\(AA\)}\tabularnewline\hrulemedium
\multicolumn{1}{BcB}{0}&\multicolumn{1}{cB}{0}&\multicolumn{1}{cB}{0}\tabularnewline[0pt]
\multicolumn{1}{BcB}{1}&\multicolumn{1}{cB}{1}&\multicolumn{1}{cB}{1}\tabularnewline\hrulethick
\end{tabular}
}%
\end{center}%
\(AA\) is equivalent to \(A\).\end{divisionsolutioneg}%
\begin{divisionsolutioneg}{2.4.5.31}{}{g:exercise:idp228300280}%
\par\smallskip%
\noindent\hypertarget{g:solution:idp228299768-main}{}\begin{center}%
{\tabularfont%
\begin{tabular}{Bccc}\hrulethick
\multicolumn{1}{BcB}{\(A\)}&\multicolumn{1}{cB}{\(A\)}&\multicolumn{1}{cB}{\(A+A\)}\tabularnewline\hrulemedium
\multicolumn{1}{BcB}{0}&\multicolumn{1}{cB}{0}&\multicolumn{1}{cB}{0}\tabularnewline[0pt]
\multicolumn{1}{BcB}{1}&\multicolumn{1}{cB}{1}&\multicolumn{1}{cB}{1}\tabularnewline\hrulethick
\end{tabular}
}%
\end{center}%
\(A+A\) is equivalent to \(A\).\end{divisionsolutioneg}%
\begin{divisionsolutioneg}{2.4.5.32}{}{g:exercise:idp228306552}%
\par\smallskip%
\noindent\hypertarget{g:solution:idp228302840-main}{}\begin{center}%
{\tabularfont%
\begin{tabular}{Bccc}\hrulethick
\multicolumn{1}{BcB}{\(A\)}&\multicolumn{1}{cB}{\(0\)}&\multicolumn{1}{cB}{\(A+0\)}\tabularnewline\hrulemedium
\multicolumn{1}{BcB}{0}&\multicolumn{1}{cB}{0}&\multicolumn{1}{cB}{0}\tabularnewline[0pt]
\multicolumn{1}{BcB}{1}&\multicolumn{1}{cB}{0}&\multicolumn{1}{cB}{1}\tabularnewline\hrulethick
\end{tabular}
}%
\end{center}%
\(A+0\) is equivalent to \(A\).\end{divisionsolutioneg}%
\begin{divisionsolutioneg}{2.4.5.33}{}{g:exercise:idp228310904}%
\par\smallskip%
\noindent\hypertarget{g:solution:idp228311416-main}{}\begin{center}%
{\tabularfont%
\begin{tabular}{Bcccc}\hrulethick
\multicolumn{1}{BcB}{\(A\)}&\multicolumn{1}{cB}{\(B\)}&\multicolumn{1}{cB}{\(AB\)}&\multicolumn{1}{cB}{\(A+AB\)}\tabularnewline\hrulemedium
\multicolumn{1}{BcB}{0}&\multicolumn{1}{cB}{0}&\multicolumn{1}{cB}{0}&\multicolumn{1}{cB}{0}\tabularnewline[0pt]
\multicolumn{1}{BcB}{0}&\multicolumn{1}{cB}{1}&\multicolumn{1}{cB}{0}&\multicolumn{1}{cB}{0}\tabularnewline[0pt]
\multicolumn{1}{BcB}{1}&\multicolumn{1}{cB}{0}&\multicolumn{1}{cB}{0}&\multicolumn{1}{cB}{1}\tabularnewline[0pt]
\multicolumn{1}{BcB}{1}&\multicolumn{1}{cB}{1}&\multicolumn{1}{cB}{1}&\multicolumn{1}{cB}{1}\tabularnewline\hrulethick
\end{tabular}
}%
\end{center}%
\(A+AB\) is equivalent to \(A\).\end{divisionsolutioneg}%
\begin{divisionsolutioneg}{2.4.5.34}{}{g:exercise:idp228325240}%
\par\smallskip%
\noindent\hypertarget{g:solution:idp228328440-main}{}\begin{center}%
{\tabularfont%
\begin{tabular}{Bccccc}\hrulethick
\multicolumn{1}{BcB}{\(A\)}&\multicolumn{1}{cB}{\(B\)}&\multicolumn{1}{cB}{\(\overline{A}\)}&\multicolumn{1}{cB}{\(\overline{A}+B\)}&\multicolumn{1}{cB}{\(A(\overline{A}+B)\)}\tabularnewline\hrulemedium
\multicolumn{1}{BcB}{0}&\multicolumn{1}{cB}{0}&\multicolumn{1}{cB}{1}&\multicolumn{1}{cB}{1}&\multicolumn{1}{cB}{0}\tabularnewline[0pt]
\multicolumn{1}{BcB}{0}&\multicolumn{1}{cB}{1}&\multicolumn{1}{cB}{1}&\multicolumn{1}{cB}{1}&\multicolumn{1}{cB}{0}\tabularnewline[0pt]
\multicolumn{1}{BcB}{1}&\multicolumn{1}{cB}{0}&\multicolumn{1}{cB}{0}&\multicolumn{1}{cB}{0}&\multicolumn{1}{cB}{0}\tabularnewline[0pt]
\multicolumn{1}{BcB}{1}&\multicolumn{1}{cB}{1}&\multicolumn{1}{cB}{0}&\multicolumn{1}{cB}{1}&\multicolumn{1}{cB}{1}\tabularnewline\hrulethick
\end{tabular}
}%
\end{center}%
\(A(\overline{A}+B)\) is true only if both \(A\) and \(B\) are true, so the statement is equivalent to \(AB\).\end{divisionsolutioneg}%
\end{exercisegroup}
\par\medskip\noindent
\end{solutions-subsection}
\end{sectionptx}
%
%
\typeout{************************************************}
\typeout{Section 2.5 Laws of Logic}
\typeout{************************************************}
%
\begin{sectionptx}{Laws of Logic}{}{Laws of Logic}{}{}{x:section:sec-logic-laws}
\index{laws of logic}%
\begin{introduction}{}%
You may have noticed some common patterns running through some of the exercises by now.  Let's examine those patterns in more detail.%
\par
First, let us look at the connections between the two sets of symbols we've used so far. \begin{center}%
{\tabularfont%
\begin{tabular}{Bcccccc}\hrulemedium
\multicolumn{1}{BcB}{Logic}&\multicolumn{1}{cB}{\(p{\wedge} q\)}&\multicolumn{1}{cB}{\(p{\vee} q\)}&\multicolumn{1}{cB}{\(\sim\!{p}\)}&\multicolumn{1}{cB}{\(F\)}&\multicolumn{1}{cB}{\(T\)}\tabularnewline\hrulemedium
\multicolumn{1}{BcB}{Boolean Algebra}&\multicolumn{1}{cB}{\(AB\)}&\multicolumn{1}{cB}{\(A+B\)}&\multicolumn{1}{cB}{\(\overline{A}\)}&\multicolumn{1}{cB}{0}&\multicolumn{1}{cB}{1}\tabularnewline\hrulemedium
\end{tabular}
}%
\end{center}%
%
\par
In each case, we have symbols for negation, \emph{or}, and \emph{and}.  There are also equivalences with False\slash{}True for logic and 0\slash{}1 (or off\slash{}on) for Boolean algebra and logic circuits.  Let's see what else they have in common.%
\end{introduction}%
%
%
\typeout{************************************************}
\typeout{Subsection 2.5.1 Identity Laws}
\typeout{************************************************}
%
\begin{subsectionptx}{Identity Laws}{}{Identity Laws}{}{}{x:subsection:ssec-identity-laws}
Consider the following truth table for the logical symbols. \begin{center}%
{\tabularfont%
\begin{tabular}{Bccccccc}\hrulethick
\multicolumn{1}{BcB}{\(p\)}&\multicolumn{1}{cB}{\(0\)}&\multicolumn{1}{cB}{\(1\)}&\multicolumn{1}{cB}{\(p{\wedge} 0\)}&\multicolumn{1}{cB}{\(p{\vee} 0\)}&\multicolumn{1}{cB}{\(p{\wedge} 1\)}&\multicolumn{1}{cB}{\(p{\vee} 1\)}\tabularnewline\hrulemedium
\multicolumn{1}{BcB}{0}&\multicolumn{1}{cB}{0}&\multicolumn{1}{cB}{1}&\multicolumn{1}{cB}{0}&\multicolumn{1}{cB}{0}&\multicolumn{1}{cB}{0}&\multicolumn{1}{cB}{1}\tabularnewline[0pt]
\multicolumn{1}{BcB}{1}&\multicolumn{1}{cB}{0}&\multicolumn{1}{cB}{1}&\multicolumn{1}{cB}{0}&\multicolumn{1}{cB}{1}&\multicolumn{1}{cB}{1}&\multicolumn{1}{cB}{1}\tabularnewline\hrulethick
\end{tabular}
}%
\end{center}%
%
\par
You can replace the logic symbols in this table with the corresponding Boolean algebra notation.  From this, we can deduce the following \terminology{identity laws}: \index{laws of logic!identity laws}\index{identity laws} \begin{assemblage}{Identity Laws.}{x:assemblage:assemblage-identity-laws}%
\begin{sidebyside}{2}{0.2}{0.2}{0.2}%
\begin{sbspanel}{0.2}%
\resizebox{\ifdim\width > \linewidth\linewidth\else\width\fi}{!}{%
{\centering%
{\tabularfont%
\begin{tabular}{l}
\alert{Logic}\tabularnewline\hrulemedium
\end{tabular}
}%
\par}
}%
%
\begin{align*}
p{\wedge} 0 \amp\Leftrightarrow 0\\
p{\vee} 0\amp\Leftrightarrow p\\
p{\wedge} 1\amp\Leftrightarrow p\\
p{\vee} 1\amp\Leftrightarrow 1
\end{align*}
%
\end{sbspanel}%
\begin{sbspanel}{0.2}%
\par
\resizebox{\ifdim\width > \linewidth\linewidth\else\width\fi}{!}{%
{\centering%
{\tabularfont%
\begin{tabular}{l}
\alert{Boolean}\tabularnewline\hrulemedium
\end{tabular}
}%
\par}
}%
%
\begin{align*}
A\cdot 0\amp = 0\\
A+0\amp = A\\
A\cdot 1\amp = A\\
A+1\amp = 1
\end{align*}
%
\end{sbspanel}%
\end{sidebyside}%
%
\end{assemblage}
 It is interesting to note that the first three Boolean identity laws look like laws of normal arithmetic with numbers.  The fourth law \emph{does not} resemble any arithmetic law.  This is a clue that Boolean algebra is \emph{not} equivalent to the algebra associated with the real numbers.%
\end{subsectionptx}
%
%
\typeout{************************************************}
\typeout{Subsection 2.5.2 Idempotent Laws}
\typeout{************************************************}
%
\begin{subsectionptx}{Idempotent Laws}{}{Idempotent Laws}{}{}{x:subsection:ssec-idempotent-laws}
We deduce the \terminology{idempotent laws} in the same way. \index{laws of logic!idempotent laws}\index{idempotent laws} Consider the following table, given using logic notation: \begin{center}%
{\tabularfont%
\begin{tabular}{Bccc}\hrulethick
\multicolumn{1}{BcB}{\(p\)}&\multicolumn{1}{cB}{\(p{\wedge} p\)}&\multicolumn{1}{cB}{\(p{\vee} p\)}\tabularnewline\hrulemedium
\multicolumn{1}{BcB}{0}&\multicolumn{1}{cB}{0}&\multicolumn{1}{cB}{0}\tabularnewline\hrulemedium
\multicolumn{1}{BcB}{1}&\multicolumn{1}{cB}{1}&\multicolumn{1}{cB}{1}\tabularnewline\hrulethick
\end{tabular}
}%
\end{center}%
 Boolean algebra notation could have been used in this table, instead.%
\par
From this we can deduce the following: \begin{assemblage}{Idempotent Laws.}{x:assemblage:assemblage-idempotent-laws}%
\begin{sidebyside}{2}{0.2}{0.2}{0.2}%
\begin{sbspanel}{0.2}%
\resizebox{\ifdim\width > \linewidth\linewidth\else\width\fi}{!}{%
{\centering%
{\tabularfont%
\begin{tabular}{l}
\alert{Logic}\tabularnewline\hrulemedium
\end{tabular}
}%
\par}
}%
%
\begin{align*}
p{\wedge} p \amp\Leftrightarrow p\\
p{\vee} p\amp\Leftrightarrow p
\end{align*}
%
\end{sbspanel}%
\begin{sbspanel}{0.2}%
\par
\resizebox{\ifdim\width > \linewidth\linewidth\else\width\fi}{!}{%
{\centering%
{\tabularfont%
\begin{tabular}{l}
\alert{Boolean}\tabularnewline\hrulemedium
\end{tabular}
}%
\par}
}%
%
\begin{align*}
A\cdot A\amp = A\\
A+A\amp = A
\end{align*}
%
\end{sbspanel}%
\end{sidebyside}%
%
\end{assemblage}
%
\end{subsectionptx}
%
%
\typeout{************************************************}
\typeout{Subsection 2.5.3 Complement Laws}
\typeout{************************************************}
%
\begin{subsectionptx}{Complement Laws}{}{Complement Laws}{}{}{x:subsection:ssec-complement-laws}
Similarly, truth tables can be used to deduce the \terminology{complement laws}: \index{laws of logic!complement laws}\index{complement laws} \begin{assemblage}{Complement Laws.}{x:assemblage:assemblage-complement-laws}%
\begin{sidebyside}{2}{0.2}{0.2}{0.2}%
\begin{sbspanel}{0.2}%
\resizebox{\ifdim\width > \linewidth\linewidth\else\width\fi}{!}{%
{\centering%
{\tabularfont%
\begin{tabular}{l}
\alert{Logic}\tabularnewline\hrulemedium
\end{tabular}
}%
\par}
}%
%
\begin{align*}
\sim\!(\sim\!{p}) \amp\Leftrightarrow p\\
p\,{\wedge}\,\sim\!{p}\amp\Leftrightarrow 0\\
p\,{\vee}\,\sim\!{p}\amp\Leftrightarrow 1
\end{align*}
%
\end{sbspanel}%
\begin{sbspanel}{0.2}%
\par
\resizebox{\ifdim\width > \linewidth\linewidth\else\width\fi}{!}{%
{\centering%
{\tabularfont%
\begin{tabular}{l}
\alert{Boolean}\tabularnewline\hrulemedium
\end{tabular}
}%
\par}
}%
%
\begin{align*}
\overline{\overline{A}}\amp =A\\
A\cdot\overline{A}\amp = 0\\
A+\overline{A}\amp = 1
\end{align*}
%
\end{sbspanel}%
\end{sidebyside}%
%
\end{assemblage}
%
\end{subsectionptx}
%
%
\typeout{************************************************}
\typeout{Subsection 2.5.4 Commutative Laws}
\typeout{************************************************}
%
\begin{subsectionptx}{Commutative Laws}{}{Commutative Laws}{}{}{x:subsection:ssec-commutative-laws}
The \terminology{commutative laws} state that the ordering of symbols is not significant for the logic operations \emph{and} and \emph{or}, and also for the corresponding operations in Boolean algebra.  The commutative law does not apply to all operations.  More on that later. \index{laws of logic!commutative laws}\index{commutative laws} \begin{assemblage}{Commutative Laws.}{x:assemblage:assemblage-commutative-laws}%
\begin{sidebyside}{2}{0.2}{0.2}{0.2}%
\begin{sbspanel}{0.2}%
\resizebox{\ifdim\width > \linewidth\linewidth\else\width\fi}{!}{%
{\centering%
{\tabularfont%
\begin{tabular}{l}
\alert{Logic}\tabularnewline\hrulemedium
\end{tabular}
}%
\par}
}%
%
\begin{align*}
p{\wedge} q \amp\Leftrightarrow q{\wedge} p\\
p {\vee} q\amp\Leftrightarrow q{\vee} p
\end{align*}
%
\end{sbspanel}%
\begin{sbspanel}{0.2}%
\par
\resizebox{\ifdim\width > \linewidth\linewidth\else\width\fi}{!}{%
{\centering%
{\tabularfont%
\begin{tabular}{l}
\alert{Boolean}\tabularnewline\hrulemedium
\end{tabular}
}%
\par}
}%
%
\begin{align*}
A\cdot B\amp =B\cdot A\\
A+B\amp =B+A
\end{align*}
%
\end{sbspanel}%
\end{sidebyside}%
%
\end{assemblage}
%
\end{subsectionptx}
%
%
\typeout{************************************************}
\typeout{Subsection 2.5.5 Associative Laws}
\typeout{************************************************}
%
\begin{subsectionptx}{Associative Laws}{}{Associative Laws}{}{}{x:subsection:ssec-associative-laws}
\hyperlink{x:exercise:exer-logic-commutative}{Exercise~{\xreffont 2.3.3.16}, p.\,\pageref{x:exercise:exer-logic-commutative}} illustrates the \terminology{associative law} for \emph{or}.  You could construct a similar truth table for \emph{and}.  The associative laws in Boolean algebra form look just like the associative laws of multiplication and addition of real numbers.  Remember, though, that Boolean algebra is not the same as the algebra of real numbers! \index{laws of logic!associative laws}\index{associative laws} \begin{assemblage}{Associative Laws.}{x:assemblage:assemblage-associative-laws}%
\begin{sidebyside}{2}{0.1}{0.1}{0.2}%
\begin{sbspanel}{0.3}%
\resizebox{\ifdim\width > \linewidth\linewidth\else\width\fi}{!}{%
{\centering%
{\tabularfont%
\begin{tabular}{l}
\alert{Logic}\tabularnewline\hrulemedium
\end{tabular}
}%
\par}
}%
%
\begin{align*}
(p{\wedge} q){\wedge} r \amp\Leftrightarrow p{\wedge}(q{\wedge} r)\\
(p{\vee} q){\vee} r\amp\Leftrightarrow p{\vee}(q{\vee} r)
\end{align*}
%
\end{sbspanel}%
\begin{sbspanel}{0.3}%
\par
\resizebox{\ifdim\width > \linewidth\linewidth\else\width\fi}{!}{%
{\centering%
{\tabularfont%
\begin{tabular}{l}
\alert{Boolean}\tabularnewline\hrulemedium
\end{tabular}
}%
\par}
}%
%
\begin{align*}
(A\cdot B)\cdot C\amp = A\cdot(B\cdot C)\\
(A+B)+C\amp =A+(B+C)
\end{align*}
%
\end{sbspanel}%
\end{sidebyside}%
%
\end{assemblage}
%
\end{subsectionptx}
%
%
\typeout{************************************************}
\typeout{Subsection 2.5.6 Summary So Far}
\typeout{************************************************}
%
\begin{subsectionptx}{Summary So Far}{}{Summary So Far}{}{}{x:subsection:ssec-basic-logic-laws}
Here are all the laws identified so far, collected into one table for easy reference.  In the next subsection, we will examine how these laws are used to simplify logical\slash{}Boolean expressions \emph{without} having to resort to constructing truth tables. \begin{center}%
{\tabularfont%
\begin{tabular}{Bccc}\hrulethick
\multicolumn{1}{BcB}{Law}&\multicolumn{1}{cB}{Logic}&\multicolumn{1}{cB}{Boolean Algebra}\tabularnewline\hrulemedium
\multicolumn{1}{BcB}{Identity}&\multicolumn{1}{cB}{\(p{\wedge} 1\Leftrightarrow p\)}&\multicolumn{1}{cB}{\(A\cdot 1=A\)}\tabularnewline[0pt]
\multicolumn{1}{BcB}{}&\multicolumn{1}{cB}{\(p{\vee} 1\Leftrightarrow 1\)}&\multicolumn{1}{cB}{\(A+1=1\)}\tabularnewline[0pt]
\multicolumn{1}{BcB}{}&\multicolumn{1}{cB}{\(p{\wedge} 0\Leftrightarrow 0\)}&\multicolumn{1}{cB}{\(A\cdot 0=0\)}\tabularnewline[0pt]
\multicolumn{1}{BcB}{}&\multicolumn{1}{cB}{\(p{\vee} 0\Leftrightarrow p\)}&\multicolumn{1}{cB}{\(A+0=A\)}\tabularnewline\hrulemedium
\multicolumn{1}{BcB}{Idempotent}&\multicolumn{1}{cB}{\(p{\wedge} p\Leftrightarrow p\)}&\multicolumn{1}{cB}{\(A\cdot A=A\)}\tabularnewline[0pt]
\multicolumn{1}{BcB}{}&\multicolumn{1}{cB}{\(p{\vee} p\Leftrightarrow p\)}&\multicolumn{1}{cB}{\(A+A=A\)}\tabularnewline\hrulemedium
\multicolumn{1}{BcB}{Complement}&\multicolumn{1}{cB}{\(\sim\!(\sim\!{p})\Leftrightarrow p\)}&\multicolumn{1}{cB}{\(\overline{\overline{A}}=A\)}\tabularnewline[0pt]
\multicolumn{1}{BcB}{}&\multicolumn{1}{cB}{\(p\,{\wedge}\,\sim\!{p}\Leftrightarrow 0 \)}&\multicolumn{1}{cB}{\(A\cdot\overline{A}=0\)}\tabularnewline[0pt]
\multicolumn{1}{BcB}{}&\multicolumn{1}{cB}{\(p\,{\vee}\,\sim\!{p}\Leftrightarrow 1\)}&\multicolumn{1}{cB}{\(A+\overline{A}=1\)}\tabularnewline\hrulemedium
\multicolumn{1}{BcB}{Commutative}&\multicolumn{1}{cB}{\(p{\wedge} q\Leftrightarrow q{\wedge} p\)}&\multicolumn{1}{cB}{\(A\cdot B=B\cdot A\)}\tabularnewline[0pt]
\multicolumn{1}{BcB}{}&\multicolumn{1}{cB}{\(p{\vee} q\Leftrightarrow q{\vee} p\)}&\multicolumn{1}{cB}{\(A+B=B+A\)}\tabularnewline\hrulemedium
\multicolumn{1}{BcB}{Associative}&\multicolumn{1}{cB}{\((p{\wedge} q){\wedge} r\Leftrightarrow p{\wedge}(q{\wedge} r)\)}&\multicolumn{1}{cB}{\((A\cdot B)\cdot C=A\cdot(B\cdot C)\)}\tabularnewline[0pt]
\multicolumn{1}{BcB}{}&\multicolumn{1}{cB}{\((p{\vee} q){\vee} r\Leftrightarrow p{\vee}(q{\vee} r)\)}&\multicolumn{1}{cB}{\((A+B)+C=A+(B+C)\)}\tabularnewline\hrulethick
\end{tabular}
}%
\end{center}%
%
\end{subsectionptx}
%
%
\typeout{************************************************}
\typeout{Subsection 2.5.7 Simplifying Logical Expressions}
\typeout{************************************************}
%
\begin{subsectionptx}{Simplifying Logical Expressions}{}{Simplifying Logical Expressions}{}{}{x:subsection:ssec-simplifying-logical-expr1}
\index{logical equivalence!using laws of logic}%
\index{logical proposition!simplifying with laws of logic}%
We use the laws of logic to simplify logical expressions or to prove logical equivalence without resorting to truth tables.%
\par
Suppose we wish to simplify \((p{\wedge} q){\vee}(\sim\!{q}\,{\wedge}\, 0){\vee}(\sim\!{r}\,{\wedge}\,r)\).  Note that this would require a truth table with 8 rows to show all combinations of \(p\), \(q\), and \(r\).  Less work is required to simplify the expression using the laws of logic.  You can think of the laws of logic as summary statements of truth tables.%
\par
The procedure for simplifying an expression using the laws of logic is to simplify each piece of the expression using a single law, then write the name of the law you are using to one side. Writing the name of the law is \alert{required}, and \alert{not optional}!  If you are using more than one law, then use a separate line for each law\slash{}step.%
\par
Let's use the laws of logic to simplify the statement given earlier: \begin{center}%
{\tabularfont%
\begin{tabular}{cc}
\((p{\wedge} 1){\vee}(\sim\!{q}\,{\wedge}\, 0){\vee}(\sim\!{r}\,{\wedge} r)\)&\multicolumn{1}{l}{}\tabularnewline[0pt]
\(p{\vee} 0{\vee}(\sim\!{r}\,{\wedge}\,r)\)&\multicolumn{1}{l}{identity}\tabularnewline[0pt]
\((p{\vee} 0){\vee}(\sim\!{r}\,{\wedge}\,r)\)&\multicolumn{1}{l}{associative}\tabularnewline[0pt]
\(p{\vee}(\sim\!{r}\,{\wedge}\,r)\)&\multicolumn{1}{l}{identity}\tabularnewline[0pt]
\(p{\vee}(r\,{\wedge}\,\sim\!{r})\)&\multicolumn{1}{l}{commutative}\tabularnewline[0pt]
\(p{\vee} 0\)&\multicolumn{1}{l}{complement}\tabularnewline[0pt]
\(p\)&\multicolumn{1}{l}{identity}
\end{tabular}
}%
\end{center}%
 So we see that \((p{\wedge} q){\vee}(\sim\!{q}\,{\wedge}\, 0){\vee}(\sim\!{r}\,{\wedge}\,r)\Leftrightarrow p\).%
\par
The simplification can be done another way: \begin{center}%
{\tabularfont%
\begin{tabular}{cc}
\((p{\wedge} 1){\vee}(\sim\!{q}\,{\wedge} \,0){\vee}(\sim\!{r}\,{\wedge}\,r)\)&\tabularnewline[0pt]
\(p{\vee} 0{\vee}(\sim\!{r}\,{\wedge}\,r)\)&\multicolumn{1}{l}{identity}\tabularnewline[0pt]
\(p{\vee} 0{\vee} 0\)&\multicolumn{1}{l}{complement}\tabularnewline[0pt]
\(p{\vee} 0\)&\multicolumn{1}{l}{definition of \emph{or}}\tabularnewline[0pt]
\(p\)&\multicolumn{1}{l}{identity}
\end{tabular}
}%
\end{center}%
 The conclusion is the same, as we should expect!%
\begin{example}{}{x:example:examp-simpl-logic-laws}%
Simplify \((p\,{\vee}\,\sim\!{p}){\wedge}(\sim\!{p}\,{\vee}\,\sim\!{p})\).\par\smallskip%
\noindent\textbf{\blocktitlefont Answer}.\label{g:answer:idp228611432}{}\hypertarget{g:answer:idp228611432}{}\quad{}\begin{center}%
{\tabularfont%
\begin{tabular}{cc}
\((p\,{\vee}\sim\!{p}){\wedge}(\sim\!{p}\,{\vee}\sim\!{p})\)&\multicolumn{1}{l}{}\tabularnewline[0pt]
\(1{\wedge}(\sim\!{p}\,{\vee}\,\sim\!{p})\)&\multicolumn{1}{l}{complement}\tabularnewline[0pt]
\(1{\wedge}(\sim\!{p})\)&\multicolumn{1}{l}{idempotent}\tabularnewline[0pt]
\(\sim\!{p}\)&\multicolumn{1}{l}{identity}
\end{tabular}
}%
\end{center}%
 (If you applied the laws correctly but in a different order or combination, you should still come to the same, correct conclusion.)\end{example}
\end{subsectionptx}
%
%
\typeout{************************************************}
\typeout{Exercises 2.5.8 Exercises}
\typeout{************************************************}
%
\begin{exercises-subsection}{Exercises}{}{Exercises}{}{}{x:exercises:exers-sec-logic-laws}
\begin{divisionexercise}{1}{}{}{g:exercise:idp228614120}%
Which of the following statements is always true? %
\begin{enumerate}[label=(\alph*)]
\item{}\(\ \)Darth Vader is both evil and not evil.%
\item{}\(\ \)Darth Vader is both evil and evil.%
\item{}\(\ \)Darth Vader is either evil or evil.%
\item{}\(\ \)Darth Vader is either evil or not evil.%
\end{enumerate}
\end{divisionexercise}%
\begin{divisionexercise}{2}{}{}{g:exercise:idp228617832}%
Which of the following statements is always false? %
\begin{enumerate}[label=(\alph*)]
\item{}\(\ \)The roadrunner has escaped from the wily coyote and he has not escaped from the wily coyote.%
\item{}\(\ \)The roadrunner has escaped from the wily coyote and he has escaped from the wily coyote.%
\item{}\(\ \)The roadrunner has escaped from the wily coyote or he has not escaped from the wily coyote.%
\item{}\(\ \)The roadrunner has escaped from the wily coyote or he has escaped from the wily coyote.%
\end{enumerate}
\end{divisionexercise}%
\begin{divisionexercise}{3}{}{}{g:exercise:idp228625896}%
Use a truth table to prove each of the two idempotent laws.\end{divisionexercise}%
\begin{divisionexercise}{4}{}{}{g:exercise:idp228636648}%
Use a truth table to prove the four identity laws.\end{divisionexercise}%
\par\medskip\noindent%
\textbf{Exercise Group.}\space\space%
Name the law of logic used in the following.  Note that the variable names have been changed, but that the law is still valid.\begin{exercisegroup}
\begin{divisionexerciseeg}{5}{}{}{g:exercise:idp228646120}%
\(\ \sim\!{q}{\vee} 1\Leftrightarrow 1\)\end{divisionexerciseeg}%
\begin{divisionexerciseeg}{6}{}{}{g:exercise:idp228648424}%
\(\ \overline{\overline{B}}=B\)\end{divisionexerciseeg}%
\begin{divisionexerciseeg}{7}{}{}{g:exercise:idp228645352}%
\(\ \sim\!{r}{\wedge} r\Leftrightarrow 0\)\end{divisionexerciseeg}%
\begin{divisionexerciseeg}{8}{}{}{g:exercise:idp228649320}%
\(\ \sim\!{q}{\vee} 0\Leftrightarrow\sim\!{q}\)\end{divisionexerciseeg}%
\begin{divisionexerciseeg}{9}{}{}{g:exercise:idp228656872}%
\(\ \overline{B}\cdot 1=\overline{B}\)\end{divisionexerciseeg}%
\begin{divisionexerciseeg}{10}{}{}{g:exercise:idp228661480}%
\(\ q{\vee} q\Leftrightarrow q\)\end{divisionexerciseeg}%
\begin{divisionexerciseeg}{11}{}{}{g:exercise:idp228660456}%
\(\ AB+\overline{AB}=1\)\end{divisionexerciseeg}%
\begin{divisionexerciseeg}{12}{}{}{g:exercise:idp228655464}%
\(\ (\sim\!{p}{\wedge} q){\wedge}\,\sim\!{q}\Leftrightarrow\sim\!{p}{\wedge}(q\,{\wedge}\sim\!{q})\)\end{divisionexerciseeg}%
\end{exercisegroup}
\par\medskip\noindent
\par\medskip\noindent%
\textbf{Exercise Group.}\space\space%
Simplify the given expression, and state the name of the law you used.  You should be able to do these in one step.\begin{exercisegroup}
\begin{divisionexerciseeg}{13}{}{}{g:exercise:idp228659560}%
\(\ r{\vee} 0\)\end{divisionexerciseeg}%
\begin{divisionexerciseeg}{14}{}{}{g:exercise:idp228661224}%
\(\ C+\overline{C}\)\end{divisionexerciseeg}%
\begin{divisionexerciseeg}{15}{}{}{g:exercise:idp228654440}%
\(\ \sim\!(\sim\!{r})\)\end{divisionexerciseeg}%
\begin{divisionexerciseeg}{16}{}{}{g:exercise:idp228654696}%
\(\ \overline{A}+\overline{A}\)\end{divisionexerciseeg}%
\begin{divisionexerciseeg}{17}{}{}{g:exercise:idp228668520}%
\(\ \overline{B}\cdot 1\)\end{divisionexerciseeg}%
\end{exercisegroup}
\par\medskip\noindent
\par\medskip\noindent%
\textbf{Exercise Group.}\space\space%
A note about the solutions provided for the remaining problems: There may be several different ways to simplify.  Also, you might take steps in a different order.  If you are concerned about the solutions given in this textbook, contact your instructor.  (Also, the steps involving the commutative and associative laws are not explicitly written.)%
\par
Use the laws of logic to simplify the following logical expressions.  If you're completely stuck, try using a truth table instead.%
\begin{exercisegroup}
\begin{divisionexerciseeg}{18}{}{}{g:exercise:idp228665064}%
\(\ (p{\wedge} p){\vee}(q\,{\wedge}\sim\!{q})\)\end{divisionexerciseeg}%
\begin{divisionexerciseeg}{19}{}{}{g:exercise:idp228668392}%
\(\ (p{\vee} p){\wedge} (q{\vee} 0)\)\end{divisionexerciseeg}%
\begin{divisionexerciseeg}{20}{}{}{g:exercise:idp228677480}%
\(\ p{\vee}(q\,{\wedge}\sim\!{q})\)\end{divisionexerciseeg}%
\end{exercisegroup}
\par\medskip\noindent
\par\medskip\noindent%
\textbf{Exercise Group.}\space\space%
Use the laws of logic to simplify the following Boolean expressions.  If you're completely stuck, try using a truth table instead.\begin{exercisegroup}
\begin{divisionexerciseeg}{21}{}{}{g:exercise:idp228675048}%
\(\ (A+A)\cdot(B+\overline{B})\)\end{divisionexerciseeg}%
\begin{divisionexerciseeg}{22}{}{}{g:exercise:idp228678120}%
\(\ B\cdot 0+A\cdot A\)\end{divisionexerciseeg}%
\begin{divisionexerciseeg}{23}{}{}{g:exercise:idp228690152}%
\(\ (B+\overline{B})(A+1)\)\end{divisionexerciseeg}%
\begin{divisionexerciseeg}{24}{}{}{g:exercise:idp228693864}%
\(\ AB\overline{B}\) (Remember, \(AB\) and \(A\cdot B\) are two different ways to write the same expression.)\end{divisionexerciseeg}%
\end{exercisegroup}
\par\medskip\noindent
\par\medskip\noindent%
\textbf{Exercise Group.}\space\space%
Prove the following Boolean expressions are equivalent using the laws of logic. On each line of your proof, manipulate only one side of the statement unless you are performing the same manipulation on both sides. If you're completely stuck, try using a truth table.\begin{exercisegroup}
\begin{divisionexerciseeg}{25}{}{}{g:exercise:idp228699368}%
\(\ (A\overline{A})\overline{B}=A(B\overline{B})\)\end{divisionexerciseeg}%
\begin{divisionexerciseeg}{26}{}{}{g:exercise:idp228704104}%
\(\ B\cdot 1+A\overline{A}=\overline{\overline{B}\cdot 1}\)\end{divisionexerciseeg}%
\begin{divisionexerciseeg}{27}{}{}{g:exercise:idp228717800}%
\(\ (A+0)(B+\overline{B})=A\)\end{divisionexerciseeg}%
\begin{divisionexerciseeg}{28}{}{}{g:exercise:idp228725352}%
\(\ AA+\overline{B}\,\overline{B}=A+\overline{B}\)\end{divisionexerciseeg}%
\end{exercisegroup}
\par\medskip\noindent
\end{exercises-subsection}
%
%
\typeout{************************************************}
\typeout{Solutions 2.5.9 Solutions to Section~{\xreffont\ref*{x:section:sec-logic-laws}} Exercises.}
\typeout{************************************************}
%
\begin{solutions-subsection}{Solutions to Section~{\xreffont\ref*{x:section:sec-logic-laws}} Exercises.}{}{Solutions to Section~{\xreffont\ref*{x:section:sec-logic-laws}} Exercises.}{}{}{g:solutions:idp228734184}
\par\medskip
\noindent\textbf{\normalsize{}2.5.8\space\textperiodcentered\space{}Exercises}
\begin{divisionsolution}{2.5.8.1}{}{g:exercise:idp228614120}%
\par\smallskip%
\noindent\hypertarget{g:solution:idp228619496-main}{}Statement (d) is true because \(p\,{\vee} \sim\!{p}\Leftrightarrow 1\).  Note that you must assess each statement based on its logical structure, and not on the truth value of the instance.\end{divisionsolution}%
\begin{divisionsolution}{2.5.8.2}{}{g:exercise:idp228617832}%
\par\smallskip%
\noindent\hypertarget{g:solution:idp228624744-main}{}Statement (a) is, symbolically, \(p\,{\wedge}\,\sim\!{p}\), which is logically equivalent to 0. So statement (a) is false regardless of whether roadrunner has or has not escaped from the wily coyote.\end{divisionsolution}%
\begin{divisionsolution}{2.5.8.3}{}{g:exercise:idp228625896}%
\par\smallskip%
\noindent\hypertarget{g:solution:idp228628072-main}{}The proof amounts to comparing the first and last columns of each of these two tables: \begin{sidebyside}{2}{0.15}{0.15}{0.3}%
\begin{sbspanel}{0.2}%
\resizebox{\ifdim\width > \linewidth\linewidth\else\width\fi}{!}{%
{\centering%
{\tabularfont%
\begin{tabular}{Bccc}\hrulethick
\multicolumn{1}{BcB}{\(p\)}&\multicolumn{1}{cB}{\(p\)}&\multicolumn{1}{cB}{\(p{\vee} p\)}\tabularnewline\hrulemedium
\multicolumn{1}{BcB}{0}&\multicolumn{1}{cB}{0}&\multicolumn{1}{cB}{0}\tabularnewline[0pt]
\multicolumn{1}{BcB}{1}&\multicolumn{1}{cB}{1}&\multicolumn{1}{cB}{1}\tabularnewline\hrulethick
\end{tabular}
}%
\par}
}%
\end{sbspanel}%
\begin{sbspanel}{0.2}%
\resizebox{\ifdim\width > \linewidth\linewidth\else\width\fi}{!}{%
{\centering%
{\tabularfont%
\begin{tabular}{Bccc}\hrulethick
\multicolumn{1}{BcB}{\(p\)}&\multicolumn{1}{cB}{\(p\)}&\multicolumn{1}{cB}{\(p{\wedge} p\)}\tabularnewline\hrulemedium
\multicolumn{1}{BcB}{0}&\multicolumn{1}{cB}{0}&\multicolumn{1}{cB}{0}\tabularnewline[0pt]
\multicolumn{1}{BcB}{1}&\multicolumn{1}{cB}{1}&\multicolumn{1}{cB}{1}\tabularnewline\hrulethick
\end{tabular}
}%
\par}
}%
\end{sbspanel}%
\end{sidebyside}%
\end{divisionsolution}%
\begin{divisionsolution}{2.5.8.4}{}{g:exercise:idp228636648}%
\par\smallskip%
\noindent\hypertarget{g:solution:idp228635496-main}{}The four identity laws follow from the following observations:  The \(p{\wedge} 0\) column is the same as \(0\), the \(p{\vee} 0\) and the \(p{\wedge} 1\) columns are the same as \(p\), and the \(p{\vee} 1\) column is the same as \(1\). \begin{center}%
{\tabularfont%
\begin{tabular}{Bccccccc}\hrulethick
\multicolumn{1}{BcB}{\(p\)}&\multicolumn{1}{cB}{\(0\)}&\multicolumn{1}{cB}{\(1\)}&\multicolumn{1}{cB}{\(p{\wedge} 0\)}&\multicolumn{1}{cB}{\(p{\vee} 0\)}&\multicolumn{1}{cB}{\(p{\wedge} 1\)}&\multicolumn{1}{cB}{\(p{\vee} 1\)}\tabularnewline\hrulemedium
\multicolumn{1}{BcB}{0}&\multicolumn{1}{cB}{0}&\multicolumn{1}{cB}{1}&\multicolumn{1}{cB}{0}&\multicolumn{1}{cB}{0}&\multicolumn{1}{cB}{0}&\multicolumn{1}{cB}{1}\tabularnewline[0pt]
\multicolumn{1}{BcB}{1}&\multicolumn{1}{cB}{0}&\multicolumn{1}{cB}{1}&\multicolumn{1}{cB}{0}&\multicolumn{1}{cB}{1}&\multicolumn{1}{cB}{1}&\multicolumn{1}{cB}{1}\tabularnewline\hrulethick
\end{tabular}
}%
\end{center}%
\end{divisionsolution}%
\begin{exercisegroup}
\begin{divisionsolutioneg}{2.5.8.5}{}{g:exercise:idp228646120}%
\par\smallskip%
\noindent\hypertarget{g:solution:idp228647144-main}{}Identity\end{divisionsolutioneg}%
\begin{divisionsolutioneg}{2.5.8.6}{}{g:exercise:idp228648424}%
\par\smallskip%
\noindent\hypertarget{g:solution:idp228653032-main}{}Complement\end{divisionsolutioneg}%
\begin{divisionsolutioneg}{2.5.8.7}{}{g:exercise:idp228645352}%
\par\smallskip%
\noindent\hypertarget{g:solution:idp228649064-main}{}Complement\end{divisionsolutioneg}%
\begin{divisionsolutioneg}{2.5.8.8}{}{g:exercise:idp228649320}%
\par\smallskip%
\noindent\hypertarget{g:solution:idp228658408-main}{}Identity\end{divisionsolutioneg}%
\begin{divisionsolutioneg}{2.5.8.9}{}{g:exercise:idp228656872}%
\par\smallskip%
\noindent\hypertarget{g:solution:idp228659176-main}{}Identity\end{divisionsolutioneg}%
\begin{divisionsolutioneg}{2.5.8.10}{}{g:exercise:idp228661480}%
\par\smallskip%
\noindent\hypertarget{g:solution:idp228660200-main}{}Idempotent\end{divisionsolutioneg}%
\begin{divisionsolutioneg}{2.5.8.11}{}{g:exercise:idp228660456}%
\par\smallskip%
\noindent\hypertarget{g:solution:idp228657128-main}{}Complement\end{divisionsolutioneg}%
\begin{divisionsolutioneg}{2.5.8.12}{}{g:exercise:idp228655464}%
\par\smallskip%
\noindent\hypertarget{g:solution:idp228659432-main}{}Associative\end{divisionsolutioneg}%
\end{exercisegroup}
\par\medskip\noindent
\begin{exercisegroup}
\begin{divisionsolutioneg}{2.5.8.13}{}{g:exercise:idp228659560}%
\par\smallskip%
\noindent\hypertarget{g:solution:idp228655848-main}{}\(\ r\), identity\end{divisionsolutioneg}%
\begin{divisionsolutioneg}{2.5.8.14}{}{g:exercise:idp228661224}%
\par\smallskip%
\noindent\hypertarget{g:solution:idp228653672-main}{}\(\ 1\), complement\end{divisionsolutioneg}%
\begin{divisionsolutioneg}{2.5.8.15}{}{g:exercise:idp228654440}%
\par\smallskip%
\noindent\hypertarget{g:solution:idp228659048-main}{}\(\ r\), complement\end{divisionsolutioneg}%
\begin{divisionsolutioneg}{2.5.8.16}{}{g:exercise:idp228654696}%
\par\smallskip%
\noindent\hypertarget{g:solution:idp228656616-main}{}\(\ \overline{A}\), idempotent\end{divisionsolutioneg}%
\begin{divisionsolutioneg}{2.5.8.17}{}{g:exercise:idp228668520}%
\par\smallskip%
\noindent\hypertarget{g:solution:idp228663912-main}{}\(\ \overline{B}\), identity\end{divisionsolutioneg}%
\end{exercisegroup}
\par\medskip\noindent
\begin{exercisegroup}
\begin{divisionsolutioneg}{2.5.8.18}{}{g:exercise:idp228665064}%
\par\smallskip%
\noindent\hypertarget{g:solution:idp228666472-main}{}\begin{center}%
{\tabularfont%
\begin{tabular}{lll}
\multicolumn{1}{c}{\((p{\wedge} p){\vee}(q\,{\wedge}\sim\!{q})\)}&\(\Leftrightarrow p{\vee}(q\,{\wedge}\sim\!{q})\)&Idempotent\tabularnewline[0pt]
&\(\Leftrightarrow p{\vee} 0\)&Complement\tabularnewline[0pt]
&\(p\)&Identity
\end{tabular}
}%
\end{center}%
\end{divisionsolutioneg}%
\begin{divisionsolutioneg}{2.5.8.19}{}{g:exercise:idp228668392}%
\par\smallskip%
\noindent\hypertarget{g:solution:idp228669288-main}{}\begin{center}%
{\tabularfont%
\begin{tabular}{lll}
\multicolumn{1}{c}{\((p{\vee} p){\wedge}(q{\vee} 0)\)}&\(\Leftrightarrow p{\wedge}(q{\vee} 0)\)&Idempotent\tabularnewline[0pt]
&\(\Leftrightarrow p{\wedge} q\)&Identity
\end{tabular}
}%
\end{center}%
\end{divisionsolutioneg}%
\begin{divisionsolutioneg}{2.5.8.20}{}{g:exercise:idp228677480}%
\par\smallskip%
\noindent\hypertarget{g:solution:idp228672360-main}{}\begin{center}%
{\tabularfont%
\begin{tabular}{lll}
\multicolumn{1}{c}{\(p{\vee}(q\,{\wedge}\sim\!{q})\)}&\(\Leftrightarrow p{\vee} 0\)&Complement\tabularnewline[0pt]
&\(\Leftrightarrow p\)&Identity
\end{tabular}
}%
\end{center}%
\end{divisionsolutioneg}%
\end{exercisegroup}
\par\medskip\noindent
\begin{exercisegroup}
\begin{divisionsolutioneg}{2.5.8.21}{}{g:exercise:idp228675048}%
\par\smallskip%
\noindent\hypertarget{g:solution:idp228675944-main}{}\begin{center}%
{\tabularfont%
\begin{tabular}{lll}
\multicolumn{1}{c}{\((A+A)(B+\overline{B})\)}&\(= A(B+\overline{B})\)&Idempotent\tabularnewline[0pt]
&\(= A\cdot 1\)&Complement\tabularnewline[0pt]
&\(= A\)&Identity
\end{tabular}
}%
\end{center}%
\end{divisionsolutioneg}%
\begin{divisionsolutioneg}{2.5.8.22}{}{g:exercise:idp228678120}%
\par\smallskip%
\noindent\hypertarget{g:solution:idp228682856-main}{}\begin{center}%
{\tabularfont%
\begin{tabular}{lll}
\multicolumn{1}{c}{\(B\cdot 0+AA\)}&\(= 0+AA\)&Identity\tabularnewline[0pt]
&\(= 0+A\)&Idempotent\tabularnewline[0pt]
&\(= A\)&Identity
\end{tabular}
}%
\end{center}%
\end{divisionsolutioneg}%
\begin{divisionsolutioneg}{2.5.8.23}{}{g:exercise:idp228690152}%
\par\smallskip%
\noindent\hypertarget{g:solution:idp228690664-main}{}\begin{center}%
{\tabularfont%
\begin{tabular}{lll}
\multicolumn{1}{c}{\((B+\overline{B})(A+1)\)}&\(= 1\cdot(A+1)\)&Complement\tabularnewline[0pt]
&\(= 1\cdot 1\)&Identity\tabularnewline[0pt]
&\(= 1\)&Definition of \emph{and}.
\end{tabular}
}%
\end{center}%
\end{divisionsolutioneg}%
\begin{divisionsolutioneg}{2.5.8.24}{}{g:exercise:idp228693864}%
\par\smallskip%
\noindent\hypertarget{g:solution:idp228686696-main}{}\begin{center}%
{\tabularfont%
\begin{tabular}{lll}
\multicolumn{1}{c}{\(AB\overline{B}\)}&\(=A\cdot 0\)&Complement\tabularnewline[0pt]
&\(=0\)&Identity
\end{tabular}
}%
\end{center}%
\end{divisionsolutioneg}%
\end{exercisegroup}
\par\medskip\noindent
\begin{exercisegroup}
\begin{divisionsolutioneg}{2.5.8.25}{}{g:exercise:idp228699368}%
\par\smallskip%
\noindent\hypertarget{g:solution:idp228699496-main}{}\begin{center}%
{\tabularfont%
\begin{tabular}{lll}
\multicolumn{1}{r}{\((A\overline{A})\overline{B}\)}&\(=A(B\overline{B})\)&\tabularnewline[0pt]
\multicolumn{1}{r}{\(0\cdot\overline{B}\)}&\(=A(B\overline{B})\)&Complement\tabularnewline[0pt]
\multicolumn{1}{r}{\(0\cdot\overline{B}\)}&\(=A\cdot 0\)&Complement\tabularnewline[0pt]
\multicolumn{1}{r}{\(0\)}&\(=A\cdot 0\)&Identity\tabularnewline[0pt]
\multicolumn{1}{r}{\(0\)}&\(=0\)&Identity
\end{tabular}
}%
\end{center}%
\end{divisionsolutioneg}%
\begin{divisionsolutioneg}{2.5.8.26}{}{g:exercise:idp228704104}%
\par\smallskip%
\noindent\hypertarget{g:solution:idp228705000-main}{}\begin{center}%
{\tabularfont%
\begin{tabular}{lll}
\multicolumn{1}{r}{\(B\cdot 1+A\overline{A}\)}&\(=\overline{\overline{B}\cdot 1}\)&\tabularnewline[0pt]
\multicolumn{1}{r}{\(B+A\overline{A}\)}&\(=\overline{\overline{B}}\)&Identity\tabularnewline[0pt]
\multicolumn{1}{r}{\(B+0\)}&\(=\overline{\overline{B}}\)&Complement\tabularnewline[0pt]
\multicolumn{1}{r}{\(B\)}&\(=\overline{\overline{B}}\)&Identity\tabularnewline[0pt]
\multicolumn{1}{r}{\(B\)}&\(=B\)&Complement
\end{tabular}
}%
\end{center}%
\end{divisionsolutioneg}%
\begin{divisionsolutioneg}{2.5.8.27}{}{g:exercise:idp228717800}%
\par\smallskip%
\noindent\hypertarget{g:solution:idp228712680-main}{}\begin{center}%
{\tabularfont%
\begin{tabular}{lll}
\multicolumn{1}{r}{\((A+0)(B+\overline{B})\)}&\(=A\)&\tabularnewline[0pt]
\multicolumn{1}{r}{\(A(B+\overline{B})\)}&\(=A\)&Identity\tabularnewline[0pt]
\multicolumn{1}{r}{\(A\cdot 1\)}&\(=A\)&Complement\tabularnewline[0pt]
\multicolumn{1}{r}{\(A\)}&\(=A\)&Identity
\end{tabular}
}%
\end{center}%
\end{divisionsolutioneg}%
\begin{divisionsolutioneg}{2.5.8.28}{}{g:exercise:idp228725352}%
\par\smallskip%
\noindent\hypertarget{g:solution:idp228726376-main}{}\begin{center}%
{\tabularfont%
\begin{tabular}{lll}
\multicolumn{1}{r}{\(AA+B\overline{B}\)}&\(=A+\overline{B}\)&\tabularnewline[0pt]
\multicolumn{1}{r}{\(A+\overline{B}\)}&\(=A+\overline{B}\)&Idempotent
\end{tabular}
}%
\end{center}%
\end{divisionsolutioneg}%
\end{exercisegroup}
\par\medskip\noindent
\end{solutions-subsection}
\end{sectionptx}
%
%
\typeout{************************************************}
\typeout{Section 2.6 More Laws of Logic}
\typeout{************************************************}
%
\begin{sectionptx}{More Laws of Logic}{}{More Laws of Logic}{}{}{x:section:sec-more-logic-laws}
%
%
\typeout{************************************************}
\typeout{Subsection 2.6.1 De Morgan's Laws}
\typeout{************************************************}
%
\begin{subsectionptx}{De Morgan's Laws}{}{De Morgan's Laws}{}{}{x:subsection:ssec-demorgans-laws}
Consider the following truth table: \begin{center}%
{\tabularfont%
\begin{tabular}{Bccccccc}\hrulethick
\multicolumn{1}{BcB}{\(p\)}&\multicolumn{1}{cB}{\(q\)}&\multicolumn{1}{cB}{\(p{\wedge} q\)}&\multicolumn{1}{cB}{\(\sim\!(p{\wedge} q)\)}&\multicolumn{1}{cB}{\(\sim\!{p}\)}&\multicolumn{1}{cB}{\(\sim\!{q}\)}&\multicolumn{1}{cB}{\(\sim\!{p}\,{\vee}\,\sim\!{q}\)}\tabularnewline\hrulemedium
\multicolumn{1}{BcB}{0}&\multicolumn{1}{cB}{0}&\multicolumn{1}{cB}{0}&\multicolumn{1}{cB}{1}&\multicolumn{1}{cB}{1}&\multicolumn{1}{cB}{1}&\multicolumn{1}{cB}{1}\tabularnewline[0pt]
\multicolumn{1}{BcB}{0}&\multicolumn{1}{cB}{1}&\multicolumn{1}{cB}{0}&\multicolumn{1}{cB}{1}&\multicolumn{1}{cB}{1}&\multicolumn{1}{cB}{0}&\multicolumn{1}{cB}{1}\tabularnewline[0pt]
\multicolumn{1}{BcB}{1}&\multicolumn{1}{cB}{0}&\multicolumn{1}{cB}{0}&\multicolumn{1}{cB}{1}&\multicolumn{1}{cB}{0}&\multicolumn{1}{cB}{1}&\multicolumn{1}{cB}{1}\tabularnewline[0pt]
\multicolumn{1}{BcB}{1}&\multicolumn{1}{cB}{1}&\multicolumn{1}{cB}{1}&\multicolumn{1}{cB}{0}&\multicolumn{1}{cB}{0}&\multicolumn{1}{cB}{0}&\multicolumn{1}{cB}{0}\tabularnewline\hrulethick
\end{tabular}
}%
\end{center}%
%
\par
From this table we can see that%
\begin{equation*}
\sim\!(p{\wedge} q)\Leftrightarrow\sim\!{p}\,{\vee}\sim\!{q}.
\end{equation*}
We can construct a similar table to show that%
\begin{equation*}
\sim\!(p{\vee} q)\Leftrightarrow\sim\!{p}\,{\wedge}\sim\!{q}.
\end{equation*}
These are called \terminology{De Morgan's Laws}. \index{laws of logic!De Morgan's laws}\index{De Morgan's laws} Constructing the tables using Boolean algebra notation yields similar statements.%
\par
\begin{assemblage}{De Morgan's Laws.}{x:assemblage:assemblage-demorgans-laws}%
\begin{sidebyside}{2}{0.15}{0.15}{0.1}%
\begin{sbspanel}{0.3}%
\resizebox{\ifdim\width > \linewidth\linewidth\else\width\fi}{!}{%
{\centering%
{\tabularfont%
\begin{tabular}{l}
\alert{Logic}\tabularnewline\hrulemedium
\end{tabular}
}%
\par}
}%
%
\begin{align*}
\sim\!(p{\wedge} q)\amp\Leftrightarrow\,\sim\!{p}\,{\vee}\sim\!{q}\\
\sim\!(p{\vee} q)\amp\Leftrightarrow\,\sim\!{p}\,{\wedge}\sim\!{q}.
\end{align*}
%
\end{sbspanel}%
\begin{sbspanel}{0.3}%
\par
\resizebox{\ifdim\width > \linewidth\linewidth\else\width\fi}{!}{%
{\centering%
{\tabularfont%
\begin{tabular}{l}
\alert{Boolean}\tabularnewline\hrulemedium
\end{tabular}
}%
\par}
}%
%
\begin{align*}
\overline{A\cdot B}\amp =\overline{A}+\overline{B}\\
\overline{A+B}\amp =\overline{A}\cdot\overline{B}
\end{align*}
%
\end{sbspanel}%
\end{sidebyside}%
%
\end{assemblage}
%
\end{subsectionptx}
%
%
\typeout{************************************************}
\typeout{Subsection 2.6.2 Distributive Laws}
\typeout{************************************************}
%
\begin{subsectionptx}{Distributive Laws}{}{Distributive Laws}{}{}{x:subsection:ssec-distributive-laws}
In the same way, we can construct truth tables to prove the \terminology{distributive laws}, in both logic and Boolean algebra notation. \index{laws of logic!distributive laws}\index{distributive laws} Note carefully the forms of these laws. \begin{assemblage}{Distributive Laws.}{x:assemblage:assemblage-distributive-laws}%
\begin{sidebyside}{2}{0.1}{0.1}{0.2}%
\begin{sbspanel}{0.3}%
\resizebox{\ifdim\width > \linewidth\linewidth\else\width\fi}{!}{%
{\centering%
{\tabularfont%
\begin{tabular}{l}
\alert{Logic}\tabularnewline\hrulemedium
\end{tabular}
}%
\par}
}%
%
\begin{align*}
p{\wedge}(q{\vee} r)\amp\Leftrightarrow(p{\wedge} q){\vee}(p{\wedge} r)\\
p{\vee}(q{\wedge} r)\amp\Leftrightarrow(p{\vee} q){\wedge}(p{\vee} r)
\end{align*}
%
\end{sbspanel}%
\begin{sbspanel}{0.3}%
\par
\resizebox{\ifdim\width > \linewidth\linewidth\else\width\fi}{!}{%
{\centering%
{\tabularfont%
\begin{tabular}{l}
\alert{Boolean}\tabularnewline\hrulemedium
\end{tabular}
}%
\par}
}%
%
\begin{align*}
A(B+C)\amp =AB+AC\\
A+BC\amp =(A+B)(A+C)
\end{align*}
%
\end{sbspanel}%
\end{sidebyside}%
%
\end{assemblage}
 The first of the distributive laws in Boolean notation looks like the usual one from arithmetic with real numbers.  The second one, however, looks decidedly \emph{unlike} any rules from arithmetic.  You must not let the familiar rules of arithmetic distract you from the laws of logic.%
\end{subsectionptx}
%
%
\typeout{************************************************}
\typeout{Subsection 2.6.3 Absorption Laws}
\typeout{************************************************}
%
\begin{subsectionptx}{Absorption Laws}{}{Absorption Laws}{}{}{x:subsection:ssec-absorption-laws}
The \terminology{absorption laws} can be verified using truth tables, or by use of the laws of logic already stated. \index{laws of logic!absorption laws}\index{absorption laws} \begin{assemblage}{Absorption Laws.}{x:assemblage:assemblage-absorption-laws}%
\begin{sidebyside}{2}{0.1}{0.1}{0.2}%
\begin{sbspanel}{0.3}%
\resizebox{\ifdim\width > \linewidth\linewidth\else\width\fi}{!}{%
{\centering%
{\tabularfont%
\begin{tabular}{l}
\alert{Logic}\tabularnewline\hrulemedium
\end{tabular}
}%
\par}
}%
%
\begin{align*}
p{\wedge}(p{\vee} q)\amp\Leftrightarrow p\\
p{\wedge}(\sim\!{p}\,{\vee}\,q)\amp\Leftrightarrow\,p{\wedge} q\\
p{\vee}(p{\wedge} q)\amp\Leftrightarrow\,p\\
p{\vee}(\sim\!{p}\,{\wedge}\,q)\amp\Leftrightarrow\,p{\vee} q
\end{align*}
%
\end{sbspanel}%
\begin{sbspanel}{0.3}%
\par
\resizebox{\ifdim\width > \linewidth\linewidth\else\width\fi}{!}{%
{\centering%
{\tabularfont%
\begin{tabular}{l}
\alert{Boolean}\tabularnewline\hrulemedium
\end{tabular}
}%
\par}
}%
%
\begin{align*}
A(A+B)\amp=A\\
A(\overline{A}+B)\amp=AB\\
A+AB\amp=A\\
A+\overline{A}{}B\amp=A+B
\end{align*}
%
\end{sbspanel}%
\end{sidebyside}%
%
\end{assemblage}
%
\end{subsectionptx}
%
%
\typeout{************************************************}
\typeout{Subsection 2.6.4 Summary So Far}
\typeout{************************************************}
%
\begin{subsectionptx}{Summary So Far}{}{Summary So Far}{}{}{x:subsection:ssec-summary-logic-laws}
Here are all the laws of logic discussed so far: \index{laws of logic!summary} \begin{center}%
{\tabularfont%
\begin{tabular}{Bccc}\hrulethick
\multicolumn{1}{BcB}{Law}&\multicolumn{1}{cB}{Logic}&\multicolumn{1}{cB}{Boolean Algebra}\tabularnewline\hrulemedium
\multicolumn{1}{BcB}{Identity}&\multicolumn{1}{cB}{\(p {\wedge}{} 1 {\Leftrightarrow}{} p\)}&\multicolumn{1}{cB}{\(A \cdot 1 = A\)}\tabularnewline[0pt]
\multicolumn{1}{BcB}{}&\multicolumn{1}{cB}{\(p {\vee}{}1 {\Leftrightarrow}{}  1\)}&\multicolumn{1}{cB}{\(A+1 = 1\)}\tabularnewline[0pt]
\multicolumn{1}{BcB}{}&\multicolumn{1}{cB}{\(p {\wedge}{} 0 {\Leftrightarrow}{}  0\)}&\multicolumn{1}{cB}{\(A \cdot 0 = 0\)}\tabularnewline[0pt]
\multicolumn{1}{BcB}{}&\multicolumn{1}{cB}{\(p {\vee}{}0 {\Leftrightarrow}{}  p\)}&\multicolumn{1}{cB}{\(A+0 = A\)}\tabularnewline\hrulemedium
\multicolumn{1}{BcB}{Idempotent}&\multicolumn{1}{cB}{\(p {\wedge}{} p {\Leftrightarrow}{}  p\)}&\multicolumn{1}{cB}{\(AA = A\)}\tabularnewline[0pt]
\multicolumn{1}{BcB}{}&\multicolumn{1}{cB}{\(p {\vee}{} p {\Leftrightarrow}{}  p\)}&\multicolumn{1}{cB}{\(A+A = A\)}\tabularnewline\hrulemedium
\multicolumn{1}{BcB}{Complement}&\multicolumn{1}{cB}{\(\sim\sim\!p  {\Leftrightarrow}{} p\)}&\multicolumn{1}{cB}{\(\overline{\overline{A}{}} = A\)}\tabularnewline[0pt]
\multicolumn{1}{BcB}{}&\multicolumn{1}{cB}{\(p {\wedge}{} \sim\!{p}  {\Leftrightarrow}{}  0\)}&\multicolumn{1}{cB}{\(A\overline{A}{} = 0\)}\tabularnewline[0pt]
\multicolumn{1}{BcB}{}&\multicolumn{1}{cB}{\(p {\vee}{}\sim\!{p}  {\Leftrightarrow}{}  1\)}&\multicolumn{1}{cB}{\(A+ \overline{A}{} = 1\)}\tabularnewline\hrulemedium
\multicolumn{1}{BcB}{Commutative}&\multicolumn{1}{cB}{\(p {\wedge}{} q {\Leftrightarrow}{}  q {\wedge}{} p\)}&\multicolumn{1}{cB}{\(AB = BA\)}\tabularnewline[0pt]
\multicolumn{1}{BcB}{}&\multicolumn{1}{cB}{\(p {\vee}{}q {\Leftrightarrow}{}   q {\vee}{} p\)}&\multicolumn{1}{cB}{\(A+ B = B + A\)}\tabularnewline\hrulemedium
\multicolumn{1}{BcB}{Associative}&\multicolumn{1}{cB}{\((p {\wedge}{} q) {\wedge}{} r {\Leftrightarrow}{}  p {\wedge}{} (q {\wedge}{} r) \)}&\multicolumn{1}{cB}{\((AB)C = A(BC) \)}\tabularnewline[0pt]
\multicolumn{1}{BcB}{}&\multicolumn{1}{cB}{\((p {\vee}{} q) {\vee}{} r {\Leftrightarrow}{}  p {\vee}{} (q {\vee}{} r) \)}&\multicolumn{1}{cB}{\((A+B)+C = A+(B+C) \)}\tabularnewline\hrulemedium
\multicolumn{1}{BcB}{De Morgan's}&\multicolumn{1}{cB}{\(\sim\!{(p {\wedge}{} q)} {\Leftrightarrow}{} \sim\!{p}{} {\vee}{} \sim\!{q}{}\)}&\multicolumn{1}{cB}{\(\overline{AB} = \overline{A}{} + \overline{B}{}\)}\tabularnewline[0pt]
\multicolumn{1}{BcB}{}&\multicolumn{1}{cB}{\(\sim\!{(p {\vee}{} q)} {\Leftrightarrow}{} \sim\!{p}{} {\wedge}{} \sim\!{q}{}\)}&\multicolumn{1}{cB}{\(\overline{A+B} = \overline{A}{} ~ \overline{B}{}\)}\tabularnewline\hrulemedium
\multicolumn{1}{BcB}{Distributive}&\multicolumn{1}{cB}{\(p{\wedge}{}(q{\vee}{}r) {\Leftrightarrow}{} (p{\wedge}{}q){\vee}{}(p{\wedge}{}r)\)}&\multicolumn{1}{cB}{\(A(B+C) = AB + AC\)}\tabularnewline[0pt]
\multicolumn{1}{BcB}{}&\multicolumn{1}{cB}{\(p{\vee}{}(q{\wedge}{}r) {\Leftrightarrow}{} (p{\vee}{}q){\wedge}{}(p{\vee}{}r)\)}&\multicolumn{1}{cB}{\(A+BC = (A+B)(A+C)\)}\tabularnewline\hrulemedium
\multicolumn{1}{BcB}{Absorption}&\multicolumn{1}{cB}{\(p{\wedge}{}(p{\vee}{}q) {\Leftrightarrow}{} p\)}&\multicolumn{1}{cB}{\(A(A+B) = A\)}\tabularnewline[0pt]
\multicolumn{1}{BcB}{}&\multicolumn{1}{cB}{\(p{\wedge}{}(\sim\!{p}{}{\vee}{}q) {\Leftrightarrow}{}  p{\wedge}{}q\)}&\multicolumn{1}{cB}{\(A(\overline{A}{} + B) = AB\)}\tabularnewline[0pt]
\multicolumn{1}{BcB}{}&\multicolumn{1}{cB}{\(p{\vee}{}(p{\wedge}{}q) {\Leftrightarrow}{}  p\)}&\multicolumn{1}{cB}{\(A + AB = A\)}\tabularnewline[0pt]
\multicolumn{1}{BcB}{}&\multicolumn{1}{cB}{\(p{\vee}{}(\sim\!{p}{}{\wedge}{}q) {\Leftrightarrow}{}  p{\vee}{}q\)}&\multicolumn{1}{cB}{\(A+ \overline{A}{}B = A+ B \)}\tabularnewline\hrulethick
\end{tabular}
}%
\end{center}%
 The next sub-section contains examples of the usage of these laws.%
\end{subsectionptx}
%
%
\typeout{************************************************}
\typeout{Subsection 2.6.5 Simplifying Logical Expressions}
\typeout{************************************************}
%
\begin{subsectionptx}{Simplifying Logical Expressions}{}{Simplifying Logical Expressions}{}{}{x:subsection:ssec-simplifying-logical-expr2}
\index{logical equivalence!using laws of logic}%
\index{logical proposition!simplifying with laws of logic}%
The clarity of your simplification or of your proof is greatly improved if you use only one law in each line of your work, and also state the law you have used.  To help you build good habits, this is \alert{required} in your coursework.  It is not merely a suggestion.%
\begin{example}{}{x:example:exmpl-simpl-dist1}%
Simplify \((p{\wedge}{}q){\vee}(p{\wedge}\sim\!{q})\).\par\smallskip%
\noindent\textbf{\blocktitlefont Answer}.\label{g:answer:idp228875240}{}\hypertarget{g:answer:idp228875240}{}\quad{}\begin{center}%
{\tabularfont%
\begin{tabular}{cc}
\((p{\wedge}{}q) {\vee}{}(p{\wedge}{}\sim\!{q}{})\)&\multicolumn{1}{l}{}\tabularnewline[0pt]
\(p{\wedge}{}(q{\vee}{}\sim\!{q}{}) \)&\multicolumn{1}{l}{distributive}\tabularnewline[0pt]
\(p{\wedge}{}1 \)&\multicolumn{1}{l}{complement}\tabularnewline[0pt]
\(p  \)&\multicolumn{1}{l}{identity}
\end{tabular}
}%
\end{center}%
\end{example}
\begin{example}{}{x:example:exmpl-simpl-dist2}%
Simplify \(AB(\overline{A}{} + \overline{B}{} )\).\par\smallskip%
\noindent\textbf{\blocktitlefont Answer}.\label{g:answer:idp228883688}{}\hypertarget{g:answer:idp228883688}{}\quad{}\begin{center}%
{\tabularfont%
\begin{tabular}{cc}
\(AB(\overline{A}{} +\overline{B}{})\)&\multicolumn{1}{l}{}\tabularnewline[0pt]
\(AB\overline{A}{}  + AB\overline{B}{}   \)&\multicolumn{1}{l}{distributive}\tabularnewline[0pt]
\(A\overline{A}{} B + AB\overline{B}{}  \)&\multicolumn{1}{l}{commutative}\tabularnewline[0pt]
\(0 \cdot B      + A\cdot 0  \)&\multicolumn{1}{l}{complement}\tabularnewline[0pt]
\(0        +       0    \)&\multicolumn{1}{l}{identity}\tabularnewline[0pt]
\(0            \)&\multicolumn{1}{l}{definition of ``or''}
\end{tabular}
}%
\end{center}%
\end{example}
Note that for many of these exercises, there is more than one way to answer.  Another equally valivd simplification of the statement in the previous example looks like the following: \begin{center}%
{\tabularfont%
\begin{tabular}{cc}
\(AB(\overline{A}{} +\overline{B}{})\)&\multicolumn{1}{l}{}\tabularnewline[0pt]
\(AB   \overline{AB} \)&\multicolumn{1}{l}{De Morgan's}\tabularnewline[0pt]
\(0 \)&\multicolumn{1}{l}{complement}
\end{tabular}
}%
\end{center}%
 This is a much shorter answer, but does require a flash of insight at the \((\overline{A} + \overline{B})\) pattern.%
\end{subsectionptx}
%
%
\typeout{************************************************}
\typeout{Subsection 2.6.6 Proofs}
\typeout{************************************************}
%
\begin{subsectionptx}{Proofs}{}{Proofs}{}{}{x:subsection:ssec-proofs}
\index{proofs!using laws of logic}%
\index{laws of logic!proofs}%
\index{proposition!proofs}%
When tasked to prove that two logic statements are equivalent, look to see if one of the statements is simpler than the other.  Try to simplify the more complicated statement until it is identical to the simpler statement.  Sometimes, you will have to simplify both statements into identical forms.%
\begin{example}{}{x:example:exmpl-proof1}%
Show that \(\overline{A+\overline{A}\,\overline{B}}=\overline{A}B\).\par\smallskip%
\noindent\textbf{\blocktitlefont Answer}.\label{g:answer:idp228901480}{}\hypertarget{g:answer:idp228901480}{}\quad{}Let's examine the left-had side and simplify it. \begin{center}%
{\tabularfont%
\begin{tabular}{llll}
\multicolumn{1}{r}{\(\overline{A+\overline{A}\,\overline{B}}\)}&\(=\)&\(\overline{A}B\)&\tabularnewline[0pt]
\multicolumn{1}{r}{\(\overline{A+\overline{B}}\)}&\(=\)&\(\overline{A}B\)&absorption\tabularnewline[0pt]
\multicolumn{1}{r}{\(\overline{A}\,\overline{\overline{B}}\)}&\(=\)&\(\overline{A}B\)&De Morgan's\tabularnewline[0pt]
\multicolumn{1}{r}{\(\overline{A}B\)}&\(=\)&\(\overline{A}B\)&complement
\end{tabular}
}%
\end{center}%
The left-hand side has been shown to be equivalent to the origintal right-hand side.  This completes the proof.%
\end{example}
\begin{example}{}{x:example:exmpl-proof2}%
Show that \((A+\overline{C})(\overline{C}+AB)=AB+\overline{C}\).\par\smallskip%
\noindent\textbf{\blocktitlefont Answer}.\label{g:answer:idp228913256}{}\hypertarget{g:answer:idp228913256}{}\quad{}The left side is more complicated, so let's examine it to see what simplification is possible. \begin{center}%
{\tabularfont%
\begin{tabular}{llll}
\multicolumn{1}{r}{\((A+\overline{C})(\overline{C}+AB)\)}&\(=\)&\(AB+\overline{C}\)&\tabularnewline[0pt]
\multicolumn{1}{r}{\((\overline{C}+A)(\overline{C}+AB)\)}&\(=\)&\(AB+\overline{C}\)&commutative\tabularnewline[0pt]
\multicolumn{1}{r}{\(\overline{C}+A(AB)\)}&\(=\)&\(AB=\overline{C}\)&distributive\tabularnewline[0pt]
\multicolumn{1}{r}{\(\overline{C}+(AA)B\)}&\(=\)&\(AB=\overline{C}\)&associative\tabularnewline[0pt]
\multicolumn{1}{r}{\(\overline{C}+AB\)}&\(=\)&\(AB+\overline{C}\)&idempotent\tabularnewline[0pt]
\multicolumn{1}{r}{\(AB+\overline{C}\)}&\(=\)&\(AB+\overline{C}\)&commutative
\end{tabular}
}%
\end{center}%
 QED.  (QED is short for the Latin phrase \emph{quod erat demonstrandum}, which means \emph{it has been demonstrated}.)\end{example}
\end{subsectionptx}
%
%
\typeout{************************************************}
\typeout{Exercises 2.6.7 Exercises}
\typeout{************************************************}
%
\begin{exercises-subsection}{Exercises}{}{Exercises}{}{}{x:exercises:exers-sec-more-logic-laws}
\par\medskip\noindent%
\textbf{Exercise Group.}\space\space%
Let \(p\) be the statement \emph{Rich is seven feet tall} and \(q\) be \emph{Susan has brown hair.}  Translate the following English sentences into logical notation.  Then, use one of the laws of logic to write an equivalent logical expression.  Finally, translate your new expression back into an English sentence.\begin{exercisegroup}
\begin{divisionexerciseeg}{1}{}{}{g:exercise:idp228928488}%
\(\ \)Rich is seven feet tall or he is seven feet tall.\end{divisionexerciseeg}%
\begin{divisionexerciseeg}{2}{}{}{g:exercise:idp228930408}%
\(\ \)Susan has brown hair and she has brown hair.\end{divisionexerciseeg}%
\begin{divisionexerciseeg}{3}{}{}{g:exercise:idp228924008}%
\(\ \)Either Rich is not seven feet tall or Susan does not have brown hair.\end{divisionexerciseeg}%
\begin{divisionexerciseeg}{4}{}{}{g:exercise:idp228934760}%
\(\ \)It is not true that Rich is seven feet tall and Susan has brown\end{divisionexerciseeg}%
\begin{divisionexerciseeg}{5}{}{}{g:exercise:idp228937576}%
\(\ \)It is not true that Rich is seven feet tall or Susan has brown hair.\end{divisionexerciseeg}%
\begin{divisionexerciseeg}{6}{}{}{g:exercise:idp228936680}%
\(\ \)Rich is not seven feet tall and Susan does not have brown hair.\end{divisionexerciseeg}%
\begin{divisionexerciseeg}{7}{}{}{g:exercise:idp228936168}%
\(\ \)Rich is seven feet tall and Susan has brown hair.\end{divisionexerciseeg}%
\begin{divisionexerciseeg}{8}{}{}{g:exercise:idp228935912}%
\(\ \)Susan has brown hair or Rich is seven feet tall.\end{divisionexerciseeg}%
\end{exercisegroup}
\par\medskip\noindent
\par\medskip\noindent%
\textbf{Exercise Group.}\space\space%
Name the law of logic used in the following.  note that the variables have changed, but that the law is still valid.\begin{exercisegroup}
\begin{divisionexerciseeg}{9}{}{}{g:exercise:idp228936808}%
\(\ g\sim\!{(q {\vee}{} r)} {\Leftrightarrow} \sim\!{q}{} {\wedge}{} \sim\!{r}{}\)\end{divisionexerciseeg}%
\begin{divisionexerciseeg}{10}{}{}{g:exercise:idp228938600}%
\(\ \overline{B}{} (B + \overline{A}{}) = \overline{B}{}~\overline{A}{} \)\end{divisionexerciseeg}%
\begin{divisionexerciseeg}{11}{}{}{g:exercise:idp228944104}%
\(\ (p {\wedge}{} q) {\vee}{} (p {\wedge} \sim\!{q}{}) {\Leftrightarrow}{} p {\wedge}{} (q {\vee}{}\sim\!{q}{})\)\end{divisionexerciseeg}%
\begin{divisionexerciseeg}{12}{}{}{g:exercise:idp228945384}%
\(\ \overline{\overline{A}{} + C} = A\overline{C}{}\)\end{divisionexerciseeg}%
\begin{divisionexerciseeg}{13}{}{}{g:exercise:idp228944744}%
\(\ B + A\overline{C}{} =(B+A)(B+\overline{C}{})\)\end{divisionexerciseeg}%
\begin{divisionexerciseeg}{14}{}{}{g:exercise:idp228945000}%
\(\ \sim\!{p}{} {\vee}{} (p {\wedge}{} r) {\Leftrightarrow}{} \sim\!{p}{}{\vee}{}r\)\end{divisionexerciseeg}%
\end{exercisegroup}
\par\medskip\noindent
\par\medskip\noindent%
\textbf{Exercise Group.}\space\space%
Simplify the given expression, and state the name of the law you used.  You should be able to do these in a single step.\begin{exercisegroup}
\begin{divisionexerciseeg}{15}{}{}{g:exercise:idp228942952}%
\(\ \overline{A}{}+ A\overline{B}{}\)\end{divisionexerciseeg}%
\begin{divisionexerciseeg}{16}{}{}{g:exercise:idp228942312}%
\(\ \overline{AB} + \overline{AB}\)\end{divisionexerciseeg}%
\begin{divisionexerciseeg}{17}{}{}{g:exercise:idp228946920}%
\(\ (A+B)(B+C)\)\end{divisionexerciseeg}%
\begin{divisionexerciseeg}{18}{}{}{g:exercise:idp228941288}%
\(\ q {\vee}{}(q {\wedge}{} r)\)\end{divisionexerciseeg}%
\begin{divisionexerciseeg}{19}{}{}{g:exercise:idp228943208}%
\(\ \overline{C}{} + C\)\end{divisionexerciseeg}%
\begin{divisionexerciseeg}{20}{}{}{g:exercise:idp228948584}%
\(\ \overline{\overline{A}{} ~ \overline{B}{}}\)\end{divisionexerciseeg}%
\end{exercisegroup}
\par\medskip\noindent
\par\medskip\noindent%
\textbf{Exercise Group.}\space\space%
For the following exercises, let \(p\) be \emph{The moon is made of green cheese} and \(q\) be \emph{The earth is made of green cheese.}  Translate the following English sentences into logical notation.  Then, use one of the laws of logic to write an equivalent logical expression.  Finally, translate your new expression back into an English sentence.\begin{exercisegroup}
\begin{divisionexerciseeg}{21}{}{}{g:exercise:idp228956136}%
\(\ \)Either the moon is made of green cheese or both the moon and the earth are made of green cheese.\end{divisionexerciseeg}%
\begin{divisionexerciseeg}{22}{}{}{g:exercise:idp228951912}%
\(\ \)The earth is made of green cheese and either the earth or the moon is made of green cheese.\end{divisionexerciseeg}%
\begin{divisionexerciseeg}{23}{}{}{g:exercise:idp228953704}%
\(\ \)Either the earth is made of green cheese while the moon is not, or the moon is made of green cheese.\end{divisionexerciseeg}%
\begin{divisionexerciseeg}{24}{}{}{g:exercise:idp228949480}%
\(\ \)The earth is made of green cheese and either the moon is made of green cheese or the earth is not.\end{divisionexerciseeg}%
\begin{divisionexerciseeg}{25}{}{}{g:exercise:idp228959976}%
\(\ \)Remembering that \({\oplus}{}\) is \emph{exclusive or}, show that \(A{\oplus}{}B = \overline{A}{}B + A\overline{B}{}\) by using a truth table.\end{divisionexerciseeg}%
\begin{divisionexerciseeg}{26}{}{}{g:exercise:idp228978664}%
\(\ \)The \mono{NAND} gate (not-AND) has the following truth table.  Use DeMorgan's laws to find an equivalent Boolean expression using only OR and NOT, and show that your expression has the same truth table. \begin{center}%
{\tabularfont%
\begin{tabular}{ccc}\hrulethick
\(A\)&\(B\)&\(A\)\mono{NAND}\(B=\overline{AB}\)\tabularnewline[0pt]
0&0&1\tabularnewline[0pt]
0&1&1\tabularnewline[0pt]
1&0&1\tabularnewline[0pt]
1&1&0\tabularnewline\hrulethick
\end{tabular}
}%
\end{center}%
\end{divisionexerciseeg}%
\end{exercisegroup}
\par\medskip\noindent
\par\medskip\noindent%
\textbf{Exercise Group.}\space\space%
Simplify the following Boolean expressions using the laws of logic.  If you're stuck, try using a truth table.\begin{exercisegroup}
\begin{divisionexerciseeg}{27}{}{}{g:exercise:idp228998888}%
\(\ A + \overline{C}{} + B + \overline{A}{} + \overline{B}{}\)\end{divisionexerciseeg}%
\begin{divisionexerciseeg}{28}{}{}{g:exercise:idp229011304}%
\(\ A + \overline{B}{} + A + B + A\)\end{divisionexerciseeg}%
\begin{divisionexerciseeg}{29}{}{}{g:exercise:idp229006440}%
\(\ \overline{\overline{A}{}\,\overline{B}{}}\)\end{divisionexerciseeg}%
\begin{divisionexerciseeg}{30}{}{}{g:exercise:idp229006696}%
\(\ \overline{\overline{A}{} + \overline{B}{}}\)\end{divisionexerciseeg}%
\begin{divisionexerciseeg}{31}{}{}{g:exercise:idp229005928}%
\(\ \overline{A}{}  + B + A\overline{B}{}\)\end{divisionexerciseeg}%
\begin{divisionexerciseeg}{32}{}{}{g:exercise:idp229007464}%
\(\ A\overline{B}{}~\overline{C}{}  + A\overline{B}{}C\)\end{divisionexerciseeg}%
\begin{divisionexerciseeg}{33}{}{}{g:exercise:idp229012968}%
\(\ \overline{A}{} BC + \overline{A}{}B\overline{C}{}  + \overline{A}{} B\overline{D}{} + \overline{A}{}BD\)\end{divisionexerciseeg}%
\begin{divisionexerciseeg}{34}{}{}{g:exercise:idp229009896}%
\(\ AB + A + \overline{AB}\)\end{divisionexerciseeg}%
\begin{divisionexerciseeg}{35}{}{}{g:exercise:idp229011048}%
\(\ A + \overline{B}{}CD + \overline{B}{}\)\end{divisionexerciseeg}%
\begin{divisionexerciseeg}{36}{}{}{g:exercise:idp229011176}%
\(\ \overline{A}{}~ \overline{B}{} (A + B)\)\end{divisionexerciseeg}%
\begin{divisionexerciseeg}{37}{}{}{g:exercise:idp229013992}%
\(\ (\overline{A}{} + \overline{B}{})(A + B)\)\end{divisionexerciseeg}%
\begin{divisionexerciseeg}{38}{}{}{g:exercise:idp229021800}%
\(\ A + \overline{A}{} B + \overline{B}{}C \)\end{divisionexerciseeg}%
\begin{divisionexerciseeg}{39}{}{}{g:exercise:idp229019880}%
\(\ B(A + C) + \overline{A}{} B\overline{C}{}\)\end{divisionexerciseeg}%
\begin{divisionexerciseeg}{40}{}{}{g:exercise:idp229013864}%
\(\ (A + B + C)(A + B + \overline{C}{})\)\end{divisionexerciseeg}%
\end{exercisegroup}
\par\medskip\noindent
\par\medskip\noindent%
\textbf{Exercise Group.}\space\space%
Prove each of the following using the laws of logic.  If you're stuck, try using a truth table.\begin{exercisegroup}
\begin{divisionexerciseeg}{41}{}{}{g:exercise:idp229015656}%
\(\ B\overline{B}{} + AA = A\)\end{divisionexerciseeg}%
\begin{divisionexerciseeg}{42}{}{}{g:exercise:idp229029480}%
\(\ \overline{A}{} (B + \overline{B}{}) = \overline{A}{}\)\end{divisionexerciseeg}%
\begin{divisionexerciseeg}{43}{}{}{g:exercise:idp229032040}%
\(\ ABC + AB\overline{C}{} = AB\)\end{divisionexerciseeg}%
\begin{divisionexerciseeg}{44}{}{}{g:exercise:idp229053544}%
\(\ AB + \overline{AB}C = AB + C\)\end{divisionexerciseeg}%
\begin{divisionexerciseeg}{45}{}{}{g:exercise:idp229055336}%
\(\ A + AB + ABC = A\)\end{divisionexerciseeg}%
\begin{divisionexerciseeg}{46}{}{}{g:exercise:idp229063144}%
\(\ \overline{A}{} C + A\overline{B}{}C = \overline{A}{} C + \overline{B}{}C\)\end{divisionexerciseeg}%
\begin{divisionexerciseeg}{47}{}{}{g:exercise:idp229092952}%
\(\ \overline{AB}(A + B) = \overline{A}{} B + A\overline{B}{}\)\end{divisionexerciseeg}%
\begin{divisionexerciseeg}{48}{}{}{g:exercise:idp229098712}%
\(\ \overline{\overline{\overline{A}{}BC}+D}=\overline{A}{}BC\overline{D}{}\)\end{divisionexerciseeg}%
\begin{divisionexerciseeg}{49}{}{}{g:exercise:idp229105624}%
\(\ A~\overline{B}{}~\overline{\overline{A}{}~\overline{C}{}}=A~\overline{B}{}\)\end{divisionexerciseeg}%
\end{exercisegroup}
\par\medskip\noindent
\end{exercises-subsection}
%
%
\typeout{************************************************}
\typeout{Solutions 2.6.8 Solutions to Section~{\xreffont\ref*{x:section:sec-more-logic-laws}} Exercises.}
\typeout{************************************************}
%
\begin{solutions-subsection}{Solutions to Section~{\xreffont\ref*{x:section:sec-more-logic-laws}} Exercises.}{}{Solutions to Section~{\xreffont\ref*{x:section:sec-more-logic-laws}} Exercises.}{}{}{g:solutions:idp229124824}
\par\medskip
\noindent\textbf{\normalsize{}2.6.7\space\textperiodcentered\space{}Exercises}
\begin{exercisegroup}
\begin{divisionsolutioneg}{2.6.7.1}{}{g:exercise:idp228928488}%
\par\smallskip%
\noindent\hypertarget{g:solution:idp228928616-main}{}\(p {\vee}{} p  {\Leftrightarrow}{}  p\).  Rich is seven feet tall.\end{divisionsolutioneg}%
\begin{divisionsolutioneg}{2.6.7.2}{}{g:exercise:idp228930408}%
\par\smallskip%
\noindent\hypertarget{g:solution:idp228927208-main}{}\(q  {\wedge}{}  q  {\Leftrightarrow}{}  q\).  Susan has brown hair.\end{divisionsolutioneg}%
\begin{divisionsolutioneg}{2.6.7.3}{}{g:exercise:idp228924008}%
\par\smallskip%
\noindent\hypertarget{g:solution:idp228934504-main}{}\(\sim\!{p}{} {\vee}{} \sim\!{q}{}  {\Leftrightarrow}{}  \sim\!{(p {\wedge}{} q)}\).  It is not the case that Rich is seven feet tall and Susan has brown hair.\end{divisionsolutioneg}%
\begin{divisionsolutioneg}{2.6.7.4}{}{g:exercise:idp228934760}%
\par\smallskip%
\noindent\hypertarget{g:solution:idp228939880-main}{}\(\sim\!{(p {\wedge}{} q)} {\Leftrightarrow}{} \sim\!{p}{} {\vee}{}  \sim\!{q}{}\).  Rich is not seven feet tall or Susan does not have brown hair.\end{divisionsolutioneg}%
\begin{divisionsolutioneg}{2.6.7.5}{}{g:exercise:idp228937576}%
\par\smallskip%
\noindent\hypertarget{g:solution:idp228934376-main}{}\(\sim\!{(p {\vee}{} q)}  {\Leftrightarrow}{}  \sim\!{p}{} {\wedge}{}  \sim\!{q}{}\).  Rich is not seven feet tall and Susan does not have brown hair.\end{divisionsolutioneg}%
\begin{divisionsolutioneg}{2.6.7.6}{}{g:exercise:idp228936680}%
\par\smallskip%
\noindent\hypertarget{g:solution:idp228936552-main}{}\(\sim\!{p}{} {\wedge}{}  \sim\!{q}{}    {\Leftrightarrow}{}  \sim\!{(p {\vee}{} q)}\).  It is not the case that Rich is seven feet tall or Susan has brown hair.\end{divisionsolutioneg}%
\begin{divisionsolutioneg}{2.6.7.7}{}{g:exercise:idp228936168}%
\par\smallskip%
\noindent\hypertarget{g:solution:idp228939240-main}{}\(p {\wedge}{} q  {\Leftrightarrow}{}  q {\wedge}{} p\).  Susan has brown hair and Rich is seven feet tall.\end{divisionsolutioneg}%
\begin{divisionsolutioneg}{2.6.7.8}{}{g:exercise:idp228935912}%
\par\smallskip%
\noindent\hypertarget{g:solution:idp228938728-main}{}\(q {\vee}{} p  {\Leftrightarrow}{}  p {\vee}{} q\).  Rich is seven feet tall or Susan has brown hair.\end{divisionsolutioneg}%
\end{exercisegroup}
\par\medskip\noindent
\begin{exercisegroup}
\begin{divisionsolutioneg}{2.6.7.9}{}{g:exercise:idp228936808}%
\par\smallskip%
\noindent\hypertarget{g:solution:idp228938344-main}{}De Morgan's\end{divisionsolutioneg}%
\begin{divisionsolutioneg}{2.6.7.10}{}{g:exercise:idp228938600}%
\par\smallskip%
\noindent\hypertarget{g:solution:idp228932328-main}{}absorption\end{divisionsolutioneg}%
\begin{divisionsolutioneg}{2.6.7.11}{}{g:exercise:idp228944104}%
\par\smallskip%
\noindent\hypertarget{g:solution:idp228946152-main}{}distributive\end{divisionsolutioneg}%
\begin{divisionsolutioneg}{2.6.7.12}{}{g:exercise:idp228945384}%
\par\smallskip%
\noindent\hypertarget{g:solution:idp228943848-main}{}De Morgan's\end{divisionsolutioneg}%
\begin{divisionsolutioneg}{2.6.7.13}{}{g:exercise:idp228944744}%
\par\smallskip%
\noindent\hypertarget{g:solution:idp228940136-main}{}distributive\end{divisionsolutioneg}%
\begin{divisionsolutioneg}{2.6.7.14}{}{g:exercise:idp228945000}%
\par\smallskip%
\noindent\hypertarget{g:solution:idp228945640-main}{}absorption\end{divisionsolutioneg}%
\end{exercisegroup}
\par\medskip\noindent
\begin{exercisegroup}
\begin{divisionsolutioneg}{2.6.7.15}{}{g:exercise:idp228942952}%
\par\smallskip%
\noindent\hypertarget{g:solution:idp228945128-main}{}\(\ \overline{A}{}+ \overline{B}{}\), absorption\end{divisionsolutioneg}%
\begin{divisionsolutioneg}{2.6.7.16}{}{g:exercise:idp228942312}%
\par\smallskip%
\noindent\hypertarget{g:solution:idp228947944-main}{}\(\ \overline{AB}\), idempotent\end{divisionsolutioneg}%
\begin{divisionsolutioneg}{2.6.7.17}{}{g:exercise:idp228946920}%
\par\smallskip%
\noindent\hypertarget{g:solution:idp228947560-main}{}\(\ B+AC \), distributive\end{divisionsolutioneg}%
\begin{divisionsolutioneg}{2.6.7.18}{}{g:exercise:idp228941288}%
\par\smallskip%
\noindent\hypertarget{g:solution:idp228941544-main}{}\(\ q\), absorption\end{divisionsolutioneg}%
\begin{divisionsolutioneg}{2.6.7.19}{}{g:exercise:idp228943208}%
\par\smallskip%
\noindent\hypertarget{g:solution:idp228954088-main}{}\(\ 1\), complement\end{divisionsolutioneg}%
\begin{divisionsolutioneg}{2.6.7.20}{}{g:exercise:idp228948584}%
\par\smallskip%
\noindent\hypertarget{g:solution:idp228952936-main}{}\(\ A+B\), De Morgan's\end{divisionsolutioneg}%
\end{exercisegroup}
\par\medskip\noindent
\begin{exercisegroup}
\begin{divisionsolutioneg}{2.6.7.21}{}{g:exercise:idp228956136}%
\par\smallskip%
\noindent\hypertarget{g:solution:idp228955368-main}{}\(\ p {\vee}{} (p {\wedge}{} q) {\Leftrightarrow}{}  p\). The moon is made of green cheese.\end{divisionsolutioneg}%
\begin{divisionsolutioneg}{2.6.7.22}{}{g:exercise:idp228951912}%
\par\smallskip%
\noindent\hypertarget{g:solution:idp228952296-main}{}\(\ q {\wedge}{} (q {\vee}{}p) {\Leftrightarrow}{}  q\). The earth is made of green cheese.\end{divisionsolutioneg}%
\begin{divisionsolutioneg}{2.6.7.23}{}{g:exercise:idp228953704}%
\par\smallskip%
\noindent\hypertarget{g:solution:idp228952680-main}{}\(\ (q {\wedge}{} \sim\!{p}{}) {\vee}{} p  {\Leftrightarrow}{}  p {\vee}{} (\sim\!{p}{} {\wedge}{} q) {\Leftrightarrow}{} p {\vee}{} q\). (Note:  I'm using the commutative laws to rearrange things)   The moon or the earth is made of green cheese.\end{divisionsolutioneg}%
\begin{divisionsolutioneg}{2.6.7.24}{}{g:exercise:idp228949480}%
\par\smallskip%
\noindent\hypertarget{g:solution:idp228955112-main}{}\(\ \ q  {\wedge}{}  (p {\vee}{}  \sim\!{q}{}  )  {\Leftrightarrow}{}  q  {\wedge}{}  p\). The earth and the moon are made of green cheese.\end{divisionsolutioneg}%
\begin{divisionsolutioneg}{2.6.7.25}{}{g:exercise:idp228959976}%
\par\smallskip%
\noindent\hypertarget{g:solution:idp228959080-main}{}\begin{center}%
{\tabularfont%
\begin{tabular}{Bcccccccc}\hrulethick
\multicolumn{1}{BcB}{\(A\)}&\multicolumn{1}{cB}{\(B\)}&\multicolumn{1}{cB}{\(A\oplus B\)}&\multicolumn{1}{cB}{\(\overline{A}\)}&\multicolumn{1}{cB}{\(\overline{B}\)}&\multicolumn{1}{cB}{\(\overline{A}B\)}&\multicolumn{1}{cB}{\(A\overline{B}\)}&\multicolumn{1}{cB}{\(\overline{A}B+A\overline{B}\)}\tabularnewline\hrulemedium
\multicolumn{1}{BcB}{0}&\multicolumn{1}{cB}{0}&\multicolumn{1}{cB}{0}&\multicolumn{1}{cB}{1}&\multicolumn{1}{cB}{1}&\multicolumn{1}{cB}{0}&\multicolumn{1}{cB}{0}&\multicolumn{1}{cB}{0}\tabularnewline\hrulemedium
\multicolumn{1}{BcB}{0}&\multicolumn{1}{cB}{1}&\multicolumn{1}{cB}{1}&\multicolumn{1}{cB}{1}&\multicolumn{1}{cB}{0}&\multicolumn{1}{cB}{1}&\multicolumn{1}{cB}{0}&\multicolumn{1}{cB}{1}\tabularnewline\hrulemedium
\multicolumn{1}{BcB}{1}&\multicolumn{1}{cB}{0}&\multicolumn{1}{cB}{1}&\multicolumn{1}{cB}{0}&\multicolumn{1}{cB}{1}&\multicolumn{1}{cB}{0}&\multicolumn{1}{cB}{1}&\multicolumn{1}{cB}{1}\tabularnewline\hrulemedium
\multicolumn{1}{BcB}{1}&\multicolumn{1}{cB}{1}&\multicolumn{1}{cB}{0}&\multicolumn{1}{cB}{0}&\multicolumn{1}{cB}{0}&\multicolumn{1}{cB}{0}&\multicolumn{1}{cB}{0}&\multicolumn{1}{cB}{0}\tabularnewline\hrulethick
\end{tabular}
}%
\end{center}%
\end{divisionsolutioneg}%
\begin{divisionsolutioneg}{2.6.7.26}{}{g:exercise:idp228978664}%
\par\smallskip%
\noindent\hypertarget{g:solution:idp228986472-main}{}By De Morgan's law, \(\overline{AB}=\overline{A}+\overline{B}\)\begin{center}%
{\tabularfont%
\begin{tabular}{Bcccccc}\hrulethick
\multicolumn{1}{BcB}{\(A\)}&\multicolumn{1}{cB}{\(B\)}&\multicolumn{1}{cB}{\(A\)\mono{NAND}\(B=\overline{AB}\)}&\multicolumn{1}{cB}{\(\overline{A}\)}&\multicolumn{1}{cB}{\(\overline{B}\)}&\multicolumn{1}{cB}{\(\overline{A}+\overline{B}\)}\tabularnewline\hrulemedium
\multicolumn{1}{BcB}{0}&\multicolumn{1}{cB}{0}&\multicolumn{1}{cB}{1}&\multicolumn{1}{cB}{1}&\multicolumn{1}{cB}{1}&\multicolumn{1}{cB}{1}\tabularnewline\hrulemedium
\multicolumn{1}{BcB}{0}&\multicolumn{1}{cB}{1}&\multicolumn{1}{cB}{1}&\multicolumn{1}{cB}{1}&\multicolumn{1}{cB}{0}&\multicolumn{1}{cB}{1}\tabularnewline\hrulemedium
\multicolumn{1}{BcB}{1}&\multicolumn{1}{cB}{0}&\multicolumn{1}{cB}{1}&\multicolumn{1}{cB}{0}&\multicolumn{1}{cB}{1}&\multicolumn{1}{cB}{1}\tabularnewline\hrulemedium
\multicolumn{1}{BcB}{1}&\multicolumn{1}{cB}{1}&\multicolumn{1}{cB}{0}&\multicolumn{1}{cB}{0}&\multicolumn{1}{cB}{0}&\multicolumn{1}{cB}{0}\tabularnewline\hrulethick
\end{tabular}
}%
\end{center}%
\end{divisionsolutioneg}%
\end{exercisegroup}
\par\medskip\noindent
\begin{exercisegroup}
\begin{divisionsolutioneg}{2.6.7.27}{}{g:exercise:idp228998888}%
\par\smallskip%
\noindent\hypertarget{g:solution:idp228999528-main}{}\(\ 1\)\end{divisionsolutioneg}%
\begin{divisionsolutioneg}{2.6.7.28}{}{g:exercise:idp229011304}%
\par\smallskip%
\noindent\hypertarget{g:solution:idp229006568-main}{}\(\ 1\)\end{divisionsolutioneg}%
\begin{divisionsolutioneg}{2.6.7.29}{}{g:exercise:idp229006440}%
\par\smallskip%
\noindent\hypertarget{g:solution:idp229009384-main}{}\(\ A+B\)\end{divisionsolutioneg}%
\begin{divisionsolutioneg}{2.6.7.30}{}{g:exercise:idp229006696}%
\par\smallskip%
\noindent\hypertarget{g:solution:idp229006824-main}{}\(\ AB\)\end{divisionsolutioneg}%
\begin{divisionsolutioneg}{2.6.7.31}{}{g:exercise:idp229005928}%
\par\smallskip%
\noindent\hypertarget{g:solution:idp229011816-main}{}\(\ 1\)\end{divisionsolutioneg}%
\begin{divisionsolutioneg}{2.6.7.32}{}{g:exercise:idp229007464}%
\par\smallskip%
\noindent\hypertarget{g:solution:idp229007208-main}{}\(\ A\overline{B}{}\)\end{divisionsolutioneg}%
\begin{divisionsolutioneg}{2.6.7.33}{}{g:exercise:idp229012968}%
\par\smallskip%
\noindent\hypertarget{g:solution:idp229007720-main}{}\(\ \overline{A}{}B\)\end{divisionsolutioneg}%
\begin{divisionsolutioneg}{2.6.7.34}{}{g:exercise:idp229009896}%
\par\smallskip%
\noindent\hypertarget{g:solution:idp229008872-main}{}\(\ 1\)\end{divisionsolutioneg}%
\begin{divisionsolutioneg}{2.6.7.35}{}{g:exercise:idp229011048}%
\par\smallskip%
\noindent\hypertarget{g:solution:idp229010280-main}{}\(\ A+\overline{B}{}\)\end{divisionsolutioneg}%
\begin{divisionsolutioneg}{2.6.7.36}{}{g:exercise:idp229011176}%
\par\smallskip%
\noindent\hypertarget{g:solution:idp229013480-main}{}\(\ 0\)\end{divisionsolutioneg}%
\begin{divisionsolutioneg}{2.6.7.37}{}{g:exercise:idp229013992}%
\par\smallskip%
\noindent\hypertarget{g:solution:idp229018088-main}{}\(\ A\overline{B}{}+\overline{A}{}B\)\end{divisionsolutioneg}%
\begin{divisionsolutioneg}{2.6.7.38}{}{g:exercise:idp229021800}%
\par\smallskip%
\noindent\hypertarget{g:solution:idp229021928-main}{}\(\ A+B+C\)\end{divisionsolutioneg}%
\begin{divisionsolutioneg}{2.6.7.39}{}{g:exercise:idp229019880}%
\par\smallskip%
\noindent\hypertarget{g:solution:idp229021032-main}{}\(\ B\)\end{divisionsolutioneg}%
\begin{divisionsolutioneg}{2.6.7.40}{}{g:exercise:idp229013864}%
\par\smallskip%
\noindent\hypertarget{g:solution:idp229017320-main}{}\(\ A+B\)\end{divisionsolutioneg}%
\end{exercisegroup}
\par\medskip\noindent
\begin{exercisegroup}
\begin{divisionsolutioneg}{2.6.7.41}{}{g:exercise:idp229015656}%
\par\smallskip%
\noindent\hypertarget{g:solution:idp229014248-main}{}\begin{center}%
{\tabularfont%
\begin{tabular}{llll}
\multicolumn{1}{r}{\(B\overline{B}{}+AA\)}&\(=\)&\(A\)&\(\)\tabularnewline[0pt]
\multicolumn{1}{r}{\(0+AA\)}&\(=\)&\(A\)&complement\tabularnewline[0pt]
\multicolumn{1}{r}{\(AA\)}&\(=\)&\(A\)&identity\tabularnewline[0pt]
\multicolumn{1}{r}{\(A\)}&\(=\)&\(A\)&idempotent
\end{tabular}
}%
\end{center}%
\end{divisionsolutioneg}%
\begin{divisionsolutioneg}{2.6.7.42}{}{g:exercise:idp229029480}%
\par\smallskip%
\noindent\hypertarget{g:solution:idp229022312-main}{}\begin{center}%
{\tabularfont%
\begin{tabular}{llll}
\multicolumn{1}{r}{\(\overline{A}{}(B+\overline{B}{})\)}&\(=\)&\(\overline{A}{}\)&\tabularnewline[0pt]
\multicolumn{1}{r}{\(\overline{A}{}(1)\)}&\(=\)&\(\overline{A}{}\)&complement\tabularnewline[0pt]
\multicolumn{1}{r}{\(\overline{A}{}\)}&\(=\)&\(\overline{A}{}\)&identity
\end{tabular}
}%
\end{center}%
\end{divisionsolutioneg}%
\begin{divisionsolutioneg}{2.6.7.43}{}{g:exercise:idp229032040}%
\par\smallskip%
\noindent\hypertarget{g:solution:idp229033064-main}{}\begin{center}%
{\tabularfont%
\begin{tabular}{llll}
\multicolumn{1}{r}{\(ABC+AB\overline{C}{}\)}&\(=\)&\(AB\)&\tabularnewline[0pt]
\multicolumn{1}{r}{\(AB(C+\overline{C}{})\)}&\(=\)&\(AB\)&distributive\tabularnewline[0pt]
\multicolumn{1}{r}{\(AB(1)\)}&\(=\)&\(AB\)&complement\tabularnewline[0pt]
\multicolumn{1}{r}{\(AB\)}&\(=\)&\(AB\)&identity
\end{tabular}
}%
\end{center}%
\end{divisionsolutioneg}%
\begin{divisionsolutioneg}{2.6.7.44}{}{g:exercise:idp229053544}%
\par\smallskip%
\noindent\hypertarget{g:solution:idp229049576-main}{}\begin{center}%
{\tabularfont%
\begin{tabular}{llll}
\multicolumn{1}{r}{\(AB+\overline{AB}C\)}&\(=\)&\(AB+C\)&\tabularnewline[0pt]
\multicolumn{1}{r}{\((AB)+(\overline{AB})C\)}&\(=\)&\(AB+C\)&associative (can skip this step)\tabularnewline[0pt]
\multicolumn{1}{r}{\(AB+C\)}&\(=\)&\(AB+C\)&absorption
\end{tabular}
}%
\end{center}%
\end{divisionsolutioneg}%
\begin{divisionsolutioneg}{2.6.7.45}{}{g:exercise:idp229055336}%
\par\smallskip%
\noindent\hypertarget{g:solution:idp229055464-main}{}\begin{center}%
{\tabularfont%
\begin{tabular}{llll}
\multicolumn{1}{r}{\(A+AB+ABC\)}&\(=\)&\(A\)&\tabularnewline[0pt]
\multicolumn{1}{r}{\(A+ABC\)}&\(=\)&\(A\)&absorption\tabularnewline[0pt]
\multicolumn{1}{r}{\(A+A(BC)\)}&\(=\)&\(A\)&associative (can skip)\tabularnewline[0pt]
\multicolumn{1}{r}{\(A\)}&\(=\)&\(A\)&absorption
\end{tabular}
}%
\end{center}%
\end{divisionsolutioneg}%
\begin{divisionsolutioneg}{2.6.7.46}{}{g:exercise:idp229063144}%
\par\smallskip%
\noindent\hypertarget{g:solution:idp229065064-main}{}\begin{center}%
{\tabularfont%
\begin{tabular}{llll}
\multicolumn{1}{r}{\(\overline{A}{}C+A\overline{B}{}C\)}&\(=\)&\(\overline{A}{}C+\overline{B}{}C\)&\tabularnewline[0pt]
\multicolumn{1}{r}{\((\overline{A}{}+A\overline{B}{})C\)}&\(=\)&\(\overline{A}{}C+\overline{B}{}C\)&distributive\tabularnewline[0pt]
\multicolumn{1}{r}{\((\overline{A}{}+\overline{B}{})C\)}&\(=\)&\(\overline{A}{}C+\overline{B}{}C\)&absorption\tabularnewline[0pt]
\multicolumn{1}{r}{\(\overline{A}{}C+\overline{B}{}C\)}&\(=\)&\(\overline{A}{}C+\overline{B}{}C\)&distributive
\end{tabular}
}%
\end{center}%
\end{divisionsolutioneg}%
\begin{divisionsolutioneg}{2.6.7.47}{}{g:exercise:idp229092952}%
\par\smallskip%
\noindent\hypertarget{g:solution:idp229093464-main}{}\begin{center}%
{\tabularfont%
\begin{tabular}{llll}
\multicolumn{1}{r}{\(\overline{AB}(A+B)\)}&\(=\)&\(\overline{A}{}B+A\overline{B}{}\)&\tabularnewline[0pt]
\multicolumn{1}{r}{\((\overline{A}{}+\overline{B}{})(A+B)\)}&\(=\)&\(\overline{A}{}B+A\overline{B}{}\)&De Morgan's\tabularnewline[0pt]
\multicolumn{1}{r}{\(\overline{A}{}A+\overline{A}{}B+\overline{B}{}A+\overline{B}{}B\)}&\(=\)&\(\overline{A}{}B+A\overline{B}{}\)&distributive\tabularnewline[0pt]
\multicolumn{1}{r}{\(0+\overline{A}{}B+\overline{B}{}A+0\)}&\(=\)&\(\overline{A}{}B+A\overline{B}{}\)&complement\tabularnewline[0pt]
\multicolumn{1}{r}{\(\overline{A}{}B+\overline{B}{}A\)}&\(=\)&\(\overline{A}{}B+A\overline{B}{}\)&identity
\end{tabular}
}%
\end{center}%
\end{divisionsolutioneg}%
\begin{divisionsolutioneg}{2.6.7.48}{}{g:exercise:idp229098712}%
\par\smallskip%
\noindent\hypertarget{g:solution:idp229102680-main}{}\begin{center}%
{\tabularfont%
\begin{tabular}{llll}
\multicolumn{1}{r}{\(\overline{\overline{\overline{A}{}BC}+D}\)}&\(=\)&\(\overline{A}{}BC\overline{D}\)&\tabularnewline[0pt]
\multicolumn{1}{r}{\(\overline{\overline{\overline{A}{}BC}}\overline{D}\)}&\(=\)&\(\overline{A}{}BC\overline{D}\)&De Morgan's\tabularnewline[0pt]
\multicolumn{1}{r}{\(\overline{A}{}BC\overline{D}\)}&\(=\)&\(\overline{A}{}BC\overline{D}\)&complement
\end{tabular}
}%
\end{center}%
\end{divisionsolutioneg}%
\begin{divisionsolutioneg}{2.6.7.49}{}{g:exercise:idp229105624}%
\par\smallskip%
\noindent\hypertarget{g:solution:idp229107928-main}{}\begin{center}%
{\tabularfont%
\begin{tabular}{llll}
\multicolumn{1}{r}{\(A\overline{B}{}\,\overline{\overline{A}{}\,\overline{C}{}}\)}&\(=\)&\(A\overline{B}{}\)&\tabularnewline[0pt]
\multicolumn{1}{r}{\(A\overline{B}{}(A+C)\)}&\(=\)&\(A\overline{B}{}\)&De Morgan's\tabularnewline[0pt]
\multicolumn{1}{r}{\(A\overline{B}{}A+A\overline{B}{}C\)}&\(=\)&\(A\overline{B}{}\)&distributive\tabularnewline[0pt]
\multicolumn{1}{r}{\(A\overline{B}{}+A\overline{B}{}C\)}&\(=\)&\(A\overline{B}{}\)&idempotent\tabularnewline[0pt]
\multicolumn{1}{r}{\(A\overline{B}{}\)}&\(=\)&\(A\overline{B}{}\)&absorption
\end{tabular}
}%
\end{center}%
\end{divisionsolutioneg}%
\end{exercisegroup}
\par\medskip\noindent
\end{solutions-subsection}
\end{sectionptx}
%
%
\typeout{************************************************}
\typeout{Section 2.7 The Conditional}
\typeout{************************************************}
%
\begin{sectionptx}{The Conditional}{}{The Conditional}{}{}{x:section:sec-conditional}
%
%
\typeout{************************************************}
\typeout{Subsection 2.7.1 The Conditional Connective}
\typeout{************************************************}
%
\begin{subsectionptx}{The Conditional Connective}{}{The Conditional Connective}{}{}{x:subsection:ssec-conditional-connective}
Suppose we have two propositions, \(p\) and \(q\).  Remembering that connectives are operations which join two or more porpositions (like \emph{and} and \emph{or}), the \terminology{conditional connective} is \begin{quote}%
If \(p\), then \(q\).\end{quote}
%
\par
In symbols, this is written as \(\conditional\), and when red aloud, you say \(p\) \emph{implies} \(q\). \index{conditional}\index{implies}\index{hypothesis}\index{conclusion} The first proposition in the conditional is called the \terminology{hypothesis}, and the second is called the \terminology{conclusion}.%
\par
There are other ways to state the conditional.  The following are equivalent statements:%
\begin{enumerate}
\item{}\(p\) implies \(q\)%
\item{}\(q\), if \(p\)%
\item{}\(p\) is sufficient for \(q\)%
\item{}\(q\) is necessary for \(p\)%
\item{}\(p\) only if \(q\)%
\end{enumerate}
%
\par
We'll be using only the forms \emph{if} \(p\) \emph{then} \(q\) and \(p\) \emph{implies} \(q\) in this course.%
\par
Experience has shown that the meaning of the conditional is best understood by means of examples.  Suppose you have an insurance contract\footnote{My (PW) thanks go to my colleague Gilles Cazelais for providing the idea for this example.\label{g:fn:idp229137752}} which reads: \begin{quote}%
If your house burns down, then the insurance company will pay you \textdollar{}1000 000.\end{quote}
 Let us further suppose your house burns down.  Under the contract, the unsurance company must give you one million dollars.  If it doesn't, then the contract has been violated.  But if your house doesn't burn down and the company doesn't give you any money, the contract has not been violated.  If your house doesn't burn down and out of boundless generosity the company gives you one million dollars anyway, the contract has not been violated.  The only circumstances under which the contract is violated is whn your house does burn down but the company fails to give you one million dollars.  This leads to the following truth table.%
\par
\begin{center}%
{\tabularfont%
\begin{tabular}{Bccc}\hrulethick
\multicolumn{1}{BcB}{House burns down?}&\multicolumn{1}{cB}{Company pays you \textdollar{}1000 000}&\multicolumn{1}{cB}{The contract is honoured.}\tabularnewline\hrulemedium
\multicolumn{1}{BcB}{no}&\multicolumn{1}{cB}{no}&\multicolumn{1}{cB}{yes}\tabularnewline[0pt]
\multicolumn{1}{BcB}{no}&\multicolumn{1}{cB}{yes}&\multicolumn{1}{cB}{yes}\tabularnewline[0pt]
\multicolumn{1}{BcB}{yes}&\multicolumn{1}{cB}{no}&\multicolumn{1}{cB}{no}\tabularnewline[0pt]
\multicolumn{1}{BcB}{yes}&\multicolumn{1}{cB}{yes}&\multicolumn{1}{cB}{yes}\tabularnewline\hrulethick
\end{tabular}
}%
\end{center}%
%
\par
To generalize to the symbolic propositions \(p\) and \(q\), \begin{center}%
{\tabularfont%
\begin{tabular}{Bccc}\hrulethick
\multicolumn{1}{BcB}{\(p\)}&\multicolumn{1}{cB}{\(q\)}&\multicolumn{1}{cB}{\(p{\rightarrow} q\)}\tabularnewline\hrulemedium
\multicolumn{1}{BcB}{F}&\multicolumn{1}{cB}{F}&\multicolumn{1}{cB}{T}\tabularnewline[0pt]
\multicolumn{1}{BcB}{F}&\multicolumn{1}{cB}{T}&\multicolumn{1}{cB}{T}\tabularnewline[0pt]
\multicolumn{1}{BcB}{T}&\multicolumn{1}{cB}{F}&\multicolumn{1}{cB}{F}\tabularnewline[0pt]
\multicolumn{1}{BcB}{T}&\multicolumn{1}{cB}{T}&\multicolumn{1}{cB}{T}\tabularnewline\hrulethick
\end{tabular}
}%
\end{center}%
%
\par
What does this mean?  It means that if \(p\) is true and \(q\) is false, then the implication \(p {\rightarrow} q\) cannot also be true.  It also means that if \(p{\rightarrow} q\) is true, then you cannot have \(p\) true and \(q\) false at the same time.%
\par
Let's look at another example.  Suppose that the following conditional is true: ``If Barney is a dog, then Barney has four legs.''  This means that if the first proposition, ``Barney is a dog,'' is true, then only one conclusion may be reached, that the second is true and Barney has four legs.  However, if \(p\) is false (that is, Barney is not a dog), then our conditional \emph{doesn't have anything to say} about the number of legs Barney may have.  If Barney is a snake, octopus, bug, person, or pond, then Barney may not have four legs.  If Barney is cat, giraffe, wooly mammoth, or a table, then Barney may have four legs.  Our coditional does not deal with what you may conclude when \(p\) is false (that is, when Barney is not a dog). \begin{example}{}{x:example:exmpl-cond1}%
Suppose that the statement ``If Jason sleeps in, he will be late for class'' is true.  Answer the following questions. %
\begin{enumerate}
\item{}Jason sleeps in.  Is he late for class?%
\item{}Jason does not sleep in.  Is he late for class?%
\item{}Jason is late for class.  Did he sleep in?%
\item{}Jason is not late for class.  Did he sleep in?%
\end{enumerate}
\par\smallskip%
\noindent\textbf{\blocktitlefont Answer}.\label{g:answer:idp229165528}{}\hypertarget{g:answer:idp229165528}{}\quad{}%
\begin{enumerate}
\item{}Yes.%
\item{}Maybe.  Perhaps he ran into traffic or encountered a bear.  Remember: the conditional has \emph{nothing to say} when the first proposition is false.%
\item{}Maybe. Again, there may be another reason for his lateness.%
\item{}No.%
\end{enumerate}
\end{example}
%
\end{subsectionptx}
%
%
\typeout{************************************************}
\typeout{Subsection 2.7.2 The Converse}
\typeout{************************************************}
%
\begin{subsectionptx}{The Converse}{}{The Converse}{}{}{x:subsection:ssec-converse}
If \(p\conditional\) is true, can we conclude that \(\converse\) is also true?  The statment \(\converse\) is called the \terminology{converse} of the statement \(\conditional\). \index{converse} Let's use our previous example again: \begin{quote}%
Original statement: If Barney is a dog, then Barney has four legs.%
\par
Converse: If Barney has four legs, then Barney is a dog.%
\end{quote}
%
\par
The truth of the original statement does \emph{not} guarantee that the converse is also true.  Suppose Barney is a cat.  Then the original statement is still true (it has nothing to say about Barney since Barney is a cat), but the converse is not true.  Barney has four legs, but Barney is not a dog.%
\par
Here is the truth table for the conditional and the converse: \begin{center}%
{\tabularfont%
\begin{tabular}{Bcccc}\hrulethick
\multicolumn{1}{BcB}{\(p\)}&\multicolumn{1}{cB}{\(q\)}&\multicolumn{1}{cB}{\(\conditional\)}&\multicolumn{1}{cB}{\(\converse\)}\tabularnewline\hrulemedium
\multicolumn{1}{BcB}{F}&\multicolumn{1}{cB}{F}&\multicolumn{1}{cB}{T}&\multicolumn{1}{cB}{T}\tabularnewline[0pt]
\multicolumn{1}{BcB}{F}&\multicolumn{1}{cB}{T}&\multicolumn{1}{cB}{T}&\multicolumn{1}{cB}{F}\tabularnewline[0pt]
\multicolumn{1}{BcB}{T}&\multicolumn{1}{cB}{F}&\multicolumn{1}{cB}{F}&\multicolumn{1}{cB}{T}\tabularnewline[0pt]
\multicolumn{1}{BcB}{T}&\multicolumn{1}{cB}{T}&\multicolumn{1}{cB}{T}&\multicolumn{1}{cB}{T}\tabularnewline\hrulethick
\end{tabular}
}%
\end{center}%
%
\par
Here's how to fill in the columns for any logical expression containing the conditional connective (\({\rightarrow}\)).  Let's call the propositions \emph{first} and \emph{second} so we don't get confused with \(p\) and \(q\).  For the conditional statement first\({\rightarrow}\) second, it will be be true for all cases \emph{except} when the first is true and the second is false.  So for \(\converse\), look for the row in which \(q\) is true and (rows 2 and 4) \(p\) is false (rows 1 and 2).  Then \(q\) is true and \(p\) is false only for row 2.  Therefore enter \emph{F} in the second row, and enter \(T\) in the remaining rows of the column for \(\converse\).  This table shows that the conditional and the converse are \emph{not} logically equivalent.%
\end{subsectionptx}
%
%
\typeout{************************************************}
\typeout{Subsection 2.7.3 The contrapositive}
\typeout{************************************************}
%
\begin{subsectionptx}{The contrapositive}{}{The contrapositive}{}{}{x:subsection:ssec-contrapositive}
The statement \(\contrapositive\) is the \terminology{contrapositive} of \(\conditional\). \index{contrapositive} Let's consider our example again: \begin{quote}%
Original statement: If Barney is a dog, then Barney has four legs.%
\par
Contrapositive: If Barney does not have four legs, then Barney is not a dog.%
\end{quote}
 We've kind of been assuming that we we commonly think of as four-legged animals actually have all four legs (i.e. no congenital or accidental losses).  With that in mind, the contrapositive appears to be equivalent to the original statement.  Let's consider the truth table. This time, we'll use 0 instead of F, and 1 instead of T for no other reason than to get used to using both notational systems interchangeably: \begin{center}%
{\tabularfont%
\begin{tabular}{Bcccccc}\hrulethick
\multicolumn{1}{BcB}{\(p\)}&\multicolumn{1}{cB}{\(q\)}&\multicolumn{1}{cB}{\(\sim\!{p}\)}&\multicolumn{1}{cB}{\(\sim\!{q}\)}&\multicolumn{1}{cB}{\(\conditional\)}&\multicolumn{1}{cB}{\(\contrapositive\)}\tabularnewline\hrulemedium
\multicolumn{1}{BcB}{0}&\multicolumn{1}{cB}{0}&\multicolumn{1}{cB}{1}&\multicolumn{1}{cB}{1}&\multicolumn{1}{cB}{1}&\multicolumn{1}{cB}{1}\tabularnewline[0pt]
\multicolumn{1}{BcB}{0}&\multicolumn{1}{cB}{1}&\multicolumn{1}{cB}{1}&\multicolumn{1}{cB}{0}&\multicolumn{1}{cB}{1}&\multicolumn{1}{cB}{1}\tabularnewline[0pt]
\multicolumn{1}{BcB}{1}&\multicolumn{1}{cB}{0}&\multicolumn{1}{cB}{0}&\multicolumn{1}{cB}{1}&\multicolumn{1}{cB}{0}&\multicolumn{1}{cB}{0}\tabularnewline[0pt]
\multicolumn{1}{BcB}{1}&\multicolumn{1}{cB}{1}&\multicolumn{1}{cB}{0}&\multicolumn{1}{cB}{0}&\multicolumn{1}{cB}{1}&\multicolumn{1}{cB}{1}\tabularnewline\hrulethick
\end{tabular}
}%
\end{center}%
%
\par
Remember, you enter a 1 unless the first statement is true and the second statement is false.  Note that in rows 1 and 3, the first (\(\sim\!{q}\)) is true.  In rows 3 and 4, the second (\(\sim\!{p}\)) is false.  So a 0 is entered in the last column in row 3.  We already know that the conditional \(\conditional\) is also false under those conditions.  The last two columns of this table are identical, so we see that \emph{the conditional and the contrapositive are logically equivalent}. \begin{example}{}{x:example:exmpl-contra1}%
Write the contrapositive for the statement, "If today is sunny, Jason will work in the garden."\par\smallskip%
\noindent\textbf{\blocktitlefont Answer}.\label{g:answer:idp229215320}{}\hypertarget{g:answer:idp229215320}{}\quad{}To get the contrapositive, negate the first and second statements and then reverse the order.  So the contrapositive is "If Jason is not working in the garden today, then it is not sunny."\end{example}
%
\end{subsectionptx}
%
%
\typeout{************************************************}
\typeout{Subsection 2.7.4 The inverse}
\typeout{************************************************}
%
\begin{subsectionptx}{The inverse}{}{The inverse}{}{}{x:subsection:ssec-inverse}
Given the conditional statement \(\conditional\), the statement \(\inverse\) is called the \terminology{inverse}. \index{inverse} Once again, let's look at our example: \begin{quote}%
Original statement: If Barney is a dog, then Barney has four legs.%
\par
Inverse: If Barney is not a dog, then Barney does not have four legs.%
\end{quote}
 From this example, we see that the original statement and its inverse are \emph{not} logically equivalent.%
\par
Here is the truth table for the inverse: \begin{center}%
{\tabularfont%
\begin{tabular}{Bcccccc}\hrulethick
\multicolumn{1}{BcB}{\(p\)}&\multicolumn{1}{cB}{\(q\)}&\multicolumn{1}{cB}{\(\sim\!{p}\)}&\multicolumn{1}{cB}{\(\sim\!{q}\)}&\multicolumn{1}{cB}{\(\conditional\)}&\multicolumn{1}{cB}{\(\inverse\)}\tabularnewline\hrulemedium
\multicolumn{1}{BcB}{0}&\multicolumn{1}{cB}{0}&\multicolumn{1}{cB}{1}&\multicolumn{1}{cB}{1}&\multicolumn{1}{cB}{1}&\multicolumn{1}{cB}{1}\tabularnewline[0pt]
\multicolumn{1}{BcB}{0}&\multicolumn{1}{cB}{1}&\multicolumn{1}{cB}{1}&\multicolumn{1}{cB}{0}&\multicolumn{1}{cB}{1}&\multicolumn{1}{cB}{0}\tabularnewline[0pt]
\multicolumn{1}{BcB}{1}&\multicolumn{1}{cB}{0}&\multicolumn{1}{cB}{0}&\multicolumn{1}{cB}{1}&\multicolumn{1}{cB}{0}&\multicolumn{1}{cB}{1}\tabularnewline[0pt]
\multicolumn{1}{BcB}{1}&\multicolumn{1}{cB}{1}&\multicolumn{1}{cB}{0}&\multicolumn{1}{cB}{0}&\multicolumn{1}{cB}{1}&\multicolumn{1}{cB}{1}\tabularnewline\hrulethick
\end{tabular}
}%
\end{center}%
 \begin{example}{}{x:example:exmpl-inv1}%
Draw the truth tables for the conditional (\(\conditional\)), the converse (\(\converse\)), the inverse (\(\inverse\)), and the contrapositive (\(\contrapositive\)).  Identify any logically equivalent statements.\par\smallskip%
\noindent\textbf{\blocktitlefont Answer}.\label{g:answer:idp229242840}{}\hypertarget{g:answer:idp229242840}{}\quad{}Here is the truth table: \begin{center}%
{\tabularfont%
\begin{tabular}{Bcccccccc}\hrulethick
\multicolumn{1}{BcB}{\(p\)}&\multicolumn{1}{cB}{\(q\)}&\multicolumn{1}{cB}{\(\sim\!{p}\)}&\multicolumn{1}{cB}{\(\sim\!{q}\)}&\multicolumn{1}{cB}{\(\conditional\)}&\multicolumn{1}{cB}{\(\converse\)}&\multicolumn{1}{cB}{\(\inverse\)}&\multicolumn{1}{cB}{\(\contrapositive\)}\tabularnewline\hrulemedium
\multicolumn{1}{BcB}{0}&\multicolumn{1}{cB}{0}&\multicolumn{1}{cB}{1}&\multicolumn{1}{cB}{1}&\multicolumn{1}{cB}{1}&\multicolumn{1}{cB}{1}&\multicolumn{1}{cB}{1}&\multicolumn{1}{cB}{1}\tabularnewline[0pt]
\multicolumn{1}{BcB}{0}&\multicolumn{1}{cB}{1}&\multicolumn{1}{cB}{1}&\multicolumn{1}{cB}{0}&\multicolumn{1}{cB}{1}&\multicolumn{1}{cB}{0}&\multicolumn{1}{cB}{0}&\multicolumn{1}{cB}{1}\tabularnewline[0pt]
\multicolumn{1}{BcB}{1}&\multicolumn{1}{cB}{0}&\multicolumn{1}{cB}{0}&\multicolumn{1}{cB}{1}&\multicolumn{1}{cB}{0}&\multicolumn{1}{cB}{1}&\multicolumn{1}{cB}{1}&\multicolumn{1}{cB}{0}\tabularnewline[0pt]
\multicolumn{1}{BcB}{1}&\multicolumn{1}{cB}{1}&\multicolumn{1}{cB}{0}&\multicolumn{1}{cB}{0}&\multicolumn{1}{cB}{1}&\multicolumn{1}{cB}{1}&\multicolumn{1}{cB}{1}&\multicolumn{1}{cB}{1}\tabularnewline\hrulethick
\end{tabular}
}%
\end{center}%
 Since the 5th and 8th columns are identical, \emph{the conditional is logically equivalent to the contrapositive}.  Since the 6th and 7th columns are identical, \emph{the converse is logically equivalent to the inverse.}\end{example}
%
\end{subsectionptx}
%
%
\typeout{************************************************}
\typeout{Subsection 2.7.5 The ``or'' form of the Conditional}
\typeout{************************************************}
%
\begin{subsectionptx}{The ``or'' form of the Conditional}{}{The ``or'' form of the Conditional}{}{}{x:subsection:ssec-alt-conditional}
The conditional \(\conditional\) can be rewritten in an alternate form in terms of the basic connectives. \index{conditional!``or'' form}\index{\emph{or} form of conditional} Consider the truth table below: \begin{center}%
{\tabularfont%
\begin{tabular}{Bccccc}\hrulethick
\multicolumn{1}{BcB}{\(p\)}&\multicolumn{1}{cB}{\(q\)}&\multicolumn{1}{cB}{\(\sim\!{p}\)}&\multicolumn{1}{cB}{\(\sim\!{p} {\vee}{} q\)}&\multicolumn{1}{cB}{\(\conditional\)}\tabularnewline\hrulemedium
\multicolumn{1}{BcB}{0}&\multicolumn{1}{cB}{0}&\multicolumn{1}{cB}{1}&\multicolumn{1}{cB}{1}&\multicolumn{1}{cB}{1}\tabularnewline[0pt]
\multicolumn{1}{BcB}{0}&\multicolumn{1}{cB}{1}&\multicolumn{1}{cB}{1}&\multicolumn{1}{cB}{1}&\multicolumn{1}{cB}{1}\tabularnewline[0pt]
\multicolumn{1}{BcB}{1}&\multicolumn{1}{cB}{0}&\multicolumn{1}{cB}{0}&\multicolumn{1}{cB}{0}&\multicolumn{1}{cB}{0}\tabularnewline[0pt]
\multicolumn{1}{BcB}{1}&\multicolumn{1}{cB}{1}&\multicolumn{1}{cB}{0}&\multicolumn{1}{cB}{1}&\multicolumn{1}{cB}{1}\tabularnewline\hrulethick
\end{tabular}
}%
\end{center}%
 This means that ``If Barney is a dog, then Barney has four legs'' is logically equivalent to ``Either Barney is not a dog, or Barney has four legs''. \begin{example}{}{x:example:exmpl-altcond1}%
Consider the conditional \(\conditional\).  Is the converse (\(\converse\)) logically equivelant to \(\sim\!{p}{\vee} q\), \(\sim\!{p}{\wedge} q\), \(p{\wedge}\sim\!{q}\), or \(p{\wedge}\sim\!{q}\)?\par\smallskip%
\noindent\textbf{\blocktitlefont Answer}.\label{g:answer:idp229280984}{}\hypertarget{g:answer:idp229280984}{}\quad{}Here's the truth table: \begin{center}%
{\tabularfont%
\begin{tabular}{Bccccccccc}\hrulethick
\multicolumn{1}{BcB}{\(p\)}&\multicolumn{1}{cB}{\(q\)}&\multicolumn{1}{cB}{\(\sim\!{p}\)}&\multicolumn{1}{cB}{\(\sim\!{q}\)}&\multicolumn{1}{cB}{\(\converse{}\)}&\multicolumn{1}{cB}{\(\sim\!{p} {\vee}{} q\)}&\multicolumn{1}{cB}{\(\sim\!{p} {\wedge}{} q\)}&\multicolumn{1}{cB}{\(p {\vee}{} \sim\!{q}\)}&\multicolumn{1}{cB}{\(p {\wedge}{} \sim\!{q}\)}\tabularnewline\hrulemedium
\multicolumn{1}{BcB}{0}&\multicolumn{1}{cB}{0}&\multicolumn{1}{cB}{1}&\multicolumn{1}{cB}{1}&\multicolumn{1}{cB}{1}&\multicolumn{1}{cB}{1}&\multicolumn{1}{cB}{0}&\multicolumn{1}{cB}{1}&\multicolumn{1}{cB}{0}\tabularnewline[0pt]
\multicolumn{1}{BcB}{0}&\multicolumn{1}{cB}{1}&\multicolumn{1}{cB}{1}&\multicolumn{1}{cB}{0}&\multicolumn{1}{cB}{0}&\multicolumn{1}{cB}{1}&\multicolumn{1}{cB}{1}&\multicolumn{1}{cB}{0}&\multicolumn{1}{cB}{0}\tabularnewline[0pt]
\multicolumn{1}{BcB}{1}&\multicolumn{1}{cB}{0}&\multicolumn{1}{cB}{0}&\multicolumn{1}{cB}{1}&\multicolumn{1}{cB}{1}&\multicolumn{1}{cB}{0}&\multicolumn{1}{cB}{0}&\multicolumn{1}{cB}{1}&\multicolumn{1}{cB}{1}\tabularnewline[0pt]
\multicolumn{1}{BcB}{1}&\multicolumn{1}{cB}{1}&\multicolumn{1}{cB}{0}&\multicolumn{1}{cB}{0}&\multicolumn{1}{cB}{1}&\multicolumn{1}{cB}{1}&\multicolumn{1}{cB}{0}&\multicolumn{1}{cB}{1}&\multicolumn{1}{cB}{0}\tabularnewline\hrulethick
\end{tabular}
}%
\end{center}%
 As can be seen from the table, the columns for \(\converse\) and \(p{\vee}\sim\!{q}\) are identical, so these two expressions are logically equivalent.\end{example}
%
\end{subsectionptx}
%
%
\typeout{************************************************}
\typeout{Subsection 2.7.6 De Morgan's Law and the contrapositive}
\typeout{************************************************}
%
\begin{subsectionptx}{De Morgan's Law and the contrapositive}{}{De Morgan's Law and the contrapositive}{}{}{x:subsection:ssec-demorgan-and-contrapos}
Consider the conditional \((p {\wedge}{} q {\rightarrow}{} r\).  The contrapositive would be \(\sim\!{r} {\rightarrow}{} \sim\!{(p {\wedge}{} q)}\).  Applying De Morgan's Law gives \(\sim\!{r} {\rightarrow}{} (\sim\!{p} {\vee}{} \sim\!{q})\).  Notice the change from the \emph{and} in the conditional to the \emph{or} in the modified contrapositive.  Forgetting to make that change is an easy trap to fall into. \index{De Morgan's laws!and contrapositive}\index{contrapositive!and De Morgan's laws} \begin{example}{}{x:example:exmpl-demorgan-contrap1}%
Consider the conditional ``If Jason sleeps in or gets caught in traffic, he will be late for class.''  What is the contrapositive?  Use De Morgan's Law to find your answer.\par\smallskip%
\noindent\textbf{\blocktitlefont Answer}.\label{g:answer:idp229311704}{}\hypertarget{g:answer:idp229311704}{}\quad{}The conditional here is%
\begin{equation*}
(\text{sleep or traffic}){\rightarrow}\text{ late}
\end{equation*}
The contrapositive is then%
\begin{align*}
\sim\!{\text{late}} \amp {\rightarrow} \sim\!{\text{sleep or traffic}}\\
\amp \sim\!{\text{sleep}}\text{ and }\sim\!{\text{traffic}}
\end{align*}
Therefore, the contrapositive is ``If Jason is not late for class, then he didn't sleep in \emph{and} he didn't get caught in traffic.''  The \emph{or} changes into \emph{and}, as indicated by De Morgan's Law.\end{example}
%
\end{subsectionptx}
%
%
\typeout{************************************************}
\typeout{Exercises 2.7.7 Exercises}
\typeout{************************************************}
%
\begin{exercises-subsection}{Exercises}{}{Exercises}{}{}{x:exercises:exers-sec-conditional}
\par\medskip\noindent%
\textbf{Exercise Group.}\space\space%
In the following exercises, let \(p\) denote \emph{The movie was popular} and \(q\) denote \emph{The movie will make a lot of money.}  Translate the following propositions into English sentences.\begin{exercisegroup}
\begin{divisionexerciseeg}{1}{}{}{g:exercise:idp229313624}%
\(\ \conditional\)\end{divisionexerciseeg}%
\begin{divisionexerciseeg}{2}{}{}{g:exercise:idp229315544}%
\(\ \inverse\)\end{divisionexerciseeg}%
\begin{divisionexerciseeg}{3}{}{}{g:exercise:idp229320536}%
\(\ \contrapositive\)\end{divisionexerciseeg}%
\begin{divisionexerciseeg}{4}{}{}{g:exercise:idp229319768}%
\(\ \converse\)\end{divisionexerciseeg}%
\begin{divisionexerciseeg}{5}{}{}{g:exercise:idp229318872}%
\(\ \sim\!{p}{\vee} q\)\end{divisionexerciseeg}%
\begin{divisionexerciseeg}{6}{}{}{g:exercise:idp229319896}%
\(\ p{\wedge}\sim\!{q}\)\end{divisionexerciseeg}%
\end{exercisegroup}
\par\medskip\noindent
\par\medskip\noindent%
\textbf{Exercise Group.}\space\space%
In the following exercises, let \(p\) denote \emph{Jason eats a burger for dinner} and \(q\) denote \emph{Jason is too full for dessert.}  Translate the following sentences into logical symbols.\begin{exercisegroup}
\begin{divisionexerciseeg}{7}{}{}{g:exercise:idp229322200}%
\(\ \)If Jason eats a burger for dinner, he will be too full for dessert.\end{divisionexerciseeg}%
\begin{divisionexerciseeg}{8}{}{}{g:exercise:idp229323736}%
\(\ \)If Jason does not eat a burger for dinner, he will not be too full for dessert.\end{divisionexerciseeg}%
\begin{divisionexerciseeg}{9}{}{}{g:exercise:idp229325016}%
\(\ \)If Jason is too full for dessert, then he ate a burger for dinner.\end{divisionexerciseeg}%
\begin{divisionexerciseeg}{10}{}{}{g:exercise:idp229325912}%
\(\ \)If Jason is not too full for dessert, then he did not eat a burger for dinner.\end{divisionexerciseeg}%
\begin{divisionexerciseeg}{11}{}{}{g:exercise:idp229326552}%
\(\ \)If Jason is too full for dessert, then he did not eat a burger for dinner.\end{divisionexerciseeg}%
\begin{divisionexerciseeg}{12}{}{}{g:exercise:idp229325144}%
\(\ \)Jason being too full for dessert implies that he ate a burger for dinner.\end{divisionexerciseeg}%
\begin{divisionexerciseeg}{13}{}{}{g:exercise:idp229328216}%
\(\ \)Jason not being too full for dessert implies that he did not eat a burger for dinner.\end{divisionexerciseeg}%
\begin{divisionexerciseeg}{14}{}{}{g:exercise:idp229329496}%
\(\ \)Jason not eating a berger for dinner implies that he will not be too full for dessert.\end{divisionexerciseeg}%
\begin{divisionexerciseeg}{15}{}{}{g:exercise:idp229326808}%
\(\ \)Jason eating a burger for dinner implies that he will be too full for dessert.\end{divisionexerciseeg}%
\begin{divisionexerciseeg}{16}{}{}{g:exercise:idp229327064}%
\(\ \)Either Jason does not eat a burger for dinner of he will be too full for dessert.\end{divisionexerciseeg}%
\begin{divisionexerciseeg}{17}{}{}{g:exercise:idp229325272}%
\(\ \)Either Jason is not too full for dessert or he ate a burger for dinner.\end{divisionexerciseeg}%
\begin{divisionexerciseeg}{18}{}{}{g:exercise:idp229334616}%
\(\ \)Either Jason is too full for dessert or he did not eat a burger for dinner.\end{divisionexerciseeg}%
\end{exercisegroup}
\par\medskip\noindent
\begin{divisionexercise}{19}{}{}{g:exercise:idp229333592}%
\(\ \)The following conditional statement is true: If Jason is eaten by nears, he will not finish his marking.  Given that, answer the following questions. %
\begin{enumerate}[label=(\alph*)]
\item{}Jason is eaten by bears.  Did he finish his marking?%
\item{}Jason is not eaten by bears.  Did he finish his marking?%
\item{}Jason finished his marking.  Was he eaten by bears?%
\item{}Jason did not finish his marking.  Was he eaten by bears?%
\end{enumerate}
\end{divisionexercise}%
\begin{divisionexercise}{20}{}{}{g:exercise:idp229071576}%
\(\ \)The following conditional statement is true: If Rich is asleep, then he is not playing ping-pong.  Given that, answer the following questions. %
\begin{enumerate}[label=(\alph*)]
\item{}Rich is playing ping-pong.  Is he asleep?%
\item{}Rich is asleep.  Is he playing ping-pong?%
\item{}Rich is not asleep.  Is he playing ping-pong?%
\item{}Rich is not playing ping-pong.  Is he asleep?%
\end{enumerate}
\end{divisionexercise}%
\par\medskip\noindent%
\textbf{Exercise Group.}\space\space%
Of course, for the previous questions, I chose situations in which you can use common sense to determine the answer.  However, the true test of whether you understand the concept is to replace the above propositions by complete nonsense.\begin{exercisegroup}
\begin{divisionexerciseeg}{21}{}{}{g:exercise:idp229074904}%
\(\ \)The following conditional statement is true: If ettercaps are green, then toves are slithy.  Given that, answer the following questions. %
\begin{enumerate}[label=(\alph*)]
\item{}Toves are slithy.  Are ettercaps green?%
\item{}Toves are not slithy.  Are ettercaps green?%
\item{}Ettercaps are green.  Are toves slithy?%
\item{}Ettercaps are red.  Are toves slithy?%
\end{enumerate}
\end{divisionexerciseeg}%
\begin{divisionexerciseeg}{22}{}{}{g:exercise:idp229074776}%
\(\ \)The following conditional statement is true: If the hare reads the Times Colonist, the tortoise will take out the recycling.  Given that, answer the following questions. %
\begin{enumerate}[label=(\alph*)]
\item{}The hare does not read the Times Colonist.  Will the tortoise take out the recycling?%
\item{}The hare reads the Times Colonist.  Will the tortoise take out the recycling?%
\item{}The tortoise takes out the recycling.  Does the hare read the Times Colonist?%
\item{}The tortoise is not taking out the recycling.  Does the hare read the Times Colonist?%
\end{enumerate}
\end{divisionexerciseeg}%
\end{exercisegroup}
\par\medskip\noindent
\par\medskip\noindent%
\textbf{Exercise Group.}\space\space%
Given the conditional statement \emph{If the frattling is non-responsive, then the runges must be strunking}, write the corresponding English sentences for the following.\begin{exercisegroup}
\begin{divisionexerciseeg}{23}{}{}{g:exercise:idp229494232}%
\(\ \)The contrapositive (\(\contrapositive\))\end{divisionexerciseeg}%
\begin{divisionexerciseeg}{24}{}{}{g:exercise:idp229496152}%
\(\ \)The converse (\(\converse\))\end{divisionexerciseeg}%
\begin{divisionexerciseeg}{25}{}{}{g:exercise:idp229493336}%
\(\ \)The inverse (\(\inverse\))\end{divisionexerciseeg}%
\begin{divisionexerciseeg}{26}{}{}{g:exercise:idp229493592}%
\(\ \)The ``or'' form (\(\sim\!{p}{\vee} q\))\end{divisionexerciseeg}%
\end{exercisegroup}
\par\medskip\noindent
\begin{divisionexercise}{27}{}{}{g:exercise:idp229504344}%
\(\ \)Given the conditional statement, \emph{If Bossy is mooing, she must be a cow,} which of the four following statements is the contrapositive (\(\contrapositive\))? %
\begin{enumerate}[label=(\alph*)]
\item{}If Bossy is not a cow, she is not mooing.%
\item{}If Bossy is a cow, then she is mooing.%
\item{}If Bossy is mooing, then she must be a cow.%
\item{}if Bossy is not mooing, then she must not be a cow.%
\end{enumerate}
\end{divisionexercise}%
\begin{divisionexercise}{28}{}{}{g:exercise:idp229498072}%
\(\ \)Given the conditional statement, \emph{If Bossy is mooing, she must be a cow,} which of the four following statements is the converse (\(\converse\))? %
\begin{enumerate}[label=(\alph*)]
\item{}If Bossy is not a cow, she is not mooing.%
\item{}If Bossy is a cow, then she is mooing.%
\item{}If Bossy is mooing, then she must be a cow.%
\item{}if Bossy is not mooing, then she must not be a cow.%
\end{enumerate}
\end{divisionexercise}%
\begin{divisionexercise}{29}{}{}{g:exercise:idp229497560}%
\(\ \)Given the conditional statement, \emph{If Bossy is mooing, she must be a cow,} which of the four following statements is also true? %
\begin{enumerate}[label=(\alph*)]
\item{}If Bossy is not a cow, she is not mooing.%
\item{}If Bossy is a cow, then she is mooing.%
\item{}If Bossy is mooing, then she must be a cow.%
\item{}if Bossy is not mooing, then she must not be a cow.%
\end{enumerate}
\end{divisionexercise}%
\begin{divisionexercise}{30}{}{}{g:exercise:idp229511768}%
\(\ \)Which of the following is the correct ``or'' form form the conditional \emph{If Bossy is mooing, then she must be a cow}? %
\begin{enumerate}[label=(\alph*)]
\item{}Bossy is a cow or she is not mooing.%
\item{}Bossy is not a cow or she is not mooing.%
\item{}Bossy is not a cow or she is mooing.%
\item{}Bossy is a cow or she is mooing.%
\end{enumerate}
\end{divisionexercise}%
\begin{divisionexercise}{31}{}{}{g:exercise:idp229508056}%
\(\ \)If the statement \emph{If Bossy is mooing, then she must be a cow} is a true statement, which of the collowing cannot occur? %
\begin{enumerate}[label=(\alph*)]
\item{}Bossy is mooing and she is a cow.%
\item{}Bossy is mooing and she is not a cow.%
\item{}Bossy is not mooing and she is not a cow.%
\item{}Bossy is not mooing and she is a cow.%
\end{enumerate}
\end{divisionexercise}%
\begin{divisionexercise}{32}{}{}{g:exercise:idp229506648}%
\(\ \)Consider the following ``or'' form statement, \emph{Either Superman has a cape or he cannot fly.}  Which of the following is the correct form of the corresponding conditional? %
\begin{enumerate}[label=(\alph*)]
\item{}If Superman does not have a cape, then he cannot fly.%
\item{}If Superman has a cape, then he can fly.%
\item{}If Superman can fly, then he has a cape.%
\item{}If Superman cannot fly, then he doesn't have a cape.%
\end{enumerate}
\end{divisionexercise}%
\begin{divisionexercise}{33}{}{}{g:exercise:idp229517912}%
\(\ \)Consider the conditional \emph{If John has the flu or misses the bus, he will be late for work.}  Which of the following is the corresponding contrapositive statement \((\contrapositive)\)? %
\begin{enumerate}[label=(\alph*)]
\item{}If John is late for work, then he had the flu or missed the bus.%
\item{}If John is late for work, then he did not have the flu or did not miss the bus.%
\item{}If John is not late for work, then he did not have the flu or did not miss the bus.%
\item{}If John is not late for work, then he did not have the flu and did not miss the bus.%
\end{enumerate}
\end{divisionexercise}%
\begin{divisionexercise}{34}{}{}{g:exercise:idp229518040}%
\(\ \)Consider the conditional \emph{If Rich doesn't show his work or makes a mistake, then he will not get full credit.}  Which of the following is the corresponding contrapositive statement \((\contrapositive)\)? %
\begin{enumerate}[label=(\alph*)]
\item{}If Rich received full credit, then he showed his work and did not make a mistake.%
\item{}If Rich received full credit, the he showed his work or did not make a mistake.%
\item{}If Rich did not get full credit, then he didn't show his work and made a mistake.%
\item{}If Rich did not get full credit, then he didn't show his work or made a mistake.%
\end{enumerate}
\end{divisionexercise}%
\begin{divisionexercise}{35}{}{}{g:exercise:idp229523928}%
\(\ \)Consider the conditional \emph{If Jason is late and has not called his wife, she will be worried.}  Which of the following is the corresponding contrapositive statement \((\contrapositive)\)? %
\begin{enumerate}[label=(\alph*)]
\item{}If Jason's wife is not worried, then he is not late and did call her.%
\item{}If Jason's wife is not worried, then he is not late or did call her.%
\item{}If Jason's wife is worried, then he is late and has not called her.%
\item{}If Jason's wife is not worried, then he is late and did not call her.%
\end{enumerate}
\end{divisionexercise}%
\begin{divisionexercise}{36}{}{}{g:exercise:idp229523416}%
\(\ \)Consider the conditional \emph{If grunkles are circular, then runges are square and triptrops are blue.}  Which of the following is the corresponding contrapositive statement \((\contrapositive)\)? %
\begin{enumerate}[label=(\alph*)]
\item{}If runges are not square and triptrops are not blue, then grunkles are not circular.%
\item{}If runges are not square or triptrops are not blue, then grunkles are circular.%
\item{}If runges are not square or triptrops are not blue, then grunkles are not circular.%
\item{}If runges are not square and triptrops are not blue, then grunkles are circular.%
\end{enumerate}
\end{divisionexercise}%
\end{exercises-subsection}
%
%
\typeout{************************************************}
\typeout{Solutions 2.7.8 Solutions to Section~{\xreffont\ref*{x:section:sec-conditional}} Exercises.}
\typeout{************************************************}
%
\begin{solutions-subsection}{Solutions to Section~{\xreffont\ref*{x:section:sec-conditional}} Exercises.}{}{Solutions to Section~{\xreffont\ref*{x:section:sec-conditional}} Exercises.}{}{}{g:solutions:idp229534808}
\par\medskip
\noindent\textbf{\normalsize{}2.7.7\space\textperiodcentered\space{}Exercises}
\begin{exercisegroup}
\begin{divisionsolutioneg}{2.7.7.1}{}{g:exercise:idp229313624}%
\par\smallskip%
\noindent\hypertarget{g:solution:idp229316696-main}{}If the movie was popular, then it will make a lot of money.  (Or:  The movie's popularity implies that it will make a lot of money.)\end{divisionsolutioneg}%
\begin{divisionsolutioneg}{2.7.7.2}{}{g:exercise:idp229315544}%
\par\smallskip%
\noindent\hypertarget{g:solution:idp229322712-main}{}If the movie was not popular, then it will not make a lot of money.\end{divisionsolutioneg}%
\begin{divisionsolutioneg}{2.7.7.3}{}{g:exercise:idp229320536}%
\par\smallskip%
\noindent\hypertarget{g:solution:idp229318616-main}{}If the movie did not make a lot of money, then the movie was not popular.\end{divisionsolutioneg}%
\begin{divisionsolutioneg}{2.7.7.4}{}{g:exercise:idp229319768}%
\par\smallskip%
\noindent\hypertarget{g:solution:idp229318488-main}{}If the movie makes a lot of money, then it is popular.\end{divisionsolutioneg}%
\begin{divisionsolutioneg}{2.7.7.5}{}{g:exercise:idp229318872}%
\par\smallskip%
\noindent\hypertarget{g:solution:idp229319128-main}{}The movie was not popular or it made a lot of money.\end{divisionsolutioneg}%
\begin{divisionsolutioneg}{2.7.7.6}{}{g:exercise:idp229319896}%
\par\smallskip%
\noindent\hypertarget{g:solution:idp229317464-main}{}The movie was popular and it did not make a lot of money.\end{divisionsolutioneg}%
\end{exercisegroup}
\par\medskip\noindent
\begin{exercisegroup}
\begin{divisionsolutioneg}{2.7.7.7}{}{g:exercise:idp229322200}%
\par\smallskip%
\noindent\hypertarget{g:solution:idp229317976-main}{}\(\conditional\)\end{divisionsolutioneg}%
\begin{divisionsolutioneg}{2.7.7.8}{}{g:exercise:idp229323736}%
\par\smallskip%
\noindent\hypertarget{g:solution:idp229324632-main}{}\(\inverse\)\end{divisionsolutioneg}%
\begin{divisionsolutioneg}{2.7.7.9}{}{g:exercise:idp229325016}%
\par\smallskip%
\noindent\hypertarget{g:solution:idp229326040-main}{}\(\converse \)\end{divisionsolutioneg}%
\begin{divisionsolutioneg}{2.7.7.10}{}{g:exercise:idp229325912}%
\par\smallskip%
\noindent\hypertarget{g:solution:idp229325656-main}{}\(\contrapositive\)\end{divisionsolutioneg}%
\begin{divisionsolutioneg}{2.7.7.11}{}{g:exercise:idp229326552}%
\par\smallskip%
\noindent\hypertarget{g:solution:idp229329880-main}{}\(\, q {\rightarrow}{} \sim\!{p}{}\)\end{divisionsolutioneg}%
\begin{divisionsolutioneg}{2.7.7.12}{}{g:exercise:idp229325144}%
\par\smallskip%
\noindent\hypertarget{g:solution:idp229329240-main}{}\(\converse\)\end{divisionsolutioneg}%
\begin{divisionsolutioneg}{2.7.7.13}{}{g:exercise:idp229328216}%
\par\smallskip%
\noindent\hypertarget{g:solution:idp229328344-main}{}\(\contrapositive \)\end{divisionsolutioneg}%
\begin{divisionsolutioneg}{2.7.7.14}{}{g:exercise:idp229329496}%
\par\smallskip%
\noindent\hypertarget{g:solution:idp229330008-main}{}\(\inverse\)\end{divisionsolutioneg}%
\begin{divisionsolutioneg}{2.7.7.15}{}{g:exercise:idp229326808}%
\par\smallskip%
\noindent\hypertarget{g:solution:idp229330648-main}{}\(\conditional\)\end{divisionsolutioneg}%
\begin{divisionsolutioneg}{2.7.7.16}{}{g:exercise:idp229327064}%
\par\smallskip%
\noindent\hypertarget{g:solution:idp229331800-main}{}\(\sim\!{p}{}{\vee} q \)\end{divisionsolutioneg}%
\begin{divisionsolutioneg}{2.7.7.17}{}{g:exercise:idp229325272}%
\par\smallskip%
\noindent\hypertarget{g:solution:idp229335000-main}{}\(\sim\!{q}{\vee} p\)\end{divisionsolutioneg}%
\begin{divisionsolutioneg}{2.7.7.18}{}{g:exercise:idp229334616}%
\par\smallskip%
\noindent\hypertarget{g:solution:idp229334488-main}{}\(\, q{\vee}\sim\!{p}{}\)\end{divisionsolutioneg}%
\end{exercisegroup}
\par\medskip\noindent
\begin{divisionsolution}{2.7.7.19}{}{g:exercise:idp229333592}%
\par\smallskip%
\noindent\hypertarget{g:solution:idp229336792-main}{}%
\begin{enumerate}[label=(\alph*)]
\item{}No%
\item{}Maybe%
\item{}No%
\item{}Maybe%
\end{enumerate}
\end{divisionsolution}%
\begin{divisionsolution}{2.7.7.20}{}{g:exercise:idp229071576}%
\par\smallskip%
\noindent\hypertarget{g:solution:idp229073624-main}{}%
\begin{enumerate}[label=(\alph*)]
\item{}No%
\item{}No%
\item{}Maybe%
\item{}Maybe%
\end{enumerate}
\end{divisionsolution}%
\begin{exercisegroup}
\begin{divisionsolutioneg}{2.7.7.21}{}{g:exercise:idp229074904}%
\par\smallskip%
\noindent\hypertarget{g:solution:idp229072472-main}{}%
\begin{enumerate}[label=(\alph*)]
\item{}Maybe%
\item{}No%
\item{}Yes%
\item{}Maybe%
\end{enumerate}
\end{divisionsolutioneg}%
\begin{divisionsolutioneg}{2.7.7.22}{}{g:exercise:idp229074776}%
\par\smallskip%
\noindent\hypertarget{g:solution:idp229490776-main}{}%
\begin{enumerate}[label=(\alph*)]
\item{}Maybe%
\item{}Yes%
\item{}Maybe%
\item{}No%
\end{enumerate}
\end{divisionsolutioneg}%
\end{exercisegroup}
\par\medskip\noindent
\begin{exercisegroup}
\begin{divisionsolutioneg}{2.7.7.23}{}{g:exercise:idp229494232}%
\par\smallskip%
\noindent\hypertarget{g:solution:idp229496280-main}{}If the runges are not strunking, then the frattling must be responsive.\end{divisionsolutioneg}%
\begin{divisionsolutioneg}{2.7.7.24}{}{g:exercise:idp229496152}%
\par\smallskip%
\noindent\hypertarget{g:solution:idp229493720-main}{}If the runges are strunking, then the frattling is non-responsive.\end{divisionsolutioneg}%
\begin{divisionsolutioneg}{2.7.7.25}{}{g:exercise:idp229493336}%
\par\smallskip%
\noindent\hypertarget{g:solution:idp229491416-main}{}If the frattling is responsive, then the runges must not be strunking.\end{divisionsolutioneg}%
\begin{divisionsolutioneg}{2.7.7.26}{}{g:exercise:idp229493592}%
\par\smallskip%
\noindent\hypertarget{g:solution:idp229498840-main}{}The frattling is responsive or the runges are strunking.\end{divisionsolutioneg}%
\end{exercisegroup}
\par\medskip\noindent
\begin{divisionsolution}{2.7.7.27}{}{g:exercise:idp229504344}%
\par\smallskip%
\noindent\hypertarget{g:solution:idp229498584-main}{}\(\ \)(a)\end{divisionsolution}%
\begin{divisionsolution}{2.7.7.28}{}{g:exercise:idp229498072}%
\par\smallskip%
\noindent\hypertarget{g:solution:idp229498456-main}{}\(\ \)(b)\end{divisionsolution}%
\begin{divisionsolution}{2.7.7.29}{}{g:exercise:idp229497560}%
\par\smallskip%
\noindent\hypertarget{g:solution:idp229510360-main}{}\(\ \)(a)\end{divisionsolution}%
\begin{divisionsolution}{2.7.7.30}{}{g:exercise:idp229511768}%
\par\smallskip%
\noindent\hypertarget{g:solution:idp229509464-main}{}\(\ \)(a)\end{divisionsolution}%
\begin{divisionsolution}{2.7.7.31}{}{g:exercise:idp229508056}%
\par\smallskip%
\noindent\hypertarget{g:solution:idp229507544-main}{}\(\ \)(b)\end{divisionsolution}%
\begin{divisionsolution}{2.7.7.32}{}{g:exercise:idp229506648}%
\par\smallskip%
\noindent\hypertarget{g:solution:idp229520856-main}{}\(\ \)If you let \(p\) be \emph{Superman does not have a cape} then (a).  If instead you let \(p\) be \emph{Superman cannot fly} then (c).\end{divisionsolution}%
\begin{divisionsolution}{2.7.7.33}{}{g:exercise:idp229517912}%
\par\smallskip%
\noindent\hypertarget{g:solution:idp229515224-main}{}\(\ \)(d)\end{divisionsolution}%
\begin{divisionsolution}{2.7.7.34}{}{g:exercise:idp229518040}%
\par\smallskip%
\noindent\hypertarget{g:solution:idp229523800-main}{}\(\ \)(a)\end{divisionsolution}%
\begin{divisionsolution}{2.7.7.35}{}{g:exercise:idp229523928}%
\par\smallskip%
\noindent\hypertarget{g:solution:idp229526872-main}{}\(\ \)(b)\end{divisionsolution}%
\begin{divisionsolution}{2.7.7.36}{}{g:exercise:idp229523416}%
\par\smallskip%
\noindent\hypertarget{g:solution:idp229530712-main}{}\(\ \)(c)\end{divisionsolution}%
\end{solutions-subsection}
\end{sectionptx}
%
%
\typeout{************************************************}
\typeout{Section 2.8 The Biconditional}
\typeout{************************************************}
%
\begin{sectionptx}{The Biconditional}{}{The Biconditional}{}{}{x:section:sec-biconditional}
%
%
\typeout{************************************************}
\typeout{Subsection 2.8.1 The Biconditional Connective}
\typeout{************************************************}
%
\begin{subsectionptx}{The Biconditional Connective}{}{The Biconditional Connective}{}{}{x:subsection:ssec-bicond-connective}
Consider the conditional \emph{If you live in Victoria, then you live in BC.}  Remembering that the conditional has nothing to say if the first proposition is false, then it is possible for you to not live in Victoria but to still live in BC.  It is also possible to not live in Vicoria and also not live in BC.%
\par
Let's now consider the conditional \emph{If the temperature outside is below \(0^{\circ}\text{C}\), then it is freezing outside.}  If I were to use this sentence in everyday English, I probably mean \emph{If the temperature outside is below \(0^{\circ}\text{C}\), then it is freezing outide \alert{and} if the temperature outside is \alert{not} below \(0^{\circ}\text{C}\), then it is \alert{not} freezing outside.}  So we could probably do with a new kind of connective that means \emph{If \(p\) then \(q\) \alert{and} if \(\sim\!{p}{}\) then \(\sim\!{q}{}\)}. \index{biconditional} The new connective is called the \terminology{biconditional}, and is written \(\biconditional\)%
\par
Here is the truth table for the biconditional: \begin{center}%
{\tabularfont%
\begin{tabular}{Bccc}\hrulethick
\multicolumn{1}{BcB}{\(p\)}&\multicolumn{1}{cB}{\(q\)}&\multicolumn{1}{cB}{\(p {\leftrightarrow}{} q\)}\tabularnewline\hrulemedium
\multicolumn{1}{BcB}{0}&\multicolumn{1}{cB}{0}&\multicolumn{1}{cB}{1}\tabularnewline[0pt]
\multicolumn{1}{BcB}{0}&\multicolumn{1}{cB}{1}&\multicolumn{1}{cB}{0}\tabularnewline[0pt]
\multicolumn{1}{BcB}{1}&\multicolumn{1}{cB}{0}&\multicolumn{1}{cB}{0}\tabularnewline[0pt]
\multicolumn{1}{BcB}{1}&\multicolumn{1}{cB}{1}&\multicolumn{1}{cB}{1}\tabularnewline\hrulethick
\end{tabular}
}%
\end{center}%
%
\par
So if \(p\) and \(q\) have the same truth value, then the biconditional is true and otherwise it is false.  (It is the negation of the \emph{exclusive or}, \(p{\oplus} q\).)%
\par
There are a number of ways to specify the biconditional in English:%
\begin{enumerate}
\item{}If and only if \(p\), then \(q\).%
\item{}\(p\) if and only if \(q\).%
\item{}If \(p\), then \(q\), and vice versa.%
\item{}If \(p\), then \(q\), and if \(\sim\!{p}{}\) then \(\sim\!{q}{}\).%
\end{enumerate}
We'll mostly be using the first two in this list. \begin{example}{}{x:example:exmpl-bicond1}%
Draw the truth tables for \(\biconditional\) and \((\conditional){\wedge}(\converse)\).  Are they logically equivalent?  Also, are \(\biconditional\) and \((\conditional){\wedge}(\inverse)\) logically equivalent?\par\smallskip%
\noindent\textbf{\blocktitlefont Answer}.\label{g:answer:idp229561560}{}\hypertarget{g:answer:idp229561560}{}\quad{}Here's the truth table: \begin{center}%
{\tabularfont%
\begin{tabular}{Bcccccccccc}\hrulethick
\multicolumn{1}{BcB}{\(p\)}&\multicolumn{1}{cB}{\(q\)}&\multicolumn{1}{cB}{\(\biconditional\)}&\multicolumn{1}{cB}{\(\sim\!{p}{}\)}&\multicolumn{1}{cB}{\(\sim\!{q}{}\)}&\multicolumn{1}{cB}{\(\conditional\)}&\multicolumn{1}{cB}{\(\converse\)}&\multicolumn{1}{cB}{\(\inverse\)}&\multicolumn{1}{cB}{\((\conditional){\wedge}(\converse)\)}&\multicolumn{1}{cB}{\((\conditional){\wedge}(\inverse)\)}\tabularnewline\hrulemedium
\multicolumn{1}{BcB}{0}&\multicolumn{1}{cB}{0}&\multicolumn{1}{cB}{1}&\multicolumn{1}{cB}{1}&\multicolumn{1}{cB}{1}&\multicolumn{1}{cB}{1}&\multicolumn{1}{cB}{1}&\multicolumn{1}{cB}{1}&\multicolumn{1}{cB}{1}&\multicolumn{1}{cB}{1}\tabularnewline[0pt]
\multicolumn{1}{BcB}{0}&\multicolumn{1}{cB}{1}&\multicolumn{1}{cB}{0}&\multicolumn{1}{cB}{1}&\multicolumn{1}{cB}{0}&\multicolumn{1}{cB}{1}&\multicolumn{1}{cB}{0}&\multicolumn{1}{cB}{0}&\multicolumn{1}{cB}{0}&\multicolumn{1}{cB}{0}\tabularnewline[0pt]
\multicolumn{1}{BcB}{1}&\multicolumn{1}{cB}{0}&\multicolumn{1}{cB}{0}&\multicolumn{1}{cB}{0}&\multicolumn{1}{cB}{1}&\multicolumn{1}{cB}{0}&\multicolumn{1}{cB}{1}&\multicolumn{1}{cB}{1}&\multicolumn{1}{cB}{0}&\multicolumn{1}{cB}{0}\tabularnewline[0pt]
\multicolumn{1}{BcB}{1}&\multicolumn{1}{cB}{1}&\multicolumn{1}{cB}{1}&\multicolumn{1}{cB}{0}&\multicolumn{1}{cB}{0}&\multicolumn{1}{cB}{1}&\multicolumn{1}{cB}{1}&\multicolumn{1}{cB}{1}&\multicolumn{1}{cB}{1}&\multicolumn{1}{cB}{1}\tabularnewline\hrulethick
\end{tabular}
}%
\end{center}%
 By looking at the relevant columns (3, 9, and 10), we can see that all three of these logical expressions are equivalent.  (We might have anticipated this, if we noticed earlier that the converse and the inverse are logically equivalent to each other.)\end{example}
 \begin{example}{}{x:example:exmpl-bicond2}%
Consider the following conditional statements. %
\begin{enumerate}
\item{}If two lines are perpendicular, then the angle between them is \(90^{\circ}\).%
\item{}If a polygon is a right triangle, then it has three sides.%
\end{enumerate}
 Which of these sentences would still be true if it were written in the form of the biconditional?\par\smallskip%
\noindent\textbf{\blocktitlefont Answer}.\label{g:answer:idp229586648}{}\hypertarget{g:answer:idp229586648}{}\quad{}%
\begin{enumerate}
\item{}would still be true since if two lines are not perpendicular, then the angle between them is not \(90^{\circ}\).%
\item{}would not be true since there are many triangles that aren't right triangles.%
\end{enumerate}
\end{example}
 \begin{example}{}{x:example:exmpl-bicond3}%
The following biconditional statement is true: ``If and only if Jason completes his marking, he will not feel guilty.''  Given that, answer the following questions: %
\begin{enumerate}
\item{}Jason feels guilty.  Did he finish his marking?%
\item{}Jason does not feel guilty.  Did he finish his marking?%
\item{}Jason finished his marking.  Does he feel guilty?%
\item{}Jason did not finish his marking.  Does he feel guilty?%
\end{enumerate}
\par\smallskip%
\noindent\textbf{\blocktitlefont Answer}.\label{g:answer:idp229593944}{}\hypertarget{g:answer:idp229593944}{}\quad{}Let \(p=\)``Jason finishes his marking'' and \(q=\)``Jason does not feel guilty.''%
\begin{enumerate}
\item{}The statement \(q\) is false, so \(p\) must also be false (truth of the biconditional requires that both have the same truth value). So Jason did not finish his marking and the answer is \emph{no}.%
\item{}\(q\) is true so \(p\) is true.  Yes.%
\item{}\(p\) is true so \(q\) is true.  And since \(q\) is \emph{not guilty} the answer is \emph{no}.%
\item{}\(p\) is false so \(q\) is false. And feeling ``not not guilty'' is the same as feeling guilty, so the answer is \emph{yes.}%
\end{enumerate}
\end{example}
%
\end{subsectionptx}
%
%
\typeout{************************************************}
\typeout{Subsection 2.8.2 Programming Applications}
\typeout{************************************************}
%
\begin{subsectionptx}{Programming Applications}{}{Programming Applications}{}{}{x:subsection:ssec-programming-apps}
The if-then statement is very common in programming.  In pseudocode, it usually takes the form%
\begin{codedisplay}

              if x > 3 then print "Hello World"
            
\end{codedisplay}
%
\par
When you are debugging, it is tempting to think that this particular code fragment behaves more like the biconditional: if "Hello World" was output, was \(x\gt 3\)?  It's tempting to think so, but what if "Hello World" was printed because of some other command?  What then can we conclude about \(x\)?%
\par
Let's examine this in more detail.  Recall that if \(\conditional\) is true and \(q\) is true, we cannot conclude anything about \(p\).  Now consider the following piece of pseudocode:%
\begin{codedisplay}

              x=4
              y=0
              if x > 3 then y = 5
              print "y = ", y
            
\end{codedisplay}
The output will be "y=5".%
\par
Will the following pseudocode produce the same output?%
\begin{codedisplay}

              x=2
              y=5
              if x > 3 then y=5
              print "y = ",y
            
\end{codedisplay}
It will.  The conditional statement did not change the value of \(y\), but the value was set to 5 initially and so the output will be "y=5".  Once again, knowing that \(q\) is true does not allow us to conclude anything about \(p\) from the conditional \(\conditional\).  However, if the output was "y=4" or any other value not equal to 5, we can draw the conclusion that \(x\) was not greater than 3.  if \(\conditional\) is true and \(q\) is false, we know with certainty that \(p\) is false also.%
\par
The if-then-else construction behaves in a similar fashion.  Consider the following code fragment:%
\begin{codedisplay}

              if x > 3 then
                y = 5
              else
                z = 7
              print "y = ",y,"z = ",z
            
\end{codedisplay}
Only if the output tells you that \(y\neq 5\) or \(z\neq 7\) will you know with certainty something about \(x\).%
\par
For special cases, the if-then-else construction can yield more information.  Consider the following:%
\begin{codedisplay}

              if x > 3 then
                y = 5
              else
                y = 7
              print "y = ",y
            
\end{codedisplay}
Since this piece of pseudocode assigns different values (5 or 7) to the \alert{same} variable \(y\), finding out the resulting value of \(y\) will determine whether \(x\) was greater than 3.  In this special case, the if-then behaves like the biconditional: if \(y=5\) then you know that \(x \gt 3\), and if \(y\ne 5\), then \(x\leq 3\).%
\end{subsectionptx}
%
%
\typeout{************************************************}
\typeout{Exercises 2.8.3 Exercises}
\typeout{************************************************}
%
\begin{exercises-subsection}{Exercises}{}{Exercises}{}{}{x:exercises:exers-sec-biconditional}
\par\medskip\noindent%
\textbf{Exercise Group.}\space\space%
Write out the truth tables for the following logical expressions.  (You might want to do them all in one or two big tables, with columns for each question.)\begin{exercisegroup}
\begin{divisionexerciseeg}{1}{}{}{g:exercise:idp229616728}%
\(\conditional\)\end{divisionexerciseeg}%
\begin{divisionexerciseeg}{2}{}{}{g:exercise:idp229624664}%
\(\inverse\)\end{divisionexerciseeg}%
\begin{divisionexerciseeg}{3}{}{}{g:exercise:idp229630808}%
\(\contrapositive\)\end{divisionexerciseeg}%
\begin{divisionexerciseeg}{4}{}{}{g:exercise:idp229648344}%
\(\converse \)\end{divisionexerciseeg}%
\begin{divisionexerciseeg}{5}{}{}{g:exercise:idp229653080}%
\(\sim\!{p}{\vee} q \)\end{divisionexerciseeg}%
\begin{divisionexerciseeg}{6}{}{}{g:exercise:idp229674840}%
\(p{\wedge}\sim\!{q} \)\end{divisionexerciseeg}%
\begin{divisionexerciseeg}{7}{}{}{g:exercise:idp229681752}%
\(\biconditional \)\end{divisionexerciseeg}%
\begin{divisionexerciseeg}{8}{}{}{g:exercise:idp229691224}%
\(\sim\!{p}{\leftrightarrow}\sim\!{q} \)\end{divisionexerciseeg}%
\begin{divisionexerciseeg}{9}{}{}{g:exercise:idp229708632}%
\(p{\oplus} q \)\end{divisionexerciseeg}%
\begin{divisionexerciseeg}{10}{}{}{g:exercise:idp229717720}%
\(p{\vee}\sim\!{q} \)\end{divisionexerciseeg}%
\begin{divisionexerciseeg}{11}{}{}{g:exercise:idp229718872}%
\(\sim\!{p}{\oplus}\sim\!{q} \)\end{divisionexerciseeg}%
\begin{divisionexerciseeg}{12}{}{}{g:exercise:idp229741400}%
\((\conditional){\wedge}(\converse) \)\end{divisionexerciseeg}%
\begin{divisionexerciseeg}{13}{}{}{g:exercise:idp229743448}%
\((\conditional){\vee}(\converse) \)\end{divisionexerciseeg}%
\begin{divisionexerciseeg}{14}{}{}{g:exercise:idp229830488}%
\((\conditional){\wedge}(\inverse) \)\end{divisionexerciseeg}%
\begin{divisionexerciseeg}{15}{}{}{g:exercise:idp229835480}%
\((\conditional){\vee}(\inverse) \)\end{divisionexerciseeg}%
\end{exercisegroup}
\par\medskip\noindent
\begin{divisionexercise}{16}{}{}{g:exercise:idp229848920}%
Looking at your results for questions 1-15, which expressions are logically equivalent to \(\biconditional\)?\end{divisionexercise}%
\begin{divisionexercise}{17}{}{}{g:exercise:idp229843160}%
Looking at your results for questions 1-15, which expressions are logically equivalent to \(\conditional\)?\end{divisionexercise}%
\begin{divisionexercise}{18}{}{}{g:exercise:idp229851480}%
Looking at your results for questions 1-15, which expressions are logically equivalent to \(\converse\)?\end{divisionexercise}%
\par\medskip\noindent%
\textbf{Exercise Group.}\space\space%
Consider the following conditional statements.  Hopefully you agree that each makes a certain amount of sense.  However, if they were rewritten as \emph{biconditional} statements, would they continue to make sense?  Answer true or false.\begin{exercisegroup}
\begin{divisionexerciseeg}{19}{}{}{g:exercise:idp229849944}%
If Barney is a dog, then Barney has four legs.\end{divisionexerciseeg}%
\begin{divisionexerciseeg}{20}{}{}{g:exercise:idp229851096}%
If Rich is asleep, then he is not playing ping-pong.\end{divisionexerciseeg}%
\begin{divisionexerciseeg}{21}{}{}{g:exercise:idp229850968}%
If Alycia gets 90\% or better as her final mark, then she will get an A+.\end{divisionexerciseeg}%
\begin{divisionexerciseeg}{22}{}{}{g:exercise:idp229850328}%
If Bossy is mooing, then she is a cow.\end{divisionexerciseeg}%
\begin{divisionexerciseeg}{23}{}{}{g:exercise:idp229850712}%
If Pat sleeps in, then she is late for class.\end{divisionexerciseeg}%
\begin{divisionexerciseeg}{24}{}{}{g:exercise:idp229860184}%
If Jason does not pay his bill on time, then he will be charged a late fee.\end{divisionexerciseeg}%
\begin{divisionexerciseeg}{25}{}{}{g:exercise:idp229859416}%
If Susan bought her computer less than a year ago, her warranty is still in effect.\end{divisionexerciseeg}%
\begin{divisionexerciseeg}{26}{}{}{g:exercise:idp229863640}%
If Raymond eats a burger for dinner, they will be too full for dessert.\end{divisionexerciseeg}%
\end{exercisegroup}
\par\medskip\noindent
\par\medskip\noindent%
\textbf{Exercise Group.}\space\space%
In the following exercises, let \(p\) denote "Jason eats a burger for dinner" and let \(q\) denote "Jason is too full for dessert."  Translate the following sentences into symbolic logical statements.\begin{exercisegroup}
\begin{divisionexerciseeg}{27}{}{}{g:exercise:idp229861336}%
If and only if Jason eats a burger for dinner, he will be too full for dessert.\end{divisionexerciseeg}%
\begin{divisionexerciseeg}{28}{}{}{g:exercise:idp229858264}%
Jason will not be too full for dessert if and only if he did not eat a burger for dinner.\end{divisionexerciseeg}%
\begin{divisionexerciseeg}{29}{}{}{g:exercise:idp229858520}%
If Jason eats a burger for dinner, then he will be too full for dessert.\end{divisionexerciseeg}%
\begin{divisionexerciseeg}{30}{}{}{g:exercise:idp229864408}%
If Jason is not too full for dessert, then he did not eat a burger for dinner.\end{divisionexerciseeg}%
\end{exercisegroup}
\par\medskip\noindent
\par\medskip\noindent%
\textbf{Exercise Group.}\space\space%
Are the following two sentences biconditional statements?  (In other words, could you replace them by an equivalent \textasciigrave{}\textasciigrave{}if and only if'{}'{} construction?)\begin{exercisegroup}
\begin{divisionexerciseeg}{31}{}{}{g:exercise:idp229858392}%
If Frank does not pay his bill on time, then he will be charged a late charge, and if he does pay his bill on time, he will not be charged a late charge.\end{divisionexerciseeg}%
\begin{divisionexerciseeg}{32}{}{}{g:exercise:idp229859800}%
If Alycia gets 90\% or better as her final mark, she will get an A+, and if she gets an A+, then she got 90\% or better as her final mark.\end{divisionexerciseeg}%
\end{exercisegroup}
\par\medskip\noindent
\begin{divisionexercise}{33}{}{}{g:exercise:idp229872856}%
If and only if Jason is eaten by bears, he will not finish his marking.  Given that, answer the following questions. %
\begin{enumerate}[label=(\alph*)]
\item{}Jason is eaten by bears.  Did he finish his marking?%
\item{}Jason is not eaten by bears.  Did he finish his marking?%
\item{}Jason finished his marking.  Was he eaten by bears?%
\item{}Jason did not finish his marking.  Whas he eaten by bears?%
\end{enumerate}
\end{divisionexercise}%
\begin{divisionexercise}{34}{}{}{g:exercise:idp229869528}%
The following conditional statement is true: If Rich is asleep, then he is not paying ping-pong and vice versa.  Given that, answer the following questions. %
\begin{enumerate}[label=(\alph*)]
\item{}Rich is playing ping-pong.  Is he asleep?%
\item{}Rich is asleep.  Is he playing ping-pong?%
\item{}Rich is not asleep.  Is he playing ping-pong?%
\item{}Rich is not playing ping-pong.  Is he asleep?%
\end{enumerate}
\end{divisionexercise}%
\begin{divisionexercise}{35}{}{}{g:exercise:idp229870680}%
The following conditional statement is true: Ettercaps are green if and only if toves are slithy.  Given that, answer the following questions. %
\begin{enumerate}[label=(\alph*)]
\item{}Toves are slithy.  Are ettercaps green?%
\item{}Toves are not slithy.  Are ettercaps green?%
\item{}Ettercaps are green.  Are toves slithy?%
\item{}Ettercaps are red.  Are toves slithy?%
\end{enumerate}
\end{divisionexercise}%
\begin{divisionexercise}{36}{}{}{g:exercise:idp229880024}%
If the statement "If and only if Superman has a cape, then he can fly" is a true statement, which of the following cannot occur?  (You may choose more than one.) %
\begin{enumerate}[label=(\alph*)]
\item{}Superman has a cape and he can fly.%
\item{}Superman has a cape and he cannot fly.%
\item{}Superman does not have a cape and cannot fly.%
\item{}Superman does not have a cape and can fly.%
\end{enumerate}
\end{divisionexercise}%
\end{exercises-subsection}
%
%
\typeout{************************************************}
\typeout{Solutions 2.8.4 Solutions to Section~{\xreffont\ref*{x:section:sec-biconditional}} Exercises.}
\typeout{************************************************}
%
\begin{solutions-subsection}{Solutions to Section~{\xreffont\ref*{x:section:sec-biconditional}} Exercises.}{}{Solutions to Section~{\xreffont\ref*{x:section:sec-biconditional}} Exercises.}{}{}{g:solutions:idp229879640}
\par\medskip
\noindent\textbf{\normalsize{}2.8.3\space\textperiodcentered\space{}Exercises}
\begin{exercisegroup}
\begin{divisionsolutioneg}{2.8.3.1}{}{g:exercise:idp229616728}%
\par\smallskip%
\noindent\hypertarget{g:solution:idp229614296-main}{}\begin{center}%
{\tabularfont%
\begin{tabular}{Bccc}\hrulethick
\multicolumn{1}{BcB}{\(p\)}&\multicolumn{1}{cB}{\(q\)}&\multicolumn{1}{cB}{\(\conditional\)}\tabularnewline\hrulemedium
\multicolumn{1}{BcB}{0}&\multicolumn{1}{cB}{0}&\multicolumn{1}{cB}{1}\tabularnewline[0pt]
\multicolumn{1}{BcB}{0}&\multicolumn{1}{cB}{1}&\multicolumn{1}{cB}{1}\tabularnewline[0pt]
\multicolumn{1}{BcB}{1}&\multicolumn{1}{cB}{0}&\multicolumn{1}{cB}{0}\tabularnewline[0pt]
\multicolumn{1}{BcB}{1}&\multicolumn{1}{cB}{1}&\multicolumn{1}{cB}{1}\tabularnewline\hrulethick
\end{tabular}
}%
\end{center}%
\end{divisionsolutioneg}%
\begin{divisionsolutioneg}{2.8.3.2}{}{g:exercise:idp229624664}%
\par\smallskip%
\noindent\hypertarget{g:solution:idp229620440-main}{}\begin{center}%
{\tabularfont%
\begin{tabular}{Bccccc}\hrulethick
\multicolumn{1}{BcB}{\(p\)}&\multicolumn{1}{cB}{\(q\)}&\multicolumn{1}{cB}{\(\sim\!{p}\)}&\multicolumn{1}{cB}{\(\sim\!{q}\)}&\multicolumn{1}{cB}{\(\inverse\)}\tabularnewline\hrulemedium
\multicolumn{1}{BcB}{0}&\multicolumn{1}{cB}{0}&\multicolumn{1}{cB}{1}&\multicolumn{1}{cB}{1}&\multicolumn{1}{cB}{1}\tabularnewline[0pt]
\multicolumn{1}{BcB}{0}&\multicolumn{1}{cB}{1}&\multicolumn{1}{cB}{1}&\multicolumn{1}{cB}{0}&\multicolumn{1}{cB}{0}\tabularnewline[0pt]
\multicolumn{1}{BcB}{1}&\multicolumn{1}{cB}{0}&\multicolumn{1}{cB}{0}&\multicolumn{1}{cB}{1}&\multicolumn{1}{cB}{1}\tabularnewline[0pt]
\multicolumn{1}{BcB}{1}&\multicolumn{1}{cB}{1}&\multicolumn{1}{cB}{0}&\multicolumn{1}{cB}{0}&\multicolumn{1}{cB}{1}\tabularnewline\hrulethick
\end{tabular}
}%
\end{center}%
\end{divisionsolutioneg}%
\begin{divisionsolutioneg}{2.8.3.3}{}{g:exercise:idp229630808}%
\par\smallskip%
\noindent\hypertarget{g:solution:idp229640920-main}{}\begin{center}%
{\tabularfont%
\begin{tabular}{Bccccc}\hrulethick
\multicolumn{1}{BcB}{\(p\)}&\multicolumn{1}{cB}{\(q\)}&\multicolumn{1}{cB}{\(\sim\!{p}\)}&\multicolumn{1}{cB}{\(\sim\!{q}\)}&\multicolumn{1}{cB}{\(\contrapositive\)}\tabularnewline\hrulemedium
\multicolumn{1}{BcB}{0}&\multicolumn{1}{cB}{0}&\multicolumn{1}{cB}{1}&\multicolumn{1}{cB}{1}&\multicolumn{1}{cB}{1}\tabularnewline[0pt]
\multicolumn{1}{BcB}{0}&\multicolumn{1}{cB}{1}&\multicolumn{1}{cB}{1}&\multicolumn{1}{cB}{0}&\multicolumn{1}{cB}{1}\tabularnewline[0pt]
\multicolumn{1}{BcB}{1}&\multicolumn{1}{cB}{0}&\multicolumn{1}{cB}{0}&\multicolumn{1}{cB}{1}&\multicolumn{1}{cB}{0}\tabularnewline[0pt]
\multicolumn{1}{BcB}{1}&\multicolumn{1}{cB}{1}&\multicolumn{1}{cB}{0}&\multicolumn{1}{cB}{0}&\multicolumn{1}{cB}{1}\tabularnewline\hrulethick
\end{tabular}
}%
\end{center}%
\end{divisionsolutioneg}%
\begin{divisionsolutioneg}{2.8.3.4}{}{g:exercise:idp229648344}%
\par\smallskip%
\noindent\hypertarget{g:solution:idp229648600-main}{}\begin{center}%
{\tabularfont%
\begin{tabular}{Bccc}\hrulethick
\multicolumn{1}{BcB}{\(p\)}&\multicolumn{1}{cB}{\(q\)}&\multicolumn{1}{cB}{\(\converse\)}\tabularnewline\hrulemedium
\multicolumn{1}{BcB}{0}&\multicolumn{1}{cB}{0}&\multicolumn{1}{cB}{1}\tabularnewline[0pt]
\multicolumn{1}{BcB}{0}&\multicolumn{1}{cB}{1}&\multicolumn{1}{cB}{0}\tabularnewline[0pt]
\multicolumn{1}{BcB}{1}&\multicolumn{1}{cB}{0}&\multicolumn{1}{cB}{1}\tabularnewline[0pt]
\multicolumn{1}{BcB}{1}&\multicolumn{1}{cB}{1}&\multicolumn{1}{cB}{1}\tabularnewline\hrulethick
\end{tabular}
}%
\end{center}%
\end{divisionsolutioneg}%
\begin{divisionsolutioneg}{2.8.3.5}{}{g:exercise:idp229653080}%
\par\smallskip%
\noindent\hypertarget{g:solution:idp229656664-main}{}\begin{center}%
{\tabularfont%
\begin{tabular}{Bcccc}\hrulethick
\multicolumn{1}{BcB}{\(p\)}&\multicolumn{1}{cB}{\(q\)}&\multicolumn{1}{cB}{\(\sim\!{p}\)}&\multicolumn{1}{cB}{\(\sim\!{p}{\vee} q\)}\tabularnewline\hrulemedium
\multicolumn{1}{BcB}{0}&\multicolumn{1}{cB}{0}&\multicolumn{1}{cB}{1}&\multicolumn{1}{cB}{1}\tabularnewline[0pt]
\multicolumn{1}{BcB}{0}&\multicolumn{1}{cB}{1}&\multicolumn{1}{cB}{1}&\multicolumn{1}{cB}{1}\tabularnewline[0pt]
\multicolumn{1}{BcB}{1}&\multicolumn{1}{cB}{0}&\multicolumn{1}{cB}{0}&\multicolumn{1}{cB}{0}\tabularnewline[0pt]
\multicolumn{1}{BcB}{1}&\multicolumn{1}{cB}{1}&\multicolumn{1}{cB}{0}&\multicolumn{1}{cB}{1}\tabularnewline\hrulethick
\end{tabular}
}%
\end{center}%
\end{divisionsolutioneg}%
\begin{divisionsolutioneg}{2.8.3.6}{}{g:exercise:idp229674840}%
\par\smallskip%
\noindent\hypertarget{g:solution:idp229674072-main}{}\begin{center}%
{\tabularfont%
\begin{tabular}{Bcccc}\hrulethick
\multicolumn{1}{BcB}{\(p\)}&\multicolumn{1}{cB}{\(q\)}&\multicolumn{1}{cB}{\(\sim\!{q}\)}&\multicolumn{1}{cB}{\(p{\wedge}\sim\!{q}\)}\tabularnewline\hrulemedium
\multicolumn{1}{BcB}{0}&\multicolumn{1}{cB}{0}&\multicolumn{1}{cB}{1}&\multicolumn{1}{cB}{0}\tabularnewline[0pt]
\multicolumn{1}{BcB}{0}&\multicolumn{1}{cB}{1}&\multicolumn{1}{cB}{0}&\multicolumn{1}{cB}{0}\tabularnewline[0pt]
\multicolumn{1}{BcB}{1}&\multicolumn{1}{cB}{0}&\multicolumn{1}{cB}{1}&\multicolumn{1}{cB}{1}\tabularnewline[0pt]
\multicolumn{1}{BcB}{1}&\multicolumn{1}{cB}{1}&\multicolumn{1}{cB}{0}&\multicolumn{1}{cB}{0}\tabularnewline\hrulethick
\end{tabular}
}%
\end{center}%
\end{divisionsolutioneg}%
\begin{divisionsolutioneg}{2.8.3.7}{}{g:exercise:idp229681752}%
\par\smallskip%
\noindent\hypertarget{g:solution:idp229682008-main}{}\begin{center}%
{\tabularfont%
\begin{tabular}{Bccc}\hrulethick
\multicolumn{1}{BcB}{\(p\)}&\multicolumn{1}{cB}{\(q\)}&\multicolumn{1}{cB}{\(\biconditional\)}\tabularnewline\hrulemedium
\multicolumn{1}{BcB}{0}&\multicolumn{1}{cB}{0}&\multicolumn{1}{cB}{1}\tabularnewline[0pt]
\multicolumn{1}{BcB}{0}&\multicolumn{1}{cB}{1}&\multicolumn{1}{cB}{0}\tabularnewline[0pt]
\multicolumn{1}{BcB}{1}&\multicolumn{1}{cB}{0}&\multicolumn{1}{cB}{0}\tabularnewline[0pt]
\multicolumn{1}{BcB}{1}&\multicolumn{1}{cB}{1}&\multicolumn{1}{cB}{1}\tabularnewline\hrulethick
\end{tabular}
}%
\end{center}%
\end{divisionsolutioneg}%
\begin{divisionsolutioneg}{2.8.3.8}{}{g:exercise:idp229691224}%
\par\smallskip%
\noindent\hypertarget{g:solution:idp229686744-main}{}\begin{center}%
{\tabularfont%
\begin{tabular}{Bccccc}\hrulethick
\multicolumn{1}{BcB}{\(p\)}&\multicolumn{1}{cB}{\(q\)}&\multicolumn{1}{cB}{\(\sim\!{p}\)}&\multicolumn{1}{cB}{\(\sim\!{q}\)}&\multicolumn{1}{cB}{\(\sim\!{p}{\leftrightarrow}\sim\!{q}\)}\tabularnewline\hrulemedium
\multicolumn{1}{BcB}{0}&\multicolumn{1}{cB}{0}&\multicolumn{1}{cB}{1}&\multicolumn{1}{cB}{1}&\multicolumn{1}{cB}{1}\tabularnewline[0pt]
\multicolumn{1}{BcB}{0}&\multicolumn{1}{cB}{1}&\multicolumn{1}{cB}{1}&\multicolumn{1}{cB}{0}&\multicolumn{1}{cB}{0}\tabularnewline[0pt]
\multicolumn{1}{BcB}{1}&\multicolumn{1}{cB}{0}&\multicolumn{1}{cB}{0}&\multicolumn{1}{cB}{1}&\multicolumn{1}{cB}{0}\tabularnewline[0pt]
\multicolumn{1}{BcB}{1}&\multicolumn{1}{cB}{1}&\multicolumn{1}{cB}{0}&\multicolumn{1}{cB}{0}&\multicolumn{1}{cB}{1}\tabularnewline\hrulethick
\end{tabular}
}%
\end{center}%
\end{divisionsolutioneg}%
\begin{divisionsolutioneg}{2.8.3.9}{}{g:exercise:idp229708632}%
\par\smallskip%
\noindent\hypertarget{g:solution:idp229705688-main}{}\begin{center}%
{\tabularfont%
\begin{tabular}{Bccc}\hrulethick
\multicolumn{1}{BcB}{\(p\)}&\multicolumn{1}{cB}{\(q\)}&\multicolumn{1}{cB}{\(p{\oplus} q\)}\tabularnewline\hrulemedium
\multicolumn{1}{BcB}{0}&\multicolumn{1}{cB}{0}&\multicolumn{1}{cB}{0}\tabularnewline[0pt]
\multicolumn{1}{BcB}{0}&\multicolumn{1}{cB}{1}&\multicolumn{1}{cB}{1}\tabularnewline[0pt]
\multicolumn{1}{BcB}{1}&\multicolumn{1}{cB}{0}&\multicolumn{1}{cB}{1}\tabularnewline[0pt]
\multicolumn{1}{BcB}{1}&\multicolumn{1}{cB}{1}&\multicolumn{1}{cB}{0}\tabularnewline\hrulethick
\end{tabular}
}%
\end{center}%
\end{divisionsolutioneg}%
\begin{divisionsolutioneg}{2.8.3.10}{}{g:exercise:idp229717720}%
\par\smallskip%
\noindent\hypertarget{g:solution:idp229716952-main}{}\begin{center}%
{\tabularfont%
\begin{tabular}{Bcccc}\hrulethick
\multicolumn{1}{BcB}{\(p\)}&\multicolumn{1}{cB}{\(q\)}&\multicolumn{1}{cB}{\(\sim\!{q}\)}&\multicolumn{1}{cB}{\(p{\vee}\sim\!{q}\)}\tabularnewline\hrulemedium
\multicolumn{1}{BcB}{0}&\multicolumn{1}{cB}{0}&\multicolumn{1}{cB}{1}&\multicolumn{1}{cB}{1}\tabularnewline[0pt]
\multicolumn{1}{BcB}{0}&\multicolumn{1}{cB}{1}&\multicolumn{1}{cB}{0}&\multicolumn{1}{cB}{0}\tabularnewline[0pt]
\multicolumn{1}{BcB}{1}&\multicolumn{1}{cB}{0}&\multicolumn{1}{cB}{1}&\multicolumn{1}{cB}{1}\tabularnewline[0pt]
\multicolumn{1}{BcB}{1}&\multicolumn{1}{cB}{1}&\multicolumn{1}{cB}{0}&\multicolumn{1}{cB}{1}\tabularnewline\hrulethick
\end{tabular}
}%
\end{center}%
\end{divisionsolutioneg}%
\begin{divisionsolutioneg}{2.8.3.11}{}{g:exercise:idp229718872}%
\par\smallskip%
\noindent\hypertarget{g:solution:idp229725016-main}{}\begin{center}%
{\tabularfont%
\begin{tabular}{Bccccc}\hrulethick
\multicolumn{1}{BcB}{\(p\)}&\multicolumn{1}{cB}{\(q\)}&\multicolumn{1}{cB}{\(\sim\!{p}\)}&\multicolumn{1}{cB}{\(\sim\!{q}\)}&\multicolumn{1}{cB}{\(\sim\!{p}{\oplus}\sim\!{q}\)}\tabularnewline\hrulemedium
\multicolumn{1}{BcB}{0}&\multicolumn{1}{cB}{0}&\multicolumn{1}{cB}{1}&\multicolumn{1}{cB}{1}&\multicolumn{1}{cB}{0}\tabularnewline[0pt]
\multicolumn{1}{BcB}{0}&\multicolumn{1}{cB}{1}&\multicolumn{1}{cB}{1}&\multicolumn{1}{cB}{0}&\multicolumn{1}{cB}{1}\tabularnewline[0pt]
\multicolumn{1}{BcB}{1}&\multicolumn{1}{cB}{0}&\multicolumn{1}{cB}{0}&\multicolumn{1}{cB}{1}&\multicolumn{1}{cB}{1}\tabularnewline[0pt]
\multicolumn{1}{BcB}{1}&\multicolumn{1}{cB}{1}&\multicolumn{1}{cB}{0}&\multicolumn{1}{cB}{0}&\multicolumn{1}{cB}{0}\tabularnewline\hrulethick
\end{tabular}
}%
\end{center}%
\end{divisionsolutioneg}%
\begin{divisionsolutioneg}{2.8.3.12}{}{g:exercise:idp229741400}%
\par\smallskip%
\noindent\hypertarget{g:solution:idp229736536-main}{}\begin{center}%
{\tabularfont%
\begin{tabular}{Bccc}\hrulethick
\multicolumn{1}{BcB}{\(p\)}&\multicolumn{1}{cB}{\(q\)}&\multicolumn{1}{cB}{\((\conditional){\wedge}(\converse)\)}\tabularnewline\hrulemedium
\multicolumn{1}{BcB}{0}&\multicolumn{1}{cB}{0}&\multicolumn{1}{cB}{1}\tabularnewline[0pt]
\multicolumn{1}{BcB}{0}&\multicolumn{1}{cB}{1}&\multicolumn{1}{cB}{0}\tabularnewline[0pt]
\multicolumn{1}{BcB}{1}&\multicolumn{1}{cB}{0}&\multicolumn{1}{cB}{0}\tabularnewline[0pt]
\multicolumn{1}{BcB}{1}&\multicolumn{1}{cB}{1}&\multicolumn{1}{cB}{1}\tabularnewline\hrulethick
\end{tabular}
}%
\end{center}%
\end{divisionsolutioneg}%
\begin{divisionsolutioneg}{2.8.3.13}{}{g:exercise:idp229743448}%
\par\smallskip%
\noindent\hypertarget{g:solution:idp229486936-main}{}\begin{center}%
{\tabularfont%
\begin{tabular}{Bccc}\hrulethick
\multicolumn{1}{BcB}{\(p\)}&\multicolumn{1}{cB}{\(q\)}&\multicolumn{1}{cB}{\((\conditional){\vee}(\converse)\)}\tabularnewline\hrulemedium
\multicolumn{1}{BcB}{0}&\multicolumn{1}{cB}{0}&\multicolumn{1}{cB}{1}\tabularnewline[0pt]
\multicolumn{1}{BcB}{0}&\multicolumn{1}{cB}{1}&\multicolumn{1}{cB}{1}\tabularnewline[0pt]
\multicolumn{1}{BcB}{1}&\multicolumn{1}{cB}{0}&\multicolumn{1}{cB}{1}\tabularnewline[0pt]
\multicolumn{1}{BcB}{1}&\multicolumn{1}{cB}{1}&\multicolumn{1}{cB}{1}\tabularnewline\hrulethick
\end{tabular}
}%
\end{center}%
\end{divisionsolutioneg}%
\begin{divisionsolutioneg}{2.8.3.14}{}{g:exercise:idp229830488}%
\par\smallskip%
\noindent\hypertarget{g:solution:idp229827672-main}{}\begin{center}%
{\tabularfont%
\begin{tabular}{Bccc}\hrulethick
\multicolumn{1}{BcB}{\(p\)}&\multicolumn{1}{cB}{\(q\)}&\multicolumn{1}{cB}{\((\conditional){\wedge}(\inverse)\)}\tabularnewline\hrulemedium
\multicolumn{1}{BcB}{0}&\multicolumn{1}{cB}{0}&\multicolumn{1}{cB}{1}\tabularnewline[0pt]
\multicolumn{1}{BcB}{0}&\multicolumn{1}{cB}{1}&\multicolumn{1}{cB}{0}\tabularnewline[0pt]
\multicolumn{1}{BcB}{1}&\multicolumn{1}{cB}{0}&\multicolumn{1}{cB}{0}\tabularnewline[0pt]
\multicolumn{1}{BcB}{1}&\multicolumn{1}{cB}{1}&\multicolumn{1}{cB}{1}\tabularnewline\hrulethick
\end{tabular}
}%
\end{center}%
\end{divisionsolutioneg}%
\begin{divisionsolutioneg}{2.8.3.15}{}{g:exercise:idp229835480}%
\par\smallskip%
\noindent\hypertarget{g:solution:idp229837912-main}{}\begin{center}%
{\tabularfont%
\begin{tabular}{Bccc}\hrulethick
\multicolumn{1}{BcB}{\(p\)}&\multicolumn{1}{cB}{\(q\)}&\multicolumn{1}{cB}{\((\conditional){\vee}(\inverse)\)}\tabularnewline\hrulemedium
\multicolumn{1}{BcB}{0}&\multicolumn{1}{cB}{0}&\multicolumn{1}{cB}{1}\tabularnewline[0pt]
\multicolumn{1}{BcB}{0}&\multicolumn{1}{cB}{1}&\multicolumn{1}{cB}{1}\tabularnewline[0pt]
\multicolumn{1}{BcB}{1}&\multicolumn{1}{cB}{0}&\multicolumn{1}{cB}{1}\tabularnewline[0pt]
\multicolumn{1}{BcB}{1}&\multicolumn{1}{cB}{1}&\multicolumn{1}{cB}{1}\tabularnewline\hrulethick
\end{tabular}
}%
\end{center}%
\end{divisionsolutioneg}%
\end{exercisegroup}
\par\medskip\noindent
\begin{divisionsolution}{2.8.3.16}{}{g:exercise:idp229848920}%
\par\smallskip%
\noindent\hypertarget{g:solution:idp229842648-main}{}By comparing the results of 1-15, we see that the following expressions are logically equivalent to \(\biconditional\): %
\begin{enumerate}[label=(\alph*)]
\item{}\(\displaystyle \sim\!{p}{\leftrightarrow}\sim\!{q}\)%
\item{}\(\displaystyle (\conditional){\wedge}(\converse)\)%
\item{}\(\displaystyle (\conditional){\wedge}(\inverse)\)%
\end{enumerate}
 (You may or may not have noticed that it is also equivalent to \(\sim\!{p}{\oplus} q\).)\end{divisionsolution}%
\begin{divisionsolution}{2.8.3.17}{}{g:exercise:idp229843160}%
\par\smallskip%
\noindent\hypertarget{g:solution:idp229844568-main}{}By comparing the results of 1-15, we see that the following expressions are logically equivalent to \(\conditional\): %
\begin{enumerate}[label=(\alph*)]
\item{}\(\displaystyle \contrapositive\)%
\item{}\(\displaystyle \sim\!{p}{\vee} q\)%
\end{enumerate}
\end{divisionsolution}%
\begin{divisionsolution}{2.8.3.18}{}{g:exercise:idp229851480}%
\par\smallskip%
\noindent\hypertarget{g:solution:idp229855064-main}{}By comparing the results of 1-15, we see that the following expressions are logically equivalent to \(\converse\): %
\begin{enumerate}[label=(\alph*)]
\item{}\(\displaystyle \inverse\)%
\item{}\(\displaystyle p{\vee}\sim\!{q}\)%
\end{enumerate}
\end{divisionsolution}%
\begin{exercisegroup}
\begin{divisionsolutioneg}{2.8.3.19}{}{g:exercise:idp229849944}%
\par\smallskip%
\noindent\hypertarget{g:solution:idp229857496-main}{}False\end{divisionsolutioneg}%
\begin{divisionsolutioneg}{2.8.3.20}{}{g:exercise:idp229851096}%
\par\smallskip%
\noindent\hypertarget{g:solution:idp229852504-main}{}False\end{divisionsolutioneg}%
\begin{divisionsolutioneg}{2.8.3.21}{}{g:exercise:idp229850968}%
\par\smallskip%
\noindent\hypertarget{g:solution:idp229849816-main}{}True\end{divisionsolutioneg}%
\begin{divisionsolutioneg}{2.8.3.22}{}{g:exercise:idp229850328}%
\par\smallskip%
\noindent\hypertarget{g:solution:idp229853272-main}{}False\end{divisionsolutioneg}%
\begin{divisionsolutioneg}{2.8.3.23}{}{g:exercise:idp229850712}%
\par\smallskip%
\noindent\hypertarget{g:solution:idp229865176-main}{}False\end{divisionsolutioneg}%
\begin{divisionsolutioneg}{2.8.3.24}{}{g:exercise:idp229860184}%
\par\smallskip%
\noindent\hypertarget{g:solution:idp229859160-main}{}True\end{divisionsolutioneg}%
\begin{divisionsolutioneg}{2.8.3.25}{}{g:exercise:idp229859416}%
\par\smallskip%
\noindent\hypertarget{g:solution:idp229860312-main}{}True\end{divisionsolutioneg}%
\begin{divisionsolutioneg}{2.8.3.26}{}{g:exercise:idp229863640}%
\par\smallskip%
\noindent\hypertarget{g:solution:idp229861592-main}{}False\end{divisionsolutioneg}%
\end{exercisegroup}
\par\medskip\noindent
\begin{exercisegroup}
\begin{divisionsolutioneg}{2.8.3.27}{}{g:exercise:idp229861336}%
\par\smallskip%
\noindent\hypertarget{g:solution:idp229860440-main}{}\(\biconditional\)\end{divisionsolutioneg}%
\begin{divisionsolutioneg}{2.8.3.28}{}{g:exercise:idp229858264}%
\par\smallskip%
\noindent\hypertarget{g:solution:idp229862616-main}{}\(\sim\!{q}{\leftrightarrow}\sim\!{p}\)\end{divisionsolutioneg}%
\begin{divisionsolutioneg}{2.8.3.29}{}{g:exercise:idp229858520}%
\par\smallskip%
\noindent\hypertarget{g:solution:idp229862872-main}{}\(\conditional\)\end{divisionsolutioneg}%
\begin{divisionsolutioneg}{2.8.3.30}{}{g:exercise:idp229864408}%
\par\smallskip%
\noindent\hypertarget{g:solution:idp229864024-main}{}\(\contrapositive\)\end{divisionsolutioneg}%
\end{exercisegroup}
\par\medskip\noindent
\begin{exercisegroup}
\begin{divisionsolutioneg}{2.8.3.31}{}{g:exercise:idp229858392}%
\par\smallskip%
\noindent\hypertarget{g:solution:idp229865560-main}{}Yes\end{divisionsolutioneg}%
\begin{divisionsolutioneg}{2.8.3.32}{}{g:exercise:idp229859800}%
\par\smallskip%
\noindent\hypertarget{g:solution:idp229866456-main}{}Yes\end{divisionsolutioneg}%
\end{exercisegroup}
\par\medskip\noindent
\begin{divisionsolution}{2.8.3.33}{}{g:exercise:idp229872856}%
\par\smallskip%
\noindent\hypertarget{g:solution:idp229869272-main}{}%
\begin{enumerate}[label=(\alph*)]
\item{}No%
\item{}Yes%
\item{}No%
\item{}Yes%
\end{enumerate}
\end{divisionsolution}%
\begin{divisionsolution}{2.8.3.34}{}{g:exercise:idp229869528}%
\par\smallskip%
\noindent\hypertarget{g:solution:idp229867480-main}{}%
\begin{enumerate}[label=(\alph*)]
\item{}No%
\item{}No%
\item{}Yes%
\item{}Yes%
\end{enumerate}
\end{divisionsolution}%
\begin{divisionsolution}{2.8.3.35}{}{g:exercise:idp229870680}%
\par\smallskip%
\noindent\hypertarget{g:solution:idp229871704-main}{}%
\begin{enumerate}[label=(\alph*)]
\item{}Yes%
\item{}No%
\item{}Yes%
\item{}No%
\end{enumerate}
\end{divisionsolution}%
\begin{divisionsolution}{2.8.3.36}{}{g:exercise:idp229880024}%
\par\smallskip%
\noindent\hypertarget{g:solution:idp229875544-main}{}(b) and (d)\end{divisionsolution}%
\end{solutions-subsection}
\end{sectionptx}
\end{chapterptx}
%
%
\typeout{************************************************}
\typeout{Chapter 3 Sequences and Series}
\typeout{************************************************}
%
\begin{chapterptx}{Sequences and Series}{}{Sequences and Series}{}{}{x:chapter:sequences-and-series}
%
%
\typeout{************************************************}
\typeout{Section 3.1 Introduction to Sequences and Series}
\typeout{************************************************}
%
\begin{sectionptx}{Introduction to Sequences and Series}{}{Introduction to Sequences and Series}{}{}{x:section:sec-intro-to-seq-and-ser}
%
%
\typeout{************************************************}
\typeout{Subsection 3.1.1 Sequences}
\typeout{************************************************}
%
\begin{subsectionptx}{Sequences}{}{Sequences}{}{}{x:subsection:ssec-sequences}
Let's start out with the definition of a sequence: \begin{definition}{}{g:definition:idp229881176}%
\index{sequence}A \terminology{sequence} is an ordered list of numbers, often with a pattern.%
\end{definition}
 In a sequence, the number of terms can be finite or infinite.  If a sequence is finite, then either the last term or the total number of terms must be specified so that it's clear where the sequence stops. \begin{example}{}{g:example:idp229874904}%
Which of the following sequences are infinite?  Which are finite? %
\begin{enumerate}[label=(\alph*)]
\item{}\(\displaystyle 7,11,14,19,\ldots\)%
\item{}\(\displaystyle 1,4,9,16,25,36,\ldots,100\)%
\item{}\(\displaystyle 4,2,1,\tfrac{1}{2},\tfrac{1}{4},\tfrac{1}{8},\tfrac{1}{16},\ldots,\tfrac{1}{256}\)%
\end{enumerate}
\par\smallskip%
\noindent\textbf{\blocktitlefont Answer}.\label{g:answer:idp229884760}{}\hypertarget{g:answer:idp229884760}{}\quad{}Sequences (b) and (c) are finite, because their last terms are given.  Sequence (a), however, goes on forever and so is infinite.\end{example}
%
\par
To begin with, let's examine some sequences in detail.  We will begin by looking at sequences that \emph{do} have a pattern. \begin{example}{}{g:example:idp229886040}%
What is the pattern for the following sequences?  What is the next term for the sequence? %
\begin{enumerate}[label=(\alph*)]
\item{}\(\displaystyle 7,11,15,19,\ldots \)%
\item{}\(\displaystyle 1, 4, 9, 16, 25, 36, \ldots 100 \)%
\item{}\(\displaystyle 4,\;2,\;1,\;\frac{1}{2},\;\frac{1}{4},\;\frac{1}{8},\;\frac{1}{{16}},\;...\;\frac{1}{{256}} \)%
\item{}\(\displaystyle 3, -6, 12, -24, \ldots \)%
\item{}\(\displaystyle 3, -6, -15, -24, \ldots \)%
\end{enumerate}
\par\smallskip%
\noindent\textbf{\blocktitlefont Answer}.\label{g:answer:idp229885016}{}\hypertarget{g:answer:idp229885016}{}\quad{}%
\begin{enumerate}[label=(\alph*)]
\item{}The pattern is that you add 4 to the previous term to get the next term.  The next term is then 23.%
\item{}The pattern is that if you say that \textasciigrave{}\textasciigrave{}1'{}'{} is the first term and \textasciigrave{}\textasciigrave{}4'{}'{} is the second term, then \(n^2\) will be the \(\nth{}\) term.  So the next term after 36 is 49.%
\item{}The pattern is to divide each term by two (or multiply by one-half) to get the next term.  So the term after \textdollar{}1\slash{}16\textdollar{} will be \textdollar{}1\slash{}32\textdollar{}.%
\item{}The pattern is to multiply each term by \textdollar{}-2\textdollar{} to get the next term.  The next term is then 48.%
\item{}The pattern is to subtract 9 from the previous term, so the next one is \(-33\).%
\end{enumerate}
\end{example}
%
\par
Note that in this previous example, the last two sequences looked very similar for three of their first four terms.  However, the third term is different so the pattern for the two sequences is not the same and subsequent terms could look very different.%
\end{subsectionptx}
%
%
\typeout{************************************************}
\typeout{Subsection 3.1.2 Notation for Sequences}
\typeout{************************************************}
%
\begin{subsectionptx}{Notation for Sequences}{}{Notation for Sequences}{}{}{x:subsection:ssec-seq-notation}
For each term in a sequence, we will use the notation of a lower-case \(a\) followed by a subscript which is called the index. \index{sequence!notation} So, depending on what we want our starting index to be, our sequence can be written as%
\begin{equation*}
a_0,a_1,a_2,\ldots,a_n
\end{equation*}
or%
\begin{equation*}
a_1,a_2,a_3,\ldots,a_n
\end{equation*}
or even%
\begin{equation*}
a_5,a_6,a_7,\ldots,a_n
\end{equation*}
In this textbook, we will use the convention that the starting index is \(m\), so our sequences can be written as%
\begin{equation*}
a_m, a_{m+1}, a_{m+2}, \ldots, a_n
\end{equation*}
%
\par
Because we are examining sequences from a computing perspective, we should be aware that computing languages don't use a single convention:  many start counting at \(m=0\), while others start at \(m=1\).\footnote{Examples of languages that have a starting index of zero are Python and the C family (C, C++, C\#).  Languages which start their index values at one include Fortran, Smalltalk, and Lua.  There are also languages such as Algol which start at a user-defined value.\label{g:fn:idp229891416}}  In this textbook we will simply specify the start value of our index for each sequence instead of using any one convention.%
\end{subsectionptx}
%
%
\typeout{************************************************}
\typeout{Subsection 3.1.3 Counting the Terms in a Sequence}
\typeout{************************************************}
%
\begin{subsectionptx}{Counting the Terms in a Sequence}{}{Counting the Terms in a Sequence}{}{}{x:subsection:ssec-counting-seq-terms}
\begin{introduction}{}%
Since it's possible to start the index for a sequence at any value, we need to be careful when determining \(k\), the total number of terms in a sequence.  The rule is:%
\begin{equation*}
\text{\#terms}=\text{last}-\text{first}+1
\end{equation*}
and since we are using the convention that \(m\) is the first index and \(n\) is the final index (or, alternatively, some index of interest), then%
\begin{equation*}
k=n-m+1
\end{equation*}
\end{introduction}%
%
%
\typeout{************************************************}
\typeout{Subsubsection  Starting with Index of One}
\typeout{************************************************}
%
\begin{subsubsectionptx}{Starting with Index of One}{}{Starting with Index of One}{}{}{x:subsubsection:sssec-index-one}
Let us consider a sequence that starts with an index of one:%
\begin{equation*}
a_1, a_2, a_3, \ldots, a_n
\end{equation*}
This convention has the advantage that if you label each term as follows:%
\begin{equation*}
\overbrace {{a_1}}^{{\rm{first}}},\overbrace {{a_2}}^{{\rm{second}}},\overbrace {{a_3}}^{{\rm{third}}},\overbrace {{a_4}}^{{\rm{fourth}}},\overbrace {{a_5}}^{{\rm{fifth}}}, \ldots ,\overbrace {{a_n}}^{{\rm{final}}}
\end{equation*}
you can see that the term \(a_5\) has an index \(n=5\) and is also the fifth term, so the number of the term (fifth) and the index (5) are consistent with each other.  This makes it more difficult to make a counting error.  Also, the total number of terms in the sequence \(a_1, a_2, a_3, \ldots, a_n\) is given by \(k = n - 1 + 1\), so \(k = n\) and is consistent with what we would expect.\footnote{In mathematics, it is most common to start counting with \(a_1\) being the first term.  Programming languages primarily designed for mathematics, such as Matlab, usually start with an index of one.\label{g:fn:idp229902552}}%
\end{subsubsectionptx}
%
%
\typeout{************************************************}
\typeout{Subsubsection  Starting with Index of Zero}
\typeout{************************************************}
%
\begin{subsubsectionptx}{Starting with Index of Zero}{}{Starting with Index of Zero}{}{}{x:subsubsection:sssec-index-zero}
However, let us now consider sequences that start with zero:%
\begin{equation*}
a_0, a_1, a_2, a_3, \ldots, a_n
\end{equation*}
Numbering the terms, we find that%
\begin{equation*}
\overbrace {{a_0}}^{{\rm{first}}},\overbrace {{a_1}}^{{\rm{second}}},\overbrace {{a_2}}^{{\rm{third}}},\overbrace {{a_3}}^{{\rm{fourth}}},\overbrace {{a_4}}^{{\rm{fifth}}}, \ldots ,\overbrace {{a_n}}^{{\rm{final}}}
\end{equation*}
and \(a_5\) is no longer the fifth term.  In fact, \(a_5\) is the \emph{sixth} term, which is why it is common in programming to separate the ``count'' of a term (first, second, third, etc.) from the index value (0 for \(a_0\), etc.).%
\par
Also, the total number of terms in \(a_0, a_1, a_2, a_3, \ldots, a_n\) is given by%
\begin{align*}
k \amp = n - m + 1\\
\amp = n - 0 + 1\\
\amp =n+1
\end{align*}
so \(k\), the ``count'' of the term is no longer equal to \(n\), the index of the final term.  So be warned:  if you are not careful with this convention, you are likely to make a type of mistake which programmers commonly call an ``off-by-one'' error.%
\end{subsubsectionptx}
%
%
\typeout{************************************************}
\typeout{Subsubsection  Starting with Index of Two or More}
\typeout{************************************************}
%
\begin{subsubsectionptx}{Starting with Index of Two or More}{}{Starting with Index of Two or More}{}{}{x:subsubsection:sssec-index-two-or-more}
As we have seen, \(a_5\) is only the fifth term in sequences that start with an index of one.  If the sequence starts at some other value, then \(a_5\) could even be the first or second term.  This does lead to a small problem in that the term \(a_n\) is commonly called the the \(\nth{}\) term in a sequence, which is only true for a starting index of one.\footnote{This leads to the awkward convention of calling \(a_0\) the \emph{zeroth} term.\label{g:fn:idp229920600}}%
\end{subsubsectionptx}
\end{subsectionptx}
%
%
\typeout{************************************************}
\typeout{Subsection 3.1.4 Defining a sequence}
\typeout{************************************************}
%
\begin{subsectionptx}{Defining a sequence}{}{Defining a sequence}{}{}{x:subsection:ssec-defining-a-sequence}
There are three ways to define a sequence: \index{sequence!defining}%
\begin{enumerate}[label=(\alph*)]
\item{}List all of the terms, or enough terms to set up the pattern.  If the sequence is finite, then either the final term or the total number of terms must be given.%
\item{}Give a general formula for the \(\nth{}\) term.%
\item{}Give a recursive formula for the \(\nth{}\) term.%
\end{enumerate}
%
\par
We have already looked at sequences defined using the first method in the examples given earlier.  Let's now examine the two types of formula, general and recursive.%
\end{subsectionptx}
%
%
\typeout{************************************************}
\typeout{Subsection 3.1.5 General Formula}
\typeout{************************************************}
%
\begin{subsectionptx}{General Formula}{}{General Formula}{}{}{x:subsection:ssec-general-formula}
A \terminology{general formula} is a formula that gives \(a_n\) as a function of \(n\) only. \index{sequence!general formula}\index{general formula} What this means is that the only variable on the right-hand-side of the general formula is the variable \(n\), and all other values in the equations are constants.%
\par
Let's look at the following examples to examine some sequences defined in this way. \begin{example}{}{g:example:idp229930712}%
Give the first four terms of the sequence given by the general formula \(a_n = 4n + 7\) for \(n \ge 0\).\par\smallskip%
\noindent\textbf{\blocktitlefont Answer}.\label{g:answer:idp229929688}{}\hypertarget{g:answer:idp229929688}{}\quad{}%
\begin{align*}
a_n \amp = 4n + 7, \text{so}\\
a_0 \amp = 4\times0 + 7 = 7\\
a_1 \amp = 4\times1 + 7 = 11\\
a_2 \amp = 4\times2 + 7 = 15\\
a_3 \amp = 4\times3 + 7 = 19
\end{align*}
The first four terms are then 7, 11, 15, and 19.  This is the same sequence that was given as part (a) in the first  example of this section.\end{example}
 \begin{example}{}{g:example:idp229930328}%
Give all terms of the sequence given by the formula \({a_n} = {\left( {\frac{1}{3}} \right)^n}\) for \(1 \le n \le 5\).\par\smallskip%
\noindent\textbf{\blocktitlefont Answer}.\label{g:answer:idp229926360}{}\hypertarget{g:answer:idp229926360}{}\quad{}This is a finite sequence, since restrictions have been placed on the values of \(n\).  The terms are then:%
\begin{align*}
{a_1} \amp = {\left( {\frac{1}{3}} \right)^1} = \frac{1}{3}\\
{a_2} \amp = {\left( {\frac{1}{3}} \right)^2} = \frac{1}{9}\\
{a_3} \amp = {\left( {\frac{1}{3}} \right)^3} = \frac{1}{{27}}\\
{a_4} \amp = {\left( {\frac{1}{3}} \right)^4} = \frac{1}{{81}}\\
{a_5} \amp = {\left( {\frac{1}{3}} \right)^5} = \frac{1}{{243}}
\end{align*}
\end{example}
%
\par
You can see from the previous examples that the general formula allows you to calculate \(a_n\) for any value of \(n\). The very useful thing about the general formula is that you don't need to know the previous term to calculate a particular term.  For instance, if you want to know \(a_{50}\) for the sequence  \(7, 11, 15, 19, \ldots\)  you can determine that the pattern is to add 4 to the previous term to get the next term.  However, to get \(a_{50}\)  you'd have to calculate \(a_{49}\) first, but \(a_{49}\) requires \(a_{48}\)  and so on.  But if you instead use the formula \({a_n} = 4n + 7\) for \(n \ge 0\)  which gives the same sequence, then \(a_{50}\) is just%
\begin{align*}
{a_n} \amp = 4n + 7\\
{a_{50}} \amp = 4 \cdot 50 + 7 = 207
\end{align*}
and there's no need to calculate preceding terms.  Handy!  \footnote{It's important to note, however, that \(a_{50}\) is \emph{not} the fiftieth term.  Because we are starting our index from zero, \(a_{50}\) is the \emph{fifty-first} term since \(k = n - m + 1 = 50 - 0 + 1 = 51\).\label{g:fn:idp229934424}}%
\end{subsectionptx}
%
%
\typeout{************************************************}
\typeout{Subsection 3.1.6 Recursive Definition}
\typeout{************************************************}
%
\begin{subsectionptx}{Recursive Definition}{}{Recursive Definition}{}{}{x:subsection:ssec-recursive-definition}
A recursive formula gives a formula for the next term in terms of the previous one. \index{recursive formula}\index{sequence!recursive formula} For example, in our old friend \(7, 11, 15, 19, \ldots\) , the next term is found by adding 4 to the previous term:  \({a_n} = {a_{n - 1}} + 4\).  However, that's not enough information to uniquely define the series because you don't know where to start.  A complete definition must include the first term also.  Therefore, the recursive definition for our old friend \(7, 11, 15, 19, \ldots\) would be%
\begin{align*}
{a_0} \amp = 7\\
{a_n} \amp = {a_{n - 1}} + 4 \text{ for } n \ge 1
\end{align*}
%
\par
Recursive definitions, then, must specify the first term (or terms, when necessary) \emph{and also} the rule which allows you to calculate the next term from the previous term or terms. \begin{example}{}{g:example:idp229945944}%
Calculate the first four terms of the sequence given by%
\begin{align*}
{a_0} \amp = 3\\
{a_n} \amp = {\left( {{a_{n - 1}} - 1} \right)^2} + 10 \text{ for } n \ge 1
\end{align*}
\par\smallskip%
\noindent\textbf{\blocktitlefont Answer}.\label{g:answer:idp229941720}{}\hypertarget{g:answer:idp229941720}{}\quad{}The first term is already given, \(a_0=3\).  Then%
\begin{align*}
{a_1} \amp = {\left( {3 - 1} \right)^2} + 10 = {2^2} + 10 = 14\\
{a_2} \amp = {\left( {14 - 1} \right)^2} + 10 = {13^2} + 10 = 179\\
{a_3} \amp = {\left( {179 - 1} \right)^2} + 10 = {178^2} + 10 = 31694
\end{align*}
So the first four terms are 3, 14, 179, 31694.\end{example}
 \begin{example}{}{g:example:idp229952600}%
Give a recursive formula for the sequence \(2, 6, 18, 54, \ldots\)\par\smallskip%
\noindent\textbf{\blocktitlefont Answer}.\label{g:answer:idp229952472}{}\hypertarget{g:answer:idp229952472}{}\quad{}The pattern is that the next term equals the previous term times three.  We can start our index at either 0 or 1, so let's choose 1.  Therefore,%
\begin{align*}
{a_1} \amp = 2\\
{a_n} = 3{a_{n - 1}} \text{ for } n \ge 2
\end{align*}
\end{example}
%
\par
Recursive definitions have the same drawback that we've seen before:  if we want to know the \(\upth{200}\) term, we need to calculate the \(\upth{199}\) first, and so on.  Only the general formula allows us to calculate each term directly without knowing the previous one.%
\end{subsectionptx}
%
%
\typeout{************************************************}
\typeout{Subsection 3.1.7 Fibonacci Sequence}
\typeout{************************************************}
%
\begin{subsectionptx}{Fibonacci Sequence}{}{Fibonacci Sequence}{}{}{x:subsection:ssec-fibonacci}
The Fibonacci sequence is likely the most famous example of a recursive sequence: \index{sequence!Fibonacci}%
\begin{equation*}
1, 1, 2, 3, 5, 8, 13, \ldots
\end{equation*}
The pattern can be quite difficult to spot -{}-{} you get the next term from the \emph{sum} of the two previous terms.  The recursive formula for this sequence is therefore%
\begin{align*}
{a_1} \amp = 1\\
{a_2} \amp = 1\\
{a_n} \amp = {a_{n - 1}} + {a_{n - 2}} \text{ for } n \ge 3
\end{align*}
Here, the first \emph{two} terms must be given to start off with so that you are then able to calculate the third term from the previous two.%
\end{subsectionptx}
%
%
\typeout{************************************************}
\typeout{Subsection 3.1.8 Series}
\typeout{************************************************}
%
\begin{subsectionptx}{Series}{}{Series}{}{}{x:subsection:ssec-series}
A series is the sum of the terms of a finite or infinite sequence.  Here are two examples:%
\begin{enumerate}[label=(\alph*)]
\item{}\(\displaystyle 16 + 20 + 24 + 28 + \ldots + 64\)%
\item{}\(\displaystyle 1 + \frac{1}{3} + \frac{1}{9} + \frac{1}{{27}} + ...\)%
\end{enumerate}
The first example is a finite series, while the second one is infinite. The classification as \emph{finite} or \emph{infinite} is based on the number of terms being summed.%
\end{subsectionptx}
%
%
\typeout{************************************************}
\typeout{Subsection 3.1.9 Notation for Series}
\typeout{************************************************}
%
\begin{subsectionptx}{Notation for Series}{}{Notation for Series}{}{}{x:subsection:ssec-notation-for-series}
The sum of the first \(k\) terms of a sequence is denoted by \(S_k\) (also sometimes called the \(\kth{}\) partial sum).  If the series is finite, it could be the sum of \emph{all} of the terms.  \(S_\infinity\) is how we write the sum of an infinite series, like the second example above. \begin{example}{}{g:example:idp229957336}%
For the series \(16 + 20 + 24 + 28 + \ldots + 64\), calculate \(S_3\) and \(S_4\).\par\smallskip%
\noindent\textbf{\blocktitlefont Answer}.\label{g:answer:idp229958616}{}\hypertarget{g:answer:idp229958616}{}\quad{}%
\begin{align*}
S_3 \amp = 16 + 20 + 24 = 60\\
S_4 \amp = 16 + 20 + 24 + 28 = 88
\end{align*}
\end{example}
%
\par
However, it's easy to see that this method becomes very cumbersome for large values of \(k\).  We'll develop some more efficient methods for particular types of series in the next two sections.%
\end{subsectionptx}
%
%
\typeout{************************************************}
\typeout{Subsection 3.1.10 Sigma Notation}
\typeout{************************************************}
%
\begin{subsectionptx}{Sigma Notation}{}{Sigma Notation}{}{}{x:subsection:ssec-sigma-notation}
It's easy to take a sequence in list form and transform it into a series by changing all of the commas to plus signs.  However, what if you are given the general formula instead?  For example, let's take \(7, 11, 15, 19, \ldots\)   which we know to be \({a_n} = 4n + 3\) for \(n \ge 1\).  Since the general form is so useful for finding \(a_n\) when \(n\) is large, it would be nice if we could retain that information while writing our sum.%
\par
To do so, we'll introduce a new notation called \terminology{sigma notation}.  It uses the Greek letter sigma (the uppercase one):  \(\Sigma\), which is commonly used to mean ``sum of''.%
\par
Let's look at an example of sigma notation and discuss what all of the parts mean.  Consider the following sum:%
\begin{equation*}
\sum\limits_{i = 1}^5 {(4i + 3} )
\end{equation*}
The letter \(i\) is an index here, and it runs from the value given at the bottom of the sigma to the number at the top of the sigma in steps of 1.  Here, \(i\) runs from 1 to 5.  We are summing, then, the value of \(4i + 3\) for each value of \(i\) as it runs from 1 to 5:%
\begin{align*}
\sum\limits_{i = 1}^5 {(4i + 3)}  \amp = \overbrace {\left( {4 \times 1 + 3} \right)}^{i = 1} + \overbrace {\left( {4 \times 2 + 3} \right)}^{i = 2} + \overbrace {\left( {4 \times 3 + 3} \right)}^{i = 3} + \overbrace {\left( {4 \times 4 + 3} \right)}^{i = 4} + \overbrace {\left( {4 \times 5 + 3} \right)}^{i = 5}\\
\amp = 7 + 11 + 15 + 19 + 23\\
\amp = 75
\end{align*}
%
\par
Let's look at more examples. \begin{example}{}{g:example:idp229972568}%
Calculate \(\sum\limits_{i = 0}^2 {(2i - 5)}\).\par\smallskip%
\noindent\textbf{\blocktitlefont Answer}.\label{g:answer:idp229976536}{}\hypertarget{g:answer:idp229976536}{}\quad{}%
\begin{align*}
\sum\limits_{i = 0}^2 {(2i - 5)} \amp = \overbrace {\left( {2 \times 0 - 5} \right)}^{i = 0} + \overbrace {\left( {2 \times 1 - 5} \right)}^{i = 1} + \overbrace {\left( {2 \times 2 - 5} \right)}^{i = 2}\\
\amp =  -5 + (- 3) +  (- 1)\\
\amp = -9
\end{align*}
\end{example}
 \begin{example}{}{g:example:idp229975256}%
Calculate \(\sum\limits_{j = 6}^9 {{{\left( {8 - j} \right)}^2}}\)\par\smallskip%
\noindent\textbf{\blocktitlefont Answer}.\label{g:answer:idp229978456}{}\hypertarget{g:answer:idp229978456}{}\quad{}%
\begin{align*}
\sum\limits_{j = 6}^9 {{{\left( {8 - j} \right)}^2}} \amp =\overbrace {\left( {8 - 6} \right)^2} ^{j = 6} + \overbrace {\left( {8 - 7} \right)^2} ^{j = 7} + \overbrace {\left( {8 - 8} \right)^2} ^{j = 8} + \overbrace {\left( {8 - 9} \right)^2}^{j = 9}\\
\amp = 4+1+0+1\\
\amp = 6
\end{align*}
\end{example}
 \begin{example}{}{g:example:idp229974872}%
Calculate \(\sum\limits_{k = 12}^{16} 3\).\par\smallskip%
\noindent\textbf{\blocktitlefont Answer}.\label{g:answer:idp229975128}{}\hypertarget{g:answer:idp229975128}{}\quad{}%
\begin{align*}
\sum\limits_{j = 12}^{16} 3 \amp = \overbrace{3}^{k=12} + \overbrace{3}^{k=13} + \overbrace{3}^{k=14} + \overbrace{3}^{k=15} + \overbrace{3}^{k=16}\\
\amp = 15
\end{align*}
\end{example}
%
\par
The tricky thing about the last one is deciding how many terms there are.  Recall that you can either write out all of the possible values of the index, or use the useful rule:%
\begin{equation*}
k = n - m + 1
\end{equation*}
and as the last example had the index running from 12 to 16, then the number of terms \(k\) is%
\begin{align*}
k \amp = 16 - 12 + 1\\
\amp = 5
\end{align*}
\begin{example}{}{g:example:idp229973464}%
Write the following series in sigma notation:%
\begin{equation*}
4 + 9 + 16 + 25 +\ldots + 100
\end{equation*}
\par\smallskip%
\noindent\textbf{\blocktitlefont Answer}.\label{g:answer:idp229982808}{}\hypertarget{g:answer:idp229982808}{}\quad{}Let's pick our index first.  If we want to be lazy, instead of starting our index at 0 or 1, we could start at 2 and our series would be%
\begin{equation*}
\sum\limits_{k = 2}^{10} {{k^2}}
\end{equation*}
Other acceptable answers would involve changing our starting point for the index to give%
\begin{equation*}
\sum\limits_{j = 1}^9 {{{\left( {j + 1} \right)}^2}}
\end{equation*}
or%
\begin{equation*}
\sum\limits_{i = 0}^8 {{{\left( {i + 2} \right)}^2}}
\end{equation*}
or even%
\begin{equation*}
\sum\limits_{l = 157}^{165} {{{\left( {l - 155} \right)}^2}}
\end{equation*}
if 157 happens to be your favourite number.\end{example}
 \begin{example}{}{g:example:idp229986392}%
Write the following sequence in sigma notation:%
\begin{equation*}
\frac{1}{3} + \frac{1}{4} + \frac{1}{5} + \frac{1}{6} + \ldots
\end{equation*}
\par\smallskip%
\noindent\textbf{\blocktitlefont Answer}.\label{g:answer:idp229983448}{}\hypertarget{g:answer:idp229983448}{}\quad{}%
\begin{equation*}
\sum\limits_{j = 3}^\infty  {\frac{1}{j}}
\end{equation*}
\end{example}
%
\par
To write an infinite series in sigma notation, you just replace the final value of the index with \(\infty\).%
\end{subsectionptx}
%
%
\typeout{************************************************}
\typeout{Exercises 3.1.11 Exercises}
\typeout{************************************************}
%
\begin{exercises-subsection}{Exercises}{}{Exercises}{}{}{g:exercises:idp229987544}
\par\medskip\noindent%
\textbf{Exercise Group.}\space\space%
Predict the next three terms of the following sequnces.\begin{exercisegroup}
\begin{divisionexerciseeg}{1}{}{}{g:exercise:idp229980504}%
\(18, 16, 14\ldots\)\end{divisionexerciseeg}%
\begin{divisionexerciseeg}{2}{}{}{g:exercise:idp229980888}%
\(1, 4, 9, 16, \ldots\)\end{divisionexerciseeg}%
\begin{divisionexerciseeg}{3}{}{}{g:exercise:idp229986008}%
\(12, 24, 48, 96, \ldots\)\end{divisionexerciseeg}%
\begin{divisionexerciseeg}{4}{}{}{g:exercise:idp229990616}%
\(144, 36, 9, \ldots\)\end{divisionexerciseeg}%
\begin{divisionexerciseeg}{5}{}{}{g:exercise:idp229995736}%
\(1,\sqrt 2 ,\sqrt 3 ,2,\sqrt 5 ,\sqrt 6 ,\ldots\)\end{divisionexerciseeg}%
\begin{divisionexerciseeg}{6}{}{}{g:exercise:idp229995352}%
\(5, -10, 20, \ldots\)\end{divisionexerciseeg}%
\begin{divisionexerciseeg}{7}{}{}{g:exercise:idp229994968}%
\(13, 25, 37, 49, \ldots\)\end{divisionexerciseeg}%
\begin{divisionexerciseeg}{8}{}{}{g:exercise:idp229989208}%
\(\frac{1}{2},\frac{1}{3},\frac{1}{4},\frac{1}{5},\ldots\)\end{divisionexerciseeg}%
\end{exercisegroup}
\par\medskip\noindent
\par\medskip\noindent%
\textbf{Exercise Group.}\space\space%
Given a formula for the general term (the \(\nth{}\) term \(a_n\) in terms of \(n\)) of the following sequences.  Use \(n = 1\) as your starting index.\begin{exercisegroup}
\begin{divisionexerciseeg}{9}{}{}{g:exercise:idp230002136}%
\(1, 4, 9, 16, \ldots\)\end{divisionexerciseeg}%
\begin{divisionexerciseeg}{10}{}{}{g:exercise:idp229999960}%
\(1,\sqrt 2 ,\sqrt 3 ,2,\sqrt 5 ,\sqrt 6 ,\ldots\)\end{divisionexerciseeg}%
\begin{divisionexerciseeg}{11}{}{}{g:exercise:idp230000344}%
\(2, 4, 6, 8, \ldots\)\end{divisionexerciseeg}%
\begin{divisionexerciseeg}{12}{}{}{g:exercise:idp230004568}%
\(\frac{1}{2},\frac{1}{3},\frac{1}{4},\frac{1}{5},...\)\end{divisionexerciseeg}%
\end{exercisegroup}
\par\medskip\noindent
\par\medskip\noindent%
\textbf{Exercise Group.}\space\space%
Find the first four terms of the following recursively defined sequences.\begin{exercisegroup}
\begin{divisionexerciseeg}{13}{}{}{g:exercise:idp229999448}%
%
\begin{align*}
a_1 \amp = 2\\
a_n \amp = a_{n - 1} + 5 \text{   for } n \ge 2
\end{align*}
\end{divisionexerciseeg}%
\begin{divisionexerciseeg}{14}{}{}{g:exercise:idp229997144}%
%
\begin{align*}
a_1 \amp = 10\\
a_n \amp = 3 a_{n - 1} \text{   for } n \ge 2
\end{align*}
\end{divisionexerciseeg}%
\begin{divisionexerciseeg}{15}{}{}{g:exercise:idp229998552}%
%
\begin{align*}
a_1 \amp = 2\\
a_2 \amp = 3\\
a_n \amp = a_{n - 1} \times a_{n - 2} \text{   for } n \ge 3
\end{align*}
\end{divisionexerciseeg}%
\begin{divisionexerciseeg}{16}{}{}{g:exercise:idp230005848}%
%
\begin{align*}
a_1 \amp = 2\\
a_n \amp =\frac{1}{a_{n - 1}} + 1 \text{   for } n \ge 2
\end{align*}
\end{divisionexerciseeg}%
\end{exercisegroup}
\par\medskip\noindent
\par\medskip\noindent%
\textbf{Exercise Group.}\space\space%
In each of the following, the general formula for the \(\nth{}\) term of a sequence is given.  Find the first four terms.\begin{exercisegroup}
\begin{divisionexerciseeg}{17}{}{}{g:exercise:idp230009816}%
\(a_n = 3n - 5\) for \(n \ge 1\)\end{divisionexerciseeg}%
\begin{divisionexerciseeg}{18}{}{}{g:exercise:idp230011992}%
\(a_n = 3^{n - 2}\) for \(n \ge 1\)\end{divisionexerciseeg}%
\begin{divisionexerciseeg}{19}{}{}{g:exercise:idp230006360}%
\(a_n = n!\) for \(n \ge 1\)\end{divisionexerciseeg}%
\begin{divisionexerciseeg}{20}{}{}{g:exercise:idp230005336}%
\(a_n = \frac{1}{n^2}\) for \(n \ge 1\)\end{divisionexerciseeg}%
\end{exercisegroup}
\par\medskip\noindent
\par\medskip\noindent%
\textbf{Exercise Group.}\space\space%
In each of the following, the general formula for the \(\nth{}\) term of a sequence is given.  Calculate the specified terms.\begin{exercisegroup}
\begin{divisionexerciseeg}{21}{}{}{g:exercise:idp230019800}%
Find \(a_7\) for the sequence \(a_n = 5\left( 2^{n + 1} \right)\) for \(n \ge 1\)\end{divisionexerciseeg}%
\begin{divisionexerciseeg}{22}{}{}{g:exercise:idp230014680}%
Find \(a_{100}\) for the sequence \(a_n = 4n + 15\) for \(n \ge 1\)\end{divisionexerciseeg}%
\begin{divisionexerciseeg}{23}{}{}{g:exercise:idp230014936}%
Find \(a_{2500}\) for the sequence \(a_n = \frac{n + 2}{n + 1}\) for \(n \ge 1\)\end{divisionexerciseeg}%
\begin{divisionexerciseeg}{24}{}{}{g:exercise:idp230014168}%
Find \(a_{10}\) for the sequence \(a_n = 2n^3\) for \(n \ge 1\)\end{divisionexerciseeg}%
\end{exercisegroup}
\par\medskip\noindent
\par\medskip\noindent%
\textbf{Exercise Group.}\space\space%
Calculate \(S_3\) and \(S_6\) for the following series.\begin{exercisegroup}
\begin{divisionexerciseeg}{25}{}{}{g:exercise:idp230026072}%
\(3 + 6 + 9 + \ldots\)\end{divisionexerciseeg}%
\begin{divisionexerciseeg}{26}{}{}{g:exercise:idp230028760}%
\(1 + 4 + 9 + 16 + \ldots\)\end{divisionexerciseeg}%
\begin{divisionexerciseeg}{27}{}{}{g:exercise:idp230026712}%
\(5 -10 + 20 - 40 + \ldots\)\end{divisionexerciseeg}%
\begin{divisionexerciseeg}{28}{}{}{g:exercise:idp230024664}%
\(5 + 3 + 1 + \ldots\)\end{divisionexerciseeg}%
\end{exercisegroup}
\par\medskip\noindent
\par\medskip\noindent%
\textbf{Exercise Group.}\space\space%
Write out each sum in full and then evaluate.\begin{exercisegroup}
\begin{divisionexerciseeg}{29}{}{}{g:exercise:idp230025176}%
\(\sum\limits_{n = 3}^7 n \)\end{divisionexerciseeg}%
\begin{divisionexerciseeg}{30}{}{}{g:exercise:idp230029272}%
\(\sum\limits_{j = 4}^{10} {{{\left( { - 1} \right)}^j}}\)\end{divisionexerciseeg}%
\begin{divisionexerciseeg}{31}{}{}{g:exercise:idp230026840}%
\(\sum\limits_{i = 0}^4 {{2^i}}\)\end{divisionexerciseeg}%
\begin{divisionexerciseeg}{32}{}{}{g:exercise:idp230021848}%
\(\sum\limits_{k = 20}^{25} {\left( {3k - 10} \right)}\)\end{divisionexerciseeg}%
\end{exercisegroup}
\par\medskip\noindent
\par\medskip\noindent%
\textbf{Exercise Group.}\space\space%
Write each series in sigma notation.  (Answers may vary.)\begin{exercisegroup}
\begin{divisionexerciseeg}{33}{}{}{g:exercise:idp230034264}%
\(1 + 8 + 27 + 64 + \ldots + 1000\)\end{divisionexerciseeg}%
\begin{divisionexerciseeg}{34}{}{}{g:exercise:idp230029656}%
\(\frac{1}{2} + \frac{1}{3} + \frac{1}{4} + \frac{1}{5} + \ldots\)\end{divisionexerciseeg}%
\begin{divisionexerciseeg}{35}{}{}{g:exercise:idp230030936}%
\(2 + 4 + 6 + 8 + \ldots\)\end{divisionexerciseeg}%
\begin{divisionexerciseeg}{36}{}{}{g:exercise:idp230036184}%
\(2 + 4 + 6 + 8\)\end{divisionexerciseeg}%
\end{exercisegroup}
\par\medskip\noindent
\par\medskip\noindent%
\textbf{Exercise Group.}\space\space%
Evil alert!  The following questions are just for those wanting a challenge.  This type of question will not be tested.\begin{exercisegroup}
\begin{divisionexerciseeg}{37}{}{}{g:exercise:idp230035032}%
(nasty)  Write the sequence \(1, 4, 9, 16, \ldots\) using a \emph{recursive} definition.\end{divisionexerciseeg}%
\begin{divisionexerciseeg}{38}{}{}{g:exercise:idp230031832}%
(thorny)  Write the sequence \(1, 2, 6, 24, \ldots\) using a \emph{general} formula.\end{divisionexerciseeg}%
\begin{divisionexerciseeg}{39}{}{}{g:exercise:idp230043096}%
(tricksy)  Consider the following sequence:%
\begin{equation*}
4, 5, 20, 100, 2000
\end{equation*}
%
\begin{enumerate}[label=(\alph*)]
\item{}What's the next term in this sequence?%
\item{}What's the recursive formula for this sequence?%
\end{enumerate}
\end{divisionexerciseeg}%
\end{exercisegroup}
\par\medskip\noindent
\end{exercises-subsection}
%
%
\typeout{************************************************}
\typeout{Solutions 3.1.12 Solutions to Section~{\xreffont\ref*{x:section:sec-intro-to-seq-and-ser}} Exercises}
\typeout{************************************************}
%
\begin{solutions-subsection}{Solutions to Section~{\xreffont\ref*{x:section:sec-intro-to-seq-and-ser}} Exercises}{}{Solutions to Section~{\xreffont\ref*{x:section:sec-intro-to-seq-and-ser}} Exercises}{}{}{g:solutions:idp230039384}
\par\medskip
\noindent\textbf{\normalsize{}3.1.11\space\textperiodcentered\space{}Exercises}
\begin{exercisegroup}
\begin{divisionsolutioneg}{3.1.11.1}{}{g:exercise:idp229980504}%
\par\smallskip%
\noindent\hypertarget{g:solution:idp229984344-main}{}\(12, 10, 8\) (pattern is to subtract 2)\end{divisionsolutioneg}%
\begin{divisionsolutioneg}{3.1.11.2}{}{g:exercise:idp229980888}%
\par\smallskip%
\noindent\hypertarget{g:solution:idp229986904-main}{}\(25, 36, 49\) (\(\nth{}\) term is equal to \(n^2\))\end{divisionsolutioneg}%
\begin{divisionsolutioneg}{3.1.11.3}{}{g:exercise:idp229986008}%
\par\smallskip%
\noindent\hypertarget{g:solution:idp229990232-main}{}\(192, 384, 768\) (multiply by 2)\end{divisionsolutioneg}%
\begin{divisionsolutioneg}{3.1.11.4}{}{g:exercise:idp229990616}%
\par\smallskip%
\noindent\hypertarget{g:solution:idp229992920-main}{}\(\frac{9}{4},\frac{9}{{16}},\frac{9}{{64}}\) (divide by 4)\end{divisionsolutioneg}%
\begin{divisionsolutioneg}{3.1.11.5}{}{g:exercise:idp229995736}%
\par\smallskip%
\noindent\hypertarget{g:solution:idp229995096-main}{}\(\sqrt 7 ,2\sqrt 2 ,3\) (\(\nth{}\) term is \(\sqrt{n}\))\end{divisionsolutioneg}%
\begin{divisionsolutioneg}{3.1.11.6}{}{g:exercise:idp229995352}%
\par\smallskip%
\noindent\hypertarget{g:solution:idp229991128-main}{}\(-40, 80, -160\) (multiply by \(-2\))\end{divisionsolutioneg}%
\begin{divisionsolutioneg}{3.1.11.7}{}{g:exercise:idp229994968}%
\par\smallskip%
\noindent\hypertarget{g:solution:idp229993560-main}{}\(61, 73, 85\) (add 12)\end{divisionsolutioneg}%
\begin{divisionsolutioneg}{3.1.11.8}{}{g:exercise:idp229989208}%
\par\smallskip%
\noindent\hypertarget{g:solution:idp229989080-main}{}\(\frac{1}{6},\frac{1}{7},\frac{1}{8}\) (denominator increases by 1)\end{divisionsolutioneg}%
\end{exercisegroup}
\par\medskip\noindent
\begin{exercisegroup}
\begin{divisionsolutioneg}{3.1.11.9}{}{g:exercise:idp230002136}%
\par\smallskip%
\noindent\hypertarget{g:solution:idp230004824-main}{}\(a_n = n^2\)\end{divisionsolutioneg}%
\begin{divisionsolutioneg}{3.1.11.10}{}{g:exercise:idp229999960}%
\par\smallskip%
\noindent\hypertarget{g:solution:idp229998808-main}{}\(a_n = \sqrt{n} \)\end{divisionsolutioneg}%
\begin{divisionsolutioneg}{3.1.11.11}{}{g:exercise:idp230000344}%
\par\smallskip%
\noindent\hypertarget{g:solution:idp229997784-main}{}\(a_n = 2n\)\end{divisionsolutioneg}%
\begin{divisionsolutioneg}{3.1.11.12}{}{g:exercise:idp230004568}%
\par\smallskip%
\noindent\hypertarget{g:solution:idp229997016-main}{}\(a_n = \frac{1}{n + 1}\)\end{divisionsolutioneg}%
\end{exercisegroup}
\par\medskip\noindent
\begin{exercisegroup}
\begin{divisionsolutioneg}{3.1.11.13}{}{g:exercise:idp229999448}%
\par\smallskip%
\noindent\hypertarget{g:solution:idp229999576-main}{}\(2, 7, 12, 17\)\end{divisionsolutioneg}%
\begin{divisionsolutioneg}{3.1.11.14}{}{g:exercise:idp229997144}%
\par\smallskip%
\noindent\hypertarget{g:solution:idp230001240-main}{}\(10, 30, 90, 270\)\end{divisionsolutioneg}%
\begin{divisionsolutioneg}{3.1.11.15}{}{g:exercise:idp229998552}%
\par\smallskip%
\noindent\hypertarget{g:solution:idp230011352-main}{}\(2, 3, 6, 18\)\end{divisionsolutioneg}%
\begin{divisionsolutioneg}{3.1.11.16}{}{g:exercise:idp230005848}%
\par\smallskip%
\noindent\hypertarget{g:solution:idp230013016-main}{}\(2,\frac{3}{2},\frac{5}{3},\frac{8}{5}\)\end{divisionsolutioneg}%
\end{exercisegroup}
\par\medskip\noindent
\begin{exercisegroup}
\begin{divisionsolutioneg}{3.1.11.17}{}{g:exercise:idp230009816}%
\par\smallskip%
\noindent\hypertarget{g:solution:idp230010328-main}{}\(-2, 1, 4, 7\)\end{divisionsolutioneg}%
\begin{divisionsolutioneg}{3.1.11.18}{}{g:exercise:idp230011992}%
\par\smallskip%
\noindent\hypertarget{g:solution:idp230008152-main}{}\(\frac{1}{3},1,3,9\)\end{divisionsolutioneg}%
\begin{divisionsolutioneg}{3.1.11.19}{}{g:exercise:idp230006360}%
\par\smallskip%
\noindent\hypertarget{g:solution:idp230007128-main}{}\(1, 2, 6, 24\)\end{divisionsolutioneg}%
\begin{divisionsolutioneg}{3.1.11.20}{}{g:exercise:idp230005336}%
\par\smallskip%
\noindent\hypertarget{g:solution:idp230005464-main}{}\(1,\frac{1}{4},\frac{1}{9},\frac{1}{16}\)\end{divisionsolutioneg}%
\end{exercisegroup}
\par\medskip\noindent
\begin{exercisegroup}
\begin{divisionsolutioneg}{3.1.11.21}{}{g:exercise:idp230019800}%
\par\smallskip%
\noindent\hypertarget{g:solution:idp230017112-main}{}\(a_7 = 1280\)\end{divisionsolutioneg}%
\begin{divisionsolutioneg}{3.1.11.22}{}{g:exercise:idp230014680}%
\par\smallskip%
\noindent\hypertarget{g:solution:idp230014296-main}{}\(a_{100} = 415\)\end{divisionsolutioneg}%
\begin{divisionsolutioneg}{3.1.11.23}{}{g:exercise:idp230014936}%
\par\smallskip%
\noindent\hypertarget{g:solution:idp230013272-main}{}\(a_{2500} = \frac{2502}{2501}\)\end{divisionsolutioneg}%
\begin{divisionsolutioneg}{3.1.11.24}{}{g:exercise:idp230014168}%
\par\smallskip%
\noindent\hypertarget{g:solution:idp230013784-main}{}\(a_{10} = 2000\)\end{divisionsolutioneg}%
\end{exercisegroup}
\par\medskip\noindent
\begin{exercisegroup}
\begin{divisionsolutioneg}{3.1.11.25}{}{g:exercise:idp230026072}%
\par\smallskip%
\noindent\hypertarget{g:solution:idp230025944-main}{}\(S_3 = 18, S_6 = 63\)\end{divisionsolutioneg}%
\begin{divisionsolutioneg}{3.1.11.26}{}{g:exercise:idp230028760}%
\par\smallskip%
\noindent\hypertarget{g:solution:idp230024920-main}{}\(S_3 = 14, S_6 = 91\)\end{divisionsolutioneg}%
\begin{divisionsolutioneg}{3.1.11.27}{}{g:exercise:idp230026712}%
\par\smallskip%
\noindent\hypertarget{g:solution:idp230029400-main}{}\(S_3 = 15, S_6 = -105\)\end{divisionsolutioneg}%
\begin{divisionsolutioneg}{3.1.11.28}{}{g:exercise:idp230024664}%
\par\smallskip%
\noindent\hypertarget{g:solution:idp230023640-main}{}\(S_3 = 9, S_6 = 0\)\end{divisionsolutioneg}%
\end{exercisegroup}
\par\medskip\noindent
\begin{exercisegroup}
\begin{divisionsolutioneg}{3.1.11.29}{}{g:exercise:idp230025176}%
\par\smallskip%
\noindent\hypertarget{g:solution:idp230022872-main}{}\(\sum\limits_{n = 3}^7 n  = 3 + 4 + 5 + 6 + 7 = 25\)\end{divisionsolutioneg}%
\begin{divisionsolutioneg}{3.1.11.30}{}{g:exercise:idp230029272}%
\par\smallskip%
\noindent\hypertarget{g:solution:idp230026584-main}{}\(\sum\limits_{j = 4}^{10} {{{\left( { - 1} \right)}^j}}  = 1 + \left( { - 1} \right) + 1 + \left( { - 1} \right) + 1 + \left( { - 1} \right) + 1 = 1\)\end{divisionsolutioneg}%
\begin{divisionsolutioneg}{3.1.11.31}{}{g:exercise:idp230026840}%
\par\smallskip%
\noindent\hypertarget{g:solution:idp230023000-main}{}\(\sum\limits_{i = 0}^4 {{2^i}}  = {2^0} + {2^1} + {2^2} + {2^3} + {2^4} = 1 + 2 + 4 + 8 + 16 = 31\)\end{divisionsolutioneg}%
\begin{divisionsolutioneg}{3.1.11.32}{}{g:exercise:idp230021848}%
\par\smallskip%
\noindent\hypertarget{g:solution:idp230021976-main}{}\(\sum\limits_{k = 20}^{25} {3k - 10}  = 50 + 53 + 56 + 59 + 62 + 65 = 345\)\end{divisionsolutioneg}%
\end{exercisegroup}
\par\medskip\noindent
\begin{exercisegroup}
\begin{divisionsolutioneg}{3.1.11.33}{}{g:exercise:idp230034264}%
\par\smallskip%
\noindent\hypertarget{g:solution:idp230037592-main}{}\(\sum\limits_{i = 1}^{10} i^3 \)\end{divisionsolutioneg}%
\begin{divisionsolutioneg}{3.1.11.34}{}{g:exercise:idp230029656}%
\par\smallskip%
\noindent\hypertarget{g:solution:idp230035544-main}{}\(\sum\limits_{j = 2}^{\infty}  \frac{1}{j}\)\end{divisionsolutioneg}%
\begin{divisionsolutioneg}{3.1.11.35}{}{g:exercise:idp230030936}%
\par\smallskip%
\noindent\hypertarget{g:solution:idp230030296-main}{}\(\sum\limits_{k = 1}^{\infty} 2k\)\end{divisionsolutioneg}%
\begin{divisionsolutioneg}{3.1.11.36}{}{g:exercise:idp230036184}%
\par\smallskip%
\noindent\hypertarget{g:solution:idp230031576-main}{}\(\sum\limits_{k = 1}^4 2k\)\end{divisionsolutioneg}%
\end{exercisegroup}
\par\medskip\noindent
\begin{exercisegroup}
\begin{divisionsolutioneg}{3.1.11.37}{}{g:exercise:idp230035032}%
\par\smallskip%
\noindent\hypertarget{g:solution:idp230036440-main}{}You could either do%
\begin{align*}
a_1 \amp = 1\\
a_n \amp = \left( \sqrt{a_{n - 1}}  + 1 \right)^2 \text{   for } n \ge 2
\end{align*}
or another possibility is%
\begin{align*}
a_1 \amp = 1\\
a_n \amp = a_{n - 1} + 2n - 1 \text{   for } n \ge 2
\end{align*}
\end{divisionsolutioneg}%
\begin{divisionsolutioneg}{3.1.11.38}{}{g:exercise:idp230031832}%
\par\smallskip%
\noindent\hypertarget{g:solution:idp230040408-main}{}\({a_n} = n!\) for \(n \ge 1\)\end{divisionsolutioneg}%
\begin{divisionsolutioneg}{3.1.11.39}{}{g:exercise:idp230043096}%
\par\smallskip%
\noindent\hypertarget{g:solution:idp230044376-main}{}%
\begin{enumerate}[label=(\alph*)]
\item{}The next term is 200 000.%
\item{}%
\begin{align*}
a_1 \amp = 4\\
a_2 \amp = 5\\
a_n \amp = a_{n - 1} \times a_{n - 2}
\end{align*}
%
\end{enumerate}
\end{divisionsolutioneg}%
\end{exercisegroup}
\par\medskip\noindent
\end{solutions-subsection}
\end{sectionptx}
%
%
\typeout{************************************************}
\typeout{Section 3.2 Arithmetic Sequences and Series}
\typeout{************************************************}
%
\begin{sectionptx}{Arithmetic Sequences and Series}{}{Arithmetic Sequences and Series}{}{}{x:section:sec-arith-seq-ser}
%
%
\typeout{************************************************}
\typeout{Subsection 3.2.1 Arithmetic Sequences}
\typeout{************************************************}
%
\begin{subsectionptx}{Arithmetic Sequences}{}{Arithmetic Sequences}{}{}{x:subsection:ssec-arith-seq}
Let's start out with a definition: \begin{definition}{}{x:definition:def-arith-seq}%
An \terminology{arithmetic sequence} is a sequence in which the next term is found by adding a constant (the \emph{common difference}\(\ d\)) to the previous term.\end{definition}
%
\par
Here are some examples of arithmetic sequences:%
\begin{enumerate}[label=(\alph*)]
\item{}\(\displaystyle 7, 11, 15, 19, \ldots \)%
\item{}\(\displaystyle 11, 4, -3, -10, \ldots -59\)%
\item{}\(\displaystyle 12, 12.3, 12.6, 12.9, \ldots\)%
\end{enumerate}
The first one has a common difference of 4, the second \(-7\), and the third 0.3.  Note that in each of them, we can find the common difference \(d\) by taking \emph{any} term and subtracting the previous term from it. \begin{example}{}{g:example:idp230053592}%
For the following sequences, state whether each of them is arithmetic. %
\begin{enumerate}[label=(\alph*)]
\item{}\(\displaystyle -3, -10, -17, -24, \ldots\)%
\item{}\(\displaystyle 4, 5, 7, 10, \ldots\)%
\item{}\(\displaystyle 2, 4, 8, 16, \ldots\)%
\item{}\(\displaystyle \frac{1}{2},\frac{1}{3},\frac{1}{4},\frac{1}{5},\ldots \frac{1}{20}\)%
\end{enumerate}
\par\smallskip%
\noindent\textbf{\blocktitlefont Answer}.\label{g:answer:idp230053464}{}\hypertarget{g:answer:idp230053464}{}\quad{}%
\begin{enumerate}[label=(\alph*)]
\item{}Yes, because the common difference \(d\) is \(-7\).%
\item{}No, because you're not adding the same number each time.%
\item{}No, because you're multiplying by 2 to get the next term, not adding.%
\item{}No, because the difference between each pair of terms is different.%
\end{enumerate}
\end{example}
%
\par
Again, you can define an arithmetic sequence in one of three ways:  by listing the terms, by giving a recursive definition, or by giving a general definition.%
\end{subsectionptx}
%
%
\typeout{************************************************}
\typeout{Subsection 3.2.2 Recursive Definitions for Arithmetic Sequences}
\typeout{************************************************}
%
\begin{subsectionptx}{Recursive Definitions for Arithmetic Sequences}{}{Recursive Definitions for Arithmetic Sequences}{}{}{x:subsection:ssec-recurdef-arith-seq}
Let's first look at an example: \begin{example}{}{g:example:idp230059224}%
Give a recursive definition for the sequence \(2, 10, 18, 26, \ldots\).\par\smallskip%
\noindent\textbf{\blocktitlefont Answer}.\label{g:answer:idp230061400}{}\hypertarget{g:answer:idp230061400}{}\quad{}Recall that a recursive definition has two parts:  listing the first term and giving the pattern.  In this case, the pattern is adding \(d = 8\) to the previous term to get the next term.  We can start our index anywhere, so let's choose zero for this example.  The recursive definition is therefore%
\begin{align*}
a_0 \amp =2\\
a_n \amp = a_{n-1}+8 \text{   for }n\ge 1
\end{align*}
\end{example}
%
\par
To generalize, the recursive formula for \emph{any} arithmetic sequence is%
\begin{align*}
a_m\amp = \lt\text{insert value here}\gt\\
a_n\amp = a_{n-1}+d \text{   for }n\ge m+1
\end{align*}
%
\end{subsectionptx}
%
%
\typeout{************************************************}
\typeout{Subsection 3.2.3 General Formulae for Arithmetic Sequences}
\typeout{************************************************}
%
\begin{subsectionptx}{General Formulae for Arithmetic Sequences}{}{General Formulae for Arithmetic Sequences}{}{}{x:subsection:ssec-genform-arith-seq}
Let's examine the previous example in more detail to see if we can recognize any patterns and come up with a general formula.  Rewriting each term, we get%
\begin{equation*}
\begin{array}{*{20}{c}}
{2,}\amp{10,}\amp{18,}\amp{26,}\amp{\ldots}\\
2\amp{2 + 8,}\amp{2 + 8\times 2,}\amp{2 + 8\times 3,}\amp{\ldots}
\end{array}
\end{equation*}
%
\par
So the \(\uprd{3}\) term equals the first plus 8 times 2, the \(\upth{4}\) term equals the first plus 8 times 3, and the \(\nth{}\) term will equal the first plus 8 times \((n - 1)\).   In other words,%
\begin{equation*}
\begin{array}{*{20}{c}}
{2,}\amp{10,}\amp{18,}\amp{26,}\amp{\ldots}\amp{a_n}\\
2\amp{2 + 8,}\amp{2 + 8 \times 2,}\amp{2 + 8 \times 3,}\amp{\ldots}\amp{2+8 \times (n-1)}
\end{array}
\end{equation*}
and so we find for this particular sequence, \(a_n=2+8 \times (n-1)\), which simplifies to \(a_n=8n-6\).%
\par
We can generalize this formula: the \(\nth{}\) term will equal the first plus \(d\) times \((n - m)\), so%
\begin{equation*}
a_n = a_m + \left( n - m \right)d \text{   where } n \ge m
\end{equation*}
for any \emph{arithmetic sequence}. \begin{example}{}{g:example:idp230067672}%
Write a general formula for the sequence \(3, 8, 13, 18, \ldots\).\par\smallskip%
\noindent\textbf{\blocktitlefont Answer}.\label{g:answer:idp230070104}{}\hypertarget{g:answer:idp230070104}{}\quad{}This sequence is arithmetic with the first term 3 and common difference 5.  Let's use a starting index of zero.%
\begin{align*}
a_n \amp = a_m + \left( n - m \right)d\\
\amp = {a_0} + \left( {n - 0} \right)d\\
\amp = 3 + \left( {n} \right)5\\
\amp = 3 + 5n\\
\amp = 5n + 3
\end{align*}
The general formula is then \(a_n = 5n + 3\) for \(n \ge 0\).\end{example}
 \begin{example}{}{g:example:idp230072920}%
What is the \(\upth{50}\) term in the sequence in the sequence \(3, 8, 13, 18, \ldots\)?\par\smallskip%
\noindent\textbf{\blocktitlefont Answer}.\label{g:answer:idp230074968}{}\hypertarget{g:answer:idp230074968}{}\quad{}This is the same sequence from the previous example.  We may then use the  formula we derived, \(a_n = 5n + 3\) for \(n \ge 0\).  But we do have to be careful about our index \(n\).  Recalling that the number of terms \(k\) is given by%
\begin{equation*}
k=n-m+1
\end{equation*}
where we used \(m = 0\) as our starting index, then%
\begin{align*}
50 \amp = n-0=1\\
n\amp = 49
\end{align*}
and so the \(\upth{50}\) term will be \(a_{49}\).%
\begin{align*}
a_n \amp = 5n + 3\\
a_{49} \amp = 5 \times 49 + 3\\
\amp = 245 + 3\\
\amp =248
\end{align*}
The \(\upth{50}\) term is 248.\end{example}
 \begin{example}{}{g:example:idp230080728}%
What is the common difference in the arithmetic sequence in which the first term is 18 and the twelfth term is \(-59\)?\par\smallskip%
\noindent\textbf{\blocktitlefont Answer}.\label{g:answer:idp230083928}{}\hypertarget{g:answer:idp230083928}{}\quad{}The easiest way to count these terms correctly is to have a starting index of one, and then the first term is \(a_1\) and the twelfth term is \(a_{12}\).%
\begin{align*}
a_n \amp = a_m + \left( n - m \right)d\\
a_{12} \amp = a_1 + \left( 12 - 1 \right)d\\
-59 \amp = 18 + \left( 12 - 1 \right)d\\
-77 \amp = 11d\\
d\amp = -7
\end{align*}
The common difference is \(-7\).\end{example}
 \begin{example}{}{g:example:idp230083160}%
Which term has a value of 404 in the sequence \(-37, -28, -19, \ldots\)?\par\smallskip%
\noindent\textbf{\blocktitlefont Answer}.\label{g:answer:idp230083800}{}\hypertarget{g:answer:idp230083800}{}\quad{}Let's use a starting index of one.  So \(a_1\) is \(-37\) and \(d\) is \(+9\).  Then we want to find the value of \(n\) for which \(a_n\) equals 404.%
\begin{align*}
a_n \amp = a_m + \left( n - m \right)d\\
\amp = a_1 + \left(n - 1 \right)d\\
404 \amp =  - 37 + \left( n - 1 \right)9\\
441 \amp = 9\left( n - 1 \right)\\
49 \amp = n - 1\\
n \amp = 50
\end{align*}
The \emph{fiftieth} term is 404.\end{example}
%
\end{subsectionptx}
%
%
\typeout{************************************************}
\typeout{Subsection 3.2.4 Arithmetic Series}
\typeout{************************************************}
%
\begin{subsectionptx}{Arithmetic Series}{}{Arithmetic Series}{}{}{x:subsection:ssec-arith-ser}
Recall that \(S_k\) is the sum of the first \(k\) terms of a series.  Let's look at a couple of examples of arithmetic series to see if we can identify any patterns.%
\par
Suppose we wish to take some partial sums of the series \(2 + 10 + 18 + 26 + \ldots\).  Let's first calculate \(S_6\).  We could just find the first six terms and add them up, but notice the following:%
\begin{equation*}
S_6 = 2 + 10 + 18 + 26 + 34 + 42
\end{equation*}
The sum of the first and last numbers is 44.  The sum of the second and second-to-last is also 44.  So is the sum of the third and third-last.  So when you take the terms in pairs, each pair has the same sum, \((a_m + a_n)\), and there are \(k/2\) pairs in total.  Then%
\begin{equation*}
S_k = \frac{k}{2}\left( a_m + a_n \right)
\end{equation*}
What if, however, there are an odd number of terms?  Let's also calculate \(S_7\):%
\begin{equation*}
S_7 = 2 + 10 + 18 + 26 + 34 + 42 + 50
\end{equation*}
The sum of the first and last is 52, as is the sum of the each ``inner pair''.  Notice that the middle, unpaired value, is \(\frac{1}{2}\) of 52.  So in a sense, the middle term is \(\frac{1}{2}\)  of a pair, for a total of \(3\frac{1}{2}\)  pairs.  But that's just \(7/2\), which is our \(k/2\) in the original formula!  So we're still good.  The relationship%
\begin{equation*}
S_k = \frac{k}{2}\left( a_m + a_n \right)
\end{equation*}
still works, for both odd and even values of \(k\).%
\par
Generalizing, we find that%
\begin{equation*}
S_k = \frac{k}{2}\left(a_m + a_n \right)
\end{equation*}
where \(k\) can be even or odd and%
\begin{equation*}
k = n - m + 1 \text{   for } n \ge m
\end{equation*}
\begin{example}{}{g:example:idp230210776}%
Find the sum of the first forty terms of the series \(2 + 10 + 18 + 26 + \ldots\).\par\smallskip%
\noindent\textbf{\blocktitlefont Answer}.\label{g:answer:idp230214744}{}\hypertarget{g:answer:idp230214744}{}\quad{}This is just the same sequence as before, with first term \(a_m = 2\) and common difference \(d = 8\).  In order to use our previous formula, however, we need to calculate the last term \(a_n\).  If we start with an index of one, then the fortieth term will be \(a_{40}\), and we will need that value to calculate \(S_{40}\).%
\begin{align*}
a_n \amp = a_m + \left( n - m \right)d\\
\amp = a_1 + \left( n - 1 \right)d\\
a_{40} \amp = 2 + 39 \times 8\\
\amp = 314
\end{align*}
So,%
\begin{align*}
S_k \amp = \frac{k}{2}\left( a_m + a_n \right)\\
S_{40} \amp = \frac{40}{2}\left( 2 + 314 \right)\\
\amp = 20 \times 316\\
\amp = 6320
\end{align*}
The sum of the first forty terms is 6320.  (Much easier than writing out the first forty terms and adding them up!)\end{example}
%
\par
In the previous example, we used the formula for \(a_n\) to calculate the last term and put its value into the formula for \(S_k\).  We could do that in a more general way:%
\begin{align*}
S_k \amp = \frac{k}{2}\left(a_m + a_n \right)\\
\amp = \frac{k}{2}\left[ a_m + \left(a_m + \left(n - m \right)d \right) \right]\\
\amp = \frac{k}{2}\left[ 2a_m + \left( n - m \right)d \right]
\end{align*}
The last expression, which gives \(S_k\) as a function of the first term, the number of terms, and the common difference, can also be used to evaluate series. \begin{example}{}{g:example:idp230223064}%
Find the sum of the first one hundred terms of the sequence \(5, -6, -17, -26, \ldots\).\par\smallskip%
\noindent\textbf{\blocktitlefont Answer}.\label{g:answer:idp230225240}{}\hypertarget{g:answer:idp230225240}{}\quad{}This sum will just be \(5 + -6 + -17 +  -26 + \ldots\), with \(a_m = 5\), \(d = -11\), and \(k = 100\).  If we start our index at one, then%
\begin{align*}
k \amp = n - m + 1\\
100 \amp = n - 1 + 1\\
n\amp = 100
\end{align*}
and we can substitute this into the equation for \(S_k\):%
\begin{align*}
S_k \amp = \frac{k}{2}\left[ 2a_m + \left( n - m \right)d \right]\\
S_{100} \amp = \frac{100}{2}\left[ 2 \times 5 + 99 \times \left(  - 11 \right) \right]\\
\amp =  - 53950
\end{align*}
\end{example}
 \begin{example}{}{g:example:idp230232280}%
Calculate \(\sum\limits_{j = 3}^{18} {5n + 10}\).\par\smallskip%
\noindent\textbf{\blocktitlefont Answer}.\label{g:answer:idp230232664}{}\hypertarget{g:answer:idp230232664}{}\quad{}The first term will be for \(j=3\) and will equal \(5(3)+10=25\).  Next is \(j=4\) and will equal \(5(4)+10=30\), \(j=5\) equaling \(5(5)=35\), and so on.   The last term will be for \(j=18\) and will equal \(5(18)+10=100\).%
\par
In other words, our series is \(25+30+35+\ldots 100\).  Is it arithmetic?  Yes, with common difference \(d = 5\).%
\par
What else do we need for our calculation?  The number of terms is%
\begin{align*}
k \amp = n-m+1\\
k \amp = 18-3+1\\
k\amp = 16
\end{align*}
Then%
\begin{align*}
S_n \amp = \frac{n}{2}\left(a_m + a_n \right)\\
S_{16} \amp = \frac{16}{2}\left(25 + 100 \right) = 1000
\end{align*}
%
\end{example}
 \begin{example}{}{g:example:idp230239960}%
Pat the math instructor asks her students to do five word problems the first week, six the second week, seven the third week, and so on, increasing the number of word problems each week by one. %
\begin{enumerate}[label=(\alph*)]
\item{}How many word problems will diligent students be doing in the last week of classes (the \(\nth{11}\) week)?%
\item{}How many word problems will diligent students have completed during the course of the term (11 weeks)?%
\end{enumerate}
\par\smallskip%
\noindent\textbf{\blocktitlefont Answer}.\label{g:answer:idp230235608}{}\hypertarget{g:answer:idp230235608}{}\quad{}%
\begin{enumerate}[label=(\alph*)]
\item{}The number of word problems is a sequence:  \(5, 6, 7, \ldots\) .  In fact, it's an arithmetic sequence with \(a_m = 5\) and \(d = 1\).  If we start our counting from one, then in the eleventh week,%
\begin{align*}
a_n \amp = a_m + \left(n - m \right)d\\
a_n \amp = a_1+\left(n-1\right)d\\
a_{11} \amp = 5+10\times 1\\
\amp = 15
\end{align*}
Diligent students will solve 15 word problems in the last week of classes.%
\item{}The \emph{total} number of word problems solved is%
\begin{align*}
S_k \amp = \frac{k}{2}\left(a_m + a_n \right)\\
S_{11} \amp = \frac{11}{2}\left(5 + 15\right)\\
\amp = 110
\end{align*}
%
\end{enumerate}
 Diligent students will have solved 110 word problems in total.\end{example}
%
\end{subsectionptx}
%
%
\typeout{************************************************}
\typeout{Subsection 3.2.5 Summary}
\typeout{************************************************}
%
\begin{subsectionptx}{Summary}{}{Summary}{}{}{x:subsection:ssec-summary-arith-ser}
For an \emph{arithmetic sequence}, the nth term is given by%
\begin{equation*}
a_n = a_m + \left(n - m \right)d \text{   for } n \ge m
\end{equation*}
For an \emph{arithmetic series}, the sum of the first \(k\) terms (\(\kth{}\) partial sum) is%
\begin{equation*}
S_k = \frac{k}{2}\left(a_m + a_n \right)
\end{equation*}
or%
\begin{equation*}
S_k = \frac{k}{2}\left[ 2a_m + \left(n - m \right)d \right]
\end{equation*}
where \(k=n-m+1\) and \(n\ge m\).%
\end{subsectionptx}
%
%
\typeout{************************************************}
\typeout{Exercises 3.2.6 Exercises}
\typeout{************************************************}
%
\begin{exercises-subsection}{Exercises}{}{Exercises}{}{}{g:exercises:idp230244056}
\par\medskip\noindent%
\textbf{Exercise Group.}\space\space%
State whether the following sequences are arithmetic or not.  If they are, state the first term and common difference.\begin{exercisegroup}
\begin{divisionexerciseeg}{1}{}{}{g:exercise:idp230243032}%
\(8, 9, 11, 13, 16, \ldots\)\end{divisionexerciseeg}%
\begin{divisionexerciseeg}{2}{}{}{g:exercise:idp230251096}%
\(-3, -10, -17, -24, \ldots\)\end{divisionexerciseeg}%
\begin{divisionexerciseeg}{3}{}{}{g:exercise:idp230254168}%
\(3, 6, 12, 24, \ldots\)\end{divisionexerciseeg}%
\begin{divisionexerciseeg}{4}{}{}{g:exercise:idp230255832}%
\(1, 2, 6, 24, \ldots\)\end{divisionexerciseeg}%
\begin{divisionexerciseeg}{5}{}{}{g:exercise:idp230254040}%
\(81, 72, 63, 54, \ldots\)\end{divisionexerciseeg}%
\begin{divisionexerciseeg}{6}{}{}{g:exercise:idp230255192}%
\(1,\frac{5}{4},\frac{3}{2},\frac{7}{4},2,\frac{9}{4},\ldots\)\end{divisionexerciseeg}%
\end{exercisegroup}
\par\medskip\noindent
\par\medskip\noindent%
\textbf{Exercise Group.}\space\space%
Give both the general formula  and the recursive formula for the \(\nth{}\) term \(a_n\) of the following arithmetic sequences.  Assume that the first term of the sequence is \(a_1\).  For the general formula, be sure to simplify your answer.\begin{exercisegroup}
\begin{divisionexerciseeg}{7}{}{}{g:exercise:idp230255448}%
\(1, 3, 5, 7, \ldots\)\end{divisionexerciseeg}%
\begin{divisionexerciseeg}{8}{}{}{g:exercise:idp230251864}%
\(5, -6, -17, -28, \ldots\)\end{divisionexerciseeg}%
\begin{divisionexerciseeg}{9}{}{}{g:exercise:idp230259160}%
\(-40, -37, -34, -31, \ldots\)\end{divisionexerciseeg}%
\begin{divisionexerciseeg}{10}{}{}{g:exercise:idp230261976}%
\(24, 28, 32, 36, \ldots\)\end{divisionexerciseeg}%
\end{exercisegroup}
\par\medskip\noindent
\par\medskip\noindent%
\textbf{Exercise Group.}\space\space%
For the following arithmetic sequences, calculate \(a_{50}\) and \(a_{261}\), assuming that the first term is \(a_1\).\begin{exercisegroup}
\begin{divisionexerciseeg}{11}{}{}{g:exercise:idp230261336}%
\(18, 16, 14, 12, \ldots\)\end{divisionexerciseeg}%
\begin{divisionexerciseeg}{12}{}{}{g:exercise:idp230260824}%
\(12, 12.3, 12.6, 12.9, \ldots\)\end{divisionexerciseeg}%
\end{exercisegroup}
\par\medskip\noindent
\par\medskip\noindent%
\textbf{Exercise Group.}\space\space%
State whether the following recursively defined sequences are arithmetic or not.  (Is there an easy way to tell?)\begin{exercisegroup}
\begin{divisionexerciseeg}{13}{}{}{g:exercise:idp230272984}%
%
\begin{align*}
a_0 \amp = 5\\
a_n\amp = a_{n-1}+4 \text{   for }n\ge 1
\end{align*}
\end{divisionexerciseeg}%
\begin{divisionexerciseeg}{14}{}{}{g:exercise:idp230275032}%
%
\begin{align*}
a_1\amp = 12\\
a_n\amp = 2a_{n-1}\text{   for }n\ge 2
\end{align*}
\end{divisionexerciseeg}%
\begin{divisionexerciseeg}{15}{}{}{g:exercise:idp230271320}%
%
\begin{align*}
a_1\amp = 75\\
a_n\amp = a_{n-1}-20\text{   for }n\ge 2
\end{align*}
\end{divisionexerciseeg}%
\begin{divisionexerciseeg}{16}{}{}{g:exercise:idp230273752}%
%
\begin{align*}
a_0\amp = 6\\
a_n\amp = a_{n-1}+1\text{   for }n\ge 1
\end{align*}
\end{divisionexerciseeg}%
\begin{divisionexerciseeg}{17}{}{}{g:exercise:idp230273240}%
%
\begin{align*}
a_0\amp = 7\\
a_n\amp = 2-a_{n-1}\text{   for }n\ge 1
\end{align*}
\end{divisionexerciseeg}%
\begin{divisionexerciseeg}{18}{}{}{g:exercise:idp230276952}%
%
\begin{align*}
a_1 \amp = 3\\
a_n\amp = \left(a_{n-1}\right)^2\text{   for }n\ge 2
\end{align*}
\end{divisionexerciseeg}%
\end{exercisegroup}
\par\medskip\noindent
\begin{divisionexercise}{19}{}{}{g:exercise:idp230277080}%
For the following sequence, calculate the \(\upst{201}\) term: \(5, 15, 25, 35, \ldots\)\end{divisionexercise}%
\begin{divisionexercise}{20}{}{}{g:exercise:idp230282200}%
For the following sequence, which term equals 137?     \(\ 1, 9, 17, 25, \ldots\)\end{divisionexercise}%
\begin{divisionexercise}{21}{}{}{g:exercise:idp230278104}%
What is the common difference for the arithmetic sequence with \(a_1 = 200\) and \(a_{12} =  - 240\)\end{divisionexercise}%
\begin{divisionexercise}{22}{}{}{g:exercise:idp230279640}%
Calculate the first term for the arithmetic sequence with common difference 7 whose sixteenth term is 102.\end{divisionexercise}%
\begin{divisionexercise}{23}{}{}{g:exercise:idp230275672}%
Calculate the first four terms of the arithmetic sequence in which the sixth term is 17 and the sixtieth term is 179.\end{divisionexercise}%
\begin{divisionexercise}{24}{}{}{g:exercise:idp230285912}%
Calculate the first four terms of the arithmetic sequence in which the one hundredth term is 403 and the sixty-fourth term is 259.\end{divisionexercise}%
\begin{divisionexercise}{25}{}{}{g:exercise:idp230283608}%
Give a general formula for the arithmetic sequence in which the twentieth term is \(-107\) and the thirty-fifth term is \(-152\).\end{divisionexercise}%
\begin{divisionexercise}{26}{}{}{g:exercise:idp230284120}%
Give a recursive formula for the arithmetic sequence in which the eleventh term is 44 and the fifty-second term is 290.\end{divisionexercise}%
\begin{divisionexercise}{27}{}{}{g:exercise:idp230289112}%
Calculate \(S_{20}\) for the series \(100 + 97 + 94 + \ldots\)\end{divisionexercise}%
\begin{divisionexercise}{28}{}{}{g:exercise:idp230291672}%
Evaluate the series \(12 + 17 + 22 + \ldots 82\).\end{divisionexercise}%
\begin{divisionexercise}{29}{}{}{g:exercise:idp230291288}%
Evaluate the series \(144 + 138 + 132 + \ldots 78\).\end{divisionexercise}%
\begin{divisionexercise}{30}{}{}{g:exercise:idp230297432}%
Calculate \(S_{100}\) for the series \(-20  - 16  - 12 - \ldots\)\end{divisionexercise}%
\begin{divisionexercise}{31}{}{}{g:exercise:idp230297816}%
Calculate the sum of the odd numbers between 100 and 500.\end{divisionexercise}%
\begin{divisionexercise}{32}{}{}{g:exercise:idp230291800}%
Find the sum of the natural numbers from 50 to 125, inclusive.\end{divisionexercise}%
\par\medskip\noindent%
\textbf{Exercise Group.}\space\space%
Calculate the following sums.\begin{exercisegroup}
\begin{divisionexerciseeg}{33}{}{}{g:exercise:idp230292440}%
\(\sum\limits_{k = 0}^{53} {(5k - 1)} \)\end{divisionexerciseeg}%
\begin{divisionexerciseeg}{34}{}{}{g:exercise:idp230295512}%
\(\sum\limits_{j = 10}^{92} {6j} \)\end{divisionexerciseeg}%
\begin{divisionexerciseeg}{35}{}{}{g:exercise:idp230295768}%
\(\sum\limits_{i = 30}^{140} {(2i + 7)} \)\end{divisionexerciseeg}%
\begin{divisionexerciseeg}{36}{}{}{g:exercise:idp230291928}%
\(\sum\limits_{k = 3}^{502} {(17 - 3k)}\)\end{divisionexerciseeg}%
\end{exercisegroup}
\par\medskip\noindent
\begin{divisionexercise}{37}{}{}{g:exercise:idp230298840}%
In a supermarket display, there are 37 cans in the bottom layer, 35 in the next layer up, 33 in the next, and so on.  How many layers are there if there are 7 cans in the top row?\end{divisionexercise}%
\begin{divisionexercise}{38}{}{}{g:exercise:idp230300632}%
In the previous problem, how many cans are there altogether?\end{divisionexercise}%
\begin{divisionexercise}{39}{}{}{g:exercise:idp230304344}%
In an old-fashioned theatre, there are 25 seats in the first row, 26 in the next, 27 in the one after, and so on.  If there are 20 rows in total, how many seats are there altogether?\end{divisionexercise}%
\end{exercises-subsection}
%
%
\typeout{************************************************}
\typeout{Solutions 3.2.7 Solutions to Section~{\xreffont\ref*{x:section:sec-arith-seq-ser}} Exercises}
\typeout{************************************************}
%
\begin{solutions-subsection}{Solutions to Section~{\xreffont\ref*{x:section:sec-arith-seq-ser}} Exercises}{}{Solutions to Section~{\xreffont\ref*{x:section:sec-arith-seq-ser}} Exercises}{}{}{g:solutions:idp230302296}
\par\medskip
\noindent\textbf{\normalsize{}3.2.6\space\textperiodcentered\space{}Exercises}
\begin{exercisegroup}
\begin{divisionsolutioneg}{3.2.6.1}{}{g:exercise:idp230243032}%
\par\smallskip%
\noindent\hypertarget{g:solution:idp230244824-main}{}not arithmetic\end{divisionsolutioneg}%
\begin{divisionsolutioneg}{3.2.6.2}{}{g:exercise:idp230251096}%
\par\smallskip%
\noindent\hypertarget{g:solution:idp230258264-main}{}yes, \(d=-7\)\end{divisionsolutioneg}%
\begin{divisionsolutioneg}{3.2.6.3}{}{g:exercise:idp230254168}%
\par\smallskip%
\noindent\hypertarget{g:solution:idp230256600-main}{}no\end{divisionsolutioneg}%
\begin{divisionsolutioneg}{3.2.6.4}{}{g:exercise:idp230255832}%
\par\smallskip%
\noindent\hypertarget{g:solution:idp230252760-main}{}no\end{divisionsolutioneg}%
\begin{divisionsolutioneg}{3.2.6.5}{}{g:exercise:idp230254040}%
\par\smallskip%
\noindent\hypertarget{g:solution:idp230256728-main}{}yes, \(d=-9\)\end{divisionsolutioneg}%
\begin{divisionsolutioneg}{3.2.6.6}{}{g:exercise:idp230255192}%
\par\smallskip%
\noindent\hypertarget{g:solution:idp230258392-main}{}yes, \(d=\frac{1}{4}\)\end{divisionsolutioneg}%
\end{exercisegroup}
\par\medskip\noindent
\begin{exercisegroup}
\begin{divisionsolutioneg}{3.2.6.7}{}{g:exercise:idp230255448}%
\par\smallskip%
\noindent\hypertarget{g:solution:idp230257112-main}{}\({a_n} = 2n - 1\) and \(\left\{ \begin{array}{l}
a_1 = 1\\
a_n = a_{n - 1} + 2
\end{array} \right.\)\end{divisionsolutioneg}%
\begin{divisionsolutioneg}{3.2.6.8}{}{g:exercise:idp230251864}%
\par\smallskip%
\noindent\hypertarget{g:solution:idp230252248-main}{}\(a_n = 16 - 11n\)	and \(\left\{ \begin{array}{l}
a_1 = 5\\
a_n = a_{n - 1} - 11
\end{array} \right.\)\end{divisionsolutioneg}%
\begin{divisionsolutioneg}{3.2.6.9}{}{g:exercise:idp230259160}%
\par\smallskip%
\noindent\hypertarget{g:solution:idp230259416-main}{}\(a_n = 3n - 43\)	and \(\left\{ \begin{array}{l}
a_1 =  - 40\\
a_n = a_{n - 1} + 3
\end{array} \right.\)\end{divisionsolutioneg}%
\begin{divisionsolutioneg}{3.2.6.10}{}{g:exercise:idp230261976}%
\par\smallskip%
\noindent\hypertarget{g:solution:idp230263640-main}{}\(a_n = 4n + 20\)	and \(\left\{ \begin{array}{l}
a_1 = 24\\
a_n = a_{n - 1} + 4
\end{array} \right.\)\end{divisionsolutioneg}%
\end{exercisegroup}
\par\medskip\noindent
\begin{exercisegroup}
\begin{divisionsolutioneg}{3.2.6.11}{}{g:exercise:idp230261336}%
\par\smallskip%
\noindent\hypertarget{g:solution:idp230266840-main}{}\({a_n} = 20 - 2n\), so \(a_{50} = -80\) and \(a_{261} = -502\)\end{divisionsolutioneg}%
\begin{divisionsolutioneg}{3.2.6.12}{}{g:exercise:idp230260824}%
\par\smallskip%
\noindent\hypertarget{g:solution:idp230264536-main}{}\(a_n = 11.7 + 0.3n\), so \(a_{50} = 26.7\) and \(a_{261} = 90\)\end{divisionsolutioneg}%
\end{exercisegroup}
\par\medskip\noindent
\begin{exercisegroup}
\begin{divisionsolutioneg}{3.2.6.13}{}{g:exercise:idp230272984}%
\par\smallskip%
\noindent\hypertarget{g:solution:idp230272600-main}{}first four terms are \(5, 9, 13, 17\), so arithmetic with \(d = 4\)\end{divisionsolutioneg}%
\begin{divisionsolutioneg}{3.2.6.14}{}{g:exercise:idp230275032}%
\par\smallskip%
\noindent\hypertarget{g:solution:idp230272344-main}{}first four terms are \(12, 24, 48, 96\), so not arithmetic\end{divisionsolutioneg}%
\begin{divisionsolutioneg}{3.2.6.15}{}{g:exercise:idp230271320}%
\par\smallskip%
\noindent\hypertarget{g:solution:idp230270936-main}{}first four terms are \(75, 55, 35, 15\), so arithmetic with \(d = -20\)\end{divisionsolutioneg}%
\begin{divisionsolutioneg}{3.2.6.16}{}{g:exercise:idp230273752}%
\par\smallskip%
\noindent\hypertarget{g:solution:idp230269912-main}{}first four terms are \(6, 7, 8, 9\), so arithmetic with \(d = 1\)\end{divisionsolutioneg}%
\begin{divisionsolutioneg}{3.2.6.17}{}{g:exercise:idp230273240}%
\par\smallskip%
\noindent\hypertarget{g:solution:idp230270424-main}{}first four terms are \(7, -5, 7, -5\), so not arithmetic\end{divisionsolutioneg}%
\begin{divisionsolutioneg}{3.2.6.18}{}{g:exercise:idp230276952}%
\par\smallskip%
\noindent\hypertarget{g:solution:idp230282584-main}{}first four terms are \(3, 9, 81, 6561\), so not arithmetic\end{divisionsolutioneg}%
\end{exercisegroup}
\par\medskip\noindent
\begin{divisionsolution}{3.2.6.19}{}{g:exercise:idp230277080}%
\par\smallskip%
\noindent\hypertarget{g:solution:idp230281432-main}{}\(a_n = 10n - 5\), so \(a_{201} = 2005\)\end{divisionsolution}%
\begin{divisionsolution}{3.2.6.20}{}{g:exercise:idp230282200}%
\par\smallskip%
\noindent\hypertarget{g:solution:idp230276696-main}{}\(a_n = 8n - 7\), so \(n = 18\)\end{divisionsolution}%
\begin{divisionsolution}{3.2.6.21}{}{g:exercise:idp230278104}%
\par\smallskip%
\noindent\hypertarget{g:solution:idp230279512-main}{}\(d=-40\)\end{divisionsolution}%
\begin{divisionsolution}{3.2.6.22}{}{g:exercise:idp230279640}%
\par\smallskip%
\noindent\hypertarget{g:solution:idp230279384-main}{}\(a_1=-3\)\end{divisionsolution}%
\begin{divisionsolution}{3.2.6.23}{}{g:exercise:idp230275672}%
\par\smallskip%
\noindent\hypertarget{g:solution:idp230280280-main}{}\(a_1=2\) and \(d=3\), so the first four terms are \(2, 5, 8, 11\)\end{divisionsolution}%
\begin{divisionsolution}{3.2.6.24}{}{g:exercise:idp230285912}%
\par\smallskip%
\noindent\hypertarget{g:solution:idp230287064-main}{}\(a_1=7\) and \(d=4\), so the first four terms are \(7, 11, 15, 19\)\end{divisionsolution}%
\begin{divisionsolution}{3.2.6.25}{}{g:exercise:idp230283608}%
\par\smallskip%
\noindent\hypertarget{g:solution:idp230286296-main}{}\(a_n =  - 3n - 47\)\end{divisionsolution}%
\begin{divisionsolution}{3.2.6.26}{}{g:exercise:idp230284120}%
\par\smallskip%
\noindent\hypertarget{g:solution:idp230288728-main}{}%
\begin{align*}
a_1\amp = -16\\
a_n\amp = a_{n-1}+6
\end{align*}
\end{divisionsolution}%
\begin{divisionsolution}{3.2.6.27}{}{g:exercise:idp230289112}%
\par\smallskip%
\noindent\hypertarget{g:solution:idp230289752-main}{}\(S_{20} = 1430\)\end{divisionsolution}%
\begin{divisionsolution}{3.2.6.28}{}{g:exercise:idp230291672}%
\par\smallskip%
\noindent\hypertarget{g:solution:idp230284632-main}{}\(S_{15} = 705\)\end{divisionsolution}%
\begin{divisionsolution}{3.2.6.29}{}{g:exercise:idp230291288}%
\par\smallskip%
\noindent\hypertarget{g:solution:idp230292824-main}{}\(S_{12} = 1332\)\end{divisionsolution}%
\begin{divisionsolution}{3.2.6.30}{}{g:exercise:idp230297432}%
\par\smallskip%
\noindent\hypertarget{g:solution:idp230292312-main}{}\(S_{100} = 17800\)\end{divisionsolution}%
\begin{divisionsolution}{3.2.6.31}{}{g:exercise:idp230297816}%
\par\smallskip%
\noindent\hypertarget{g:solution:idp230296408-main}{}\(S_{200} = 60000\)\end{divisionsolution}%
\begin{divisionsolution}{3.2.6.32}{}{g:exercise:idp230291800}%
\par\smallskip%
\noindent\hypertarget{g:solution:idp230296024-main}{}\(S_{76} = 6650\)\end{divisionsolution}%
\begin{exercisegroup}
\begin{divisionsolutioneg}{3.2.6.33}{}{g:exercise:idp230292440}%
\par\smallskip%
\noindent\hypertarget{g:solution:idp230299224-main}{}\(S_{53} = 7101\)\end{divisionsolutioneg}%
\begin{divisionsolutioneg}{3.2.6.34}{}{g:exercise:idp230295512}%
\par\smallskip%
\noindent\hypertarget{g:solution:idp230298200-main}{}\(S_{83} = 25398\)\end{divisionsolutioneg}%
\begin{divisionsolutioneg}{3.2.6.35}{}{g:exercise:idp230295768}%
\par\smallskip%
\noindent\hypertarget{g:solution:idp230293976-main}{}\(S_{111} = 19647\)\end{divisionsolutioneg}%
\begin{divisionsolutioneg}{3.2.6.36}{}{g:exercise:idp230291928}%
\par\smallskip%
\noindent\hypertarget{g:solution:idp230293080-main}{}\(S_{500} = -370,250\)\end{divisionsolutioneg}%
\end{exercisegroup}
\par\medskip\noindent
\begin{divisionsolution}{3.2.6.37}{}{g:exercise:idp230298840}%
\par\smallskip%
\noindent\hypertarget{g:solution:idp230295896-main}{}\(n=16\)\end{divisionsolution}%
\begin{divisionsolution}{3.2.6.38}{}{g:exercise:idp230300632}%
\par\smallskip%
\noindent\hypertarget{g:solution:idp230303832-main}{}\(S_{16} = 352\)\end{divisionsolution}%
\begin{divisionsolution}{3.2.6.39}{}{g:exercise:idp230304344}%
\par\smallskip%
\noindent\hypertarget{g:solution:idp230300376-main}{}\(S_{20} = 690\)\end{divisionsolution}%
\end{solutions-subsection}
\end{sectionptx}
%
%
\typeout{************************************************}
\typeout{Section 3.3 Geometric Sequences and Series}
\typeout{************************************************}
%
\begin{sectionptx}{Geometric Sequences and Series}{}{Geometric Sequences and Series}{}{}{x:section:sec-geom-seq-ser}
%
%
\typeout{************************************************}
\typeout{Subsection 3.3.1 Geometric Sequences}
\typeout{************************************************}
%
\begin{subsectionptx}{Geometric Sequences}{}{Geometric Sequences}{}{}{x:subsection:ssec-geom-seq}
Let's start out with a definition: \begin{definition}{}{x:definition:defn-geom-seq}%
A \terminology{geometric sequence} is a sequence in which the next term is found by multiplying the previous term by a constant (the \emph{common ratio}\(\ r\))\end{definition}
%
\par
Here are some examples of geometric sequences:%
\begin{enumerate}[label=(\alph*)]
\item{}\(\displaystyle 9, 18, 36, 72, \ldots\)%
\item{}\(\displaystyle 12, 18, 27, \frac{81}{2},\ldots\)%
\item{}\(\displaystyle 10, -30, 90, -270, \ldots,  -196830\)%
\item{}\(\displaystyle -3, -12, -48, -192, \ldots\)%
\item{}\(\displaystyle 48, -36, 27, \ldots\)%
\end{enumerate}
%
\par
The common ratios of each of these sequences, in order from a) to e), is \(2\), \(\frac{3}{2}\), \(-3\), \(4\), \(-\frac{3}{4}\), respectively.  Note that in each of them, we can find the common ratio \(r\) by taking \(any\) term and dividing it by the previous term.%
\par
Like any other sequences, geometric sequences can be finite or infinite.  Example c) above is finite, as the last term is specified.  The others are infinite sequences. \begin{example}{}{g:example:idp230314328}%
For each of the following sequences, state whether it is arithmetic, geometric, or neither. %
\begin{enumerate}[label=(\alph*)]
\item{}\(\displaystyle 45, 15, 5, \ldots\)%
\item{}\(\displaystyle 5, 3, 1, -1, \ldots\)%
\item{}\(\displaystyle 1, 8, 27, 64, \ldots , 1000\)%
\item{}\(\displaystyle -1, 1, -1, 1, -1, 1, \ldots\)%
\end{enumerate}
\par\smallskip%
\noindent\textbf{\blocktitlefont Answer}.\label{g:answer:idp230321368}{}\hypertarget{g:answer:idp230321368}{}\quad{}%
\begin{enumerate}[label=(\alph*)]
\item{}Geometric, because the common ratio \(r\) is \(\frac{1}{3}\)%
\item{}Arithmetic, because the common difference \(d\) is \(-2\).%
\item{}Neither, because there isn't either a common difference or ratio between terms.  (In fact, the pattern is that \(a_n = n^3\) for \(n \ge 1\).)%
\item{}Geometric, because the common ratio \(r\) is \(-1\).%
\end{enumerate}
\end{example}
%
\par
Again, you can define a geometric sequence in one of three ways:  by listing the terms, by giving a recursive definition, or by giving a general definition.%
\end{subsectionptx}
%
%
\typeout{************************************************}
\typeout{Subsection 3.3.2 Recursive Definitions for Geometric Sequences}
\typeout{************************************************}
%
\begin{subsectionptx}{Recursive Definitions for Geometric Sequences}{}{Recursive Definitions for Geometric Sequences}{}{}{x:subsection:ssec-recdef-geomseq}
Let's look at an example. \begin{example}{}{g:example:idp230317272}%
Give a recursive definition for the sequence \(2, 10, 50, 250, \ldots\)\par\smallskip%
\noindent\textbf{\blocktitlefont Answer}.\label{g:answer:idp230320216}{}\hypertarget{g:answer:idp230320216}{}\quad{}Recall that a recursive definition has two parts:  listing the first term and giving the pattern.  In this case, the pattern is multiplying the previous term by \(r = 5\)  to get the next term.  Let's use 0 as our starting index.  The recursive definition is therefore%
\begin{align*}
a_0 \amp = 2\\
a_n \amp = 5a_{n - 1} \text{   for } n \ge 1
\end{align*}
\end{example}
%
\par
Generally, the recursive definition for \emph{any} geometric sequence is%
\begin{align*}
a_m \amp = \lt \text{ insert value here } \gt\\
a_n \amp = r\times a_{n-1},\, n \ge m+1
\end{align*}
%
\end{subsectionptx}
%
%
\typeout{************************************************}
\typeout{Subsection 3.3.3 General Formulae for Geometric Sequences}
\typeout{************************************************}
%
\begin{subsectionptx}{General Formulae for Geometric Sequences}{}{General Formulae for Geometric Sequences}{}{}{x:subsection:ssec-genform-gemseq}
Let's examine the previous example in more detail to see if we can recognize any patterns and come up with a general formula.  Rewriting each term, we get%
\begin{equation*}
\begin{array}{*{20}{c}}
2,\amp 10,\amp 50,\amp 250,\amp \ldots\\
2,\amp 2 \times 5,\amp 2 \times {5^2},\amp 2 \times {5^3},\amp \ldots\\
\end{array}
\end{equation*}
So the \(\uprd{3}\) term equals the first times 5 squared, the \(\upth{4}\) term equals the first times 5 cubed, and the \(\nth{}\) term will equal the first times 5 raised to the \((n - 1)\) power.  In general, for sequences with first term \(a_m\), the \(nth{}\) term equals the first term times \(r\) raised to the \((n - m)\) power, namely%
\begin{equation*}
a_n=a_m r^{n-m}
\end{equation*}
for all \emph{geometric} sequences. \begin{example}{}{g:example:idp230326872}%
Write a general formula for the sequence \(3, 6, 12, \ldots\)\par\smallskip%
\noindent\textbf{\blocktitlefont Answer}.\label{g:answer:idp230332504}{}\hypertarget{g:answer:idp230332504}{}\quad{}This sequence is geometric with the first term 3 and common ratio 2.  If we choose \(n = 1\) for our first term, then%
\begin{align*}
a_n \amp = a_m r^{n-m}\\
\amp = a_1 r^{n-1}\\
\amp = 3\times\left(2^{n-1}\right)
\end{align*}
The general formula is then \(a_3\times 2^{n-1}\) for \(n\ge 1\).\end{example}
 \begin{example}{}{g:example:idp230338904}%
What is the \(\upth{20}\) term in the sequence in the sequence \(3, 6, 12, \ldots\)?\par\smallskip%
\noindent\textbf{\blocktitlefont Answer}.\label{g:answer:idp230339928}{}\hypertarget{g:answer:idp230339928}{}\quad{}This is the same sequence from the previous example.  We may then use the formula we derived above with \(n = 20\).%
\begin{align*}
a_n\amp = a_m r^{n-m}\\
\amp = a_1 r^{n-1}\\
a_{20}\amp = 3\times 2^{20-1}\\
\amp = 3\times 2^{19}\\
a_{20}=1,572,864
\end{align*}
The \(\upth{20}\) term is 1,572,864, which provides a nice example for how fast geometric sequences can grow, even for small values of \(r\).\end{example}
 \begin{example}{}{g:example:idp230333016}%
Write a general formula for the sequence \(8, 12, 18, 27, \ldots\)  What is the fifteenth term in this sequence?  The fiftieth?\par\smallskip%
\noindent\textbf{\blocktitlefont Answer}.\label{g:answer:idp230339544}{}\hypertarget{g:answer:idp230339544}{}\quad{}If we start our counting at \(n=1\), then the fifteenth term is \(a_{15}\) and the fiftieth term is \(a_{50}\).%
\begin{align*}
a_n\amp = a_m r^{n-m}\\
\amp = a_1 r^{n-1}\\
a_n \amp = 8\left(\frac{3}{2}\right)^{n-1}\\
a_{15}\amp = 8\left(\frac{3}{2}\right)^{14}\approx 2335.43\\
a_{50}\amp = 8\left(\frac{3}{2}\right)^{50}\approx 3.40065\times 10^9
\end{align*}
So the general formula is \(a_n = 8\left( \frac{3}{2} \right)^{n - 1}\) for \(n \ge 1\) and the fifteenth and fiftieth terms are approximately 2335.43 and \(3.4 \times 10^9\), respectively.\end{example}
%
\end{subsectionptx}
%
%
\typeout{************************************************}
\typeout{Subsection 3.3.4 Geometric Series}
\typeout{************************************************}
%
\begin{subsectionptx}{Geometric Series}{}{Geometric Series}{}{}{x:subsection:ssec-geom-ser}
Recall that \(S_k\) is the sum of the first \(k\) terms of a series.  Let's look at how a formula for \(S_k\) is derived, using a series that starts with \(n=1\).%
\begin{align*}
S_k \amp = \amp{a_1} \amp+ \amp{a_2} \amp+ \amp{a_3} \amp+ \amp{a_4} \amp+ \amp\ldots \amp+ \amp{a_{n - 2}} \amp+ \amp{a_{n - 1}} \amp+ \amp{a_k}\\
S_k \amp = \amp{a_1} \amp+ \amp{a_1}r \amp+ \amp{a_1}{r^2} \amp+ \amp{a_1}{r^3} \amp+ \amp\ldots \amp+ \amp{a_1}{r^{k - 3}} \amp+ \amp{a_1}{r^{k - 2}} \amp+ \amp{a_1}{r^{k - 1}}
\end{align*}
Let's take that last expression for \(S_k\) and multiply it by \(-r\) to get%
\begin{equation*}
- r{S_k} =  - {a_1}r - {a_1}{r^2} - {a_1}{r^3} - {a_1}{r^4} - ... - {a_1}{r^{k - 2}} - {a_1}{r^{k - 1}} - {a_1}{r^k}
\end{equation*}
Then if we add the rows for \(S_k\) and \(-rS_k\), we get%
\begin{align*}
{S_k} \amp= \amp {a_1} \amp+ {a_1}r \amp+ {a_1}{r^2} \amp+ {a_1}{r^3} \amp+ \ldots \amp+ {a_1}{r^{k - 3}} \amp+ {a_1}{r^{k - 2}} \amp+ {a_1}{r^{k - 1}}\\
- r{S_k} \amp=\amp\amp  - {a_1}r \amp- {a_1}{r^2} \amp- {a_1}{r^3} \amp- {a_1}{r^4} \amp- \ldots \amp- {a_1}{r^{k - 2}} \amp- {a_1}{r^{k - 1}} \amp- {a_1}{r^k}\\
\hline\\
{S_k} - r{S_k} \amp=\amp {a_1} \amp\amp\amp\amp\amp\amp\amp\amp- {a_1}{r^k}
\end{align*}
since all of the terms in between these two (\(a_1\) and \(a_1r^k\)) combine pair-wise to zero by subtraction.  Then%
\begin{align*}
{S_k} - r{S_k}\amp = {a_1}\left( {1 - {r^k}} \right)\\
{S_k}\left( {1 - r} \right) \amp = {a_1}\left( {1 - {r^k}} \right)
\end{align*}
and so%
\begin{equation*}
{S_k} = \frac{{{a_1}\left( {1 - {r^k}} \right)}}{{\left( {1 - r} \right)}}
\end{equation*}
To generalize, the formula for the sum of the first \(n\) terms for \emph{any geometric series}  that starts with first term \(a_m\) is%
\begin{equation*}
{S_k} = \frac{{{a_m}\left( {1 - {r^k}} \right)}}{{\left( {1 - r} \right)}}
\end{equation*}
\begin{example}{}{g:example:idp230350936}%
Find the sum of the first 20 terms of the series \(3 + 6 + 12 + \ldots\)\par\smallskip%
\noindent\textbf{\blocktitlefont Answer}.\label{g:answer:idp230352216}{}\hypertarget{g:answer:idp230352216}{}\quad{}This is a geometric series with \(a_m = 3\) and \(r = 2\).  We want to find \(S_{20}\).%
\begin{align*}
{S_k} \amp = \frac{{{a_m}\left( {1 - {r^k}} \right)}}{{\left( {1 - r} \right)}}\\
{S_{20}} \amp = \frac{{3\left( {1 - {2^{20}}} \right)}}{{\left( {1 - 2} \right)}} = 3,145,725
\end{align*}
The sum of the first 20 terms is 3,145,725.\end{example}
 \begin{example}{}{g:example:idp230363480}%
Find the sum of the first forty terms of the series \(8 - 12 + 18 - 27 \ldots\).\par\smallskip%
\noindent\textbf{\blocktitlefont Answer}.\label{g:answer:idp230359768}{}\hypertarget{g:answer:idp230359768}{}\quad{}This is a geometric series with \(a_m = 8\) and \(r = -\frac{3}{2}\).  We want to find \(S_{40}\).%
\begin{align*}
{S_k} \amp = \frac{{{a_m}\left( {1 - {r^k}} \right)}}{{\left( {1 - r} \right)}}\\
{S_{20}} \amp = \frac{{8\left( {1 - {{\left( { - 1.5} \right)}^{40}}} \right)}}{{\left( {1 - \left( { - 1.5} \right)} \right)}} =  - 3.53835 \times {10^7}
\end{align*}
The sum of the first forty terms is approximately \(-3.54 \times{} 10^7\).\end{example}
%
\end{subsectionptx}
%
%
\typeout{************************************************}
\typeout{Subsection 3.3.5 Sum of an Infinite Geometric Series}
\typeout{************************************************}
%
\begin{subsectionptx}{Sum of an Infinite Geometric Series}{}{Sum of an Infinite Geometric Series}{}{}{x:subsection:ssec-sum-geom-ser}
Let's take a look at the infinite series \(\frac{1}{2} + \frac{1}{4} + \frac{1}{8} + \frac{1}{{16}} + \ldots\) What happens when we try to evaluate this sum using the \(S_k\) formula?  We can put \(a_m = \frac{1}{2}\), \(r = \frac{1}{2}\), but can we put \(n = \infty\) into the formula?  It is important to note that \(\infty\) is \emph{not a number}.  Instead, it is a symbol that represents the characteristic of being unbounded, or ``going on forever''.%
\par
Let's take a closer look at the behaviour of \((\frac{1}{2})^n\) for large values of \(n\).  As \(n\) gets larger, the fraction \({\left( {\frac{1}{2}} \right)^n} = \frac{1}{{{2^n}}}\) gets ever smaller.  In fact, as \(n\) grows without bound (sometimes written \(n\to\infty\)), \((\frac{1}{2})^n \) will approach zero.%
\par
This is true for any \(r\) provided that \(|r| \lt 1\).  (If you're not familiar with the absolute value bars, an equivalent expression is that \(-1 \lt r \lt 1\).)%
\par
Recalling that%
\begin{equation*}
{S_k} = \frac{{{a_m}\left( {1 - {r^k}} \right)}}{{\left( {1 - r} \right)}}
\end{equation*}
and letting the \(r^n\) term go to zero, then%
\begin{equation*}
{S_\infty } = \frac{{{a_m}}}{{1 - r}} \text{ for } -1 \lt r \lt 1
\end{equation*}
for any \emph{infinite geometric series} with \(a_m\) the first term, provided that \(r\) meets the restriction above.%
\par
Let's now revisit the series that started this discussion and evaluate it in the following example. \begin{example}{}{g:example:idp230372952}%
Evaluate \(\frac{1}{2} + \frac{1}{4} + \frac{1}{8} + \frac{1}{{16}} +\ldots\)\par\smallskip%
\noindent\textbf{\blocktitlefont Answer}.\label{g:answer:idp230367832}{}\hypertarget{g:answer:idp230367832}{}\quad{}This series is geometric with \(a_m = \frac{1}{2}\) and \(r = \frac{1}{2}\).  Then%
\begin{equation*}
{S_\infty } = \frac{{{a_m}}}{{1 - r}} = \frac{{1/2}}{{1 - 1/2}} = \frac{{1/2}}{{1/2}} = 1
\end{equation*}
The sum of this series is 1.\end{example}
 \begin{example}{}{g:example:idp230374488}%
Evaluate \(24 + 16 + \frac{32}{3}+ \ldots\)\par\smallskip%
\noindent\textbf{\blocktitlefont Answer}.\label{g:answer:idp230375256}{}\hypertarget{g:answer:idp230375256}{}\quad{}This series is geometric with \(a_m = 24\) and \(r = \frac{2}{3}\).%
\begin{equation*}
{S_\infty } = \frac{{{a_m}}}{{1 - r}} = \frac{{24}}{{1 - \frac{2}{3}}} = \frac{{24}}{{\frac{1}{3}}} = 24 \times \frac{3}{1} = 72
\end{equation*}
\end{example}
 \begin{example}{}{g:example:idp230376408}%
Evaluate \(24 - 16 + \frac{32}{3}+ \ldots\)\par\smallskip%
\noindent\textbf{\blocktitlefont Answer}.\label{g:answer:idp230379608}{}\hypertarget{g:answer:idp230379608}{}\quad{}This series is identical to the previous one except that \(r\) is now negative:  \(a_m = 24\) and \(r = -\frac{2}{3}\).%
\begin{equation*}
{S_\infty } = \frac{{{a_m}}}{{1 - r}} = \frac{{24}}{{1 - \left( { - \frac{2}{3}} \right)}} = \frac{{24}}{{1 + \frac{2}{3}}} = \frac{{24}}{{\frac{5}{3}}} = 24 \times \frac{3}{5} = \frac{{72}}{5} = 14.4
\end{equation*}
\end{example}
 \begin{example}{}{g:example:idp230380120}%
Evaluate \(12 + 18 + 27 + \ldots\).\par\smallskip%
\noindent\textbf{\blocktitlefont Answer}.\label{g:answer:idp230376920}{}\hypertarget{g:answer:idp230376920}{}\quad{}This series is geometric with \(a_m = 12\) and \(r = \frac{3}{2}\).  You may already realize what's going on, but in case you don't, let's naively put the values into the formula and see what we get:%
\begin{equation*}
{S_\infty } = \frac{{{a_m}}}{{1 - r}} = \frac{{12}}{{1 - \frac{3}{2}}} = \frac{{12}}{{ - \frac{1}{2}}} = 12 \times  - \frac{2}{1} =  - 24
\end{equation*}
Wait!  How can the sum of a bunch of positive number be negative?  The answer is that our restriction for \(r\) is that it must be between \(-1\) and 1, but \(r = 1.5\).  Because \(r\) does not satisfy the restriction, \emph{we cannot use} the above formula for \(S_\infty\).  Indeed, if you add up a bunch of positive numbers that are increasing as you go up, you can see that the sum just keeps getting bigger as we add more terms.  You could then either say that the sum is infinite or ``does not exist''.  These two statements are not \emph{quite} equivalent.  While it is correct to say that the sum \emph{does not exist}, it is \emph{more descriptive} to state that the sum is infinite (or to write ``\(=+\infty\)'').  There could be a sum that grows without bound ``in the negative direction'', in which case it would be more descriptive to write ``\(=-\infty\)'' than simply to say that the sum does not exist.\end{example}
%
\par
Let's look at three specific cases:%
\begin{enumerate}[label=(\alph*)]
\item{}\(\displaystyle 12+18+27+\ldots\)%
\item{}\(\displaystyle -12-18-27-\ldots\)%
\item{}\(\displaystyle 12-18+27+\ldots\)%
\end{enumerate}
Each term in (a) is getting more positive, so the sum of that sequence does not exist but writing ``\(=+\infty\)'' is more descriptive.  Each term in (b) is getting more and more negative, so the sum of that sequence does not exist, but writing ``\(=-\infty\)'' is more descriptive.  But in the last case, the sum oscillates back and forth:  \(S_1 = 12\), \(S_2 = -6\), \(S_3 = 21\), \(S_4 = -19.5\), and so on.  The sign of \(S_k\) is either positive or negative depending on whether the number of terms you've added is even or odd.  In this case, all we can say is that the sum does not exist. \begin{example}{}{g:example:idp230391128}%
Evaluate \(\sum\limits_{j = 0}^\infty  {27{{\left( {\frac{1}{3}} \right)}^j}}\).\par\smallskip%
\noindent\textbf{\blocktitlefont Answer}.\label{g:answer:idp230391512}{}\hypertarget{g:answer:idp230391512}{}\quad{}The best place to start is to figure out the first few terms to determine the pattern:%
\begin{align*}
\text{when } j = 0,\quad \amp 27{\left( {\frac{1}{3}} \right)^0} = 27 \times 1 = 27\\
\text{when } j = 1,\quad \amp 27{\left( {\frac{1}{3}} \right)^1} = 27 \times \frac{1}{3} = 9\\
\text{when } j = 2,\quad \amp 27{\left( {\frac{1}{3}} \right)^2} = 27 \times \frac{1}{{{3^2}}} = 3
\end{align*}
so our sequence is \(27, 9, 3, \ldots\)  This is geometric with \(a_m = 27\) and \(r = \frac{1}{3}\).  Then%
\begin{equation*}
{S_\infty } = \frac{{{a_m}}}{{1 - r}}
= \frac{{27}}{{1 - \frac{1}{3}}}
= \frac{{27}}{{\frac{2}{3}}}
= 27 \times \frac{3}{2}
= \frac{{81}}{2}
= 40.5
\end{equation*}
\end{example}
 \begin{example}{}{g:example:idp230395096}%
Evaluate \(\sum\limits_{k = 5}^\infty  {\;\frac{1}{2}} k\).\par\smallskip%
\noindent\textbf{\blocktitlefont Answer}.\label{g:answer:idp230393688}{}\hypertarget{g:answer:idp230393688}{}\quad{}As in the previous example, let's figure out the first few terms to determine the pattern:%
\begin{align*}
\text{when } k = 5,\quad \amp \frac{1}{2}k = \frac{1}{2}5 = 2.5\\
\text{when } k = 6,\quad \amp \frac{1}{2}k = \frac{1}{2}6 = 3\\
\text{when } k = 7,\quad \amp \frac{1}{2}k = \frac{1}{2}7 = 3.5
\end{align*}
so our sequence is \(2.5, 3, 3.5, \ldots\).  Wait!  This sequence is arithmetic!  Not only that, but the numbers are increasing.  So the sum of the \emph{series} does not exist due to unbounded growth in the positive direction. You can write ``does not exist'' or, if you prefer, the more descriptive ``\(=+\infty\)''.\end{example}
%
\end{subsectionptx}
%
%
\typeout{************************************************}
\typeout{Subsection 3.3.6 Repeating Decimals}
\typeout{************************************************}
%
\begin{subsectionptx}{Repeating Decimals}{}{Repeating Decimals}{}{}{x:subsection:ssec-rep-decimals}
Let's examine \(0.\overline 7\) in some detail to see what we find:%
\begin{align*}
0.\overline 7 \amp = 0.777777777...\\
\amp = 0.7 + 0.07 + 0.007 + 0.0007 + \ldots
\end{align*}
But this is just the sum of an infinite series with \(a_m = 0.7\) and \(r = 0.1\).  Rewriting \(a_1\) and \(r\) in fraction form (you'll see why in a minute) gives \(a_m = \frac{7}{10}\) and \(r =\frac{1}{10}\).  Then%
\begin{equation*}
{S_\infty } = \frac{{{a_m}}}{{1 - r}} = \frac{{\frac{7}{{10}}}}{{1 - \frac{1}{{10}}}} = \frac{{\frac{7}{{10}}}}{{\frac{9}{{10}}}} = \frac{7}{{10}} \times \frac{{10}}{9} = \frac{7}{9}
\end{equation*}
So \(0.\overline 7 = 7/9\).  Interesting! \begin{example}{}{g:example:idp230400344}%
Find an exact fraction for \(0.\overline 6 \).\par\smallskip%
\noindent\textbf{\blocktitlefont Answer}.\label{g:answer:idp230401240}{}\hypertarget{g:answer:idp230401240}{}\quad{}%
\begin{align*}
0.\overline 6  \amp = 0.66666666\ldots\\
\amp = 0.6 + 0.06 + 0.006 + 0.0006 + \ldots
\end{align*}
But this is just the sum of an infinite geometric series with \(a_m = \frac{6}{10}\) and \(r = \frac{1}{10}\).  Then%
\begin{equation*}
{S_\infty } = \frac{{{a_m}}}{{1 - r}} = \frac{{\frac{6}{{10}}}}{{1 - \frac{1}{{10}}}} = \frac{{\frac{6}{{10}}}}{{\frac{9}{{10}}}} = \frac{6}{{10}} \times \frac{{10}}{9} = \frac{6}{9} = \frac{2}{3}
\end{equation*}
So \(0.\overline 6 = 2/3\).\end{example}
 \begin{example}{}{g:example:idp230408152}%
Find an exact fraction for \(0.\overline {18}\).\par\smallskip%
\noindent\textbf{\blocktitlefont Answer}.\label{g:answer:idp230410328}{}\hypertarget{g:answer:idp230410328}{}\quad{}%
\begin{align*}
0.\overline {18}  \amp = 0.1818181818\ldots\\
\amp  = 0.18 + 0.0018 + 0.000018 + \ldots
\end{align*}
But this is just the sum of an infinite geometric series with \(a_m = \frac{18}{100}\) and \(r = \frac{1}{100}\).  Then%
\begin{equation*}
{S_\infty } = \frac{{{a_m}}}{{1 - r}} = \frac{{\frac{{18}}{{100}}}}{{1 - \frac{1}{{100}}}} = \frac{{\frac{{18}}{{100}}}}{{\frac{{99}}{{100}}}} = \frac{{18}}{{100}} \times \frac{{100}}{{99}} = \frac{{18}}{{99}} = \frac{2}{{11}}
\end{equation*}
So \(0.\overline {18} = 2/11\).\end{example}
%
\end{subsectionptx}
%
%
\typeout{************************************************}
\typeout{Subsection 3.3.7 Summary}
\typeout{************************************************}
%
\begin{subsectionptx}{Summary}{}{Summary}{}{}{x:subsection:ssec-summary-gemseqser}
For a \emph{geometric} sequence, the \(\nth{}\) term is given by%
\begin{equation*}
{a_n} = {a_m}{r^{n - m}}
\end{equation*}
for \(n \ge m\)%
\par
For a \emph{geometric} series, the sum of the first \(k\) terms (\(\kth{}\) partial sum) is%
\begin{equation*}
{S_k} = \frac{{{a_m}\left( {1 - {r^k}} \right)}}{{\left( {1 - r} \right)}}
\end{equation*}
where \(a_m\) is the first term.%
\par
For an \emph{infinite geometric} series, the sum is%
\begin{equation*}
{S_\infty } = \frac{{{a_m}}}{{1 - r}}
\end{equation*}
provided that  \(-1 \lt r \lt 1\) and \(a_m\) is the first term.%
\end{subsectionptx}
%
%
\typeout{************************************************}
\typeout{Exercises 3.3.8 Exercises}
\typeout{************************************************}
%
\begin{exercises-subsection}{Exercises}{}{Exercises}{}{}{g:exercises:idp230419928}
\par\medskip\noindent%
\textbf{Exercise Group.}\space\space%
State whether the following sequences are geometric or not.  If they are, state the first term and common ratio.\begin{exercisegroup}
\begin{divisionexerciseeg}{1}{}{}{g:exercise:idp230421208}%
\(8, 9, 11, 13, 16, \ldots\)\end{divisionexerciseeg}%
\begin{divisionexerciseeg}{2}{}{}{g:exercise:idp230417112}%
\(-3, -10, -17, -24, \ldots\)\end{divisionexerciseeg}%
\begin{divisionexerciseeg}{3}{}{}{g:exercise:idp230420824}%
\(3, 6, 12, 24, \ldots\)\end{divisionexerciseeg}%
\begin{divisionexerciseeg}{4}{}{}{g:exercise:idp230419416}%
\(1, 2, 6, 24, \ldots\)\end{divisionexerciseeg}%
\begin{divisionexerciseeg}{5}{}{}{g:exercise:idp230427864}%
\(81, 72, 63, 54, \ldots\)\end{divisionexerciseeg}%
\begin{divisionexerciseeg}{6}{}{}{g:exercise:idp230430552}%
\(72, 48, 32,\ldots\)\end{divisionexerciseeg}%
\end{exercisegroup}
\par\medskip\noindent
\par\medskip\noindent%
\textbf{Exercise Group.}\space\space%
Give both the general formula and the recursive formula for the \(\nth{}\) term \(a_n\) of the following sequences.  Use the convention \(n \ge 1\).\begin{exercisegroup}
\begin{divisionexerciseeg}{7}{}{}{g:exercise:idp230422872}%
\(1, 3, 9, 27, \ldots\)\end{divisionexerciseeg}%
\begin{divisionexerciseeg}{8}{}{}{g:exercise:idp230430168}%
\(64, 16, 4, 1, \ldots\)\end{divisionexerciseeg}%
\begin{divisionexerciseeg}{9}{}{}{g:exercise:idp230425432}%
\(3, -6, 12, -24, \ldots\)\end{divisionexerciseeg}%
\begin{divisionexerciseeg}{10}{}{}{g:exercise:idp230425048}%
\(24, 2.4, 0.24, \ldots\)\end{divisionexerciseeg}%
\end{exercisegroup}
\par\medskip\noindent
\par\medskip\noindent%
\textbf{Exercise Group.}\space\space%
For the following sequences, calculate \(a_{50}\) and \(a_{261}\), assuming that the first term is \(a_1\).\begin{exercisegroup}
\begin{divisionexerciseeg}{11}{}{}{g:exercise:idp230431448}%
\(12, 18, 27,\ldots\)\end{divisionexerciseeg}%
\begin{divisionexerciseeg}{12}{}{}{g:exercise:idp230438232}%
\(12, 8, \frac{{16}}{3}, \ldots\)\end{divisionexerciseeg}%
\end{exercisegroup}
\par\medskip\noindent
\par\medskip\noindent%
\textbf{Exercise Group.}\space\space%
State whether the following recursively defined sequences are geometric or not.  (Is there an easy way to tell?)\begin{exercisegroup}
\begin{divisionexerciseeg}{13}{}{}{g:exercise:idp230437976}%
%
\begin{align*}
a_1\amp = 5\\
a_n \amp = a_{n - 1} + 4   \text{   for } n \ge 2
\end{align*}
\end{divisionexerciseeg}%
\begin{divisionexerciseeg}{14}{}{}{g:exercise:idp230436568}%
%
\begin{align*}
a_0 \amp = 12\\
a_n \amp = 2a_{n-1} \text{   for } n\ge 1
\end{align*}
\end{divisionexerciseeg}%
\begin{divisionexerciseeg}{15}{}{}{g:exercise:idp230432344}%
%
\begin{align*}
a_0\amp = 75\\
a_n\amp = 10a_{n-1} \text{   for } n\ge 1
\end{align*}
\end{divisionexerciseeg}%
\begin{divisionexerciseeg}{16}{}{}{g:exercise:idp230443864}%
%
\begin{align*}
a_1 \amp = 7\\
a_n\amp = 2-a_{n-1} \text{   for } n\ge 2
\end{align*}
\end{divisionexerciseeg}%
\begin{divisionexerciseeg}{17}{}{}{g:exercise:idp230442840}%
%
\begin{align*}
a_1 \amp = 8\\
a_n \amp = -a_{n-1} \text{   for } n\ge 2
\end{align*}
\end{divisionexerciseeg}%
\begin{divisionexerciseeg}{18}{}{}{g:exercise:idp230446808}%
%
\begin{align*}
a_0 \amp = 3\\
a_n\amp = (a_{n-1})^2 \text{   for } n\ge 1
\end{align*}
\end{divisionexerciseeg}%
\end{exercisegroup}
\par\medskip\noindent
\begin{divisionexercise}{19}{}{}{g:exercise:idp230445400}%
For the following sequence, calculate the \(\upst{201}\) term:  \(5, 15, 45, \ldots\)\end{divisionexercise}%
\begin{divisionexercise}{20}{}{}{g:exercise:idp230446040}%
For the following sequence, calculate the \(\upth{20}\) term:  \(7, -14, 28, \ldots\)\end{divisionexercise}%
\begin{divisionexercise}{21}{}{}{g:exercise:idp230448344}%
Calculate \(S_{20}\) for the series \(100 + 50 + 25 + \ldots\)\end{divisionexercise}%
\begin{divisionexercise}{22}{}{}{g:exercise:idp230455000}%
Calculate \(S_{20}\) for the series \(100 + 200 + 400 + \ldots\)\end{divisionexercise}%
\par\medskip\noindent%
\textbf{Exercise Group.}\space\space%
Calculate the sum, if it exists, for the following series.\begin{exercisegroup}
\begin{divisionexerciseeg}{23}{}{}{g:exercise:idp230452696}%
\(- 6 + 4 - \frac{8}{3} + \ldots\)\end{divisionexerciseeg}%
\begin{divisionexerciseeg}{24}{}{}{g:exercise:idp230450648}%
\(100 + 50 + 25 + \ldots\)\end{divisionexerciseeg}%
\begin{divisionexerciseeg}{25}{}{}{g:exercise:idp230451416}%
\(100 + 200 + 400 + \ldots\)\end{divisionexerciseeg}%
\begin{divisionexerciseeg}{26}{}{}{g:exercise:idp230454232}%
\(12 + 3 + \frac{3}{4}+ \ldots\)\end{divisionexerciseeg}%
\end{exercisegroup}
\par\medskip\noindent
\par\medskip\noindent%
\textbf{Exercise Group.}\space\space%
Calculate the following sums, if they exist.\begin{exercisegroup}
\begin{divisionexerciseeg}{27}{}{}{g:exercise:idp230457688}%
\(\sum\limits_{k = 0}^{10} {{2^{k + 2}}}\)\end{divisionexerciseeg}%
\begin{divisionexerciseeg}{28}{}{}{g:exercise:idp230457048}%
\({\sum\limits_{j = 1}^\infty  {15\left( {\frac{3}{5}} \right)} ^j}\)\end{divisionexerciseeg}%
\begin{divisionexerciseeg}{29}{}{}{g:exercise:idp230456152}%
\(\sum\limits_{i = 2}^\infty  {25{{\left( {0.1} \right)}^j}}\)\end{divisionexerciseeg}%
\begin{divisionexerciseeg}{30}{}{}{g:exercise:idp230463064}%
\(\sum\limits_{i = 0}^\infty  {4{{\left( { - 3} \right)}^i}}\)\end{divisionexerciseeg}%
\end{exercisegroup}
\par\medskip\noindent
\begin{divisionexercise}{31}{}{}{g:exercise:idp230462168}%
If the number of vampires in Transylvania doubles every month, then how many vampires will be in Transylvania in 3 years, starting from one individual?  Comment on your result if the total population of Transylvania is 2 million people.\end{divisionexercise}%
\begin{divisionexercise}{32}{}{}{g:exercise:idp230459992}%
As I was going to St. Ives, I met a man with seven wives.  Each wife had seven sacks.  Each sack had seven cats.  Each cat had seven kits.  Kits, cats, sacks, wives: does this form a geometric sequence?\end{divisionexercise}%
\begin{divisionexercise}{33}{}{}{g:exercise:idp230465880}%
The paper used in the photocopier by Pat's office is said to be 0.097\textasciitilde{}mm thick.  If it is folded over repeatedly, doubling its thickness each time, how thick will the paper be if it's folded 7 times?  Bonus:  why, then, were the Mythbusters having so many problems trying to fold the paper this many times?\end{divisionexercise}%
\end{exercises-subsection}
%
%
\typeout{************************************************}
\typeout{Solutions 3.3.9 Solutions to Section~{\xreffont\ref*{x:section:sec-geom-seq-ser}} Exercises}
\typeout{************************************************}
%
\begin{solutions-subsection}{Solutions to Section~{\xreffont\ref*{x:section:sec-geom-seq-ser}} Exercises}{}{Solutions to Section~{\xreffont\ref*{x:section:sec-geom-seq-ser}} Exercises}{}{}{g:solutions:idp230209240}
\par\medskip
\noindent\textbf{\normalsize{}3.3.8\space\textperiodcentered\space{}Exercises}
\begin{exercisegroup}
\begin{divisionsolutioneg}{3.3.8.1}{}{g:exercise:idp230421208}%
\par\smallskip%
\noindent\hypertarget{g:solution:idp230416984-main}{}no\end{divisionsolutioneg}%
\begin{divisionsolutioneg}{3.3.8.2}{}{g:exercise:idp230417112}%
\par\smallskip%
\noindent\hypertarget{g:solution:idp230418264-main}{}no\end{divisionsolutioneg}%
\begin{divisionsolutioneg}{3.3.8.3}{}{g:exercise:idp230420824}%
\par\smallskip%
\noindent\hypertarget{g:solution:idp230418520-main}{}yes, \(r=2\)\end{divisionsolutioneg}%
\begin{divisionsolutioneg}{3.3.8.4}{}{g:exercise:idp230419416}%
\par\smallskip%
\noindent\hypertarget{g:solution:idp230421592-main}{}no\end{divisionsolutioneg}%
\begin{divisionsolutioneg}{3.3.8.5}{}{g:exercise:idp230427864}%
\par\smallskip%
\noindent\hypertarget{g:solution:idp230424152-main}{}no\end{divisionsolutioneg}%
\begin{divisionsolutioneg}{3.3.8.6}{}{g:exercise:idp230430552}%
\par\smallskip%
\noindent\hypertarget{g:solution:idp230429016-main}{}yes, \(r=\frac{2}{3}\)\end{divisionsolutioneg}%
\end{exercisegroup}
\par\medskip\noindent
\begin{exercisegroup}
\begin{divisionsolutioneg}{3.3.8.7}{}{g:exercise:idp230422872}%
\par\smallskip%
\noindent\hypertarget{g:solution:idp230424280-main}{}\({a_n} = {\left( 3 \right)^{n - 1}}\) and%
\begin{equation*}
\left\{ \begin{array}{l}
{a_1} = 1\\
{a_n} = 3{a_{n - 1}}
\end{array} \right.
\end{equation*}
\end{divisionsolutioneg}%
\begin{divisionsolutioneg}{3.3.8.8}{}{g:exercise:idp230430168}%
\par\smallskip%
\noindent\hypertarget{g:solution:idp230424536-main}{}\({a_n} = 64{\left( {\frac{1}{4}} \right)^{n - 1}}\) and%
\begin{equation*}
\left\{ \begin{array}{l}
{a_1} = 64\\
{a_n} = \frac{{{a_{n - 1}}}}{4}
\end{array} \right.
\end{equation*}
\end{divisionsolutioneg}%
\begin{divisionsolutioneg}{3.3.8.9}{}{g:exercise:idp230425432}%
\par\smallskip%
\noindent\hypertarget{g:solution:idp230424792-main}{}\({a_n} = 3{\left( { - 2} \right)^{n - 1}}\) and%
\begin{equation*}
\left\{ \begin{array}{l}
{a_1} = 3\\
{a_n} =  - 2{a_{n - 1}}
\end{array} \right.
\end{equation*}
\end{divisionsolutioneg}%
\begin{divisionsolutioneg}{3.3.8.10}{}{g:exercise:idp230425048}%
\par\smallskip%
\noindent\hypertarget{g:solution:idp230430424-main}{}\({a_n} = 24{\left( {0.1} \right)^{n - 1}}\) and%
\begin{equation*}
\left\{ \begin{array}{l}
{a_1} = 24\\
{a_n} = 0.1 \times {a_{n - 1}}
\end{array} \right.
\end{equation*}
\end{divisionsolutioneg}%
\end{exercisegroup}
\par\medskip\noindent
\begin{exercisegroup}
\begin{divisionsolutioneg}{3.3.8.11}{}{g:exercise:idp230431448}%
\par\smallskip%
\noindent\hypertarget{g:solution:idp230438488-main}{}\({a_n} = 12{\left( {\frac{3}{2}} \right)^{n - 1}}\), so \(a_{50} \approx 5.1\times10^9\) and \(a_{261} \approx 7.3\times10^{46}\)\end{divisionsolutioneg}%
\begin{divisionsolutioneg}{3.3.8.12}{}{g:exercise:idp230438232}%
\par\smallskip%
\noindent\hypertarget{g:solution:idp230434392-main}{}\({a_n} = 12{\left( {\frac{2}{3}} \right)^{n - 1}}\), so \(a_{50} \approx 2.8\times10^{-8}\) and \(a_{261} \approx 1.97\times10^{-45}\)\end{divisionsolutioneg}%
\end{exercisegroup}
\par\medskip\noindent
\begin{exercisegroup}
\begin{divisionsolutioneg}{3.3.8.13}{}{g:exercise:idp230437976}%
\par\smallskip%
\noindent\hypertarget{g:solution:idp230435544-main}{}no\end{divisionsolutioneg}%
\begin{divisionsolutioneg}{3.3.8.14}{}{g:exercise:idp230436568}%
\par\smallskip%
\noindent\hypertarget{g:solution:idp230437464-main}{}yes, with \(r=2\)\end{divisionsolutioneg}%
\begin{divisionsolutioneg}{3.3.8.15}{}{g:exercise:idp230432344}%
\par\smallskip%
\noindent\hypertarget{g:solution:idp230443608-main}{}yes, with \(r=10\)\end{divisionsolutioneg}%
\begin{divisionsolutioneg}{3.3.8.16}{}{g:exercise:idp230443864}%
\par\smallskip%
\noindent\hypertarget{g:solution:idp230445016-main}{}no\end{divisionsolutioneg}%
\begin{divisionsolutioneg}{3.3.8.17}{}{g:exercise:idp230442840}%
\par\smallskip%
\noindent\hypertarget{g:solution:idp230446552-main}{}yes, with \(r=-1\)\end{divisionsolutioneg}%
\begin{divisionsolutioneg}{3.3.8.18}{}{g:exercise:idp230446808}%
\par\smallskip%
\noindent\hypertarget{g:solution:idp230447320-main}{}no\end{divisionsolutioneg}%
\end{exercisegroup}
\par\medskip\noindent
\begin{divisionsolution}{3.3.8.19}{}{g:exercise:idp230445400}%
\par\smallskip%
\noindent\hypertarget{g:solution:idp230439896-main}{}\({a_n} = 5{\left( 3 \right)^{n - 1}}\), so \(a_{201} = 5(3)^{200} = 1.33 \times 10^{96}\)\end{divisionsolution}%
\begin{divisionsolution}{3.3.8.20}{}{g:exercise:idp230446040}%
\par\smallskip%
\noindent\hypertarget{g:solution:idp230447064-main}{}\({a_n} = 7{\left( { - 2} \right)^{n - 1}}\), so \(a_{20} = 7(-2)^{19} = -3,670,016\)\end{divisionsolution}%
\begin{divisionsolution}{3.3.8.21}{}{g:exercise:idp230448344}%
\par\smallskip%
\noindent\hypertarget{g:solution:idp230450136-main}{}\(S_{20} = 200\) (the exact answer is \(\frac{{26214375}}{{131072}}\) or 199.99980926513671875, but if you round to three decimals, the answer is 200.000)\end{divisionsolution}%
\begin{divisionsolution}{3.3.8.22}{}{g:exercise:idp230455000}%
\par\smallskip%
\noindent\hypertarget{g:solution:idp230448856-main}{}\(S_{20} = 104,857,500\)\end{divisionsolution}%
\begin{exercisegroup}
\begin{divisionsolutioneg}{3.3.8.23}{}{g:exercise:idp230452696}%
\par\smallskip%
\noindent\hypertarget{g:solution:idp230452824-main}{}\({S_\infty } = \frac{{{a_1}}}{{1 - r}} = \frac{{ - 6}}{{1 - \left( { - 2/3} \right)}} =  - \frac{{18}}{5} =  - 3.6\)\end{divisionsolutioneg}%
\begin{divisionsolutioneg}{3.3.8.24}{}{g:exercise:idp230450648}%
\par\smallskip%
\noindent\hypertarget{g:solution:idp230449624-main}{}\(S_\infinity = 200\)\end{divisionsolutioneg}%
\begin{divisionsolutioneg}{3.3.8.25}{}{g:exercise:idp230451416}%
\par\smallskip%
\noindent\hypertarget{g:solution:idp230453080-main}{}\(S_\infinity\) does not exist (\(r \gt 1\))\end{divisionsolutioneg}%
\begin{divisionsolutioneg}{3.3.8.26}{}{g:exercise:idp230454232}%
\par\smallskip%
\noindent\hypertarget{g:solution:idp230447704-main}{}\(S_\infinity = 16\)\end{divisionsolutioneg}%
\end{exercisegroup}
\par\medskip\noindent
\begin{exercisegroup}
\begin{divisionsolutioneg}{3.3.8.27}{}{g:exercise:idp230457688}%
\par\smallskip%
\noindent\hypertarget{g:solution:idp230457816-main}{}\({S_{11}} = {2^2} + {2^3} + {2^4} + ... + {2^{12}} = \frac{{{a_1}(1 - {r^n})}}{{1 - r}} = \frac{{{2^2}(1 - {2^{11}})}}{{1 - 2}}= 8188\)\end{divisionsolutioneg}%
\begin{divisionsolutioneg}{3.3.8.28}{}{g:exercise:idp230457048}%
\par\smallskip%
\noindent\hypertarget{g:solution:idp230463192-main}{}\(S_\infinity  = 22.5\)\end{divisionsolutioneg}%
\begin{divisionsolutioneg}{3.3.8.29}{}{g:exercise:idp230456152}%
\par\smallskip%
\noindent\hypertarget{g:solution:idp230461784-main}{}\(S_\infinity = \frac{5}{{18}} = 0.2\overline 7\)\end{divisionsolutioneg}%
\begin{divisionsolutioneg}{3.3.8.30}{}{g:exercise:idp230463064}%
\par\smallskip%
\noindent\hypertarget{g:solution:idp230462680-main}{}\(S_\infinity\) does not exist (\(r \lt -1\))\end{divisionsolutioneg}%
\end{exercisegroup}
\par\medskip\noindent
\begin{divisionsolution}{3.3.8.31}{}{g:exercise:idp230462168}%
\par\smallskip%
\noindent\hypertarget{g:solution:idp230456536-main}{}3 years is 36 months, so we have a 36-term sequence starting with \(1, 2, 4, 8, \ldots\)  The \(\nth{}\) term will be \({a_n} = 1{\left( 2 \right)^{n - 1}}\), so the \(\upth{36}\) term will be \({a_{36}} = 1{\left( 2 \right)^{35}} = 34,359,738,368\), which is a tad larger than the total population of Transylvania.\end{divisionsolution}%
\begin{divisionsolution}{3.3.8.32}{}{g:exercise:idp230459992}%
\par\smallskip%
\noindent\hypertarget{g:solution:idp230460120-main}{}1 man%
\par
7 wives%
\par
\(\# \text{sacks} = \#\text{wives} \times \# \text{sacks/wife} = 7 \times 7 = 49\)%
\par
\(\# \text{cats} = \#\text{sacks} \times \# \text{cats/sack} = 49 \times 7 = 343\)%
\par
\(\# \text{kits} = \#\text{cats} \times \# \text{kits/cat} = 343 \times 7 = 2401\)%
\par
So kits, cats, sacks, and wives is the sequence \(2401, 373, 49, 7\), which is geometric with four terms:  \(a_1 = 2401\) and \(r = \frac{1}{7}\).%
\end{divisionsolution}%
\begin{divisionsolution}{3.3.8.33}{}{g:exercise:idp230465880}%
\par\smallskip%
\noindent\hypertarget{g:solution:idp230464856-main}{}The paper is initially 0.097 mm thick with no folds.  After one fold, the thickness will be \(0.097\times2\), after two folds \(0.097\times2\times2\), etc.  So our starting term (zero folds) will be \(a_0=0.097\) and then will double with r = 2 thereafter, where \(n\) is not only the index but also the number of folds made. So \({a_n} = 0.097{\left( 2 \right)^n}\), and the term with seven folds will be \({a_7} = 0.097{\left( 2 \right)^7} = 12.416\), so we can conclude that the paper thickness will be 12.4 mm, or just over 1 cm thick. (The Mythbusters realized that the problems with paperfolding lie with the fold itself, and making the fold lie as flat as possible.  If I remember correctly, they resorted to C-clamps and hitting the fold with a hammer to flatten it.)\end{divisionsolution}%
\end{solutions-subsection}
\end{sectionptx}
\end{chapterptx}
%
\backmatter%
%
\clearpage\phantomsection%
\addcontentsline{toc}{part}{Back Matter}%
%
%% The index is here, setup is all in preamble
%% Index locators are cross-references, so same font here
{\xreffont\printindex}
%
\end{document}